
\begin{DoxyItemize}
\item \hyperlink{introduction_install}{I\-N\-S\-T\-A\-L\-L Installing Free\-Mat}  
\end{DoxyItemize}\hypertarget{introduction_install}{}\section{I\-N\-S\-T\-A\-L\-L Installing Free\-Mat}\label{introduction_install}
Section\-: \hyperlink{sec_introduction}{Introduction and Getting Started} \hypertarget{introduction_install_General}{}\subsection{Instructions}\label{introduction_install_General}
Here are the general instructions for installing Free\-Mat. First, follow the instructions listed below for the platform of interest. Then, run the \begin{DoxyVerb}-->pathtool
\end{DoxyVerb}
 which brings up the path setup tool. More documentation on the G\-U\-I elements (and how to use them) will be forthcoming. \hypertarget{introduction_install_Linux}{}\subsection{Linux}\label{introduction_install_Linux}
For Linux, Free\-Mat is now provided as a binary installation. To install it simply download the binary using your web browser, and then unpack it \begin{DoxyVerb}  tar xvfz FreeMat-\<VERSION_NUMBER\>-Linux-Binary.tar.gz
\end{DoxyVerb}
 You can then run Free\-Mat directly without any additional effort \begin{DoxyVerb}  FreeMat-\<VERSION_NUMBER\>-Linux-Binary/Contents/bin/FreeMat
\end{DoxyVerb}
 will start up Free\-Mat as an X application. If you want to run it as a command line application (to run from within an xterm), use the {\ttfamily nogui} flag \begin{DoxyVerb}  FreeMat-\<VERSION_NUMBER\>-Linux-Binary/Contents/bin/FreeMat -nogui
\end{DoxyVerb}
 If you do not want Free\-Mat to use X at all (no graphics at all), use the {\ttfamily no\-X} flag \begin{DoxyVerb}  FreeMat-\<VERSION_NUMBER\>-Linux-Binary/Contents/bin/FreeMat -noX
\end{DoxyVerb}
 For convenience, you may want to add Free\-Mat to your path. The exact mechanism for doing this depends on your shell. Assume that you have unpacked {\ttfamily Free\-Mat-\/$<$V\-E\-R\-S\-I\-O\-N\-\_\-\-N\-U\-M\-B\-E\-R$>$-\/\-Linux-\/\-Binary.\-tar.\-gz} into the directory {\ttfamily /home/myname}. Then if you use {\ttfamily csh} or its derivatives (like {\ttfamily tcsh}) you should add the following line to your {\ttfamily .cshrc} file\-: \begin{DoxyVerb}  set path=($path /home/myname/FreeMat-\<VERSION_NUMBER\>-Linux/Binary/Contents/bin)
\end{DoxyVerb}
 If you use {\ttfamily bash}, then add the following line to your {\ttfamily .bash\-\_\-profile} \begin{DoxyVerb}  PATH=$PATH:/home/myname/FreeMat-\<VERSION_NUMBER\>-Linux/Binary/Contents/bin
\end{DoxyVerb}
 If the prebuilt binary package does not work for your Linux distribution, you will need to build Free\-Mat from source (see the source section below). When you have Free\-Mat running, you can setup your path using the {\ttfamily pathtool}. Note that the {\ttfamily F\-R\-E\-E\-M\-A\-T\-\_\-\-P\-A\-T\-H} is no longer used by Free\-Mat. You must use the {\ttfamily pathtool} to adjust the path. \hypertarget{introduction_install_Windows}{}\subsection{Windows}\label{introduction_install_Windows}
For Windows, Free\-Mat is installed via a binary installer program. To use it, simply download the setup program {\ttfamily Free\-Mat-\/$<$V\-E\-R\-S\-I\-O\-N\-\_\-\-N\-U\-M\-B\-E\-R$>$-\/\-Setup.\-exe}, and double click it. Follow the instructions to do the installation, then setup your path using {\ttfamily pathtool}. \hypertarget{introduction_install_Mac}{}\subsection{O\-S X}\label{introduction_install_Mac}
For Mac O\-S X, Free\-Mat is distributed as an application bundle. To install it, simply download the compressed disk image file {\ttfamily Free\-Mat-\/$<$V\-E\-R\-S\-I\-O\-N\-\_\-\-N\-U\-M\-B\-E\-R$>$.dmg}, double click to mount the disk image, and then copy the application {\ttfamily Free\-Mat-\/$<$V\-E\-R\-S\-I\-O\-N\-\_\-\-N\-U\-M\-B\-E\-R$>$} to some convenient place. To run Free\-Mat, simply double click on the application. Run {\ttfamily pathtool} to setup your Free\-Mat path. \hypertarget{introduction_install_Source}{}\subsection{Code}\label{introduction_install_Source}
The source code build is a little more complicated than previous versions of Free\-Mat. Here are the current build instructions for all platforms. 
\begin{DoxyEnumerate}
\item Build and install Qt 4.\-3 or later -\/ {\ttfamily \href{http://trolltech.com/developer/downloads/opensource}{\tt http\-://trolltech.\-com/developer/downloads/opensource}}  
\item Install g77 or gfortran (use fink for Mac O\-S X, use {\ttfamily gcc-\/g77} package for Min\-G\-W)  
\item Download the source code {\ttfamily Free\-Mat-\/$<$V\-E\-R\-S\-I\-O\-N\-\_\-\-N\-U\-M\-B\-E\-R$>$-\/src.\-tar.\-gz}.  
\item Unpack the source code\-: {\ttfamily tar xvfz Free\-Mat-\/$<$V\-E\-R\-S\-I\-O\-N\-\_\-\-N\-U\-M\-B\-E\-R$>$-\/src.\-tar.\-gz}.  
\item For Windows, you will need to install M\-S\-Y\-S as well as M\-I\-N\-G\-W to build Free\-Mat. You will also need unzip to unpack the enclosed matio.\-zip archive. Alternately, you can cross-\/build the W\-Indows version of Free\-Mat under Linux (this is how I build it now).  
\item If you are extraordinarily lucky (or prepared), you can issue the usual ./configure, then the make and make install. This is not likely to work because of the somewhat esoteric dependencies of Free\-Mat. The configure step will probably fail and indicate what external dependencies are still needed.  
\item I assume that you are familiar with the process of installing dependencies if you are trying to build Free\-Mat from source.  
\end{DoxyEnumerate}To build a binary distributable (app bundle on the Mac, setup installer on win32, and a binary distribution on Linux), you will need to run {\ttfamily make package} instead of {\ttfamily make install}. 