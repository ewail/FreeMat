
\begin{DoxyItemize}
\item \hyperlink{debug_dbauto}{D\-B\-A\-U\-T\-O Control Dbauto Functionality}  
\item \hyperlink{debug_dbdelete}{D\-B\-D\-E\-L\-E\-T\-E Delete a Breakpoint}  
\item \hyperlink{debug_dbdown}{D\-B\-Down Move Down One Debug Level}  
\item \hyperlink{debug_dblist}{D\-B\-L\-I\-S\-T List Breakpoints}  
\item \hyperlink{debug_dbstep}{D\-B\-S\-T\-E\-P Step N Statements}  
\item \hyperlink{debug_dbstop}{D\-B\-S\-T\-O\-P}  
\item \hyperlink{debug_dbup}{D\-B\-U\-P Move Up One Debug Level}  
\item \hyperlink{debug_fdump}{F\-D\-U\-M\-P Dump Information on Function}  
\end{DoxyItemize}\hypertarget{debug_dbauto}{}\section{D\-B\-A\-U\-T\-O Control Dbauto Functionality}\label{debug_dbauto}
Section\-: \hyperlink{sec_debug}{Debugging Free\-Mat Code} \hypertarget{vtkwidgets_vtkxyplotwidget_Usage}{}\subsection{Usage}\label{vtkwidgets_vtkxyplotwidget_Usage}
The dbauto functionality in Free\-Mat allows you to debug your Free\-Mat programs. When {\ttfamily dbauto} is {\ttfamily on}, then any error that occurs while the program is running causes Free\-Mat to stop execution at that point and return you to the command line (just as if you had placed a {\ttfamily keyboard} command there). You can then examine variables, modify them, and resume execution using {\ttfamily return}. Alternately, you can exit out of all running routines via a {\ttfamily retall} statement. Note that errors that occur inside of {\ttfamily try}/{\ttfamily catch} blocks do not (by design) cause auto breakpoints. The {\ttfamily dbauto} function toggles the dbauto state of Free\-Mat. The syntax for its use is \begin{DoxyVerb}   dbauto(state)
\end{DoxyVerb}
 where {\ttfamily state} is either \begin{DoxyVerb}   dbauto('on')
\end{DoxyVerb}
 to activate dbauto, or \begin{DoxyVerb}   dbauto('off')
\end{DoxyVerb}
 to deactivate dbauto. Alternately, you can use Free\-Mat's string-\/syntax equivalence and enter \begin{DoxyVerb}   dbauto on
\end{DoxyVerb}
 or \begin{DoxyVerb}   dbauto off
\end{DoxyVerb}
 to turn dbauto on or off (respectively). Entering {\ttfamily dbauto} with no arguments returns the current state (either 'on' or 'off'). \hypertarget{debug_dbdelete}{}\section{D\-B\-D\-E\-L\-E\-T\-E Delete a Breakpoint}\label{debug_dbdelete}
Section\-: \hyperlink{sec_debug}{Debugging Free\-Mat Code} \hypertarget{vtkwidgets_vtkxyplotwidget_Usage}{}\subsection{Usage}\label{vtkwidgets_vtkxyplotwidget_Usage}
The {\ttfamily dbdelete} function deletes a breakpoint. The syntax for the {\ttfamily dbdelete} function is \begin{DoxyVerb}  dbdelete(num)
\end{DoxyVerb}
 where {\ttfamily num} is the number of the breakpoint to delete. \hypertarget{debug_dbdown}{}\section{D\-B\-Down Move Down One Debug Level}\label{debug_dbdown}
Section\-: \hyperlink{sec_debug}{Debugging Free\-Mat Code} \hypertarget{vtkwidgets_vtkxyplotwidget_Usage}{}\subsection{Usage}\label{vtkwidgets_vtkxyplotwidget_Usage}
The {\ttfamily dbdown} function moves up one level in the debug hierarchy. The syntax for the {\ttfamily dbdown} function is \begin{DoxyVerb} dbdown
\end{DoxyVerb}
 \hypertarget{debug_dblist}{}\section{D\-B\-L\-I\-S\-T List Breakpoints}\label{debug_dblist}
Section\-: \hyperlink{sec_debug}{Debugging Free\-Mat Code} \hypertarget{vtkwidgets_vtkxyplotwidget_Usage}{}\subsection{Usage}\label{vtkwidgets_vtkxyplotwidget_Usage}
List the current set of breakpoints. The syntax for the {\ttfamily dblist} is simply \begin{DoxyVerb}  dblist
\end{DoxyVerb}
 \hypertarget{debug_dbstep}{}\section{D\-B\-S\-T\-E\-P Step N Statements}\label{debug_dbstep}
Section\-: \hyperlink{sec_debug}{Debugging Free\-Mat Code} \hypertarget{vtkwidgets_vtkxyplotwidget_Usage}{}\subsection{Usage}\label{vtkwidgets_vtkxyplotwidget_Usage}
Step {\ttfamily N} statements during debug mode. The synax for this is either \begin{DoxyVerb}  dbstep(N)
\end{DoxyVerb}
 to step {\ttfamily N} statements, or \begin{DoxyVerb}  dbstep
\end{DoxyVerb}
 to step one statement. \hypertarget{debug_dbstop}{}\section{D\-B\-S\-T\-O\-P}\label{debug_dbstop}
Section\-: \hyperlink{sec_debug}{Debugging Free\-Mat Code} \hypertarget{vtkwidgets_vtkxyplotwidget_Usage}{}\subsection{Usage}\label{vtkwidgets_vtkxyplotwidget_Usage}
Set a breakpoint. The syntax for this is\-: \begin{DoxyVerb}  dbstop(funcname,linenumber)
\end{DoxyVerb}
 where {\ttfamily funcname} is the name of the function where we want to set the breakpoint, and {\ttfamily linenumber} is the line number. \hypertarget{debug_dbup}{}\section{D\-B\-U\-P Move Up One Debug Level}\label{debug_dbup}
Section\-: \hyperlink{sec_debug}{Debugging Free\-Mat Code} \hypertarget{vtkwidgets_vtkxyplotwidget_Usage}{}\subsection{Usage}\label{vtkwidgets_vtkxyplotwidget_Usage}
The {\ttfamily dbup} function moves up one level in the debug hierarchy. The syntax for the {\ttfamily dbup} function is \begin{DoxyVerb} dbup
\end{DoxyVerb}
 \hypertarget{debug_fdump}{}\section{F\-D\-U\-M\-P Dump Information on Function}\label{debug_fdump}
Section\-: \hyperlink{sec_debug}{Debugging Free\-Mat Code} \hypertarget{vtkwidgets_vtkxyplotwidget_Usage}{}\subsection{Usage}\label{vtkwidgets_vtkxyplotwidget_Usage}
Dumps information about a function (diagnostic information only) \begin{DoxyVerb}   fdump fname
\end{DoxyVerb}
 