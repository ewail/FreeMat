
\begin{DoxyItemize}
\item \hyperlink{constants_e}{E Euler Constant (Base of Natural Logarithm)}  
\item \hyperlink{constants_eps}{E\-P\-S Double Precision Floating Point Relative Machine Precision Epsilon}  
\item \hyperlink{constants_false}{F\-A\-L\-S\-E Logical False}  
\item \hyperlink{constants_feps}{F\-E\-P\-S Single Precision Floating Point Relative Machine Precision Epsilon}  
\item \hyperlink{constants_i}{I-\/\-J Square Root of Negative One}  
\item \hyperlink{constants_inf}{I\-N\-F Infinity Constant}  
\item \hyperlink{constants_pi}{P\-I Constant Pi}  
\item \hyperlink{constants_teps}{T\-E\-P\-S Type-\/based Epsilon Calculation}  
\item \hyperlink{constants_true}{T\-R\-U\-E Logical T\-R\-U\-E}  
\end{DoxyItemize}\hypertarget{constants_e}{}\section{E Euler Constant (Base of Natural Logarithm)}\label{constants_e}
Section\-: \hyperlink{sec_constants}{Base Constants} \hypertarget{vtkwidgets_vtkxyplotwidget_Usage}{}\subsection{Usage}\label{vtkwidgets_vtkxyplotwidget_Usage}
Returns a {\ttfamily double} (64-\/bit floating point number) value that represents Euler's constant, the base of the natural logarithm. Typical usage \begin{DoxyVerb}   y = e
\end{DoxyVerb}
 This value is approximately 2.\-718281828459045. \hypertarget{variables_struct_Example}{}\subsection{Example}\label{variables_struct_Example}
The following example demonstrates the use of the {\ttfamily e} function.


\begin{DoxyVerbInclude}
--> e

ans = 
    2.7183 

--> log(e)

ans = 
 1 
\end{DoxyVerbInclude}
 \hypertarget{constants_eps}{}\section{E\-P\-S Double Precision Floating Point Relative Machine Precision Epsilon}\label{constants_eps}
Section\-: \hyperlink{sec_constants}{Base Constants} \hypertarget{vtkwidgets_vtkxyplotwidget_Usage}{}\subsection{Usage}\label{vtkwidgets_vtkxyplotwidget_Usage}
Returns {\ttfamily eps}, which quantifies the relative machine precision of floating point numbers (a machine specific quantity). The syntax for {\ttfamily eps} is\-: \begin{DoxyVerb}   y = eps
   y = eps('double')
   y = eps(X)
\end{DoxyVerb}
 First form returns {\ttfamily eps} for {\ttfamily double} precision values. For most typical processors, this value is approximately {\ttfamily 2$^\wedge$-\/52}, or 2.\-2204e-\/16. Second form return {\ttfamily eps} for class {\ttfamily double} or {\ttfamily single}. Third form returns distance to the next value greater than X. \hypertarget{variables_struct_Example}{}\subsection{Example}\label{variables_struct_Example}
The following example demonstrates the use of the {\ttfamily eps} function, and one of its numerical consequences.


\begin{DoxyVerbInclude}
--> eps

ans = 
 2.2204e-16 

--> 1.0+eps

ans = 
    1.0000 

--> eps(1000.)

ans = 
 1.1369e-13 
\end{DoxyVerbInclude}
 \hypertarget{constants_false}{}\section{F\-A\-L\-S\-E Logical False}\label{constants_false}
Section\-: \hyperlink{sec_constants}{Base Constants} \hypertarget{vtkwidgets_vtkxyplotwidget_Usage}{}\subsection{Usage}\label{vtkwidgets_vtkxyplotwidget_Usage}
Returns a logical 0. The syntax for its use is \begin{DoxyVerb}   y = false
\end{DoxyVerb}
 You can also create an array of logical ones using the syntax \begin{DoxyVerb}   y = false(d1,d2,...,dn)
\end{DoxyVerb}
 or the syntax \begin{DoxyVerb}   y = false([d1,d2,...,dn])
\end{DoxyVerb}
 \hypertarget{constants_feps}{}\section{F\-E\-P\-S Single Precision Floating Point Relative Machine Precision Epsilon}\label{constants_feps}
Section\-: \hyperlink{sec_constants}{Base Constants} \hypertarget{vtkwidgets_vtkxyplotwidget_Usage}{}\subsection{Usage}\label{vtkwidgets_vtkxyplotwidget_Usage}
Returns {\ttfamily feps}, which quantifies the relative machine precision of floating point numbers (a machine specific quantity). The syntax for {\ttfamily feps} is\-: \begin{DoxyVerb}   y = feps
\end{DoxyVerb}
 which returns {\ttfamily feps} for {\ttfamily single} precision values. For most typical processors, this value is approximately {\ttfamily 2$^\wedge$-\/24}, or 5.\-9604e-\/8. \hypertarget{variables_struct_Example}{}\subsection{Example}\label{variables_struct_Example}
The following example demonstrates the use of the {\ttfamily feps} function, and one of its numerical consequences.


\begin{DoxyVerbInclude}
--> feps

ans = 
 1.1921e-07 

--> 1.0f+eps

ans = 
 1 
\end{DoxyVerbInclude}
 \hypertarget{constants_i}{}\section{I-\/\-J Square Root of Negative One}\label{constants_i}
Section\-: \hyperlink{sec_constants}{Base Constants} \hypertarget{vtkwidgets_vtkxyplotwidget_Usage}{}\subsection{Usage}\label{vtkwidgets_vtkxyplotwidget_Usage}
Returns a {\ttfamily complex} value that represents the square root of -\/1. There are two functions that return the same value\-: \begin{DoxyVerb}   y = i
\end{DoxyVerb}
 and \begin{DoxyVerb}   y = j.
\end{DoxyVerb}
 This allows either {\ttfamily i} or {\ttfamily j} to be used as loop indices. The returned value is a 32-\/bit complex value. \hypertarget{variables_struct_Example}{}\subsection{Example}\label{variables_struct_Example}
The following examples demonstrate a few calculations with {\ttfamily i}.


\begin{DoxyVerbInclude}
--> i

ans = 
   0.0000 +  1.0000i 

--> i^2

ans = 
  -1.0000 +  0.0000i 
\end{DoxyVerbInclude}


The same calculations with {\ttfamily j}\-:


\begin{DoxyVerbInclude}
--> j

ans = 
   0.0000 +  1.0000i 

--> j^2

ans = 
  -1.0000 +  0.0000i 
\end{DoxyVerbInclude}


Here is an example of how {\ttfamily i} can be used as a loop index and then recovered as the square root of -\/1.


\begin{DoxyVerbInclude}
--> accum = 0; for i=1:100; accum = accum + i; end; accum

ans = 
 5050 

--> i

ans = 
 100 

--> clear i
--> i

ans = 
   0.0000 +  1.0000i 
\end{DoxyVerbInclude}
 \hypertarget{constants_inf}{}\section{I\-N\-F Infinity Constant}\label{constants_inf}
Section\-: \hyperlink{sec_constants}{Base Constants} \hypertarget{vtkwidgets_vtkxyplotwidget_Usage}{}\subsection{Usage}\label{vtkwidgets_vtkxyplotwidget_Usage}
Returns a value that represents positive infinity for both 32 and 64-\/bit floating point values. There are several forms for the {\ttfamily Inf} function. The first form returns a double precision {\ttfamily Inf}. \begin{DoxyVerb}   y = inf
\end{DoxyVerb}
 The next form takes a class name that can be either {\ttfamily 'double'} \begin{DoxyVerb}   y = inf('double')
\end{DoxyVerb}
 or {\ttfamily 'single'}\-: \begin{DoxyVerb}   y = inf('single')
\end{DoxyVerb}
 With a single parameter it generates a square matrix of {\ttfamily inf}s. \begin{DoxyVerb}   y = inf(n)
\end{DoxyVerb}
 Alternatively, you can specify the dimensions of the array via \begin{DoxyVerb}   y = inf(m,n,p,...)
\end{DoxyVerb}
 or \begin{DoxyVerb}   y = inf([m,n,p,...])
\end{DoxyVerb}
 Finally, you can add a classname of either {\ttfamily 'single'} or {\ttfamily 'double'}. \hypertarget{transforms_svd_Function}{}\subsection{Internals}\label{transforms_svd_Function}
The infinity constant has several interesting properties. In particular\-: \[ \begin{array}{ll} \infty \times 0 & = \mathrm{NaN} \\ \infty \times a & = \infty \, \mathrm{for all} \, a > 0 \\ \infty \times a & = -\infty \, \mathrm{for all} \, a < 0 \\ \infty / \infty & = \mathrm{NaN} \\ \infty / 0 & = \infty \end{array} \] Note that infinities are not preserved under type conversion to integer types (see the examples below). \hypertarget{variables_struct_Example}{}\subsection{Example}\label{variables_struct_Example}
The following examples demonstrate the various properties of the infinity constant.


\begin{DoxyVerbInclude}
--> inf*0

ans = 
 NaN 

--> inf*2

ans = 
 Inf 

--> inf*-2

ans = 
 -Inf 

--> inf/inf

ans = 
 NaN 

--> inf/0

ans = 
 Inf 

--> inf/nan

ans = 
 NaN 
\end{DoxyVerbInclude}


Note that infinities are preserved under type conversion to floating point types (i.\-e., {\ttfamily float}, {\ttfamily double}, {\ttfamily complex} and {\ttfamily dcomplex} types), but not integer types.


\begin{DoxyVerbInclude}
--> uint32(inf)

ans = 
 4294967295 

--> complex(inf)

ans = 
 Inf 
\end{DoxyVerbInclude}
 \hypertarget{constants_pi}{}\section{P\-I Constant Pi}\label{constants_pi}
Section\-: \hyperlink{sec_constants}{Base Constants} \hypertarget{vtkwidgets_vtkxyplotwidget_Usage}{}\subsection{Usage}\label{vtkwidgets_vtkxyplotwidget_Usage}
Returns a {\ttfamily double} (64-\/bit floating point number) value that represents pi (ratio between the circumference and diameter of a circle...). Typical usage \begin{DoxyVerb}   y = pi
\end{DoxyVerb}
 This value is approximately 3.\-141592653589793. \hypertarget{variables_struct_Example}{}\subsection{Example}\label{variables_struct_Example}
The following example demonstrates the use of the {\ttfamily pi} function.


\begin{DoxyVerbInclude}
--> pi

ans = 
    3.1416 

--> cos(pi)

ans = 
 -1 
\end{DoxyVerbInclude}
 \hypertarget{constants_teps}{}\section{T\-E\-P\-S Type-\/based Epsilon Calculation}\label{constants_teps}
Section\-: \hyperlink{sec_constants}{Base Constants} \hypertarget{vtkwidgets_vtkxyplotwidget_Usage}{}\subsection{Usage}\label{vtkwidgets_vtkxyplotwidget_Usage}
Returns {\ttfamily eps} for double precision arguments and {\ttfamily feps} for single precision arguments. The syntax for {\ttfamily teps} is \begin{DoxyVerb}   y = teps(x)
\end{DoxyVerb}
 The {\ttfamily teps} function is most useful if you need to compute epsilon based on the type of the array. \hypertarget{variables_struct_Example}{}\subsection{Example}\label{variables_struct_Example}
The following example demonstrates the use of the {\ttfamily teps} function, and one of its numerical consequences.


\begin{DoxyVerbInclude}
--> teps(float(3.4))

ans = 
 1.1921e-07 

--> teps(complex(3.4+i*2))

ans = 
 2.2204e-16 
\end{DoxyVerbInclude}
 \hypertarget{constants_true}{}\section{T\-R\-U\-E Logical T\-R\-U\-E}\label{constants_true}
Section\-: \hyperlink{sec_constants}{Base Constants} \hypertarget{vtkwidgets_vtkxyplotwidget_Usage}{}\subsection{Usage}\label{vtkwidgets_vtkxyplotwidget_Usage}
Returns a logical 1. The syntax for its use is \begin{DoxyVerb}   y = true
\end{DoxyVerb}
 You can also create an array of logical ones using the syntax \begin{DoxyVerb}   y = true(d1,d2,...,dn)
\end{DoxyVerb}
 or the syntax \begin{DoxyVerb}   y = true([d1,d2,...,dn])
\end{DoxyVerb}
 