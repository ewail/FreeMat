
\begin{DoxyItemize}
\item \hyperlink{vtkviews_vtkconvertselectiondomain}{vtk\-Convert\-Selection\-Domain}  
\item \hyperlink{vtkviews_vtkdatarepresentation}{vtk\-Data\-Representation}  
\item \hyperlink{vtkviews_vtkemptyrepresentation}{vtk\-Empty\-Representation}  
\item \hyperlink{vtkviews_vtkgraphlayoutview}{vtk\-Graph\-Layout\-View}  
\item \hyperlink{vtkviews_vtkhierarchicalgraphpipeline}{vtk\-Hierarchical\-Graph\-Pipeline}  
\item \hyperlink{vtkviews_vtkhierarchicalgraphview}{vtk\-Hierarchical\-Graph\-View}  
\item \hyperlink{vtkviews_vtkicicleview}{vtk\-Icicle\-View}  
\item \hyperlink{vtkviews_vtkinteractorstyleareaselecthover}{vtk\-Interactor\-Style\-Area\-Select\-Hover}  
\item \hyperlink{vtkviews_vtkinteractorstyletreemaphover}{vtk\-Interactor\-Style\-Tree\-Map\-Hover}  
\item \hyperlink{vtkviews_vtkparallelcoordinateshistogramrepresentation}{vtk\-Parallel\-Coordinates\-Histogram\-Representation}  
\item \hyperlink{vtkviews_vtkparallelcoordinatesrepresentation}{vtk\-Parallel\-Coordinates\-Representation}  
\item \hyperlink{vtkviews_vtkparallelcoordinatesview}{vtk\-Parallel\-Coordinates\-View}  
\item \hyperlink{vtkviews_vtkrenderedgraphrepresentation}{vtk\-Rendered\-Graph\-Representation}  
\item \hyperlink{vtkviews_vtkrenderedhierarchyrepresentation}{vtk\-Rendered\-Hierarchy\-Representation}  
\item \hyperlink{vtkviews_vtkrenderedrepresentation}{vtk\-Rendered\-Representation}  
\item \hyperlink{vtkviews_vtkrenderedsurfacerepresentation}{vtk\-Rendered\-Surface\-Representation}  
\item \hyperlink{vtkviews_vtkrenderedtreearearepresentation}{vtk\-Rendered\-Tree\-Area\-Representation}  
\item \hyperlink{vtkviews_vtkrenderview}{vtk\-Render\-View}  
\item \hyperlink{vtkviews_vtktreeareaview}{vtk\-Tree\-Area\-View}  
\item \hyperlink{vtkviews_vtktreemapview}{vtk\-Tree\-Map\-View}  
\item \hyperlink{vtkviews_vtktreeringview}{vtk\-Tree\-Ring\-View}  
\item \hyperlink{vtkviews_vtkview}{vtk\-View}  
\item \hyperlink{vtkviews_vtkviewupdater}{vtk\-View\-Updater}  
\end{DoxyItemize}\hypertarget{vtkviews_vtkconvertselectiondomain}{}\section{vtk\-Convert\-Selection\-Domain}\label{vtkviews_vtkconvertselectiondomain}
Section\-: \hyperlink{sec_vtkviews}{Visualization Toolkit View Classes} \hypertarget{vtkwidgets_vtkxyplotwidget_Usage}{}\subsection{Usage}\label{vtkwidgets_vtkxyplotwidget_Usage}
vtk\-Convert\-Selection\-Domain converts a selection from one domain to another using known domain mappings. The domain mappings are described by a vtk\-Multi\-Block\-Data\-Set containing one or more vtk\-Tables.

The first input port is for the input selection (or collection of annotations in a vtk\-Annotation\-Layers object), while the second port is for the multi-\/block of mappings, and the third port is for the data that is being selected on.

If the second or third port is not set, this filter will pass the selection/annotation to the output unchanged.

The second output is the selection associated with the \char`\"{}current annotation\char`\"{} normally representing the current interactive selection.

To create an instance of class vtk\-Convert\-Selection\-Domain, simply invoke its constructor as follows \begin{DoxyVerb}  obj = vtkConvertSelectionDomain
\end{DoxyVerb}
 \hypertarget{vtkwidgets_vtkxyplotwidget_Methods}{}\subsection{Methods}\label{vtkwidgets_vtkxyplotwidget_Methods}
The class vtk\-Convert\-Selection\-Domain has several methods that can be used. They are listed below. Note that the documentation is translated automatically from the V\-T\-K sources, and may not be completely intelligible. When in doubt, consult the V\-T\-K website. In the methods listed below, {\ttfamily obj} is an instance of the vtk\-Convert\-Selection\-Domain class. 
\begin{DoxyItemize}
\item {\ttfamily string = obj.\-Get\-Class\-Name ()}  
\item {\ttfamily int = obj.\-Is\-A (string name)}  
\item {\ttfamily vtk\-Convert\-Selection\-Domain = obj.\-New\-Instance ()}  
\item {\ttfamily vtk\-Convert\-Selection\-Domain = obj.\-Safe\-Down\-Cast (vtk\-Object o)}  
\end{DoxyItemize}\hypertarget{vtkviews_vtkdatarepresentation}{}\section{vtk\-Data\-Representation}\label{vtkviews_vtkdatarepresentation}
Section\-: \hyperlink{sec_vtkviews}{Visualization Toolkit View Classes} \hypertarget{vtkwidgets_vtkxyplotwidget_Usage}{}\subsection{Usage}\label{vtkwidgets_vtkxyplotwidget_Usage}
vtk\-Data\-Representation the superclass for representations of data objects. This class itself may be instantiated and used as a representation that simply holds a connection to a pipeline.

If there are multiple representations present in a view, you should use a subclass of vtk\-Data\-Representation. The representation is responsible for taking the input pipeline connection and converting it to an object usable by a view. In the most common case, the representation will contain the pipeline necessary to convert a data object into an actor or set of actors.

The representation has a concept of a selection. If the user performs a selection operation on the view, the view forwards this on to its representations. The representation is responsible for displaying that selection in an appropriate way.

Representation selections may also be linked. The representation shares the selection by converting it into a view-\/independent format, then setting the selection on its vtk\-Annotation\-Link. Other representations sharing the same selection link instance will get the same selection from the selection link when the view is updated. The application is responsible for linking representations as appropriate by setting the same vtk\-Annotation\-Link on each linked representation.

To create an instance of class vtk\-Data\-Representation, simply invoke its constructor as follows \begin{DoxyVerb}  obj = vtkDataRepresentation
\end{DoxyVerb}
 \hypertarget{vtkwidgets_vtkxyplotwidget_Methods}{}\subsection{Methods}\label{vtkwidgets_vtkxyplotwidget_Methods}
The class vtk\-Data\-Representation has several methods that can be used. They are listed below. Note that the documentation is translated automatically from the V\-T\-K sources, and may not be completely intelligible. When in doubt, consult the V\-T\-K website. In the methods listed below, {\ttfamily obj} is an instance of the vtk\-Data\-Representation class. 
\begin{DoxyItemize}
\item {\ttfamily string = obj.\-Get\-Class\-Name ()}  
\item {\ttfamily int = obj.\-Is\-A (string name)}  
\item {\ttfamily vtk\-Data\-Representation = obj.\-New\-Instance ()}  
\item {\ttfamily vtk\-Data\-Representation = obj.\-Safe\-Down\-Cast (vtk\-Object o)}  
\item {\ttfamily vtk\-Algorithm\-Output = obj.\-Get\-Input\-Connection (int port, int index)} -\/ The annotation link for this representation. To link annotations, set the same vtk\-Annotation\-Link object in multiple representations.  
\item {\ttfamily vtk\-Annotation\-Link = obj.\-Get\-Annotation\-Link ()} -\/ The annotation link for this representation. To link annotations, set the same vtk\-Annotation\-Link object in multiple representations.  
\item {\ttfamily obj.\-Set\-Annotation\-Link (vtk\-Annotation\-Link link)} -\/ The annotation link for this representation. To link annotations, set the same vtk\-Annotation\-Link object in multiple representations.  
\item {\ttfamily obj.\-Apply\-View\-Theme (vtk\-View\-Theme )} -\/ The view calls this method when a selection occurs. The representation takes this selection and converts it into a selection on its data by calling Convert\-Selection, then calls Update\-Selection with the converted selection. Subclasses should not overrride this method, but should instead override Convert\-Selection. The optional third argument specifies whether the selection should be added to the previous selection on this representation.  
\item {\ttfamily obj.\-Select (vtk\-View view, vtk\-Selection selection)} -\/ The view calls this method when a selection occurs. The representation takes this selection and converts it into a selection on its data by calling Convert\-Selection, then calls Update\-Selection with the converted selection. Subclasses should not overrride this method, but should instead override Convert\-Selection. The optional third argument specifies whether the selection should be added to the previous selection on this representation.  
\item {\ttfamily obj.\-Select (vtk\-View view, vtk\-Selection selection, bool extend)} -\/ The view calls this method when a selection occurs. The representation takes this selection and converts it into a selection on its data by calling Convert\-Selection, then calls Update\-Selection with the converted selection. Subclasses should not overrride this method, but should instead override Convert\-Selection. The optional third argument specifies whether the selection should be added to the previous selection on this representation.  
\item {\ttfamily obj.\-Set\-Selectable (bool )} -\/ Whether this representation is able to handle a selection. Default is true.  
\item {\ttfamily bool = obj.\-Get\-Selectable ()} -\/ Whether this representation is able to handle a selection. Default is true.  
\item {\ttfamily obj.\-Selectable\-On ()} -\/ Whether this representation is able to handle a selection. Default is true.  
\item {\ttfamily obj.\-Selectable\-Off ()} -\/ Whether this representation is able to handle a selection. Default is true.  
\item {\ttfamily obj.\-Update\-Selection (vtk\-Selection selection)} -\/ Updates the selection in the selection link and fires a selection change event. Subclasses should not overrride this method, but should instead override Convert\-Selection. The optional second argument specifies whether the selection should be added to the previous selection on this representation.  
\item {\ttfamily obj.\-Update\-Selection (vtk\-Selection selection, bool extend)} -\/ Updates the selection in the selection link and fires a selection change event. Subclasses should not overrride this method, but should instead override Convert\-Selection. The optional second argument specifies whether the selection should be added to the previous selection on this representation.  
\item {\ttfamily vtk\-Algorithm\-Output = obj.\-Get\-Internal\-Annotation\-Output\-Port ()} -\/ The output port that contains the annotations whose selections are localized for a particular input data object. This should be used when connecting the internal pipelines.  
\item {\ttfamily vtk\-Algorithm\-Output = obj.\-Get\-Internal\-Annotation\-Output\-Port (int port)} -\/ The output port that contains the annotations whose selections are localized for a particular input data object. This should be used when connecting the internal pipelines.  
\item {\ttfamily vtk\-Algorithm\-Output = obj.\-Get\-Internal\-Annotation\-Output\-Port (int port, int conn)} -\/ The output port that contains the annotations whose selections are localized for a particular input data object. This should be used when connecting the internal pipelines.  
\item {\ttfamily vtk\-Algorithm\-Output = obj.\-Get\-Internal\-Selection\-Output\-Port ()} -\/ The output port that contains the selection associated with the current annotation (normally the interactive selection). This should be used when connecting the internal pipelines.  
\item {\ttfamily vtk\-Algorithm\-Output = obj.\-Get\-Internal\-Selection\-Output\-Port (int port)} -\/ The output port that contains the selection associated with the current annotation (normally the interactive selection). This should be used when connecting the internal pipelines.  
\item {\ttfamily vtk\-Algorithm\-Output = obj.\-Get\-Internal\-Selection\-Output\-Port (int port, int conn)} -\/ The output port that contains the selection associated with the current annotation (normally the interactive selection). This should be used when connecting the internal pipelines.  
\item {\ttfamily vtk\-Algorithm\-Output = obj.\-Get\-Internal\-Output\-Port ()} -\/ Retrieves an output port for the input data object at the specified port and connection index. This may be connected to the representation's internal pipeline.  
\item {\ttfamily vtk\-Algorithm\-Output = obj.\-Get\-Internal\-Output\-Port (int port)} -\/ Retrieves an output port for the input data object at the specified port and connection index. This may be connected to the representation's internal pipeline.  
\item {\ttfamily vtk\-Algorithm\-Output = obj.\-Get\-Internal\-Output\-Port (int port, int conn)} -\/ Retrieves an output port for the input data object at the specified port and connection index. This may be connected to the representation's internal pipeline.  
\item {\ttfamily obj.\-Set\-Selection\-Type (int )} -\/ Set the selection type produced by this view. This should be one of the content type constants defined in vtk\-Selection\-Node.\-h. Common values are vtk\-Selection\-Node\-::\-I\-N\-D\-I\-C\-E\-S vtk\-Selection\-Node\-::\-P\-E\-D\-I\-G\-R\-E\-E\-I\-D\-S vtk\-Selection\-Node\-::\-V\-A\-L\-U\-E\-S  
\item {\ttfamily int = obj.\-Get\-Selection\-Type ()} -\/ Set the selection type produced by this view. This should be one of the content type constants defined in vtk\-Selection\-Node.\-h. Common values are vtk\-Selection\-Node\-::\-I\-N\-D\-I\-C\-E\-S vtk\-Selection\-Node\-::\-P\-E\-D\-I\-G\-R\-E\-E\-I\-D\-S vtk\-Selection\-Node\-::\-V\-A\-L\-U\-E\-S  
\item {\ttfamily obj.\-Set\-Selection\-Array\-Names (vtk\-String\-Array names)} -\/ If a V\-A\-L\-U\-E\-S selection, the arrays used to produce a selection.  
\item {\ttfamily vtk\-String\-Array = obj.\-Get\-Selection\-Array\-Names ()} -\/ If a V\-A\-L\-U\-E\-S selection, the arrays used to produce a selection.  
\item {\ttfamily obj.\-Set\-Selection\-Array\-Name (string name)} -\/ If a V\-A\-L\-U\-E\-S selection, the array used to produce a selection.  
\item {\ttfamily string = obj.\-Get\-Selection\-Array\-Name ()} -\/ If a V\-A\-L\-U\-E\-S selection, the array used to produce a selection.  
\end{DoxyItemize}\hypertarget{vtkviews_vtkemptyrepresentation}{}\section{vtk\-Empty\-Representation}\label{vtkviews_vtkemptyrepresentation}
Section\-: \hyperlink{sec_vtkviews}{Visualization Toolkit View Classes} \hypertarget{vtkwidgets_vtkxyplotwidget_Usage}{}\subsection{Usage}\label{vtkwidgets_vtkxyplotwidget_Usage}
To create an instance of class vtk\-Empty\-Representation, simply invoke its constructor as follows \begin{DoxyVerb}  obj = vtkEmptyRepresentation
\end{DoxyVerb}
 \hypertarget{vtkwidgets_vtkxyplotwidget_Methods}{}\subsection{Methods}\label{vtkwidgets_vtkxyplotwidget_Methods}
The class vtk\-Empty\-Representation has several methods that can be used. They are listed below. Note that the documentation is translated automatically from the V\-T\-K sources, and may not be completely intelligible. When in doubt, consult the V\-T\-K website. In the methods listed below, {\ttfamily obj} is an instance of the vtk\-Empty\-Representation class. 
\begin{DoxyItemize}
\item {\ttfamily string = obj.\-Get\-Class\-Name ()}  
\item {\ttfamily int = obj.\-Is\-A (string name)}  
\item {\ttfamily vtk\-Empty\-Representation = obj.\-New\-Instance ()}  
\item {\ttfamily vtk\-Empty\-Representation = obj.\-Safe\-Down\-Cast (vtk\-Object o)}  
\item {\ttfamily vtk\-Algorithm\-Output = obj.\-Get\-Internal\-Annotation\-Output\-Port ()} -\/ Since this representation has no inputs, override superclass implementation with one that ignores \char`\"{}port\char`\"{} and \char`\"{}conn\char`\"{} and still allows it to have an annotation output.  
\item {\ttfamily vtk\-Algorithm\-Output = obj.\-Get\-Internal\-Annotation\-Output\-Port (int port)} -\/ Since this representation has no inputs, override superclass implementation with one that ignores \char`\"{}port\char`\"{} and \char`\"{}conn\char`\"{} and still allows it to have an annotation output.  
\item {\ttfamily vtk\-Algorithm\-Output = obj.\-Get\-Internal\-Annotation\-Output\-Port (int port, int conn)} -\/ Since this representation has no inputs, override superclass implementation with one that ignores \char`\"{}port\char`\"{} and \char`\"{}conn\char`\"{} and still allows it to have an annotation output.  
\end{DoxyItemize}\hypertarget{vtkviews_vtkgraphlayoutview}{}\section{vtk\-Graph\-Layout\-View}\label{vtkviews_vtkgraphlayoutview}
Section\-: \hyperlink{sec_vtkviews}{Visualization Toolkit View Classes} \hypertarget{vtkwidgets_vtkxyplotwidget_Usage}{}\subsection{Usage}\label{vtkwidgets_vtkxyplotwidget_Usage}
vtk\-Graph\-Layout\-View performs graph layout and displays a vtk\-Graph. You may color and label the vertices and edges using fields in the graph. If coordinates are already assigned to the graph vertices in your graph, set the layout strategy to Pass\-Through in this view. The default layout is Fast2\-D which is fast but not that good, for better layout set the layout to Simple2\-D or Force\-Directed. There are also tree and circle layout strategies. \-:)

.S\-E\-E A\-L\-S\-O vtk\-Fast2\-D\-Layout\-Strategy vtk\-Simple2\-D\-Layout\-Strategy vtk\-Force\-Directed\-Layout\-Strategy

.S\-E\-C\-T\-I\-O\-N Thanks Thanks a bunch to the holographic unfolding pattern.

To create an instance of class vtk\-Graph\-Layout\-View, simply invoke its constructor as follows \begin{DoxyVerb}  obj = vtkGraphLayoutView
\end{DoxyVerb}
 \hypertarget{vtkwidgets_vtkxyplotwidget_Methods}{}\subsection{Methods}\label{vtkwidgets_vtkxyplotwidget_Methods}
The class vtk\-Graph\-Layout\-View has several methods that can be used. They are listed below. Note that the documentation is translated automatically from the V\-T\-K sources, and may not be completely intelligible. When in doubt, consult the V\-T\-K website. In the methods listed below, {\ttfamily obj} is an instance of the vtk\-Graph\-Layout\-View class. 
\begin{DoxyItemize}
\item {\ttfamily string = obj.\-Get\-Class\-Name ()}  
\item {\ttfamily int = obj.\-Is\-A (string name)}  
\item {\ttfamily vtk\-Graph\-Layout\-View = obj.\-New\-Instance ()}  
\item {\ttfamily vtk\-Graph\-Layout\-View = obj.\-Safe\-Down\-Cast (vtk\-Object o)}  
\item {\ttfamily obj.\-Set\-Vertex\-Label\-Array\-Name (string name)} -\/ The array to use for vertex labeling. Default is \char`\"{}label\char`\"{}.  
\item {\ttfamily string = obj.\-Get\-Vertex\-Label\-Array\-Name ()} -\/ The array to use for vertex labeling. Default is \char`\"{}label\char`\"{}.  
\item {\ttfamily obj.\-Set\-Edge\-Label\-Array\-Name (string name)} -\/ The array to use for edge labeling. Default is \char`\"{}label\char`\"{}.  
\item {\ttfamily string = obj.\-Get\-Edge\-Label\-Array\-Name ()} -\/ The array to use for edge labeling. Default is \char`\"{}label\char`\"{}.  
\item {\ttfamily obj.\-Set\-Vertex\-Label\-Visibility (bool vis)} -\/ Whether to show vertex labels. Default is off.  
\item {\ttfamily bool = obj.\-Get\-Vertex\-Label\-Visibility ()} -\/ Whether to show vertex labels. Default is off.  
\item {\ttfamily obj.\-Vertex\-Label\-Visibility\-On ()} -\/ Whether to show vertex labels. Default is off.  
\item {\ttfamily obj.\-Vertex\-Label\-Visibility\-Off ()} -\/ Whether to show vertex labels. Default is off.  
\item {\ttfamily obj.\-Set\-Hide\-Vertex\-Labels\-On\-Interaction (bool vis)} -\/ Whether to hide vertex labels during mouse interactions. Default is off.  
\item {\ttfamily bool = obj.\-Get\-Hide\-Vertex\-Labels\-On\-Interaction ()} -\/ Whether to hide vertex labels during mouse interactions. Default is off.  
\item {\ttfamily obj.\-Hide\-Vertex\-Labels\-On\-Interaction\-On ()} -\/ Whether to hide vertex labels during mouse interactions. Default is off.  
\item {\ttfamily obj.\-Hide\-Vertex\-Labels\-On\-Interaction\-Off ()} -\/ Whether to hide vertex labels during mouse interactions. Default is off.  
\item {\ttfamily obj.\-Set\-Edge\-Visibility (bool vis)} -\/ Whether to show the edges at all. Default is on  
\item {\ttfamily bool = obj.\-Get\-Edge\-Visibility ()} -\/ Whether to show the edges at all. Default is on  
\item {\ttfamily obj.\-Edge\-Visibility\-On ()} -\/ Whether to show the edges at all. Default is on  
\item {\ttfamily obj.\-Edge\-Visibility\-Off ()} -\/ Whether to show the edges at all. Default is on  
\item {\ttfamily obj.\-Set\-Edge\-Label\-Visibility (bool vis)} -\/ Whether to show edge labels. Default is off.  
\item {\ttfamily bool = obj.\-Get\-Edge\-Label\-Visibility ()} -\/ Whether to show edge labels. Default is off.  
\item {\ttfamily obj.\-Edge\-Label\-Visibility\-On ()} -\/ Whether to show edge labels. Default is off.  
\item {\ttfamily obj.\-Edge\-Label\-Visibility\-Off ()} -\/ Whether to show edge labels. Default is off.  
\item {\ttfamily obj.\-Set\-Hide\-Edge\-Labels\-On\-Interaction (bool vis)} -\/ Whether to hide edge labels during mouse interactions. Default is off.  
\item {\ttfamily bool = obj.\-Get\-Hide\-Edge\-Labels\-On\-Interaction ()} -\/ Whether to hide edge labels during mouse interactions. Default is off.  
\item {\ttfamily obj.\-Hide\-Edge\-Labels\-On\-Interaction\-On ()} -\/ Whether to hide edge labels during mouse interactions. Default is off.  
\item {\ttfamily obj.\-Hide\-Edge\-Labels\-On\-Interaction\-Off ()} -\/ Whether to hide edge labels during mouse interactions. Default is off.  
\item {\ttfamily obj.\-Set\-Vertex\-Color\-Array\-Name (string name)} -\/ The array to use for coloring vertices. Default is \char`\"{}color\char`\"{}.  
\item {\ttfamily string = obj.\-Get\-Vertex\-Color\-Array\-Name ()} -\/ The array to use for coloring vertices. Default is \char`\"{}color\char`\"{}.  
\item {\ttfamily obj.\-Set\-Color\-Vertices (bool vis)} -\/ Whether to color vertices. Default is off.  
\item {\ttfamily bool = obj.\-Get\-Color\-Vertices ()} -\/ Whether to color vertices. Default is off.  
\item {\ttfamily obj.\-Color\-Vertices\-On ()} -\/ Whether to color vertices. Default is off.  
\item {\ttfamily obj.\-Color\-Vertices\-Off ()} -\/ Whether to color vertices. Default is off.  
\item {\ttfamily obj.\-Set\-Edge\-Color\-Array\-Name (string name)} -\/ The array to use for coloring edges. Default is \char`\"{}color\char`\"{}.  
\item {\ttfamily string = obj.\-Get\-Edge\-Color\-Array\-Name ()} -\/ The array to use for coloring edges. Default is \char`\"{}color\char`\"{}.  
\item {\ttfamily obj.\-Set\-Color\-Edges (bool vis)} -\/ Whether to color edges. Default is off.  
\item {\ttfamily bool = obj.\-Get\-Color\-Edges ()} -\/ Whether to color edges. Default is off.  
\item {\ttfamily obj.\-Color\-Edges\-On ()} -\/ Whether to color edges. Default is off.  
\item {\ttfamily obj.\-Color\-Edges\-Off ()} -\/ Whether to color edges. Default is off.  
\item {\ttfamily obj.\-Set\-Enabled\-Edges\-Array\-Name (string name)} -\/ The array to use for coloring edges.  
\item {\ttfamily string = obj.\-Get\-Enabled\-Edges\-Array\-Name ()} -\/ The array to use for coloring edges.  
\item {\ttfamily obj.\-Set\-Enable\-Edges\-By\-Array (bool vis)} -\/ Whether to color edges. Default is off.  
\item {\ttfamily int = obj.\-Get\-Enable\-Edges\-By\-Array ()} -\/ Whether to color edges. Default is off.  
\item {\ttfamily obj.\-Set\-Enabled\-Vertices\-Array\-Name (string name)} -\/ The array to use for coloring vertices.  
\item {\ttfamily string = obj.\-Get\-Enabled\-Vertices\-Array\-Name ()} -\/ The array to use for coloring vertices.  
\item {\ttfamily obj.\-Set\-Enable\-Vertices\-By\-Array (bool vis)} -\/ Whether to color vertices. Default is off.  
\item {\ttfamily int = obj.\-Get\-Enable\-Vertices\-By\-Array ()} -\/ Whether to color vertices. Default is off.  
\item {\ttfamily obj.\-Set\-Scaling\-Array\-Name (string name)} -\/ The array used for scaling (if Scaled\-Glyphs is O\-N)  
\item {\ttfamily string = obj.\-Get\-Scaling\-Array\-Name ()} -\/ The array used for scaling (if Scaled\-Glyphs is O\-N)  
\item {\ttfamily obj.\-Set\-Scaled\-Glyphs (bool arg)} -\/ Whether to use scaled glyphs or not. Default is off.  
\item {\ttfamily bool = obj.\-Get\-Scaled\-Glyphs ()} -\/ Whether to use scaled glyphs or not. Default is off.  
\item {\ttfamily obj.\-Scaled\-Glyphs\-On ()} -\/ Whether to use scaled glyphs or not. Default is off.  
\item {\ttfamily obj.\-Scaled\-Glyphs\-Off ()} -\/ Whether to use scaled glyphs or not. Default is off.  
\item {\ttfamily obj.\-Set\-Layout\-Strategy (string name)} -\/ The layout strategy to use when performing the graph layout. The possible strings are\-:
\begin{DoxyItemize}
\item \char`\"{}\-Random\char`\"{} Randomly places vertices in a box.
\item \char`\"{}\-Force Directed\char`\"{} A layout in 3\-D or 2\-D simulating forces on edges.
\item \char`\"{}\-Simple 2\-D\char`\"{} A simple 2\-D force directed layout.
\item \char`\"{}\-Clustering 2\-D\char`\"{} A 2\-D force directed layout that's just like simple 2\-D but uses some techniques to cluster better.
\item \char`\"{}\-Community 2\-D\char`\"{} A linear-\/time 2\-D layout that's just like Fast 2\-D but looks for and uses a community array to 'accentuate' clusters.
\item \char`\"{}\-Fast 2\-D\char`\"{} A linear-\/time 2\-D layout.
\item \char`\"{}\-Pass Through\char`\"{} Use locations assigned to the input.
\item \char`\"{}\-Circular\char`\"{} Places vertices uniformly on a circle.
\item \char`\"{}\-Cone\char`\"{} Cone tree layout.
\item \char`\"{}\-Span Tree\char`\"{} Span Tree Layout. Default is \char`\"{}\-Simple 2\-D\char`\"{}.  
\end{DoxyItemize}
\item {\ttfamily obj.\-Set\-Layout\-Strategy\-To\-Random ()} -\/ The layout strategy to use when performing the graph layout. The possible strings are\-:
\begin{DoxyItemize}
\item \char`\"{}\-Random\char`\"{} Randomly places vertices in a box.
\item \char`\"{}\-Force Directed\char`\"{} A layout in 3\-D or 2\-D simulating forces on edges.
\item \char`\"{}\-Simple 2\-D\char`\"{} A simple 2\-D force directed layout.
\item \char`\"{}\-Clustering 2\-D\char`\"{} A 2\-D force directed layout that's just like simple 2\-D but uses some techniques to cluster better.
\item \char`\"{}\-Community 2\-D\char`\"{} A linear-\/time 2\-D layout that's just like Fast 2\-D but looks for and uses a community array to 'accentuate' clusters.
\item \char`\"{}\-Fast 2\-D\char`\"{} A linear-\/time 2\-D layout.
\item \char`\"{}\-Pass Through\char`\"{} Use locations assigned to the input.
\item \char`\"{}\-Circular\char`\"{} Places vertices uniformly on a circle.
\item \char`\"{}\-Cone\char`\"{} Cone tree layout.
\item \char`\"{}\-Span Tree\char`\"{} Span Tree Layout. Default is \char`\"{}\-Simple 2\-D\char`\"{}.  
\end{DoxyItemize}
\item {\ttfamily obj.\-Set\-Layout\-Strategy\-To\-Force\-Directed ()} -\/ The layout strategy to use when performing the graph layout. The possible strings are\-:
\begin{DoxyItemize}
\item \char`\"{}\-Random\char`\"{} Randomly places vertices in a box.
\item \char`\"{}\-Force Directed\char`\"{} A layout in 3\-D or 2\-D simulating forces on edges.
\item \char`\"{}\-Simple 2\-D\char`\"{} A simple 2\-D force directed layout.
\item \char`\"{}\-Clustering 2\-D\char`\"{} A 2\-D force directed layout that's just like simple 2\-D but uses some techniques to cluster better.
\item \char`\"{}\-Community 2\-D\char`\"{} A linear-\/time 2\-D layout that's just like Fast 2\-D but looks for and uses a community array to 'accentuate' clusters.
\item \char`\"{}\-Fast 2\-D\char`\"{} A linear-\/time 2\-D layout.
\item \char`\"{}\-Pass Through\char`\"{} Use locations assigned to the input.
\item \char`\"{}\-Circular\char`\"{} Places vertices uniformly on a circle.
\item \char`\"{}\-Cone\char`\"{} Cone tree layout.
\item \char`\"{}\-Span Tree\char`\"{} Span Tree Layout. Default is \char`\"{}\-Simple 2\-D\char`\"{}.  
\end{DoxyItemize}
\item {\ttfamily obj.\-Set\-Layout\-Strategy\-To\-Simple2\-D ()} -\/ The layout strategy to use when performing the graph layout. The possible strings are\-:
\begin{DoxyItemize}
\item \char`\"{}\-Random\char`\"{} Randomly places vertices in a box.
\item \char`\"{}\-Force Directed\char`\"{} A layout in 3\-D or 2\-D simulating forces on edges.
\item \char`\"{}\-Simple 2\-D\char`\"{} A simple 2\-D force directed layout.
\item \char`\"{}\-Clustering 2\-D\char`\"{} A 2\-D force directed layout that's just like simple 2\-D but uses some techniques to cluster better.
\item \char`\"{}\-Community 2\-D\char`\"{} A linear-\/time 2\-D layout that's just like Fast 2\-D but looks for and uses a community array to 'accentuate' clusters.
\item \char`\"{}\-Fast 2\-D\char`\"{} A linear-\/time 2\-D layout.
\item \char`\"{}\-Pass Through\char`\"{} Use locations assigned to the input.
\item \char`\"{}\-Circular\char`\"{} Places vertices uniformly on a circle.
\item \char`\"{}\-Cone\char`\"{} Cone tree layout.
\item \char`\"{}\-Span Tree\char`\"{} Span Tree Layout. Default is \char`\"{}\-Simple 2\-D\char`\"{}.  
\end{DoxyItemize}
\item {\ttfamily obj.\-Set\-Layout\-Strategy\-To\-Clustering2\-D ()} -\/ The layout strategy to use when performing the graph layout. The possible strings are\-:
\begin{DoxyItemize}
\item \char`\"{}\-Random\char`\"{} Randomly places vertices in a box.
\item \char`\"{}\-Force Directed\char`\"{} A layout in 3\-D or 2\-D simulating forces on edges.
\item \char`\"{}\-Simple 2\-D\char`\"{} A simple 2\-D force directed layout.
\item \char`\"{}\-Clustering 2\-D\char`\"{} A 2\-D force directed layout that's just like simple 2\-D but uses some techniques to cluster better.
\item \char`\"{}\-Community 2\-D\char`\"{} A linear-\/time 2\-D layout that's just like Fast 2\-D but looks for and uses a community array to 'accentuate' clusters.
\item \char`\"{}\-Fast 2\-D\char`\"{} A linear-\/time 2\-D layout.
\item \char`\"{}\-Pass Through\char`\"{} Use locations assigned to the input.
\item \char`\"{}\-Circular\char`\"{} Places vertices uniformly on a circle.
\item \char`\"{}\-Cone\char`\"{} Cone tree layout.
\item \char`\"{}\-Span Tree\char`\"{} Span Tree Layout. Default is \char`\"{}\-Simple 2\-D\char`\"{}.  
\end{DoxyItemize}
\item {\ttfamily obj.\-Set\-Layout\-Strategy\-To\-Community2\-D ()} -\/ The layout strategy to use when performing the graph layout. The possible strings are\-:
\begin{DoxyItemize}
\item \char`\"{}\-Random\char`\"{} Randomly places vertices in a box.
\item \char`\"{}\-Force Directed\char`\"{} A layout in 3\-D or 2\-D simulating forces on edges.
\item \char`\"{}\-Simple 2\-D\char`\"{} A simple 2\-D force directed layout.
\item \char`\"{}\-Clustering 2\-D\char`\"{} A 2\-D force directed layout that's just like simple 2\-D but uses some techniques to cluster better.
\item \char`\"{}\-Community 2\-D\char`\"{} A linear-\/time 2\-D layout that's just like Fast 2\-D but looks for and uses a community array to 'accentuate' clusters.
\item \char`\"{}\-Fast 2\-D\char`\"{} A linear-\/time 2\-D layout.
\item \char`\"{}\-Pass Through\char`\"{} Use locations assigned to the input.
\item \char`\"{}\-Circular\char`\"{} Places vertices uniformly on a circle.
\item \char`\"{}\-Cone\char`\"{} Cone tree layout.
\item \char`\"{}\-Span Tree\char`\"{} Span Tree Layout. Default is \char`\"{}\-Simple 2\-D\char`\"{}.  
\end{DoxyItemize}
\item {\ttfamily obj.\-Set\-Layout\-Strategy\-To\-Fast2\-D ()} -\/ The layout strategy to use when performing the graph layout. The possible strings are\-:
\begin{DoxyItemize}
\item \char`\"{}\-Random\char`\"{} Randomly places vertices in a box.
\item \char`\"{}\-Force Directed\char`\"{} A layout in 3\-D or 2\-D simulating forces on edges.
\item \char`\"{}\-Simple 2\-D\char`\"{} A simple 2\-D force directed layout.
\item \char`\"{}\-Clustering 2\-D\char`\"{} A 2\-D force directed layout that's just like simple 2\-D but uses some techniques to cluster better.
\item \char`\"{}\-Community 2\-D\char`\"{} A linear-\/time 2\-D layout that's just like Fast 2\-D but looks for and uses a community array to 'accentuate' clusters.
\item \char`\"{}\-Fast 2\-D\char`\"{} A linear-\/time 2\-D layout.
\item \char`\"{}\-Pass Through\char`\"{} Use locations assigned to the input.
\item \char`\"{}\-Circular\char`\"{} Places vertices uniformly on a circle.
\item \char`\"{}\-Cone\char`\"{} Cone tree layout.
\item \char`\"{}\-Span Tree\char`\"{} Span Tree Layout. Default is \char`\"{}\-Simple 2\-D\char`\"{}.  
\end{DoxyItemize}
\item {\ttfamily obj.\-Set\-Layout\-Strategy\-To\-Pass\-Through ()} -\/ The layout strategy to use when performing the graph layout. The possible strings are\-:
\begin{DoxyItemize}
\item \char`\"{}\-Random\char`\"{} Randomly places vertices in a box.
\item \char`\"{}\-Force Directed\char`\"{} A layout in 3\-D or 2\-D simulating forces on edges.
\item \char`\"{}\-Simple 2\-D\char`\"{} A simple 2\-D force directed layout.
\item \char`\"{}\-Clustering 2\-D\char`\"{} A 2\-D force directed layout that's just like simple 2\-D but uses some techniques to cluster better.
\item \char`\"{}\-Community 2\-D\char`\"{} A linear-\/time 2\-D layout that's just like Fast 2\-D but looks for and uses a community array to 'accentuate' clusters.
\item \char`\"{}\-Fast 2\-D\char`\"{} A linear-\/time 2\-D layout.
\item \char`\"{}\-Pass Through\char`\"{} Use locations assigned to the input.
\item \char`\"{}\-Circular\char`\"{} Places vertices uniformly on a circle.
\item \char`\"{}\-Cone\char`\"{} Cone tree layout.
\item \char`\"{}\-Span Tree\char`\"{} Span Tree Layout. Default is \char`\"{}\-Simple 2\-D\char`\"{}.  
\end{DoxyItemize}
\item {\ttfamily obj.\-Set\-Layout\-Strategy\-To\-Circular ()} -\/ The layout strategy to use when performing the graph layout. The possible strings are\-:
\begin{DoxyItemize}
\item \char`\"{}\-Random\char`\"{} Randomly places vertices in a box.
\item \char`\"{}\-Force Directed\char`\"{} A layout in 3\-D or 2\-D simulating forces on edges.
\item \char`\"{}\-Simple 2\-D\char`\"{} A simple 2\-D force directed layout.
\item \char`\"{}\-Clustering 2\-D\char`\"{} A 2\-D force directed layout that's just like simple 2\-D but uses some techniques to cluster better.
\item \char`\"{}\-Community 2\-D\char`\"{} A linear-\/time 2\-D layout that's just like Fast 2\-D but looks for and uses a community array to 'accentuate' clusters.
\item \char`\"{}\-Fast 2\-D\char`\"{} A linear-\/time 2\-D layout.
\item \char`\"{}\-Pass Through\char`\"{} Use locations assigned to the input.
\item \char`\"{}\-Circular\char`\"{} Places vertices uniformly on a circle.
\item \char`\"{}\-Cone\char`\"{} Cone tree layout.
\item \char`\"{}\-Span Tree\char`\"{} Span Tree Layout. Default is \char`\"{}\-Simple 2\-D\char`\"{}.  
\end{DoxyItemize}
\item {\ttfamily obj.\-Set\-Layout\-Strategy\-To\-Tree ()} -\/ The layout strategy to use when performing the graph layout. The possible strings are\-:
\begin{DoxyItemize}
\item \char`\"{}\-Random\char`\"{} Randomly places vertices in a box.
\item \char`\"{}\-Force Directed\char`\"{} A layout in 3\-D or 2\-D simulating forces on edges.
\item \char`\"{}\-Simple 2\-D\char`\"{} A simple 2\-D force directed layout.
\item \char`\"{}\-Clustering 2\-D\char`\"{} A 2\-D force directed layout that's just like simple 2\-D but uses some techniques to cluster better.
\item \char`\"{}\-Community 2\-D\char`\"{} A linear-\/time 2\-D layout that's just like Fast 2\-D but looks for and uses a community array to 'accentuate' clusters.
\item \char`\"{}\-Fast 2\-D\char`\"{} A linear-\/time 2\-D layout.
\item \char`\"{}\-Pass Through\char`\"{} Use locations assigned to the input.
\item \char`\"{}\-Circular\char`\"{} Places vertices uniformly on a circle.
\item \char`\"{}\-Cone\char`\"{} Cone tree layout.
\item \char`\"{}\-Span Tree\char`\"{} Span Tree Layout. Default is \char`\"{}\-Simple 2\-D\char`\"{}.  
\end{DoxyItemize}
\item {\ttfamily obj.\-Set\-Layout\-Strategy\-To\-Cosmic\-Tree ()} -\/ The layout strategy to use when performing the graph layout. The possible strings are\-:
\begin{DoxyItemize}
\item \char`\"{}\-Random\char`\"{} Randomly places vertices in a box.
\item \char`\"{}\-Force Directed\char`\"{} A layout in 3\-D or 2\-D simulating forces on edges.
\item \char`\"{}\-Simple 2\-D\char`\"{} A simple 2\-D force directed layout.
\item \char`\"{}\-Clustering 2\-D\char`\"{} A 2\-D force directed layout that's just like simple 2\-D but uses some techniques to cluster better.
\item \char`\"{}\-Community 2\-D\char`\"{} A linear-\/time 2\-D layout that's just like Fast 2\-D but looks for and uses a community array to 'accentuate' clusters.
\item \char`\"{}\-Fast 2\-D\char`\"{} A linear-\/time 2\-D layout.
\item \char`\"{}\-Pass Through\char`\"{} Use locations assigned to the input.
\item \char`\"{}\-Circular\char`\"{} Places vertices uniformly on a circle.
\item \char`\"{}\-Cone\char`\"{} Cone tree layout.
\item \char`\"{}\-Span Tree\char`\"{} Span Tree Layout. Default is \char`\"{}\-Simple 2\-D\char`\"{}.  
\end{DoxyItemize}
\item {\ttfamily obj.\-Set\-Layout\-Strategy\-To\-Cone ()} -\/ The layout strategy to use when performing the graph layout. The possible strings are\-:
\begin{DoxyItemize}
\item \char`\"{}\-Random\char`\"{} Randomly places vertices in a box.
\item \char`\"{}\-Force Directed\char`\"{} A layout in 3\-D or 2\-D simulating forces on edges.
\item \char`\"{}\-Simple 2\-D\char`\"{} A simple 2\-D force directed layout.
\item \char`\"{}\-Clustering 2\-D\char`\"{} A 2\-D force directed layout that's just like simple 2\-D but uses some techniques to cluster better.
\item \char`\"{}\-Community 2\-D\char`\"{} A linear-\/time 2\-D layout that's just like Fast 2\-D but looks for and uses a community array to 'accentuate' clusters.
\item \char`\"{}\-Fast 2\-D\char`\"{} A linear-\/time 2\-D layout.
\item \char`\"{}\-Pass Through\char`\"{} Use locations assigned to the input.
\item \char`\"{}\-Circular\char`\"{} Places vertices uniformly on a circle.
\item \char`\"{}\-Cone\char`\"{} Cone tree layout.
\item \char`\"{}\-Span Tree\char`\"{} Span Tree Layout. Default is \char`\"{}\-Simple 2\-D\char`\"{}.  
\end{DoxyItemize}
\item {\ttfamily obj.\-Set\-Layout\-Strategy\-To\-Span\-Tree ()} -\/ The layout strategy to use when performing the graph layout. The possible strings are\-:
\begin{DoxyItemize}
\item \char`\"{}\-Random\char`\"{} Randomly places vertices in a box.
\item \char`\"{}\-Force Directed\char`\"{} A layout in 3\-D or 2\-D simulating forces on edges.
\item \char`\"{}\-Simple 2\-D\char`\"{} A simple 2\-D force directed layout.
\item \char`\"{}\-Clustering 2\-D\char`\"{} A 2\-D force directed layout that's just like simple 2\-D but uses some techniques to cluster better.
\item \char`\"{}\-Community 2\-D\char`\"{} A linear-\/time 2\-D layout that's just like Fast 2\-D but looks for and uses a community array to 'accentuate' clusters.
\item \char`\"{}\-Fast 2\-D\char`\"{} A linear-\/time 2\-D layout.
\item \char`\"{}\-Pass Through\char`\"{} Use locations assigned to the input.
\item \char`\"{}\-Circular\char`\"{} Places vertices uniformly on a circle.
\item \char`\"{}\-Cone\char`\"{} Cone tree layout.
\item \char`\"{}\-Span Tree\char`\"{} Span Tree Layout. Default is \char`\"{}\-Simple 2\-D\char`\"{}.  
\end{DoxyItemize}
\item {\ttfamily string = obj.\-Get\-Layout\-Strategy\-Name ()} -\/ The layout strategy to use when performing the graph layout. The possible strings are\-:
\begin{DoxyItemize}
\item \char`\"{}\-Random\char`\"{} Randomly places vertices in a box.
\item \char`\"{}\-Force Directed\char`\"{} A layout in 3\-D or 2\-D simulating forces on edges.
\item \char`\"{}\-Simple 2\-D\char`\"{} A simple 2\-D force directed layout.
\item \char`\"{}\-Clustering 2\-D\char`\"{} A 2\-D force directed layout that's just like simple 2\-D but uses some techniques to cluster better.
\item \char`\"{}\-Community 2\-D\char`\"{} A linear-\/time 2\-D layout that's just like Fast 2\-D but looks for and uses a community array to 'accentuate' clusters.
\item \char`\"{}\-Fast 2\-D\char`\"{} A linear-\/time 2\-D layout.
\item \char`\"{}\-Pass Through\char`\"{} Use locations assigned to the input.
\item \char`\"{}\-Circular\char`\"{} Places vertices uniformly on a circle.
\item \char`\"{}\-Cone\char`\"{} Cone tree layout.
\item \char`\"{}\-Span Tree\char`\"{} Span Tree Layout. Default is \char`\"{}\-Simple 2\-D\char`\"{}.  
\end{DoxyItemize}
\item {\ttfamily vtk\-Graph\-Layout\-Strategy = obj.\-Get\-Layout\-Strategy ()} -\/ The layout strategy to use when performing the graph layout. This signature allows an application to create a layout object directly and simply set the pointer through this method.  
\item {\ttfamily obj.\-Set\-Layout\-Strategy (vtk\-Graph\-Layout\-Strategy s)} -\/ The layout strategy to use when performing the graph layout. This signature allows an application to create a layout object directly and simply set the pointer through this method.  
\item {\ttfamily obj.\-Set\-Edge\-Layout\-Strategy (string name)} -\/ The layout strategy to use when performing the edge layout. The possible strings are\-: \char`\"{}\-Arc Parallel\char`\"{} -\/ Arc parallel edges and self loops. \char`\"{}\-Pass Through\char`\"{} -\/ Use edge routes assigned to the input. Default is \char`\"{}\-Arc Parallel\char`\"{}.  
\item {\ttfamily obj.\-Set\-Edge\-Layout\-Strategy\-To\-Arc\-Parallel ()} -\/ The layout strategy to use when performing the edge layout. The possible strings are\-: \char`\"{}\-Arc Parallel\char`\"{} -\/ Arc parallel edges and self loops. \char`\"{}\-Pass Through\char`\"{} -\/ Use edge routes assigned to the input. Default is \char`\"{}\-Arc Parallel\char`\"{}.  
\item {\ttfamily obj.\-Set\-Edge\-Layout\-Strategy\-To\-Pass\-Through ()} -\/ The layout strategy to use when performing the edge layout. The possible strings are\-: \char`\"{}\-Arc Parallel\char`\"{} -\/ Arc parallel edges and self loops. \char`\"{}\-Pass Through\char`\"{} -\/ Use edge routes assigned to the input. Default is \char`\"{}\-Arc Parallel\char`\"{}.  
\item {\ttfamily string = obj.\-Get\-Edge\-Layout\-Strategy\-Name ()} -\/ The layout strategy to use when performing the edge layout. The possible strings are\-: \char`\"{}\-Arc Parallel\char`\"{} -\/ Arc parallel edges and self loops. \char`\"{}\-Pass Through\char`\"{} -\/ Use edge routes assigned to the input. Default is \char`\"{}\-Arc Parallel\char`\"{}.  
\item {\ttfamily vtk\-Edge\-Layout\-Strategy = obj.\-Get\-Edge\-Layout\-Strategy ()} -\/ The layout strategy to use when performing the edge layout. This signature allows an application to create a layout object directly and simply set the pointer through this method.  
\item {\ttfamily obj.\-Set\-Edge\-Layout\-Strategy (vtk\-Edge\-Layout\-Strategy s)} -\/ The layout strategy to use when performing the edge layout. This signature allows an application to create a layout object directly and simply set the pointer through this method.  
\item {\ttfamily obj.\-Add\-Icon\-Type (string type, int index)} -\/ Associate the icon at index \char`\"{}index\char`\"{} in the vtk\-Texture to all vertices containing \char`\"{}type\char`\"{} as a value in the vertex attribute array specified by Icon\-Array\-Name.  
\item {\ttfamily obj.\-Clear\-Icon\-Types ()} -\/ Clear all icon mappings.  
\item {\ttfamily obj.\-Set\-Icon\-Alignment (int alignment)} -\/ Specify where the icons should be placed in relation to the vertex. See vtk\-Icon\-Glyph\-Filter.\-h for possible values.  
\item {\ttfamily obj.\-Set\-Icon\-Visibility (bool b)} -\/ Whether icons are visible (default off).  
\item {\ttfamily bool = obj.\-Get\-Icon\-Visibility ()} -\/ Whether icons are visible (default off).  
\item {\ttfamily obj.\-Icon\-Visibility\-On ()} -\/ Whether icons are visible (default off).  
\item {\ttfamily obj.\-Icon\-Visibility\-Off ()} -\/ Whether icons are visible (default off).  
\item {\ttfamily obj.\-Set\-Icon\-Array\-Name (string name)} -\/ The array used for assigning icons  
\item {\ttfamily string = obj.\-Get\-Icon\-Array\-Name ()} -\/ The array used for assigning icons  
\item {\ttfamily obj.\-Set\-Glyph\-Type (int type)} -\/ The type of glyph to use for the vertices  
\item {\ttfamily int = obj.\-Get\-Glyph\-Type ()} -\/ The type of glyph to use for the vertices  
\item {\ttfamily obj.\-Set\-Vertex\-Label\-Font\-Size (int size)} -\/ The size of the font used for vertex labeling  
\item {\ttfamily int = obj.\-Get\-Vertex\-Label\-Font\-Size ()} -\/ The size of the font used for vertex labeling  
\item {\ttfamily obj.\-Set\-Edge\-Label\-Font\-Size (int size)} -\/ The size of the font used for edge labeling  
\item {\ttfamily int = obj.\-Get\-Edge\-Label\-Font\-Size ()} -\/ The size of the font used for edge labeling  
\item {\ttfamily obj.\-Set\-Edge\-Scalar\-Bar\-Visibility (bool vis)} -\/ Whether the scalar bar for edges is visible. Default is off.  
\item {\ttfamily bool = obj.\-Get\-Edge\-Scalar\-Bar\-Visibility ()} -\/ Whether the scalar bar for edges is visible. Default is off.  
\item {\ttfamily obj.\-Set\-Vertex\-Scalar\-Bar\-Visibility (bool vis)} -\/ Whether the scalar bar for vertices is visible. Default is off.  
\item {\ttfamily bool = obj.\-Get\-Vertex\-Scalar\-Bar\-Visibility ()} -\/ Whether the scalar bar for vertices is visible. Default is off.  
\item {\ttfamily obj.\-Zoom\-To\-Selection ()} -\/ Reset the camera based on the bounds of the selected region.  
\item {\ttfamily int = obj.\-Is\-Layout\-Complete ()} -\/ Is the graph layout complete? This method is useful for when the strategy is iterative and the application wants to show the iterative progress of the graph layout See Also\-: Update\-Layout();  
\item {\ttfamily obj.\-Update\-Layout ()} -\/ This method is useful for when the strategy is iterative and the application wants to show the iterative progress of the graph layout. The application would have something like while(!\-Is\-Layout\-Complete()) \{ Update\-Layout(); \} See Also\-: Is\-Layout\-Complete();  
\end{DoxyItemize}\hypertarget{vtkviews_vtkhierarchicalgraphpipeline}{}\section{vtk\-Hierarchical\-Graph\-Pipeline}\label{vtkviews_vtkhierarchicalgraphpipeline}
Section\-: \hyperlink{sec_vtkviews}{Visualization Toolkit View Classes} \hypertarget{vtkwidgets_vtkxyplotwidget_Usage}{}\subsection{Usage}\label{vtkwidgets_vtkxyplotwidget_Usage}
vtk\-Hierarchical\-Graph\-Pipeline renders bundled edges that are meant to be viewed as an overlay on a tree. This class is not for general use, but is used in the internals of vtk\-Rendered\-Hierarchy\-Representation and vtk\-Rendered\-Tree\-Area\-Representation.

To create an instance of class vtk\-Hierarchical\-Graph\-Pipeline, simply invoke its constructor as follows \begin{DoxyVerb}  obj = vtkHierarchicalGraphPipeline
\end{DoxyVerb}
 \hypertarget{vtkwidgets_vtkxyplotwidget_Methods}{}\subsection{Methods}\label{vtkwidgets_vtkxyplotwidget_Methods}
The class vtk\-Hierarchical\-Graph\-Pipeline has several methods that can be used. They are listed below. Note that the documentation is translated automatically from the V\-T\-K sources, and may not be completely intelligible. When in doubt, consult the V\-T\-K website. In the methods listed below, {\ttfamily obj} is an instance of the vtk\-Hierarchical\-Graph\-Pipeline class. 
\begin{DoxyItemize}
\item {\ttfamily string = obj.\-Get\-Class\-Name ()}  
\item {\ttfamily int = obj.\-Is\-A (string name)}  
\item {\ttfamily vtk\-Hierarchical\-Graph\-Pipeline = obj.\-New\-Instance ()}  
\item {\ttfamily vtk\-Hierarchical\-Graph\-Pipeline = obj.\-Safe\-Down\-Cast (vtk\-Object o)}  
\item {\ttfamily vtk\-Actor = obj.\-Get\-Actor ()} -\/ The actor associated with the hierarchical graph.  
\item {\ttfamily vtk\-Actor2\-D = obj.\-Get\-Label\-Actor ()} -\/ The actor associated with the hierarchical graph.  
\item {\ttfamily obj.\-Set\-Bundling\-Strength (double strength)} -\/ The bundling strength for the bundled edges.  
\item {\ttfamily double = obj.\-Get\-Bundling\-Strength ()} -\/ The bundling strength for the bundled edges.  
\item {\ttfamily obj.\-Set\-Label\-Array\-Name (string name)} -\/ The edge label array name.  
\item {\ttfamily string = obj.\-Get\-Label\-Array\-Name ()} -\/ The edge label array name.  
\item {\ttfamily obj.\-Set\-Label\-Visibility (bool vis)} -\/ The edge label visibility.  
\item {\ttfamily bool = obj.\-Get\-Label\-Visibility ()} -\/ The edge label visibility.  
\item {\ttfamily obj.\-Label\-Visibility\-On ()} -\/ The edge label visibility.  
\item {\ttfamily obj.\-Label\-Visibility\-Off ()} -\/ The edge label visibility.  
\item {\ttfamily obj.\-Set\-Label\-Text\-Property (vtk\-Text\-Property prop)} -\/ The edge label text property.  
\item {\ttfamily vtk\-Text\-Property = obj.\-Get\-Label\-Text\-Property ()} -\/ The edge label text property.  
\item {\ttfamily obj.\-Set\-Color\-Array\-Name (string name)} -\/ The edge color array.  
\item {\ttfamily string = obj.\-Get\-Color\-Array\-Name ()} -\/ The edge color array.  
\item {\ttfamily obj.\-Set\-Color\-Edges\-By\-Array (bool vis)} -\/ Whether to color the edges by an array.  
\item {\ttfamily bool = obj.\-Get\-Color\-Edges\-By\-Array ()} -\/ Whether to color the edges by an array.  
\item {\ttfamily obj.\-Color\-Edges\-By\-Array\-On ()} -\/ Whether to color the edges by an array.  
\item {\ttfamily obj.\-Color\-Edges\-By\-Array\-Off ()} -\/ Whether to color the edges by an array.  
\item {\ttfamily obj.\-Set\-Visibility (bool vis)} -\/ The visibility of this graph.  
\item {\ttfamily bool = obj.\-Get\-Visibility ()} -\/ The visibility of this graph.  
\item {\ttfamily obj.\-Visibility\-On ()} -\/ The visibility of this graph.  
\item {\ttfamily obj.\-Visibility\-Off ()} -\/ The visibility of this graph.  
\item {\ttfamily vtk\-Selection = obj.\-Convert\-Selection (vtk\-Data\-Representation rep, vtk\-Selection sel)} -\/ Returns a new selection relevant to this graph based on an input selection and the view that this graph is contained in.  
\item {\ttfamily obj.\-Prepare\-Input\-Connections (vtk\-Algorithm\-Output graph\-Conn, vtk\-Algorithm\-Output tree\-Conn, vtk\-Algorithm\-Output ann\-Conn)} -\/ Sets the input connections for this graph. graph\-Conn is the input graph connection. tree\-Conn is the input tree connection. ann\-Conn is the annotation link connection.  
\item {\ttfamily obj.\-Apply\-View\-Theme (vtk\-View\-Theme theme)} -\/ Applies the view theme to this graph.  
\item {\ttfamily obj.\-Set\-Hover\-Array\-Name (string )} -\/ The array to use while hovering over an edge.  
\item {\ttfamily string = obj.\-Get\-Hover\-Array\-Name ()} -\/ The array to use while hovering over an edge.  
\item {\ttfamily obj.\-Set\-Spline\-Type (int type)} -\/ The spline mode to use in vtk\-Spline\-Graph\-Edges. vtk\-Spline\-Graph\-Edges\-::\-C\-U\-S\-T\-O\-M uses a vtk\-Cardinal\-Spline. vtk\-Spline\-Graph\-Edges\-::\-B\-S\-P\-L\-I\-N\-E uses a b-\/spline. The default is C\-U\-S\-T\-O\-M.  
\item {\ttfamily int = obj.\-Get\-Spline\-Type ()} -\/ The spline mode to use in vtk\-Spline\-Graph\-Edges. vtk\-Spline\-Graph\-Edges\-::\-C\-U\-S\-T\-O\-M uses a vtk\-Cardinal\-Spline. vtk\-Spline\-Graph\-Edges\-::\-B\-S\-P\-L\-I\-N\-E uses a b-\/spline. The default is C\-U\-S\-T\-O\-M.  
\item {\ttfamily obj.\-Register\-Progress (vtk\-Render\-View view)} -\/ Register progress with a view.  
\end{DoxyItemize}\hypertarget{vtkviews_vtkhierarchicalgraphview}{}\section{vtk\-Hierarchical\-Graph\-View}\label{vtkviews_vtkhierarchicalgraphview}
Section\-: \hyperlink{sec_vtkviews}{Visualization Toolkit View Classes} \hypertarget{vtkwidgets_vtkxyplotwidget_Usage}{}\subsection{Usage}\label{vtkwidgets_vtkxyplotwidget_Usage}
Takes a graph and a hierarchy (currently a tree) and lays out the graph vertices based on their categorization within the hierarchy.

.S\-E\-E A\-L\-S\-O vtk\-Graph\-Layout\-View

.S\-E\-C\-T\-I\-O\-N Thanks Thanks to the turtle with jets for feet, without you this class wouldn't have been possible.

To create an instance of class vtk\-Hierarchical\-Graph\-View, simply invoke its constructor as follows \begin{DoxyVerb}  obj = vtkHierarchicalGraphView
\end{DoxyVerb}
 \hypertarget{vtkwidgets_vtkxyplotwidget_Methods}{}\subsection{Methods}\label{vtkwidgets_vtkxyplotwidget_Methods}
The class vtk\-Hierarchical\-Graph\-View has several methods that can be used. They are listed below. Note that the documentation is translated automatically from the V\-T\-K sources, and may not be completely intelligible. When in doubt, consult the V\-T\-K website. In the methods listed below, {\ttfamily obj} is an instance of the vtk\-Hierarchical\-Graph\-View class. 
\begin{DoxyItemize}
\item {\ttfamily string = obj.\-Get\-Class\-Name ()}  
\item {\ttfamily int = obj.\-Is\-A (string name)}  
\item {\ttfamily vtk\-Hierarchical\-Graph\-View = obj.\-New\-Instance ()}  
\item {\ttfamily vtk\-Hierarchical\-Graph\-View = obj.\-Safe\-Down\-Cast (vtk\-Object o)}  
\item {\ttfamily vtk\-Data\-Representation = obj.\-Set\-Hierarchy\-From\-Input\-Connection (vtk\-Algorithm\-Output conn)} -\/ Set the tree and graph representations to the appropriate input ports.  
\item {\ttfamily vtk\-Data\-Representation = obj.\-Set\-Hierarchy\-From\-Input (vtk\-Data\-Object input)} -\/ Set the tree and graph representations to the appropriate input ports.  
\item {\ttfamily vtk\-Data\-Representation = obj.\-Set\-Graph\-From\-Input\-Connection (vtk\-Algorithm\-Output conn)} -\/ Set the tree and graph representations to the appropriate input ports.  
\item {\ttfamily vtk\-Data\-Representation = obj.\-Set\-Graph\-From\-Input (vtk\-Data\-Object input)} -\/ Set the tree and graph representations to the appropriate input ports.  
\item {\ttfamily obj.\-Set\-Graph\-Edge\-Label\-Array\-Name (string name)} -\/ The array to use for edge labeling. Default is \char`\"{}label\char`\"{}.  
\item {\ttfamily string = obj.\-Get\-Graph\-Edge\-Label\-Array\-Name ()} -\/ The array to use for edge labeling. Default is \char`\"{}label\char`\"{}.  
\item {\ttfamily obj.\-Set\-Graph\-Edge\-Label\-Visibility (bool vis)} -\/ Whether to show edge labels. Default is off.  
\item {\ttfamily bool = obj.\-Get\-Graph\-Edge\-Label\-Visibility ()} -\/ Whether to show edge labels. Default is off.  
\item {\ttfamily obj.\-Graph\-Edge\-Label\-Visibility\-On ()} -\/ Whether to show edge labels. Default is off.  
\item {\ttfamily obj.\-Graph\-Edge\-Label\-Visibility\-Off ()} -\/ Whether to show edge labels. Default is off.  
\item {\ttfamily obj.\-Set\-Graph\-Edge\-Color\-Array\-Name (string name)} -\/ The array to use for coloring edges. Default is \char`\"{}color\char`\"{}.  
\item {\ttfamily string = obj.\-Get\-Graph\-Edge\-Color\-Array\-Name ()} -\/ The array to use for coloring edges. Default is \char`\"{}color\char`\"{}.  
\item {\ttfamily obj.\-Set\-Graph\-Edge\-Color\-To\-Spline\-Fraction ()} -\/ Set the color to be the spline fraction  
\item {\ttfamily obj.\-Set\-Color\-Graph\-Edges\-By\-Array (bool vis)} -\/ Whether to color edges. Default is off.  
\item {\ttfamily bool = obj.\-Get\-Color\-Graph\-Edges\-By\-Array ()} -\/ Whether to color edges. Default is off.  
\item {\ttfamily obj.\-Color\-Graph\-Edges\-By\-Array\-On ()} -\/ Whether to color edges. Default is off.  
\item {\ttfamily obj.\-Color\-Graph\-Edges\-By\-Array\-Off ()} -\/ Whether to color edges. Default is off.  
\item {\ttfamily obj.\-Set\-Bundling\-Strength (double strength)} -\/ Set the bundling strength.  
\item {\ttfamily double = obj.\-Get\-Bundling\-Strength ()} -\/ Set the bundling strength.  
\item {\ttfamily obj.\-Set\-Graph\-Visibility (bool b)} -\/ Whether the graph edges are visible (default off).  
\item {\ttfamily bool = obj.\-Get\-Graph\-Visibility ()} -\/ Whether the graph edges are visible (default off).  
\item {\ttfamily obj.\-Graph\-Visibility\-On ()} -\/ Whether the graph edges are visible (default off).  
\item {\ttfamily obj.\-Graph\-Visibility\-Off ()} -\/ Whether the graph edges are visible (default off).  
\item {\ttfamily obj.\-Set\-Graph\-Edge\-Label\-Font\-Size (int size)} -\/ The size of the font used for edge labeling  
\item {\ttfamily int = obj.\-Get\-Graph\-Edge\-Label\-Font\-Size ()} -\/ The size of the font used for edge labeling  
\end{DoxyItemize}\hypertarget{vtkviews_vtkicicleview}{}\section{vtk\-Icicle\-View}\label{vtkviews_vtkicicleview}
Section\-: \hyperlink{sec_vtkviews}{Visualization Toolkit View Classes} \hypertarget{vtkwidgets_vtkxyplotwidget_Usage}{}\subsection{Usage}\label{vtkwidgets_vtkxyplotwidget_Usage}
vtk\-Icicle\-View shows a vtk\-Tree in horizontal layers where each vertex in the tree is represented by a bar. Child sectors are below (or above) parent sectors, and may be colored and sized by various parameters.

To create an instance of class vtk\-Icicle\-View, simply invoke its constructor as follows \begin{DoxyVerb}  obj = vtkIcicleView
\end{DoxyVerb}
 \hypertarget{vtkwidgets_vtkxyplotwidget_Methods}{}\subsection{Methods}\label{vtkwidgets_vtkxyplotwidget_Methods}
The class vtk\-Icicle\-View has several methods that can be used. They are listed below. Note that the documentation is translated automatically from the V\-T\-K sources, and may not be completely intelligible. When in doubt, consult the V\-T\-K website. In the methods listed below, {\ttfamily obj} is an instance of the vtk\-Icicle\-View class. 
\begin{DoxyItemize}
\item {\ttfamily string = obj.\-Get\-Class\-Name ()}  
\item {\ttfamily int = obj.\-Is\-A (string name)}  
\item {\ttfamily vtk\-Icicle\-View = obj.\-New\-Instance ()}  
\item {\ttfamily vtk\-Icicle\-View = obj.\-Safe\-Down\-Cast (vtk\-Object o)}  
\item {\ttfamily obj.\-Set\-Top\-To\-Bottom (bool value)} -\/ Sets whether the stacks go from top to bottom or bottom to top.  
\item {\ttfamily bool = obj.\-Get\-Top\-To\-Bottom ()} -\/ Sets whether the stacks go from top to bottom or bottom to top.  
\item {\ttfamily obj.\-Top\-To\-Bottom\-On ()} -\/ Sets whether the stacks go from top to bottom or bottom to top.  
\item {\ttfamily obj.\-Top\-To\-Bottom\-Off ()} -\/ Sets whether the stacks go from top to bottom or bottom to top.  
\item {\ttfamily obj.\-Set\-Root\-Width (double width)} -\/ Set the width of the root node  
\item {\ttfamily double = obj.\-Get\-Root\-Width ()} -\/ Set the width of the root node  
\item {\ttfamily obj.\-Set\-Layer\-Thickness (double thickness)} -\/ Set the thickness of each layer  
\item {\ttfamily double = obj.\-Get\-Layer\-Thickness ()} -\/ Set the thickness of each layer  
\item {\ttfamily obj.\-Set\-Use\-Gradient\-Coloring (bool value)} -\/ Turn on/off gradient coloring.  
\item {\ttfamily bool = obj.\-Get\-Use\-Gradient\-Coloring ()} -\/ Turn on/off gradient coloring.  
\item {\ttfamily obj.\-Use\-Gradient\-Coloring\-On ()} -\/ Turn on/off gradient coloring.  
\item {\ttfamily obj.\-Use\-Gradient\-Coloring\-Off ()} -\/ Turn on/off gradient coloring.  
\end{DoxyItemize}\hypertarget{vtkviews_vtkinteractorstyleareaselecthover}{}\section{vtk\-Interactor\-Style\-Area\-Select\-Hover}\label{vtkviews_vtkinteractorstyleareaselecthover}
Section\-: \hyperlink{sec_vtkviews}{Visualization Toolkit View Classes} \hypertarget{vtkwidgets_vtkxyplotwidget_Usage}{}\subsection{Usage}\label{vtkwidgets_vtkxyplotwidget_Usage}
The vtk\-Interactor\-Style\-Area\-Select\-Hover specifically works with pipelines that create a hierarchical tree. Such pipelines will have a vtk\-Area\-Layout filter which must be passed to this interactor style for it to function correctly. This interactor style allows only 2\-D panning and zooming, rubber band selection and provides a balloon containing the name of the vertex hovered over.

To create an instance of class vtk\-Interactor\-Style\-Area\-Select\-Hover, simply invoke its constructor as follows \begin{DoxyVerb}  obj = vtkInteractorStyleAreaSelectHover
\end{DoxyVerb}
 \hypertarget{vtkwidgets_vtkxyplotwidget_Methods}{}\subsection{Methods}\label{vtkwidgets_vtkxyplotwidget_Methods}
The class vtk\-Interactor\-Style\-Area\-Select\-Hover has several methods that can be used. They are listed below. Note that the documentation is translated automatically from the V\-T\-K sources, and may not be completely intelligible. When in doubt, consult the V\-T\-K website. In the methods listed below, {\ttfamily obj} is an instance of the vtk\-Interactor\-Style\-Area\-Select\-Hover class. 
\begin{DoxyItemize}
\item {\ttfamily string = obj.\-Get\-Class\-Name ()}  
\item {\ttfamily int = obj.\-Is\-A (string name)}  
\item {\ttfamily vtk\-Interactor\-Style\-Area\-Select\-Hover = obj.\-New\-Instance ()}  
\item {\ttfamily vtk\-Interactor\-Style\-Area\-Select\-Hover = obj.\-Safe\-Down\-Cast (vtk\-Object o)}  
\item {\ttfamily obj.\-Set\-Layout (vtk\-Area\-Layout layout)} -\/ Must be set to the vtk\-Area\-Layout used to compute the bounds of each vertex.  
\item {\ttfamily vtk\-Area\-Layout = obj.\-Get\-Layout ()} -\/ Must be set to the vtk\-Area\-Layout used to compute the bounds of each vertex.  
\item {\ttfamily obj.\-Set\-Label\-Field (string )} -\/ The name of the field to use when displaying text in the hover balloon.  
\item {\ttfamily string = obj.\-Get\-Label\-Field ()} -\/ The name of the field to use when displaying text in the hover balloon.  
\item {\ttfamily obj.\-Set\-Use\-Rectangular\-Coordinates (bool )} -\/ Determine whether or not to use rectangular coordinates instead of polar coordinates.  
\item {\ttfamily bool = obj.\-Get\-Use\-Rectangular\-Coordinates ()} -\/ Determine whether or not to use rectangular coordinates instead of polar coordinates.  
\item {\ttfamily obj.\-Use\-Rectangular\-Coordinates\-On ()} -\/ Determine whether or not to use rectangular coordinates instead of polar coordinates.  
\item {\ttfamily obj.\-Use\-Rectangular\-Coordinates\-Off ()} -\/ Determine whether or not to use rectangular coordinates instead of polar coordinates.  
\item {\ttfamily obj.\-On\-Mouse\-Move ()} -\/ Overridden from vtk\-Interactor\-Style\-Image to provide the desired interaction behavior.  
\item {\ttfamily obj.\-Set\-Interactor (vtk\-Render\-Window\-Interactor rwi)} -\/ Set the interactor that this interactor style works with.  
\item {\ttfamily obj.\-Set\-High\-Light\-Color (double r, double g, double b)} -\/ Set the color used to highlight the hovered vertex.  
\item {\ttfamily obj.\-Set\-High\-Light\-Width (double lw)} -\/ The width of the line around the hovered vertex.  
\item {\ttfamily double = obj.\-Get\-High\-Light\-Width ()} -\/ The width of the line around the hovered vertex.  
\item {\ttfamily vtk\-Id\-Type = obj.\-Get\-Id\-At\-Pos (int x, int y)} -\/ Obtain the tree vertex id at the position specified.  
\end{DoxyItemize}\hypertarget{vtkviews_vtkinteractorstyletreemaphover}{}\section{vtk\-Interactor\-Style\-Tree\-Map\-Hover}\label{vtkviews_vtkinteractorstyletreemaphover}
Section\-: \hyperlink{sec_vtkviews}{Visualization Toolkit View Classes} \hypertarget{vtkwidgets_vtkxyplotwidget_Usage}{}\subsection{Usage}\label{vtkwidgets_vtkxyplotwidget_Usage}
The vtk\-Interactor\-Style\-Tree\-Map\-Hover specifically works with pipelines that create a tree map. Such pipelines will have a vtk\-Tree\-Map\-Layout filter and a vtk\-Tree\-Map\-To\-Poly\-Data filter, both of which must be passed to this interactor style for it to function correctly. This interactor style allows only 2\-D panning and zooming, and additionally provides a balloon containing the name of the vertex hovered over, and allows the user to highlight a vertex by clicking on it.

To create an instance of class vtk\-Interactor\-Style\-Tree\-Map\-Hover, simply invoke its constructor as follows \begin{DoxyVerb}  obj = vtkInteractorStyleTreeMapHover
\end{DoxyVerb}
 \hypertarget{vtkwidgets_vtkxyplotwidget_Methods}{}\subsection{Methods}\label{vtkwidgets_vtkxyplotwidget_Methods}
The class vtk\-Interactor\-Style\-Tree\-Map\-Hover has several methods that can be used. They are listed below. Note that the documentation is translated automatically from the V\-T\-K sources, and may not be completely intelligible. When in doubt, consult the V\-T\-K website. In the methods listed below, {\ttfamily obj} is an instance of the vtk\-Interactor\-Style\-Tree\-Map\-Hover class. 
\begin{DoxyItemize}
\item {\ttfamily string = obj.\-Get\-Class\-Name ()}  
\item {\ttfamily int = obj.\-Is\-A (string name)}  
\item {\ttfamily vtk\-Interactor\-Style\-Tree\-Map\-Hover = obj.\-New\-Instance ()}  
\item {\ttfamily vtk\-Interactor\-Style\-Tree\-Map\-Hover = obj.\-Safe\-Down\-Cast (vtk\-Object o)}  
\item {\ttfamily obj.\-Set\-Layout (vtk\-Tree\-Map\-Layout layout)} -\/ Must be set to the vtk\-Tree\-Map\-Layout used to compute the bounds of each vertex for the tree map.  
\item {\ttfamily vtk\-Tree\-Map\-Layout = obj.\-Get\-Layout ()} -\/ Must be set to the vtk\-Tree\-Map\-Layout used to compute the bounds of each vertex for the tree map.  
\item {\ttfamily obj.\-Set\-Tree\-Map\-To\-Poly\-Data (vtk\-Tree\-Map\-To\-Poly\-Data filter)} -\/ Must be set to the vtk\-Tree\-Map\-To\-Poly\-Data used to convert the tree map into polydata.  
\item {\ttfamily vtk\-Tree\-Map\-To\-Poly\-Data = obj.\-Get\-Tree\-Map\-To\-Poly\-Data ()} -\/ Must be set to the vtk\-Tree\-Map\-To\-Poly\-Data used to convert the tree map into polydata.  
\item {\ttfamily obj.\-Set\-Label\-Field (string )} -\/ The name of the field to use when displaying text in the hover balloon.  
\item {\ttfamily string = obj.\-Get\-Label\-Field ()} -\/ The name of the field to use when displaying text in the hover balloon.  
\item {\ttfamily obj.\-On\-Mouse\-Move ()} -\/ Overridden from vtk\-Interactor\-Style\-Image to provide the desired interaction behavior.  
\item {\ttfamily obj.\-On\-Left\-Button\-Up ()} -\/ Overridden from vtk\-Interactor\-Style\-Image to provide the desired interaction behavior.  
\item {\ttfamily obj.\-High\-Light\-Item (vtk\-Id\-Type id)} -\/ Highlights a specific vertex.  
\item {\ttfamily obj.\-High\-Light\-Current\-Selected\-Item ()} -\/ Highlights a specific vertex.  
\item {\ttfamily obj.\-Set\-Interactor (vtk\-Render\-Window\-Interactor rwi)}  
\item {\ttfamily obj.\-Set\-High\-Light\-Color (double r, double g, double b)} -\/ Set the color used to highlight the hovered vertex.  
\item {\ttfamily obj.\-Set\-Selection\-Light\-Color (double r, double g, double b)} -\/ Set the color used to highlight the selected vertex.  
\item {\ttfamily obj.\-Set\-High\-Light\-Width (double lw)} -\/ The width of the line around the hovered vertex.  
\item {\ttfamily double = obj.\-Get\-High\-Light\-Width ()} -\/ The width of the line around the hovered vertex.  
\item {\ttfamily obj.\-Set\-Selection\-Width (double lw)} -\/ The width of the line around the selected vertex.  
\item {\ttfamily double = obj.\-Get\-Selection\-Width ()} -\/ The width of the line around the selected vertex.  
\end{DoxyItemize}\hypertarget{vtkviews_vtkparallelcoordinateshistogramrepresentation}{}\section{vtk\-Parallel\-Coordinates\-Histogram\-Representation}\label{vtkviews_vtkparallelcoordinateshistogramrepresentation}
Section\-: \hyperlink{sec_vtkviews}{Visualization Toolkit View Classes} \hypertarget{vtkwidgets_vtkxyplotwidget_Usage}{}\subsection{Usage}\label{vtkwidgets_vtkxyplotwidget_Usage}
A parallel coordinates plot represents each variable in a multivariate data set as a separate axis. Individual samples of that data set are represented as a polyline that pass through each variable axis at positions that correspond to data values. This class can generate parallel coordinates plots identical to its superclass (vtk\-Parallel\-Coordinates\-Representation) and has the same interaction styles.

In addition to the standard parallel coordinates plot, this class also can draw a histogram summary of the parallel coordinates plot. Rather than draw every row in an input data set, first it computes a 2\-D histogram for all neighboring variable axes, then it draws bar (thickness corresponds to bin size) for each bin the histogram with opacity weighted by the number of rows contained in the bin. The result is essentially a density map.

Because this emphasizes dense regions over sparse outliers, this class also uses a vtk\-Compute\-Histogram2\-D\-Outliers instance to identify outlier table rows and draws those as standard parallel coordinates lines.

To create an instance of class vtk\-Parallel\-Coordinates\-Histogram\-Representation, simply invoke its constructor as follows \begin{DoxyVerb}  obj = vtkParallelCoordinatesHistogramRepresentation
\end{DoxyVerb}
 \hypertarget{vtkwidgets_vtkxyplotwidget_Methods}{}\subsection{Methods}\label{vtkwidgets_vtkxyplotwidget_Methods}
The class vtk\-Parallel\-Coordinates\-Histogram\-Representation has several methods that can be used. They are listed below. Note that the documentation is translated automatically from the V\-T\-K sources, and may not be completely intelligible. When in doubt, consult the V\-T\-K website. In the methods listed below, {\ttfamily obj} is an instance of the vtk\-Parallel\-Coordinates\-Histogram\-Representation class. 
\begin{DoxyItemize}
\item {\ttfamily string = obj.\-Get\-Class\-Name ()}  
\item {\ttfamily int = obj.\-Is\-A (string name)}  
\item {\ttfamily vtk\-Parallel\-Coordinates\-Histogram\-Representation = obj.\-New\-Instance ()}  
\item {\ttfamily vtk\-Parallel\-Coordinates\-Histogram\-Representation = obj.\-Safe\-Down\-Cast (vtk\-Object o)}  
\item {\ttfamily obj.\-Apply\-View\-Theme (vtk\-View\-Theme theme)} -\/ Apply the theme to this view.  
\item {\ttfamily obj.\-Set\-Use\-Histograms (int )} -\/ Whether to use the histogram rendering mode or the superclass's line rendering mode  
\item {\ttfamily int = obj.\-Get\-Use\-Histograms ()} -\/ Whether to use the histogram rendering mode or the superclass's line rendering mode  
\item {\ttfamily obj.\-Use\-Histograms\-On ()} -\/ Whether to use the histogram rendering mode or the superclass's line rendering mode  
\item {\ttfamily obj.\-Use\-Histograms\-Off ()} -\/ Whether to use the histogram rendering mode or the superclass's line rendering mode  
\item {\ttfamily obj.\-Set\-Show\-Outliers (int )} -\/ Whether to compute and show outlier lines  
\item {\ttfamily int = obj.\-Get\-Show\-Outliers ()} -\/ Whether to compute and show outlier lines  
\item {\ttfamily obj.\-Show\-Outliers\-On ()} -\/ Whether to compute and show outlier lines  
\item {\ttfamily obj.\-Show\-Outliers\-Off ()} -\/ Whether to compute and show outlier lines  
\item {\ttfamily obj.\-Set\-Histogram\-Lookup\-Table\-Range (double , double )} -\/ Control over the range of the lookup table used to draw the histogram quads.  
\item {\ttfamily obj.\-Set\-Histogram\-Lookup\-Table\-Range (double a\mbox{[}2\mbox{]})} -\/ Control over the range of the lookup table used to draw the histogram quads.  
\item {\ttfamily double = obj. Get\-Histogram\-Lookup\-Table\-Range ()} -\/ Control over the range of the lookup table used to draw the histogram quads.  
\item {\ttfamily obj.\-Set\-Preferred\-Number\-Of\-Outliers (int )} -\/ Target maximum number of outliers to be drawn, although not guaranteed.  
\item {\ttfamily int = obj.\-Get\-Preferred\-Number\-Of\-Outliers ()} -\/ Target maximum number of outliers to be drawn, although not guaranteed.  
\item {\ttfamily int = obj.\-Swap\-Axis\-Positions (int position1, int position2)} -\/ Calls superclass swap, and assures that only histograms affected by the swap get recomputed.  
\item {\ttfamily int = obj.\-Set\-Range\-At\-Position (int position, double range\mbox{[}2\mbox{]})} -\/ Calls the superclass method, and assures that only the two histograms affect by this call get recomputed.  
\end{DoxyItemize}\hypertarget{vtkviews_vtkparallelcoordinatesrepresentation}{}\section{vtk\-Parallel\-Coordinates\-Representation}\label{vtkviews_vtkparallelcoordinatesrepresentation}
Section\-: \hyperlink{sec_vtkviews}{Visualization Toolkit View Classes} \hypertarget{vtkwidgets_vtkxyplotwidget_Usage}{}\subsection{Usage}\label{vtkwidgets_vtkxyplotwidget_Usage}
A parallel coordinates plot represents each variable in a multivariate data set as a separate axis. Individual samples of that data set are represented as a polyline that pass through each variable axis at positions that correspond to data values. vtk\-Parallel\-Coordinates\-Representation generates this plot when added to a vtk\-Parallel\-Coordinates\-View, which handles interaction and highlighting. Sample polylines can alternatively be represented as s-\/curves by enabling the Use\-Curves flag.

There are three selection modes\-: lasso, angle, and function. Lasso selection picks sample lines that pass through a polyline. Angle selection picks sample lines that have similar slope to a line segment. Function selection picks sample lines that are near a linear function defined on two variables. This function specified by passing two (x,y) variable value pairs.

All primitives are plotted in normalized view coordinates \mbox{[}0,1\mbox{]}.

To create an instance of class vtk\-Parallel\-Coordinates\-Representation, simply invoke its constructor as follows \begin{DoxyVerb}  obj = vtkParallelCoordinatesRepresentation
\end{DoxyVerb}
 \hypertarget{vtkwidgets_vtkxyplotwidget_Methods}{}\subsection{Methods}\label{vtkwidgets_vtkxyplotwidget_Methods}
The class vtk\-Parallel\-Coordinates\-Representation has several methods that can be used. They are listed below. Note that the documentation is translated automatically from the V\-T\-K sources, and may not be completely intelligible. When in doubt, consult the V\-T\-K website. In the methods listed below, {\ttfamily obj} is an instance of the vtk\-Parallel\-Coordinates\-Representation class. 
\begin{DoxyItemize}
\item {\ttfamily string = obj.\-Get\-Class\-Name ()}  
\item {\ttfamily int = obj.\-Is\-A (string name)}  
\item {\ttfamily vtk\-Parallel\-Coordinates\-Representation = obj.\-New\-Instance ()}  
\item {\ttfamily vtk\-Parallel\-Coordinates\-Representation = obj.\-Safe\-Down\-Cast (vtk\-Object o)}  
\item {\ttfamily obj.\-Apply\-View\-Theme (vtk\-View\-Theme theme)} -\/ Apply the theme to this view. Cell\-Color is used for line coloring and titles. Edge\-Label\-Color is used for axis color. Cell\-Opacity is used for line opacity.  
\item {\ttfamily string = obj.\-Get\-Hover\-Text (vtk\-View view, int x, int y)} -\/ Returns the hover text at an x,y location.  
\item {\ttfamily int = obj.\-Set\-Position\-And\-Size (double position, double size)} -\/ Change the position of the plot  
\item {\ttfamily int = obj.\-Get\-Position\-And\-Size (double position, double size)} -\/ Change the position of the plot  
\item {\ttfamily obj.\-Set\-Axis\-Titles (vtk\-String\-Array )} -\/ Set/\-Get the axis titles  
\item {\ttfamily obj.\-Set\-Axis\-Titles (vtk\-Algorithm\-Output )} -\/ Set/\-Get the axis titles  
\item {\ttfamily obj.\-Set\-Plot\-Title (string )} -\/ Set the title for the entire plot  
\item {\ttfamily int = obj.\-Get\-Number\-Of\-Axes ()} -\/ Get the number of axes in the plot  
\item {\ttfamily int = obj.\-Get\-Number\-Of\-Samples ()}  
\item {\ttfamily obj.\-Set\-Number\-Of\-Axis\-Labels (int num)} -\/ Set/\-Get the number of labels to display on each axis  
\item {\ttfamily int = obj.\-Get\-Number\-Of\-Axis\-Labels ()} -\/ Set/\-Get the number of labels to display on each axis  
\item {\ttfamily int = obj.\-Swap\-Axis\-Positions (int position1, int position2)} -\/ Move an axis to a particular screen position. Using these methods requires an Update() before they will work properly.  
\item {\ttfamily int = obj.\-Set\-X\-Coordinate\-Of\-Position (int position, double xcoord)} -\/ Move an axis to a particular screen position. Using these methods requires an Update() before they will work properly.  
\item {\ttfamily double = obj.\-Get\-X\-Coordinate\-Of\-Position (int axis)} -\/ Move an axis to a particular screen position. Using these methods requires an Update() before they will work properly.  
\item {\ttfamily obj.\-Get\-X\-Coordinates\-Of\-Positions (double coords)} -\/ Move an axis to a particular screen position. Using these methods requires an Update() before they will work properly.  
\item {\ttfamily int = obj.\-Get\-Position\-Near\-X\-Coordinate (double xcoord)} -\/ Move an axis to a particular screen position. Using these methods requires an Update() before they will work properly.  
\item {\ttfamily obj.\-Set\-Use\-Curves (int )} -\/ Whether or not to display using curves  
\item {\ttfamily int = obj.\-Get\-Use\-Curves ()} -\/ Whether or not to display using curves  
\item {\ttfamily obj.\-Use\-Curves\-On ()} -\/ Whether or not to display using curves  
\item {\ttfamily obj.\-Use\-Curves\-Off ()} -\/ Whether or not to display using curves  
\item {\ttfamily obj.\-Set\-Curve\-Resolution (int )} -\/ Resolution of the curves displayed, enabled by setting Use\-Curves  
\item {\ttfamily int = obj.\-Get\-Curve\-Resolution ()} -\/ Resolution of the curves displayed, enabled by setting Use\-Curves  
\item {\ttfamily double = obj.\-Get\-Line\-Opacity ()} -\/ Access plot properties  
\item {\ttfamily double = obj.\-Get\-Font\-Size ()} -\/ Access plot properties  
\item {\ttfamily double = obj. Get\-Line\-Color ()} -\/ Access plot properties  
\item {\ttfamily double = obj. Get\-Axis\-Color ()} -\/ Access plot properties  
\item {\ttfamily double = obj. Get\-Axis\-Label\-Color ()} -\/ Access plot properties  
\item {\ttfamily obj.\-Set\-Line\-Opacity (double )} -\/ Access plot properties  
\item {\ttfamily obj.\-Set\-Font\-Size (double )} -\/ Access plot properties  
\item {\ttfamily obj.\-Set\-Line\-Color (double , double , double )} -\/ Access plot properties  
\item {\ttfamily obj.\-Set\-Line\-Color (double a\mbox{[}3\mbox{]})} -\/ Access plot properties  
\item {\ttfamily obj.\-Set\-Axis\-Color (double , double , double )} -\/ Access plot properties  
\item {\ttfamily obj.\-Set\-Axis\-Color (double a\mbox{[}3\mbox{]})} -\/ Access plot properties  
\item {\ttfamily obj.\-Set\-Axis\-Label\-Color (double , double , double )} -\/ Access plot properties  
\item {\ttfamily obj.\-Set\-Axis\-Label\-Color (double a\mbox{[}3\mbox{]})} -\/ Access plot properties  
\item {\ttfamily obj.\-Set\-Angle\-Brush\-Threshold (double )} -\/ Maximum angle difference (in degrees) of selection using angle/function brushes  
\item {\ttfamily double = obj.\-Get\-Angle\-Brush\-Threshold ()} -\/ Maximum angle difference (in degrees) of selection using angle/function brushes  
\item {\ttfamily obj.\-Set\-Function\-Brush\-Threshold (double )} -\/ Maximum angle difference (in degrees) of selection using angle/function brushes  
\item {\ttfamily double = obj.\-Get\-Function\-Brush\-Threshold ()} -\/ Maximum angle difference (in degrees) of selection using angle/function brushes  
\item {\ttfamily int = obj.\-Get\-Range\-At\-Position (int position, double range\mbox{[}2\mbox{]})} -\/ Set/get the value range of the axis at a particular screen position  
\item {\ttfamily int = obj.\-Set\-Range\-At\-Position (int position, double range\mbox{[}2\mbox{]})} -\/ Set/get the value range of the axis at a particular screen position  
\item {\ttfamily obj.\-Reset\-Axes ()} -\/ Reset the axes to their default positions and orders  
\item {\ttfamily obj.\-Lasso\-Select (int brush\-Class, int brush\-Operator, vtk\-Points brush\-Points)} -\/ Do a selection of the lines. See the main description for how to use these functions. Range\-Select is currently stubbed out.  
\item {\ttfamily obj.\-Angle\-Select (int brush\-Class, int brush\-Operator, double p1, double p2)} -\/ Do a selection of the lines. See the main description for how to use these functions. Range\-Select is currently stubbed out.  
\item {\ttfamily obj.\-Function\-Select (int brush\-Class, int brush\-Operator, double p1, double p2, double q1, double q2)} -\/ Do a selection of the lines. See the main description for how to use these functions. Range\-Select is currently stubbed out.  
\item {\ttfamily obj.\-Range\-Select (int brush\-Class, int brush\-Operator, double p1, double p2)} -\/ Do a selection of the lines. See the main description for how to use these functions. Range\-Select is currently stubbed out.  
\end{DoxyItemize}\hypertarget{vtkviews_vtkparallelcoordinatesview}{}\section{vtk\-Parallel\-Coordinates\-View}\label{vtkviews_vtkparallelcoordinatesview}
Section\-: \hyperlink{sec_vtkviews}{Visualization Toolkit View Classes} \hypertarget{vtkwidgets_vtkxyplotwidget_Usage}{}\subsection{Usage}\label{vtkwidgets_vtkxyplotwidget_Usage}
This class manages interaction with the vtk\-Parallel\-Coordinates\-Representation. There are two inspection modes\-: axis manipulation and line selection. In axis manipulation mode, P\-C axes can be dragged and reordered with the L\-M\-B, axis ranges can be increased/decreased by dragging up/down with the L\-M\-B, and R\-M\-B controls zoom and pan.

In line selection mode, there are three subclasses of selections\-: lasso, angle, and function selection. Lasso selection lets the user brush a line and select all P\-C lines that pass nearby. Angle selection lets the user draw a representative line between axes and select all lines that have similar orientation. Function selection lets the user draw two representative lines between a pair of axes and select all lines that match the linear interpolation of those lines.

There are several self-\/explanatory operators for combining selections\-: A\-D\-D, S\-U\-B\-T\-R\-A\-C\-T R\-E\-P\-L\-A\-C\-E, and I\-N\-T\-E\-R\-S\-E\-C\-T.

To create an instance of class vtk\-Parallel\-Coordinates\-View, simply invoke its constructor as follows \begin{DoxyVerb}  obj = vtkParallelCoordinatesView
\end{DoxyVerb}
 \hypertarget{vtkwidgets_vtkxyplotwidget_Methods}{}\subsection{Methods}\label{vtkwidgets_vtkxyplotwidget_Methods}
The class vtk\-Parallel\-Coordinates\-View has several methods that can be used. They are listed below. Note that the documentation is translated automatically from the V\-T\-K sources, and may not be completely intelligible. When in doubt, consult the V\-T\-K website. In the methods listed below, {\ttfamily obj} is an instance of the vtk\-Parallel\-Coordinates\-View class. 
\begin{DoxyItemize}
\item {\ttfamily string = obj.\-Get\-Class\-Name ()}  
\item {\ttfamily int = obj.\-Is\-A (string name)}  
\item {\ttfamily vtk\-Parallel\-Coordinates\-View = obj.\-New\-Instance ()}  
\item {\ttfamily vtk\-Parallel\-Coordinates\-View = obj.\-Safe\-Down\-Cast (vtk\-Object o)}  
\item {\ttfamily obj.\-Set\-Brush\-Mode (int )}  
\item {\ttfamily obj.\-Set\-Brush\-Mode\-To\-Lasso ()}  
\item {\ttfamily obj.\-Set\-Brush\-Mode\-To\-Angle ()}  
\item {\ttfamily obj.\-Set\-Brush\-Mode\-To\-Function ()}  
\item {\ttfamily obj.\-Set\-Brush\-Mode\-To\-Axis\-Threshold ()}  
\item {\ttfamily int = obj.\-Get\-Brush\-Mode ()}  
\item {\ttfamily obj.\-Set\-Brush\-Operator (int )}  
\item {\ttfamily obj.\-Set\-Brush\-Operator\-To\-Add ()}  
\item {\ttfamily obj.\-Set\-Brush\-Operator\-To\-Subtract ()}  
\item {\ttfamily obj.\-Set\-Brush\-Operator\-To\-Intersect ()}  
\item {\ttfamily obj.\-Set\-Brush\-Operator\-To\-Replace ()}  
\item {\ttfamily int = obj.\-Get\-Brush\-Operator ()}  
\item {\ttfamily obj.\-Set\-Inspect\-Mode (int )}  
\item {\ttfamily obj.\-Set\-Inspect\-Mode\-To\-Manipulate\-Axes ()}  
\item {\ttfamily obj.\-Set\-Inpsect\-Mode\-To\-Select\-Data ()}  
\item {\ttfamily int = obj.\-Get\-Inspect\-Mode ()}  
\item {\ttfamily obj.\-Set\-Maximum\-Number\-Of\-Brush\-Points (int )}  
\item {\ttfamily int = obj.\-Get\-Maximum\-Number\-Of\-Brush\-Points ()}  
\item {\ttfamily obj.\-Set\-Current\-Brush\-Class (int )}  
\item {\ttfamily int = obj.\-Get\-Current\-Brush\-Class ()}  
\item {\ttfamily obj.\-Apply\-View\-Theme (vtk\-View\-Theme theme)}  
\end{DoxyItemize}\hypertarget{vtkviews_vtkrenderedgraphrepresentation}{}\section{vtk\-Rendered\-Graph\-Representation}\label{vtkviews_vtkrenderedgraphrepresentation}
Section\-: \hyperlink{sec_vtkviews}{Visualization Toolkit View Classes} \hypertarget{vtkwidgets_vtkxyplotwidget_Usage}{}\subsection{Usage}\label{vtkwidgets_vtkxyplotwidget_Usage}
To create an instance of class vtk\-Rendered\-Graph\-Representation, simply invoke its constructor as follows \begin{DoxyVerb}  obj = vtkRenderedGraphRepresentation
\end{DoxyVerb}
 \hypertarget{vtkwidgets_vtkxyplotwidget_Methods}{}\subsection{Methods}\label{vtkwidgets_vtkxyplotwidget_Methods}
The class vtk\-Rendered\-Graph\-Representation has several methods that can be used. They are listed below. Note that the documentation is translated automatically from the V\-T\-K sources, and may not be completely intelligible. When in doubt, consult the V\-T\-K website. In the methods listed below, {\ttfamily obj} is an instance of the vtk\-Rendered\-Graph\-Representation class. 
\begin{DoxyItemize}
\item {\ttfamily string = obj.\-Get\-Class\-Name ()}  
\item {\ttfamily int = obj.\-Is\-A (string name)}  
\item {\ttfamily vtk\-Rendered\-Graph\-Representation = obj.\-New\-Instance ()}  
\item {\ttfamily vtk\-Rendered\-Graph\-Representation = obj.\-Safe\-Down\-Cast (vtk\-Object o)}  
\item {\ttfamily obj.\-Set\-Vertex\-Label\-Array\-Name (string name)}  
\item {\ttfamily string = obj.\-Get\-Vertex\-Label\-Array\-Name ()}  
\item {\ttfamily obj.\-Set\-Vertex\-Label\-Priority\-Array\-Name (string name)}  
\item {\ttfamily string = obj.\-Get\-Vertex\-Label\-Priority\-Array\-Name ()}  
\item {\ttfamily obj.\-Set\-Vertex\-Label\-Visibility (bool b)}  
\item {\ttfamily bool = obj.\-Get\-Vertex\-Label\-Visibility ()}  
\item {\ttfamily obj.\-Vertex\-Label\-Visibility\-On ()}  
\item {\ttfamily obj.\-Vertex\-Label\-Visibility\-Off ()}  
\item {\ttfamily obj.\-Set\-Vertex\-Label\-Text\-Property (vtk\-Text\-Property p)}  
\item {\ttfamily vtk\-Text\-Property = obj.\-Get\-Vertex\-Label\-Text\-Property ()}  
\item {\ttfamily obj.\-Set\-Vertex\-Hover\-Array\-Name (string )}  
\item {\ttfamily string = obj.\-Get\-Vertex\-Hover\-Array\-Name ()}  
\item {\ttfamily obj.\-Set\-Hide\-Vertex\-Labels\-On\-Interaction (bool )} -\/ Whether to hide the display of vertex labels during mouse interaction. Default is off.  
\item {\ttfamily bool = obj.\-Get\-Hide\-Vertex\-Labels\-On\-Interaction ()} -\/ Whether to hide the display of vertex labels during mouse interaction. Default is off.  
\item {\ttfamily obj.\-Hide\-Vertex\-Labels\-On\-Interaction\-On ()} -\/ Whether to hide the display of vertex labels during mouse interaction. Default is off.  
\item {\ttfamily obj.\-Hide\-Vertex\-Labels\-On\-Interaction\-Off ()} -\/ Whether to hide the display of vertex labels during mouse interaction. Default is off.  
\item {\ttfamily obj.\-Set\-Edge\-Label\-Array\-Name (string name)}  
\item {\ttfamily string = obj.\-Get\-Edge\-Label\-Array\-Name ()}  
\item {\ttfamily obj.\-Set\-Edge\-Label\-Priority\-Array\-Name (string name)}  
\item {\ttfamily string = obj.\-Get\-Edge\-Label\-Priority\-Array\-Name ()}  
\item {\ttfamily obj.\-Set\-Edge\-Label\-Visibility (bool b)}  
\item {\ttfamily bool = obj.\-Get\-Edge\-Label\-Visibility ()}  
\item {\ttfamily obj.\-Edge\-Label\-Visibility\-On ()}  
\item {\ttfamily obj.\-Edge\-Label\-Visibility\-Off ()}  
\item {\ttfamily obj.\-Set\-Edge\-Label\-Text\-Property (vtk\-Text\-Property p)}  
\item {\ttfamily vtk\-Text\-Property = obj.\-Get\-Edge\-Label\-Text\-Property ()}  
\item {\ttfamily obj.\-Set\-Edge\-Hover\-Array\-Name (string )}  
\item {\ttfamily string = obj.\-Get\-Edge\-Hover\-Array\-Name ()}  
\item {\ttfamily obj.\-Set\-Hide\-Edge\-Labels\-On\-Interaction (bool )} -\/ Whether to hide the display of edge labels during mouse interaction. Default is off.  
\item {\ttfamily bool = obj.\-Get\-Hide\-Edge\-Labels\-On\-Interaction ()} -\/ Whether to hide the display of edge labels during mouse interaction. Default is off.  
\item {\ttfamily obj.\-Hide\-Edge\-Labels\-On\-Interaction\-On ()} -\/ Whether to hide the display of edge labels during mouse interaction. Default is off.  
\item {\ttfamily obj.\-Hide\-Edge\-Labels\-On\-Interaction\-Off ()} -\/ Whether to hide the display of edge labels during mouse interaction. Default is off.  
\item {\ttfamily obj.\-Set\-Vertex\-Icon\-Array\-Name (string name)}  
\item {\ttfamily string = obj.\-Get\-Vertex\-Icon\-Array\-Name ()}  
\item {\ttfamily obj.\-Set\-Vertex\-Icon\-Priority\-Array\-Name (string name)}  
\item {\ttfamily string = obj.\-Get\-Vertex\-Icon\-Priority\-Array\-Name ()}  
\item {\ttfamily obj.\-Set\-Vertex\-Icon\-Visibility (bool b)}  
\item {\ttfamily bool = obj.\-Get\-Vertex\-Icon\-Visibility ()}  
\item {\ttfamily obj.\-Vertex\-Icon\-Visibility\-On ()}  
\item {\ttfamily obj.\-Vertex\-Icon\-Visibility\-Off ()}  
\item {\ttfamily obj.\-Add\-Vertex\-Icon\-Type (string name, int type)}  
\item {\ttfamily obj.\-Clear\-Vertex\-Icon\-Types ()}  
\item {\ttfamily obj.\-Set\-Use\-Vertex\-Icon\-Type\-Map (bool b)}  
\item {\ttfamily bool = obj.\-Get\-Use\-Vertex\-Icon\-Type\-Map ()}  
\item {\ttfamily obj.\-Use\-Vertex\-Icon\-Type\-Map\-On ()}  
\item {\ttfamily obj.\-Use\-Vertex\-Icon\-Type\-Map\-Off ()}  
\item {\ttfamily obj.\-Set\-Vertex\-Icon\-Alignment (int align)}  
\item {\ttfamily int = obj.\-Get\-Vertex\-Icon\-Alignment ()}  
\item {\ttfamily obj.\-Set\-Vertex\-Selected\-Icon (int icon)}  
\item {\ttfamily int = obj.\-Get\-Vertex\-Selected\-Icon ()}  
\item {\ttfamily obj.\-Set\-Vertex\-Icon\-Selection\-Mode (int mode)} -\/ Set the mode to one of 
\begin{DoxyItemize}
\item vtk\-Apply\-Icons\-::\-S\-E\-L\-E\-C\-T\-E\-D\-\_\-\-I\-C\-O\-N -\/ use Vertex\-Selected\-Icon 
\item vtk\-Apply\-Icons\-::\-S\-E\-L\-E\-C\-T\-E\-D\-\_\-\-O\-F\-F\-S\-E\-T -\/ use Vertex\-Selected\-Icon as offset 
\item vtk\-Apply\-Icons\-::\-A\-N\-N\-O\-T\-A\-T\-I\-O\-N\-\_\-\-I\-C\-O\-N -\/ use current annotation icon 
\item vtk\-Apply\-Icons\-::\-I\-G\-N\-O\-R\-E\-\_\-\-S\-E\-L\-E\-C\-T\-I\-O\-N -\/ ignore selected elements 
\end{DoxyItemize}The default is I\-G\-N\-O\-R\-E\-\_\-\-S\-E\-L\-E\-C\-T\-I\-O\-N.  
\item {\ttfamily int = obj.\-Get\-Vertex\-Icon\-Selection\-Mode ()} -\/ Set the mode to one of 
\begin{DoxyItemize}
\item vtk\-Apply\-Icons\-::\-S\-E\-L\-E\-C\-T\-E\-D\-\_\-\-I\-C\-O\-N -\/ use Vertex\-Selected\-Icon 
\item vtk\-Apply\-Icons\-::\-S\-E\-L\-E\-C\-T\-E\-D\-\_\-\-O\-F\-F\-S\-E\-T -\/ use Vertex\-Selected\-Icon as offset 
\item vtk\-Apply\-Icons\-::\-A\-N\-N\-O\-T\-A\-T\-I\-O\-N\-\_\-\-I\-C\-O\-N -\/ use current annotation icon 
\item vtk\-Apply\-Icons\-::\-I\-G\-N\-O\-R\-E\-\_\-\-S\-E\-L\-E\-C\-T\-I\-O\-N -\/ ignore selected elements 
\end{DoxyItemize}The default is I\-G\-N\-O\-R\-E\-\_\-\-S\-E\-L\-E\-C\-T\-I\-O\-N.  
\item {\ttfamily obj.\-Set\-Vertex\-Icon\-Selection\-Mode\-To\-Selected\-Icon ()} -\/ Set the mode to one of 
\begin{DoxyItemize}
\item vtk\-Apply\-Icons\-::\-S\-E\-L\-E\-C\-T\-E\-D\-\_\-\-I\-C\-O\-N -\/ use Vertex\-Selected\-Icon 
\item vtk\-Apply\-Icons\-::\-S\-E\-L\-E\-C\-T\-E\-D\-\_\-\-O\-F\-F\-S\-E\-T -\/ use Vertex\-Selected\-Icon as offset 
\item vtk\-Apply\-Icons\-::\-A\-N\-N\-O\-T\-A\-T\-I\-O\-N\-\_\-\-I\-C\-O\-N -\/ use current annotation icon 
\item vtk\-Apply\-Icons\-::\-I\-G\-N\-O\-R\-E\-\_\-\-S\-E\-L\-E\-C\-T\-I\-O\-N -\/ ignore selected elements 
\end{DoxyItemize}The default is I\-G\-N\-O\-R\-E\-\_\-\-S\-E\-L\-E\-C\-T\-I\-O\-N.  
\item {\ttfamily obj.\-Set\-Vertex\-Icon\-Selection\-Mode\-To\-Selected\-Offset ()} -\/ Set the mode to one of 
\begin{DoxyItemize}
\item vtk\-Apply\-Icons\-::\-S\-E\-L\-E\-C\-T\-E\-D\-\_\-\-I\-C\-O\-N -\/ use Vertex\-Selected\-Icon 
\item vtk\-Apply\-Icons\-::\-S\-E\-L\-E\-C\-T\-E\-D\-\_\-\-O\-F\-F\-S\-E\-T -\/ use Vertex\-Selected\-Icon as offset 
\item vtk\-Apply\-Icons\-::\-A\-N\-N\-O\-T\-A\-T\-I\-O\-N\-\_\-\-I\-C\-O\-N -\/ use current annotation icon 
\item vtk\-Apply\-Icons\-::\-I\-G\-N\-O\-R\-E\-\_\-\-S\-E\-L\-E\-C\-T\-I\-O\-N -\/ ignore selected elements 
\end{DoxyItemize}The default is I\-G\-N\-O\-R\-E\-\_\-\-S\-E\-L\-E\-C\-T\-I\-O\-N.  
\item {\ttfamily obj.\-Set\-Vertex\-Icon\-Selection\-Mode\-To\-Annotation\-Icon ()} -\/ Set the mode to one of 
\begin{DoxyItemize}
\item vtk\-Apply\-Icons\-::\-S\-E\-L\-E\-C\-T\-E\-D\-\_\-\-I\-C\-O\-N -\/ use Vertex\-Selected\-Icon 
\item vtk\-Apply\-Icons\-::\-S\-E\-L\-E\-C\-T\-E\-D\-\_\-\-O\-F\-F\-S\-E\-T -\/ use Vertex\-Selected\-Icon as offset 
\item vtk\-Apply\-Icons\-::\-A\-N\-N\-O\-T\-A\-T\-I\-O\-N\-\_\-\-I\-C\-O\-N -\/ use current annotation icon 
\item vtk\-Apply\-Icons\-::\-I\-G\-N\-O\-R\-E\-\_\-\-S\-E\-L\-E\-C\-T\-I\-O\-N -\/ ignore selected elements 
\end{DoxyItemize}The default is I\-G\-N\-O\-R\-E\-\_\-\-S\-E\-L\-E\-C\-T\-I\-O\-N.  
\item {\ttfamily obj.\-Set\-Vertex\-Icon\-Selection\-Mode\-To\-Ignore\-Selection ()}  
\item {\ttfamily obj.\-Set\-Edge\-Icon\-Array\-Name (string name)}  
\item {\ttfamily string = obj.\-Get\-Edge\-Icon\-Array\-Name ()}  
\item {\ttfamily obj.\-Set\-Edge\-Icon\-Priority\-Array\-Name (string name)}  
\item {\ttfamily string = obj.\-Get\-Edge\-Icon\-Priority\-Array\-Name ()}  
\item {\ttfamily obj.\-Set\-Edge\-Icon\-Visibility (bool b)}  
\item {\ttfamily bool = obj.\-Get\-Edge\-Icon\-Visibility ()}  
\item {\ttfamily obj.\-Edge\-Icon\-Visibility\-On ()}  
\item {\ttfamily obj.\-Edge\-Icon\-Visibility\-Off ()}  
\item {\ttfamily obj.\-Add\-Edge\-Icon\-Type (string name, int type)}  
\item {\ttfamily obj.\-Clear\-Edge\-Icon\-Types ()}  
\item {\ttfamily obj.\-Set\-Use\-Edge\-Icon\-Type\-Map (bool b)}  
\item {\ttfamily bool = obj.\-Get\-Use\-Edge\-Icon\-Type\-Map ()}  
\item {\ttfamily obj.\-Use\-Edge\-Icon\-Type\-Map\-On ()}  
\item {\ttfamily obj.\-Use\-Edge\-Icon\-Type\-Map\-Off ()}  
\item {\ttfamily obj.\-Set\-Edge\-Icon\-Alignment (int align)}  
\item {\ttfamily int = obj.\-Get\-Edge\-Icon\-Alignment ()}  
\item {\ttfamily obj.\-Set\-Color\-Vertices\-By\-Array (bool b)}  
\item {\ttfamily bool = obj.\-Get\-Color\-Vertices\-By\-Array ()}  
\item {\ttfamily obj.\-Color\-Vertices\-By\-Array\-On ()}  
\item {\ttfamily obj.\-Color\-Vertices\-By\-Array\-Off ()}  
\item {\ttfamily obj.\-Set\-Vertex\-Color\-Array\-Name (string name)}  
\item {\ttfamily string = obj.\-Get\-Vertex\-Color\-Array\-Name ()}  
\item {\ttfamily obj.\-Set\-Color\-Edges\-By\-Array (bool b)}  
\item {\ttfamily bool = obj.\-Get\-Color\-Edges\-By\-Array ()}  
\item {\ttfamily obj.\-Color\-Edges\-By\-Array\-On ()}  
\item {\ttfamily obj.\-Color\-Edges\-By\-Array\-Off ()}  
\item {\ttfamily obj.\-Set\-Edge\-Color\-Array\-Name (string name)}  
\item {\ttfamily string = obj.\-Get\-Edge\-Color\-Array\-Name ()}  
\item {\ttfamily obj.\-Set\-Enable\-Vertices\-By\-Array (bool b)}  
\item {\ttfamily bool = obj.\-Get\-Enable\-Vertices\-By\-Array ()}  
\item {\ttfamily obj.\-Enable\-Vertices\-By\-Array\-On ()}  
\item {\ttfamily obj.\-Enable\-Vertices\-By\-Array\-Off ()}  
\item {\ttfamily obj.\-Set\-Enabled\-Vertices\-Array\-Name (string name)}  
\item {\ttfamily string = obj.\-Get\-Enabled\-Vertices\-Array\-Name ()}  
\item {\ttfamily obj.\-Set\-Enable\-Edges\-By\-Array (bool b)}  
\item {\ttfamily bool = obj.\-Get\-Enable\-Edges\-By\-Array ()}  
\item {\ttfamily obj.\-Enable\-Edges\-By\-Array\-On ()}  
\item {\ttfamily obj.\-Enable\-Edges\-By\-Array\-Off ()}  
\item {\ttfamily obj.\-Set\-Enabled\-Edges\-Array\-Name (string name)}  
\item {\ttfamily string = obj.\-Get\-Enabled\-Edges\-Array\-Name ()}  
\item {\ttfamily obj.\-Set\-Edge\-Visibility (bool b)}  
\item {\ttfamily bool = obj.\-Get\-Edge\-Visibility ()}  
\item {\ttfamily obj.\-Edge\-Visibility\-On ()}  
\item {\ttfamily obj.\-Edge\-Visibility\-Off ()}  
\item {\ttfamily obj.\-Set\-Layout\-Strategy (vtk\-Graph\-Layout\-Strategy strategy)} -\/ Set/get the graph layout strategy.  
\item {\ttfamily vtk\-Graph\-Layout\-Strategy = obj.\-Get\-Layout\-Strategy ()} -\/ Set/get the graph layout strategy.  
\item {\ttfamily obj.\-Set\-Layout\-Strategy (string name)} -\/ Get/set the layout strategy by name.  
\item {\ttfamily string = obj.\-Get\-Layout\-Strategy\-Name ()} -\/ Get/set the layout strategy by name.  
\item {\ttfamily obj.\-Set\-Layout\-Strategy\-To\-Random ()} -\/ Set predefined layout strategies.  
\item {\ttfamily obj.\-Set\-Layout\-Strategy\-To\-Force\-Directed ()} -\/ Set predefined layout strategies.  
\item {\ttfamily obj.\-Set\-Layout\-Strategy\-To\-Simple2\-D ()} -\/ Set predefined layout strategies.  
\item {\ttfamily obj.\-Set\-Layout\-Strategy\-To\-Clustering2\-D ()} -\/ Set predefined layout strategies.  
\item {\ttfamily obj.\-Set\-Layout\-Strategy\-To\-Community2\-D ()} -\/ Set predefined layout strategies.  
\item {\ttfamily obj.\-Set\-Layout\-Strategy\-To\-Fast2\-D ()} -\/ Set predefined layout strategies.  
\item {\ttfamily obj.\-Set\-Layout\-Strategy\-To\-Pass\-Through ()} -\/ Set predefined layout strategies.  
\item {\ttfamily obj.\-Set\-Layout\-Strategy\-To\-Circular ()} -\/ Set predefined layout strategies.  
\item {\ttfamily obj.\-Set\-Layout\-Strategy\-To\-Tree ()} -\/ Set predefined layout strategies.  
\item {\ttfamily obj.\-Set\-Layout\-Strategy\-To\-Cosmic\-Tree ()} -\/ Set predefined layout strategies.  
\item {\ttfamily obj.\-Set\-Layout\-Strategy\-To\-Cone ()} -\/ Set predefined layout strategies.  
\item {\ttfamily obj.\-Set\-Layout\-Strategy\-To\-Span\-Tree ()} -\/ Set the layout strategy to use coordinates from arrays. The x array must be specified. The y and z arrays are optional.  
\item {\ttfamily obj.\-Set\-Layout\-Strategy\-To\-Assign\-Coordinates (string xarr, string yarr, string zarr)} -\/ Set the layout strategy to use coordinates from arrays. The x array must be specified. The y and z arrays are optional.  
\item {\ttfamily obj.\-Set\-Layout\-Strategy\-To\-Tree (bool radial, double angle, double leaf\-Spacing, double log\-Spacing)} -\/ Set the layout strategy to a tree layout. Radial indicates whether to do a radial or standard top-\/down tree layout. The angle parameter is the angular distance spanned by the tree. Leaf spacing is a value from 0 to 1 indicating how much of the radial layout should be allocated to leaf nodes (as opposed to between tree branches). The log spacing value is a non-\/negative value where $>$ 1 will create expanding levels, $<$ 1 will create contracting levels, and = 1 makes all levels the same size. See vtk\-Tree\-Layout\-Strategy for more information.  
\item {\ttfamily obj.\-Set\-Layout\-Strategy\-To\-Cosmic\-Tree (string node\-Size\-Array\-Name, bool size\-Leaf\-Nodes\-Onlytrue, int layout\-Depth, vtk\-Id\-Type layout\-Root)} -\/ Set the layout strategy to a cosmic tree layout. node\-Size\-Array\-Name is the array used to size the circles (default is N\-U\-L\-L, which makes leaf nodes the same size). size\-Leaf\-Nodes\-Only only uses the leaf node sizes, and computes the parent size as the sum of the child sizes (default true). layout\-Depth stops layout at a certain depth (default is 0, which does the entire tree). layout\-Root is the vertex that will be considered the root node of the layout (default is -\/1, which will use the tree's root). See vtk\-Cosmic\-Tree\-Layout\-Strategy for more information.  
\item {\ttfamily obj.\-Set\-Edge\-Layout\-Strategy (vtk\-Edge\-Layout\-Strategy strategy)} -\/ Set/get the graph layout strategy.  
\item {\ttfamily vtk\-Edge\-Layout\-Strategy = obj.\-Get\-Edge\-Layout\-Strategy ()} -\/ Set/get the graph layout strategy.  
\item {\ttfamily obj.\-Set\-Edge\-Layout\-Strategy\-To\-Arc\-Parallel ()} -\/ Set/get the graph layout strategy.  
\item {\ttfamily obj.\-Set\-Edge\-Layout\-Strategy\-To\-Pass\-Through ()} -\/ Set the edge layout strategy to a geospatial arced strategy appropriate for vtk\-Geo\-View.  
\item {\ttfamily obj.\-Set\-Edge\-Layout\-Strategy\-To\-Geo (double explode\-Factor)} -\/ Set the edge layout strategy to a geospatial arced strategy appropriate for vtk\-Geo\-View.  
\item {\ttfamily obj.\-Set\-Edge\-Layout\-Strategy (string name)} -\/ Set the edge layout strategy by name.  
\item {\ttfamily string = obj.\-Get\-Edge\-Layout\-Strategy\-Name ()} -\/ Set the edge layout strategy by name.  
\item {\ttfamily obj.\-Apply\-View\-Theme (vtk\-View\-Theme theme)} -\/ Apply a theme to this representation.  
\item {\ttfamily obj.\-Set\-Glyph\-Type (int type)} -\/ Set the graph vertex glyph type.  
\item {\ttfamily int = obj.\-Get\-Glyph\-Type ()} -\/ Set the graph vertex glyph type.  
\item {\ttfamily obj.\-Set\-Scaling (bool b)} -\/ Set whether to scale vertex glyphs.  
\item {\ttfamily bool = obj.\-Get\-Scaling ()} -\/ Set whether to scale vertex glyphs.  
\item {\ttfamily obj.\-Scaling\-On ()} -\/ Set whether to scale vertex glyphs.  
\item {\ttfamily obj.\-Scaling\-Off ()} -\/ Set whether to scale vertex glyphs.  
\item {\ttfamily obj.\-Set\-Scaling\-Array\-Name (string name)} -\/ Set the glyph scaling array name.  
\item {\ttfamily string = obj.\-Get\-Scaling\-Array\-Name ()} -\/ Set the glyph scaling array name.  
\item {\ttfamily obj.\-Set\-Vertex\-Scalar\-Bar\-Visibility (bool b)} -\/ Vertex/edge scalar bar visibility.  
\item {\ttfamily bool = obj.\-Get\-Vertex\-Scalar\-Bar\-Visibility ()} -\/ Vertex/edge scalar bar visibility.  
\item {\ttfamily obj.\-Set\-Edge\-Scalar\-Bar\-Visibility (bool b)} -\/ Vertex/edge scalar bar visibility.  
\item {\ttfamily bool = obj.\-Get\-Edge\-Scalar\-Bar\-Visibility ()} -\/ Vertex/edge scalar bar visibility.  
\item {\ttfamily bool = obj.\-Is\-Layout\-Complete ()} -\/ Whether the current graph layout is complete.  
\item {\ttfamily obj.\-Update\-Layout ()} -\/ Performs another iteration on the graph layout.  
\item {\ttfamily obj.\-Compute\-Selected\-Graph\-Bounds (double bounds\mbox{[}6\mbox{]})} -\/ Compute the bounding box of the selected subgraph.  
\end{DoxyItemize}\hypertarget{vtkviews_vtkrenderedhierarchyrepresentation}{}\section{vtk\-Rendered\-Hierarchy\-Representation}\label{vtkviews_vtkrenderedhierarchyrepresentation}
Section\-: \hyperlink{sec_vtkviews}{Visualization Toolkit View Classes} \hypertarget{vtkwidgets_vtkxyplotwidget_Usage}{}\subsection{Usage}\label{vtkwidgets_vtkxyplotwidget_Usage}
To create an instance of class vtk\-Rendered\-Hierarchy\-Representation, simply invoke its constructor as follows \begin{DoxyVerb}  obj = vtkRenderedHierarchyRepresentation
\end{DoxyVerb}
 \hypertarget{vtkwidgets_vtkxyplotwidget_Methods}{}\subsection{Methods}\label{vtkwidgets_vtkxyplotwidget_Methods}
The class vtk\-Rendered\-Hierarchy\-Representation has several methods that can be used. They are listed below. Note that the documentation is translated automatically from the V\-T\-K sources, and may not be completely intelligible. When in doubt, consult the V\-T\-K website. In the methods listed below, {\ttfamily obj} is an instance of the vtk\-Rendered\-Hierarchy\-Representation class. 
\begin{DoxyItemize}
\item {\ttfamily string = obj.\-Get\-Class\-Name ()}  
\item {\ttfamily int = obj.\-Is\-A (string name)}  
\item {\ttfamily vtk\-Rendered\-Hierarchy\-Representation = obj.\-New\-Instance ()}  
\item {\ttfamily vtk\-Rendered\-Hierarchy\-Representation = obj.\-Safe\-Down\-Cast (vtk\-Object o)}  
\item {\ttfamily obj.\-Set\-Graph\-Edge\-Label\-Array\-Name (string name)} -\/  
\item {\ttfamily obj.\-Set\-Graph\-Edge\-Label\-Array\-Name (string name, int idx)} -\/  
\item {\ttfamily string = obj.\-Get\-Graph\-Edge\-Label\-Array\-Name ()} -\/  
\item {\ttfamily string = obj.\-Get\-Graph\-Edge\-Label\-Array\-Name (int idx)} -\/  
\item {\ttfamily obj.\-Set\-Graph\-Edge\-Label\-Visibility (bool vis)}  
\item {\ttfamily obj.\-Set\-Graph\-Edge\-Label\-Visibility (bool vis, int idx)}  
\item {\ttfamily bool = obj.\-Get\-Graph\-Edge\-Label\-Visibility ()}  
\item {\ttfamily bool = obj.\-Get\-Graph\-Edge\-Label\-Visibility (int idx)}  
\item {\ttfamily obj.\-Graph\-Edge\-Label\-Visibility\-On ()}  
\item {\ttfamily obj.\-Graph\-Edge\-Label\-Visibility\-Off ()}  
\item {\ttfamily obj.\-Set\-Graph\-Edge\-Color\-Array\-Name (string name)}  
\item {\ttfamily obj.\-Set\-Graph\-Edge\-Color\-Array\-Name (string name, int idx)}  
\item {\ttfamily string = obj.\-Get\-Graph\-Edge\-Color\-Array\-Name ()}  
\item {\ttfamily string = obj.\-Get\-Graph\-Edge\-Color\-Array\-Name (int idx)}  
\item {\ttfamily obj.\-Set\-Color\-Graph\-Edges\-By\-Array (bool vis)}  
\item {\ttfamily obj.\-Set\-Color\-Graph\-Edges\-By\-Array (bool vis, int idx)}  
\item {\ttfamily bool = obj.\-Get\-Color\-Graph\-Edges\-By\-Array ()}  
\item {\ttfamily bool = obj.\-Get\-Color\-Graph\-Edges\-By\-Array (int idx)}  
\item {\ttfamily obj.\-Color\-Graph\-Edges\-By\-Array\-On ()}  
\item {\ttfamily obj.\-Color\-Graph\-Edges\-By\-Array\-Off ()}  
\item {\ttfamily obj.\-Set\-Graph\-Edge\-Color\-To\-Spline\-Fraction ()}  
\item {\ttfamily obj.\-Set\-Graph\-Edge\-Color\-To\-Spline\-Fraction (int idx)}  
\item {\ttfamily obj.\-Set\-Graph\-Visibility (bool vis)}  
\item {\ttfamily obj.\-Set\-Graph\-Visibility (bool vis, int idx)}  
\item {\ttfamily bool = obj.\-Get\-Graph\-Visibility ()}  
\item {\ttfamily bool = obj.\-Get\-Graph\-Visibility (int idx)}  
\item {\ttfamily obj.\-Graph\-Visibility\-On ()}  
\item {\ttfamily obj.\-Graph\-Visibility\-Off ()}  
\item {\ttfamily obj.\-Set\-Bundling\-Strength (double strength)}  
\item {\ttfamily obj.\-Set\-Bundling\-Strength (double strength, int idx)}  
\item {\ttfamily double = obj.\-Get\-Bundling\-Strength ()}  
\item {\ttfamily double = obj.\-Get\-Bundling\-Strength (int idx)}  
\item {\ttfamily obj.\-Set\-Graph\-Edge\-Label\-Font\-Size (int size)}  
\item {\ttfamily obj.\-Set\-Graph\-Edge\-Label\-Font\-Size (int size, int idx)}  
\item {\ttfamily int = obj.\-Get\-Graph\-Edge\-Label\-Font\-Size ()}  
\item {\ttfamily int = obj.\-Get\-Graph\-Edge\-Label\-Font\-Size (int idx)}  
\end{DoxyItemize}\hypertarget{vtkviews_vtkrenderedrepresentation}{}\section{vtk\-Rendered\-Representation}\label{vtkviews_vtkrenderedrepresentation}
Section\-: \hyperlink{sec_vtkviews}{Visualization Toolkit View Classes} \hypertarget{vtkwidgets_vtkxyplotwidget_Usage}{}\subsection{Usage}\label{vtkwidgets_vtkxyplotwidget_Usage}
To create an instance of class vtk\-Rendered\-Representation, simply invoke its constructor as follows \begin{DoxyVerb}  obj = vtkRenderedRepresentation
\end{DoxyVerb}
 \hypertarget{vtkwidgets_vtkxyplotwidget_Methods}{}\subsection{Methods}\label{vtkwidgets_vtkxyplotwidget_Methods}
The class vtk\-Rendered\-Representation has several methods that can be used. They are listed below. Note that the documentation is translated automatically from the V\-T\-K sources, and may not be completely intelligible. When in doubt, consult the V\-T\-K website. In the methods listed below, {\ttfamily obj} is an instance of the vtk\-Rendered\-Representation class. 
\begin{DoxyItemize}
\item {\ttfamily string = obj.\-Get\-Class\-Name ()}  
\item {\ttfamily int = obj.\-Is\-A (string name)}  
\item {\ttfamily vtk\-Rendered\-Representation = obj.\-New\-Instance ()}  
\item {\ttfamily vtk\-Rendered\-Representation = obj.\-Safe\-Down\-Cast (vtk\-Object o)}  
\item {\ttfamily obj.\-Set\-Label\-Render\-Mode (int )} -\/ Set the label render mode. vtk\-Render\-View\-::\-Q\-T -\/ Use Qt-\/based labeler with fitted labeling and unicode support. Requires V\-T\-K\-\_\-\-U\-S\-E\-\_\-\-Q\-T to be on. vtk\-Render\-View\-::\-F\-R\-E\-E\-T\-Y\-P\-E -\/ Use standard freetype text rendering.  
\item {\ttfamily int = obj.\-Get\-Label\-Render\-Mode ()} -\/ Set the label render mode. vtk\-Render\-View\-::\-Q\-T -\/ Use Qt-\/based labeler with fitted labeling and unicode support. Requires V\-T\-K\-\_\-\-U\-S\-E\-\_\-\-Q\-T to be on. vtk\-Render\-View\-::\-F\-R\-E\-E\-T\-Y\-P\-E -\/ Use standard freetype text rendering.  
\end{DoxyItemize}\hypertarget{vtkviews_vtkrenderedsurfacerepresentation}{}\section{vtk\-Rendered\-Surface\-Representation}\label{vtkviews_vtkrenderedsurfacerepresentation}
Section\-: \hyperlink{sec_vtkviews}{Visualization Toolkit View Classes} \hypertarget{vtkwidgets_vtkxyplotwidget_Usage}{}\subsection{Usage}\label{vtkwidgets_vtkxyplotwidget_Usage}
vtk\-Rendered\-Surface\-Representation is used to show a geometric dataset in a view. The representation uses a vtk\-Geometry\-Filter to convert the dataset to polygonal data (e.\-g. volumetric data is converted to its external surface). The representation may then be added to vtk\-Render\-View.

To create an instance of class vtk\-Rendered\-Surface\-Representation, simply invoke its constructor as follows \begin{DoxyVerb}  obj = vtkRenderedSurfaceRepresentation
\end{DoxyVerb}
 \hypertarget{vtkwidgets_vtkxyplotwidget_Methods}{}\subsection{Methods}\label{vtkwidgets_vtkxyplotwidget_Methods}
The class vtk\-Rendered\-Surface\-Representation has several methods that can be used. They are listed below. Note that the documentation is translated automatically from the V\-T\-K sources, and may not be completely intelligible. When in doubt, consult the V\-T\-K website. In the methods listed below, {\ttfamily obj} is an instance of the vtk\-Rendered\-Surface\-Representation class. 
\begin{DoxyItemize}
\item {\ttfamily string = obj.\-Get\-Class\-Name ()}  
\item {\ttfamily int = obj.\-Is\-A (string name)}  
\item {\ttfamily vtk\-Rendered\-Surface\-Representation = obj.\-New\-Instance ()}  
\item {\ttfamily vtk\-Rendered\-Surface\-Representation = obj.\-Safe\-Down\-Cast (vtk\-Object o)}  
\item {\ttfamily obj.\-Set\-Cell\-Color\-Array\-Name (string array\-Name)}  
\item {\ttfamily string = obj.\-Get\-Cell\-Color\-Array\-Name ()} -\/ Apply a theme to this representation.  
\item {\ttfamily obj.\-Apply\-View\-Theme (vtk\-View\-Theme theme)} -\/ Apply a theme to this representation.  
\end{DoxyItemize}\hypertarget{vtkviews_vtkrenderedtreearearepresentation}{}\section{vtk\-Rendered\-Tree\-Area\-Representation}\label{vtkviews_vtkrenderedtreearearepresentation}
Section\-: \hyperlink{sec_vtkviews}{Visualization Toolkit View Classes} \hypertarget{vtkwidgets_vtkxyplotwidget_Usage}{}\subsection{Usage}\label{vtkwidgets_vtkxyplotwidget_Usage}
To create an instance of class vtk\-Rendered\-Tree\-Area\-Representation, simply invoke its constructor as follows \begin{DoxyVerb}  obj = vtkRenderedTreeAreaRepresentation
\end{DoxyVerb}
 \hypertarget{vtkwidgets_vtkxyplotwidget_Methods}{}\subsection{Methods}\label{vtkwidgets_vtkxyplotwidget_Methods}
The class vtk\-Rendered\-Tree\-Area\-Representation has several methods that can be used. They are listed below. Note that the documentation is translated automatically from the V\-T\-K sources, and may not be completely intelligible. When in doubt, consult the V\-T\-K website. In the methods listed below, {\ttfamily obj} is an instance of the vtk\-Rendered\-Tree\-Area\-Representation class. 
\begin{DoxyItemize}
\item {\ttfamily string = obj.\-Get\-Class\-Name ()}  
\item {\ttfamily int = obj.\-Is\-A (string name)}  
\item {\ttfamily vtk\-Rendered\-Tree\-Area\-Representation = obj.\-New\-Instance ()}  
\item {\ttfamily vtk\-Rendered\-Tree\-Area\-Representation = obj.\-Safe\-Down\-Cast (vtk\-Object o)}  
\item {\ttfamily obj.\-Set\-Label\-Render\-Mode (int mode)} -\/ Set the label render mode. Q\-T -\/ Use vtk\-Qt\-Tree\-Ring\-Labeler with fitted labeling and unicode support. Requires V\-T\-K\-\_\-\-U\-S\-E\-\_\-\-Q\-T to be on. F\-R\-E\-E\-T\-Y\-P\-E -\/ Use standard freetype text rendering.  
\item {\ttfamily obj.\-Set\-Area\-Label\-Array\-Name (string name)} -\/ The array to use for area labeling. Default is \char`\"{}label\char`\"{}.  
\item {\ttfamily string = obj.\-Get\-Area\-Label\-Array\-Name ()} -\/ The array to use for area labeling. Default is \char`\"{}label\char`\"{}.  
\item {\ttfamily obj.\-Set\-Area\-Size\-Array\-Name (string name)} -\/ The array to use for area sizes. Default is \char`\"{}size\char`\"{}.  
\item {\ttfamily string = obj.\-Get\-Area\-Size\-Array\-Name ()} -\/ The array to use for area sizes. Default is \char`\"{}size\char`\"{}.  
\item {\ttfamily obj.\-Set\-Area\-Label\-Priority\-Array\-Name (string name)} -\/ The array to use for area labeling priority. Default is \char`\"{}\-Graph\-Vertex\-Degree\char`\"{}.  
\item {\ttfamily string = obj.\-Get\-Area\-Label\-Priority\-Array\-Name ()} -\/ The array to use for area labeling priority. Default is \char`\"{}\-Graph\-Vertex\-Degree\char`\"{}.  
\item {\ttfamily obj.\-Set\-Graph\-Edge\-Label\-Array\-Name (string name)} -\/ The array to use for edge labeling. Default is \char`\"{}label\char`\"{}.  
\item {\ttfamily obj.\-Set\-Graph\-Edge\-Label\-Array\-Name (string name, int idx)} -\/ The array to use for edge labeling. Default is \char`\"{}label\char`\"{}.  
\item {\ttfamily string = obj.\-Get\-Graph\-Edge\-Label\-Array\-Name ()} -\/ The array to use for edge labeling. Default is \char`\"{}label\char`\"{}.  
\item {\ttfamily string = obj.\-Get\-Graph\-Edge\-Label\-Array\-Name (int idx)} -\/ The array to use for edge labeling. Default is \char`\"{}label\char`\"{}.  
\item {\ttfamily obj.\-Set\-Graph\-Edge\-Label\-Text\-Property (vtk\-Text\-Property tp)} -\/ The text property for the graph edge labels.  
\item {\ttfamily obj.\-Set\-Graph\-Edge\-Label\-Text\-Property (vtk\-Text\-Property tp, int idx)} -\/ The text property for the graph edge labels.  
\item {\ttfamily vtk\-Text\-Property = obj.\-Get\-Graph\-Edge\-Label\-Text\-Property ()} -\/ The text property for the graph edge labels.  
\item {\ttfamily vtk\-Text\-Property = obj.\-Get\-Graph\-Edge\-Label\-Text\-Property (int idx)} -\/ The text property for the graph edge labels.  
\item {\ttfamily obj.\-Set\-Area\-Hover\-Array\-Name (string )} -\/ The name of the array whose value appears when the mouse hovers over a rectangle in the treemap.  
\item {\ttfamily string = obj.\-Get\-Area\-Hover\-Array\-Name ()} -\/ The name of the array whose value appears when the mouse hovers over a rectangle in the treemap.  
\item {\ttfamily obj.\-Set\-Area\-Label\-Visibility (bool vis)} -\/ Whether to show area labels. Default is off.  
\item {\ttfamily bool = obj.\-Get\-Area\-Label\-Visibility ()} -\/ Whether to show area labels. Default is off.  
\item {\ttfamily obj.\-Area\-Label\-Visibility\-On ()} -\/ Whether to show area labels. Default is off.  
\item {\ttfamily obj.\-Area\-Label\-Visibility\-Off ()} -\/ Whether to show area labels. Default is off.  
\item {\ttfamily obj.\-Set\-Area\-Label\-Text\-Property (vtk\-Text\-Property tp)} -\/ The text property for the area labels.  
\item {\ttfamily vtk\-Text\-Property = obj.\-Get\-Area\-Label\-Text\-Property ()} -\/ The text property for the area labels.  
\item {\ttfamily obj.\-Set\-Graph\-Edge\-Label\-Visibility (bool vis)} -\/ Whether to show edge labels. Default is off.  
\item {\ttfamily obj.\-Set\-Graph\-Edge\-Label\-Visibility (bool vis, int idx)} -\/ Whether to show edge labels. Default is off.  
\item {\ttfamily bool = obj.\-Get\-Graph\-Edge\-Label\-Visibility ()} -\/ Whether to show edge labels. Default is off.  
\item {\ttfamily bool = obj.\-Get\-Graph\-Edge\-Label\-Visibility (int idx)} -\/ Whether to show edge labels. Default is off.  
\item {\ttfamily obj.\-Graph\-Edge\-Label\-Visibility\-On ()} -\/ Whether to show edge labels. Default is off.  
\item {\ttfamily obj.\-Graph\-Edge\-Label\-Visibility\-Off ()} -\/ Whether to show edge labels. Default is off.  
\item {\ttfamily obj.\-Set\-Area\-Color\-Array\-Name (string name)} -\/ The array to use for coloring vertices. Default is \char`\"{}color\char`\"{}.  
\item {\ttfamily string = obj.\-Get\-Area\-Color\-Array\-Name ()} -\/ The array to use for coloring vertices. Default is \char`\"{}color\char`\"{}.  
\item {\ttfamily obj.\-Set\-Color\-Areas\-By\-Array (bool vis)} -\/ Whether to color vertices. Default is off.  
\item {\ttfamily bool = obj.\-Get\-Color\-Areas\-By\-Array ()} -\/ Whether to color vertices. Default is off.  
\item {\ttfamily obj.\-Color\-Areas\-By\-Array\-On ()} -\/ Whether to color vertices. Default is off.  
\item {\ttfamily obj.\-Color\-Areas\-By\-Array\-Off ()} -\/ Whether to color vertices. Default is off.  
\item {\ttfamily obj.\-Set\-Graph\-Edge\-Color\-Array\-Name (string name)} -\/ The array to use for coloring edges. Default is \char`\"{}color\char`\"{}.  
\item {\ttfamily obj.\-Set\-Graph\-Edge\-Color\-Array\-Name (string name, int idx)} -\/ The array to use for coloring edges. Default is \char`\"{}color\char`\"{}.  
\item {\ttfamily string = obj.\-Get\-Graph\-Edge\-Color\-Array\-Name ()} -\/ The array to use for coloring edges. Default is \char`\"{}color\char`\"{}.  
\item {\ttfamily string = obj.\-Get\-Graph\-Edge\-Color\-Array\-Name (int idx)} -\/ The array to use for coloring edges. Default is \char`\"{}color\char`\"{}.  
\item {\ttfamily obj.\-Set\-Graph\-Edge\-Color\-To\-Spline\-Fraction ()} -\/ Set the color to be the spline fraction  
\item {\ttfamily obj.\-Set\-Graph\-Edge\-Color\-To\-Spline\-Fraction (int idx)} -\/ Set the color to be the spline fraction  
\item {\ttfamily obj.\-Set\-Color\-Graph\-Edges\-By\-Array (bool vis)} -\/ Whether to color edges. Default is off.  
\item {\ttfamily obj.\-Set\-Color\-Graph\-Edges\-By\-Array (bool vis, int idx)} -\/ Whether to color edges. Default is off.  
\item {\ttfamily bool = obj.\-Get\-Color\-Graph\-Edges\-By\-Array ()} -\/ Whether to color edges. Default is off.  
\item {\ttfamily bool = obj.\-Get\-Color\-Graph\-Edges\-By\-Array (int idx)} -\/ Whether to color edges. Default is off.  
\item {\ttfamily obj.\-Color\-Graph\-Edges\-By\-Array\-On ()} -\/ Whether to color edges. Default is off.  
\item {\ttfamily obj.\-Color\-Graph\-Edges\-By\-Array\-Off ()} -\/ Whether to color edges. Default is off.  
\item {\ttfamily obj.\-Set\-Graph\-Hover\-Array\-Name (string name)} -\/ The name of the array whose value appears when the mouse hovers over a graph edge.  
\item {\ttfamily obj.\-Set\-Graph\-Hover\-Array\-Name (string name, int idx)} -\/ The name of the array whose value appears when the mouse hovers over a graph edge.  
\item {\ttfamily string = obj.\-Get\-Graph\-Hover\-Array\-Name ()} -\/ The name of the array whose value appears when the mouse hovers over a graph edge.  
\item {\ttfamily string = obj.\-Get\-Graph\-Hover\-Array\-Name (int idx)} -\/ The name of the array whose value appears when the mouse hovers over a graph edge.  
\item {\ttfamily obj.\-Set\-Shrink\-Percentage (double value)} -\/ Set the region shrink percentage between 0.\-0 and 1.\-0.  
\item {\ttfamily double = obj.\-Get\-Shrink\-Percentage ()} -\/ Set the region shrink percentage between 0.\-0 and 1.\-0.  
\item {\ttfamily obj.\-Set\-Graph\-Bundling\-Strength (double strength)} -\/ Set the bundling strength.  
\item {\ttfamily obj.\-Set\-Graph\-Bundling\-Strength (double strength, int idx)} -\/ Set the bundling strength.  
\item {\ttfamily double = obj.\-Get\-Graph\-Bundling\-Strength ()} -\/ Set the bundling strength.  
\item {\ttfamily double = obj.\-Get\-Graph\-Bundling\-Strength (int idx)} -\/ Set the bundling strength.  
\item {\ttfamily obj.\-Set\-Graph\-Spline\-Type (int type, int idx)} -\/ Sets the spline type for the graph edges. vtk\-Spline\-Graph\-Edges\-::\-C\-U\-S\-T\-O\-M uses a vtk\-Cardinal\-Spline. vtk\-Spline\-Graph\-Edges\-::\-B\-S\-P\-L\-I\-N\-E uses a b-\/spline. The default is C\-U\-S\-T\-O\-M.  
\item {\ttfamily int = obj.\-Get\-Graph\-Spline\-Type (int idx)} -\/ Sets the spline type for the graph edges. vtk\-Spline\-Graph\-Edges\-::\-C\-U\-S\-T\-O\-M uses a vtk\-Cardinal\-Spline. vtk\-Spline\-Graph\-Edges\-::\-B\-S\-P\-L\-I\-N\-E uses a b-\/spline. The default is C\-U\-S\-T\-O\-M.  
\item {\ttfamily obj.\-Set\-Area\-Layout\-Strategy (vtk\-Area\-Layout\-Strategy strategy)} -\/ The layout strategy for producing spatial regions for the tree.  
\item {\ttfamily vtk\-Area\-Layout\-Strategy = obj.\-Get\-Area\-Layout\-Strategy ()} -\/ The layout strategy for producing spatial regions for the tree.  
\item {\ttfamily obj.\-Set\-Area\-To\-Poly\-Data (vtk\-Poly\-Data\-Algorithm area\-To\-Poly)} -\/ The filter for converting areas to polydata. This may e.\-g. be vtk\-Tree\-Map\-To\-Poly\-Data or vtk\-Tree\-Ring\-To\-Poly\-Data. The filter must take a vtk\-Tree as input and produce vtk\-Poly\-Data.  
\item {\ttfamily vtk\-Poly\-Data\-Algorithm = obj.\-Get\-Area\-To\-Poly\-Data ()} -\/ The filter for converting areas to polydata. This may e.\-g. be vtk\-Tree\-Map\-To\-Poly\-Data or vtk\-Tree\-Ring\-To\-Poly\-Data. The filter must take a vtk\-Tree as input and produce vtk\-Poly\-Data.  
\item {\ttfamily obj.\-Set\-Use\-Rectangular\-Coordinates (bool )} -\/ Whether the area represents radial or rectangular coordinates.  
\item {\ttfamily bool = obj.\-Get\-Use\-Rectangular\-Coordinates ()} -\/ Whether the area represents radial or rectangular coordinates.  
\item {\ttfamily obj.\-Use\-Rectangular\-Coordinates\-On ()} -\/ Whether the area represents radial or rectangular coordinates.  
\item {\ttfamily obj.\-Use\-Rectangular\-Coordinates\-Off ()} -\/ Whether the area represents radial or rectangular coordinates.  
\item {\ttfamily obj.\-Set\-Area\-Label\-Mapper (vtk\-Labeled\-Data\-Mapper mapper)} -\/ The mapper for rendering labels on areas. This may e.\-g. be vtk\-Dynamic2\-D\-Label\-Mapper or vtk\-Tree\-Map\-Label\-Mapper.  
\item {\ttfamily vtk\-Labeled\-Data\-Mapper = obj.\-Get\-Area\-Label\-Mapper ()} -\/ The mapper for rendering labels on areas. This may e.\-g. be vtk\-Dynamic2\-D\-Label\-Mapper or vtk\-Tree\-Map\-Label\-Mapper.  
\item {\ttfamily obj.\-Apply\-View\-Theme (vtk\-View\-Theme theme)} -\/ Apply the theme to this view.  
\item {\ttfamily obj.\-Set\-Edge\-Scalar\-Bar\-Visibility (bool b)} -\/ Visibility of scalar bar actor for edges.  
\item {\ttfamily bool = obj.\-Get\-Edge\-Scalar\-Bar\-Visibility ()} -\/ Visibility of scalar bar actor for edges.  
\end{DoxyItemize}\hypertarget{vtkviews_vtkrenderview}{}\section{vtk\-Render\-View}\label{vtkviews_vtkrenderview}
Section\-: \hyperlink{sec_vtkviews}{Visualization Toolkit View Classes} \hypertarget{vtkwidgets_vtkxyplotwidget_Usage}{}\subsection{Usage}\label{vtkwidgets_vtkxyplotwidget_Usage}
vtk\-Render\-View is a view which contains a vtk\-Renderer. You may add vtk\-Actors directly to the renderer, or add certain vtk\-Data\-Representation subclasses to the renderer. The render view supports drag selection with the mouse to select cells.

This class is also the parent class for any more specialized view which uses a renderer.

To create an instance of class vtk\-Render\-View, simply invoke its constructor as follows \begin{DoxyVerb}  obj = vtkRenderView
\end{DoxyVerb}
 \hypertarget{vtkwidgets_vtkxyplotwidget_Methods}{}\subsection{Methods}\label{vtkwidgets_vtkxyplotwidget_Methods}
The class vtk\-Render\-View has several methods that can be used. They are listed below. Note that the documentation is translated automatically from the V\-T\-K sources, and may not be completely intelligible. When in doubt, consult the V\-T\-K website. In the methods listed below, {\ttfamily obj} is an instance of the vtk\-Render\-View class. 
\begin{DoxyItemize}
\item {\ttfamily string = obj.\-Get\-Class\-Name ()}  
\item {\ttfamily int = obj.\-Is\-A (string name)}  
\item {\ttfamily vtk\-Render\-View = obj.\-New\-Instance ()}  
\item {\ttfamily vtk\-Render\-View = obj.\-Safe\-Down\-Cast (vtk\-Object o)}  
\item {\ttfamily vtk\-Renderer = obj.\-Get\-Renderer ()} -\/ Gets the renderer for this view.  
\item {\ttfamily vtk\-Render\-Window = obj.\-Get\-Render\-Window ()} -\/ Get a handle to the render window.  
\item {\ttfamily vtk\-Render\-Window\-Interactor = obj.\-Get\-Interactor ()} -\/ The render window interactor.  
\item {\ttfamily obj.\-Set\-Interactor (vtk\-Render\-Window\-Interactor interactor)} -\/ The render window interactor.  
\item {\ttfamily obj.\-Set\-Interactor\-Style (vtk\-Interactor\-Observer style)} -\/ The interactor style associated with the render view.  
\item {\ttfamily vtk\-Interactor\-Observer = obj.\-Get\-Interactor\-Style ()} -\/ The interactor style associated with the render view.  
\item {\ttfamily obj.\-Set\-Interaction\-Mode (int mode)} -\/ Set the interaction mode for the view. Choices are\-: vtk\-Render\-View\-::\-I\-N\-T\-E\-R\-A\-C\-T\-I\-O\-N\-\_\-\-M\-O\-D\-E\-\_\-2\-D -\/ 2\-D interactor vtk\-Render\-View\-::\-I\-N\-T\-E\-R\-A\-C\-T\-I\-O\-N\-\_\-\-M\-O\-D\-E\-\_\-3\-D -\/ 3\-D interactor  
\item {\ttfamily int = obj.\-Get\-Interaction\-Mode ()} -\/ Set the interaction mode for the view. Choices are\-: vtk\-Render\-View\-::\-I\-N\-T\-E\-R\-A\-C\-T\-I\-O\-N\-\_\-\-M\-O\-D\-E\-\_\-2\-D -\/ 2\-D interactor vtk\-Render\-View\-::\-I\-N\-T\-E\-R\-A\-C\-T\-I\-O\-N\-\_\-\-M\-O\-D\-E\-\_\-3\-D -\/ 3\-D interactor  
\item {\ttfamily obj.\-Set\-Interaction\-Mode\-To2\-D ()} -\/ Set the interaction mode for the view. Choices are\-: vtk\-Render\-View\-::\-I\-N\-T\-E\-R\-A\-C\-T\-I\-O\-N\-\_\-\-M\-O\-D\-E\-\_\-2\-D -\/ 2\-D interactor vtk\-Render\-View\-::\-I\-N\-T\-E\-R\-A\-C\-T\-I\-O\-N\-\_\-\-M\-O\-D\-E\-\_\-3\-D -\/ 3\-D interactor  
\item {\ttfamily obj.\-Set\-Interaction\-Mode\-To3\-D ()} -\/ Applies a view theme to this view.  
\item {\ttfamily obj.\-Apply\-View\-Theme (vtk\-View\-Theme theme)} -\/ Applies a view theme to this view.  
\item {\ttfamily obj.\-Set\-Transform (vtk\-Abstract\-Transform transform)} -\/ Set the view's transform. All vtk\-Rendered\-Representations added to this view should use this transform.  
\item {\ttfamily vtk\-Abstract\-Transform = obj.\-Get\-Transform ()} -\/ Set the view's transform. All vtk\-Rendered\-Representations added to this view should use this transform.  
\item {\ttfamily obj.\-Set\-Display\-Hover\-Text (bool b)} -\/ Whether the view should display hover text.  
\item {\ttfamily bool = obj.\-Get\-Display\-Hover\-Text ()} -\/ Whether the view should display hover text.  
\item {\ttfamily obj.\-Display\-Hover\-Text\-On ()} -\/ Whether the view should display hover text.  
\item {\ttfamily obj.\-Display\-Hover\-Text\-Off ()} -\/ Whether the view should display hover text.  
\item {\ttfamily obj.\-Set\-Selection\-Mode (int )} -\/ Sets the selection mode for the render view. S\-U\-R\-F\-A\-C\-E selection uses vtk\-Hardware\-Selector to perform a selection of visible cells. F\-R\-U\-S\-T\-U\-M selection just creates a view frustum selection, which will select everything in the frustum.  
\item {\ttfamily int = obj.\-Get\-Selection\-Mode\-Min\-Value ()} -\/ Sets the selection mode for the render view. S\-U\-R\-F\-A\-C\-E selection uses vtk\-Hardware\-Selector to perform a selection of visible cells. F\-R\-U\-S\-T\-U\-M selection just creates a view frustum selection, which will select everything in the frustum.  
\item {\ttfamily int = obj.\-Get\-Selection\-Mode\-Max\-Value ()} -\/ Sets the selection mode for the render view. S\-U\-R\-F\-A\-C\-E selection uses vtk\-Hardware\-Selector to perform a selection of visible cells. F\-R\-U\-S\-T\-U\-M selection just creates a view frustum selection, which will select everything in the frustum.  
\item {\ttfamily int = obj.\-Get\-Selection\-Mode ()} -\/ Sets the selection mode for the render view. S\-U\-R\-F\-A\-C\-E selection uses vtk\-Hardware\-Selector to perform a selection of visible cells. F\-R\-U\-S\-T\-U\-M selection just creates a view frustum selection, which will select everything in the frustum.  
\item {\ttfamily obj.\-Set\-Selection\-Mode\-To\-Surface ()} -\/ Sets the selection mode for the render view. S\-U\-R\-F\-A\-C\-E selection uses vtk\-Hardware\-Selector to perform a selection of visible cells. F\-R\-U\-S\-T\-U\-M selection just creates a view frustum selection, which will select everything in the frustum.  
\item {\ttfamily obj.\-Set\-Selection\-Mode\-To\-Frustum ()} -\/ Updates the representations, then calls Render() on the render window associated with this view.  
\item {\ttfamily obj.\-Render ()} -\/ Updates the representations, then calls Render() on the render window associated with this view.  
\item {\ttfamily obj.\-Reset\-Camera ()} -\/ Updates the representations, then calls Reset\-Camera() on the renderer associated with this view.  
\item {\ttfamily obj.\-Reset\-Camera\-Clipping\-Range ()} -\/ Updates the representations, then calls Reset\-Camera\-Clipping\-Range() on the renderer associated with this view.  
\item {\ttfamily obj.\-Add\-Labels (vtk\-Algorithm\-Output conn)} -\/ Add labels from an input connection with an associated text property. The output must be a vtk\-Label\-Hierarchy (normally the output of vtk\-Point\-Set\-To\-Label\-Hierarchy).  
\item {\ttfamily obj.\-Remove\-Labels (vtk\-Algorithm\-Output conn)} -\/ Remove labels from an input connection.  
\item {\ttfamily obj.\-Set\-Icon\-Texture (vtk\-Texture texture)} -\/ Set the icon sheet to use for rendering icons.  
\item {\ttfamily vtk\-Texture = obj.\-Get\-Icon\-Texture ()} -\/ Set the icon sheet to use for rendering icons.  
\item {\ttfamily obj.\-Set\-Icon\-Size (int , int )} -\/ Set the size of each icon in the icon texture.  
\item {\ttfamily obj.\-Set\-Icon\-Size (int a\mbox{[}2\mbox{]})} -\/ Set the size of each icon in the icon texture.  
\item {\ttfamily int = obj. Get\-Icon\-Size ()} -\/ Set the size of each icon in the icon texture.  
\item {\ttfamily obj.\-Set\-Label\-Placement\-Mode (int mode)} -\/ Label placement mode. N\-O\-\_\-\-O\-V\-E\-R\-L\-A\-P uses vtk\-Label\-Placement\-Mapper, which has a faster startup time and works with 2\-D or 3\-D labels. A\-L\-L displays all labels (Warning\-: This may cause incredibly slow render times on datasets with more than a few hundred labels).  
\item {\ttfamily int = obj.\-Get\-Label\-Placement\-Mode ()} -\/ Label placement mode. N\-O\-\_\-\-O\-V\-E\-R\-L\-A\-P uses vtk\-Label\-Placement\-Mapper, which has a faster startup time and works with 2\-D or 3\-D labels. A\-L\-L displays all labels (Warning\-: This may cause incredibly slow render times on datasets with more than a few hundred labels).  
\item {\ttfamily obj.\-Set\-Label\-Placement\-Mode\-To\-No\-Overlap ()} -\/ Label placement mode. N\-O\-\_\-\-O\-V\-E\-R\-L\-A\-P uses vtk\-Label\-Placement\-Mapper, which has a faster startup time and works with 2\-D or 3\-D labels. A\-L\-L displays all labels (Warning\-: This may cause incredibly slow render times on datasets with more than a few hundred labels).  
\item {\ttfamily obj.\-Set\-Label\-Placement\-Mode\-To\-All ()} -\/ Label render mode. F\-R\-E\-E\-T\-Y\-P\-E uses the freetype label rendering. Q\-T uses more advanced Qt-\/based label rendering.  
\item {\ttfamily obj.\-Set\-Label\-Render\-Mode (int mode)} -\/ Label render mode. F\-R\-E\-E\-T\-Y\-P\-E uses the freetype label rendering. Q\-T uses more advanced Qt-\/based label rendering.  
\item {\ttfamily int = obj.\-Get\-Label\-Render\-Mode ()} -\/ Label render mode. F\-R\-E\-E\-T\-Y\-P\-E uses the freetype label rendering. Q\-T uses more advanced Qt-\/based label rendering.  
\item {\ttfamily obj.\-Set\-Label\-Render\-Mode\-To\-Freetype ()} -\/ Label render mode. F\-R\-E\-E\-T\-Y\-P\-E uses the freetype label rendering. Q\-T uses more advanced Qt-\/based label rendering.  
\item {\ttfamily obj.\-Set\-Label\-Render\-Mode\-To\-Qt ()}  
\end{DoxyItemize}\hypertarget{vtkviews_vtktreeareaview}{}\section{vtk\-Tree\-Area\-View}\label{vtkviews_vtktreeareaview}
Section\-: \hyperlink{sec_vtkviews}{Visualization Toolkit View Classes} \hypertarget{vtkwidgets_vtkxyplotwidget_Usage}{}\subsection{Usage}\label{vtkwidgets_vtkxyplotwidget_Usage}
Takes a graph and a hierarchy (currently a tree) and lays out the graph vertices based on their categorization within the hierarchy.

.S\-E\-E A\-L\-S\-O vtk\-Graph\-Layout\-View

.S\-E\-C\-T\-I\-O\-N Thanks Thanks to Jason Shepherd for implementing this class

To create an instance of class vtk\-Tree\-Area\-View, simply invoke its constructor as follows \begin{DoxyVerb}  obj = vtkTreeAreaView
\end{DoxyVerb}
 \hypertarget{vtkwidgets_vtkxyplotwidget_Methods}{}\subsection{Methods}\label{vtkwidgets_vtkxyplotwidget_Methods}
The class vtk\-Tree\-Area\-View has several methods that can be used. They are listed below. Note that the documentation is translated automatically from the V\-T\-K sources, and may not be completely intelligible. When in doubt, consult the V\-T\-K website. In the methods listed below, {\ttfamily obj} is an instance of the vtk\-Tree\-Area\-View class. 
\begin{DoxyItemize}
\item {\ttfamily string = obj.\-Get\-Class\-Name ()}  
\item {\ttfamily int = obj.\-Is\-A (string name)}  
\item {\ttfamily vtk\-Tree\-Area\-View = obj.\-New\-Instance ()}  
\item {\ttfamily vtk\-Tree\-Area\-View = obj.\-Safe\-Down\-Cast (vtk\-Object o)}  
\item {\ttfamily vtk\-Data\-Representation = obj.\-Set\-Tree\-From\-Input\-Connection (vtk\-Algorithm\-Output conn)} -\/ Set the tree and graph representations to the appropriate input ports.  
\item {\ttfamily vtk\-Data\-Representation = obj.\-Set\-Tree\-From\-Input (vtk\-Tree input)} -\/ Set the tree and graph representations to the appropriate input ports.  
\item {\ttfamily vtk\-Data\-Representation = obj.\-Set\-Graph\-From\-Input\-Connection (vtk\-Algorithm\-Output conn)} -\/ Set the tree and graph representations to the appropriate input ports.  
\item {\ttfamily vtk\-Data\-Representation = obj.\-Set\-Graph\-From\-Input (vtk\-Graph input)} -\/ Set the tree and graph representations to the appropriate input ports.  
\item {\ttfamily obj.\-Set\-Area\-Label\-Array\-Name (string name)} -\/ The array to use for area labeling. Default is \char`\"{}label\char`\"{}.  
\item {\ttfamily string = obj.\-Get\-Area\-Label\-Array\-Name ()} -\/ The array to use for area labeling. Default is \char`\"{}label\char`\"{}.  
\item {\ttfamily obj.\-Set\-Area\-Size\-Array\-Name (string name)} -\/ The array to use for area sizes. Default is \char`\"{}size\char`\"{}.  
\item {\ttfamily string = obj.\-Get\-Area\-Size\-Array\-Name ()} -\/ The array to use for area sizes. Default is \char`\"{}size\char`\"{}.  
\item {\ttfamily obj.\-Set\-Label\-Priority\-Array\-Name (string name)} -\/ The array to use for area labeling priority. Default is \char`\"{}\-Graph\-Vertex\-Degree\char`\"{}.  
\item {\ttfamily string = obj.\-Get\-Label\-Priority\-Array\-Name ()} -\/ The array to use for area labeling priority. Default is \char`\"{}\-Graph\-Vertex\-Degree\char`\"{}.  
\item {\ttfamily obj.\-Set\-Edge\-Label\-Array\-Name (string name)} -\/ The array to use for edge labeling. Default is \char`\"{}label\char`\"{}.  
\item {\ttfamily string = obj.\-Get\-Edge\-Label\-Array\-Name ()} -\/ The array to use for edge labeling. Default is \char`\"{}label\char`\"{}.  
\item {\ttfamily obj.\-Set\-Area\-Hover\-Array\-Name (string name)} -\/ The name of the array whose value appears when the mouse hovers over a rectangle in the treemap. This must be a string array.  
\item {\ttfamily string = obj.\-Get\-Area\-Hover\-Array\-Name ()} -\/ The name of the array whose value appears when the mouse hovers over a rectangle in the treemap. This must be a string array.  
\item {\ttfamily obj.\-Set\-Area\-Label\-Visibility (bool vis)} -\/ Whether to show area labels. Default is off.  
\item {\ttfamily bool = obj.\-Get\-Area\-Label\-Visibility ()} -\/ Whether to show area labels. Default is off.  
\item {\ttfamily obj.\-Area\-Label\-Visibility\-On ()} -\/ Whether to show area labels. Default is off.  
\item {\ttfamily obj.\-Area\-Label\-Visibility\-Off ()} -\/ Whether to show area labels. Default is off.  
\item {\ttfamily obj.\-Set\-Edge\-Label\-Visibility (bool vis)} -\/ Whether to show edge labels. Default is off.  
\item {\ttfamily bool = obj.\-Get\-Edge\-Label\-Visibility ()} -\/ Whether to show edge labels. Default is off.  
\item {\ttfamily obj.\-Edge\-Label\-Visibility\-On ()} -\/ Whether to show edge labels. Default is off.  
\item {\ttfamily obj.\-Edge\-Label\-Visibility\-Off ()} -\/ Whether to show edge labels. Default is off.  
\item {\ttfamily obj.\-Set\-Area\-Color\-Array\-Name (string name)} -\/ The array to use for coloring vertices. Default is \char`\"{}color\char`\"{}.  
\item {\ttfamily string = obj.\-Get\-Area\-Color\-Array\-Name ()} -\/ The array to use for coloring vertices. Default is \char`\"{}color\char`\"{}.  
\item {\ttfamily obj.\-Set\-Color\-Areas (bool vis)} -\/ Whether to color vertices. Default is off.  
\item {\ttfamily bool = obj.\-Get\-Color\-Areas ()} -\/ Whether to color vertices. Default is off.  
\item {\ttfamily obj.\-Color\-Areas\-On ()} -\/ Whether to color vertices. Default is off.  
\item {\ttfamily obj.\-Color\-Areas\-Off ()} -\/ Whether to color vertices. Default is off.  
\item {\ttfamily obj.\-Set\-Edge\-Color\-Array\-Name (string name)} -\/ The array to use for coloring edges. Default is \char`\"{}color\char`\"{}.  
\item {\ttfamily string = obj.\-Get\-Edge\-Color\-Array\-Name ()} -\/ The array to use for coloring edges. Default is \char`\"{}color\char`\"{}.  
\item {\ttfamily obj.\-Set\-Edge\-Color\-To\-Spline\-Fraction ()} -\/ Set the color to be the spline fraction  
\item {\ttfamily obj.\-Set\-Shrink\-Percentage (double value)} -\/ Set the region shrink percentage between 0.\-0 and 1.\-0.  
\item {\ttfamily double = obj.\-Get\-Shrink\-Percentage ()} -\/ Set the region shrink percentage between 0.\-0 and 1.\-0.  
\item {\ttfamily obj.\-Set\-Color\-Edges (bool vis)} -\/ Whether to color edges. Default is off.  
\item {\ttfamily bool = obj.\-Get\-Color\-Edges ()} -\/ Whether to color edges. Default is off.  
\item {\ttfamily obj.\-Color\-Edges\-On ()} -\/ Whether to color edges. Default is off.  
\item {\ttfamily obj.\-Color\-Edges\-Off ()} -\/ Whether to color edges. Default is off.  
\item {\ttfamily obj.\-Set\-Bundling\-Strength (double strength)} -\/ Set the bundling strength.  
\item {\ttfamily double = obj.\-Get\-Bundling\-Strength ()} -\/ Set the bundling strength.  
\item {\ttfamily obj.\-Set\-Area\-Label\-Font\-Size (int size)} -\/ The size of the font used for area labeling  
\item {\ttfamily int = obj.\-Get\-Area\-Label\-Font\-Size ()} -\/ The size of the font used for area labeling  
\item {\ttfamily obj.\-Set\-Edge\-Label\-Font\-Size (int size)} -\/ The size of the font used for edge labeling  
\item {\ttfamily int = obj.\-Get\-Edge\-Label\-Font\-Size ()} -\/ The size of the font used for edge labeling  
\item {\ttfamily obj.\-Set\-Layout\-Strategy (vtk\-Area\-Layout\-Strategy strategy)} -\/ The layout strategy for producing spatial regions for the tree.  
\item {\ttfamily vtk\-Area\-Layout\-Strategy = obj.\-Get\-Layout\-Strategy ()} -\/ The layout strategy for producing spatial regions for the tree.  
\item {\ttfamily obj.\-Set\-Use\-Rectangular\-Coordinates (bool rect)} -\/ Whether the area represents radial or rectangular coordinates.  
\item {\ttfamily bool = obj.\-Get\-Use\-Rectangular\-Coordinates ()} -\/ Whether the area represents radial or rectangular coordinates.  
\item {\ttfamily obj.\-Use\-Rectangular\-Coordinates\-On ()} -\/ Whether the area represents radial or rectangular coordinates.  
\item {\ttfamily obj.\-Use\-Rectangular\-Coordinates\-Off ()} -\/ Whether the area represents radial or rectangular coordinates.  
\item {\ttfamily obj.\-Set\-Edge\-Scalar\-Bar\-Visibility (bool b)} -\/ Visibility of scalar bar actor for edges.  
\item {\ttfamily bool = obj.\-Get\-Edge\-Scalar\-Bar\-Visibility ()} -\/ Visibility of scalar bar actor for edges.  
\end{DoxyItemize}\hypertarget{vtkviews_vtktreemapview}{}\section{vtk\-Tree\-Map\-View}\label{vtkviews_vtktreemapview}
Section\-: \hyperlink{sec_vtkviews}{Visualization Toolkit View Classes} \hypertarget{vtkwidgets_vtkxyplotwidget_Usage}{}\subsection{Usage}\label{vtkwidgets_vtkxyplotwidget_Usage}
vtk\-Tree\-Map\-View shows a vtk\-Tree in a tree map, where each vertex in the tree is represented by a box. Child boxes are contained within the parent box, and may be colored and sized by various parameters.

To create an instance of class vtk\-Tree\-Map\-View, simply invoke its constructor as follows \begin{DoxyVerb}  obj = vtkTreeMapView
\end{DoxyVerb}
 \hypertarget{vtkwidgets_vtkxyplotwidget_Methods}{}\subsection{Methods}\label{vtkwidgets_vtkxyplotwidget_Methods}
The class vtk\-Tree\-Map\-View has several methods that can be used. They are listed below. Note that the documentation is translated automatically from the V\-T\-K sources, and may not be completely intelligible. When in doubt, consult the V\-T\-K website. In the methods listed below, {\ttfamily obj} is an instance of the vtk\-Tree\-Map\-View class. 
\begin{DoxyItemize}
\item {\ttfamily string = obj.\-Get\-Class\-Name ()}  
\item {\ttfamily int = obj.\-Is\-A (string name)}  
\item {\ttfamily vtk\-Tree\-Map\-View = obj.\-New\-Instance ()}  
\item {\ttfamily vtk\-Tree\-Map\-View = obj.\-Safe\-Down\-Cast (vtk\-Object o)}  
\item {\ttfamily obj.\-Set\-Layout\-Strategy (vtk\-Area\-Layout\-Strategy s)} -\/ Sets the treemap layout strategy  
\item {\ttfamily obj.\-Set\-Layout\-Strategy (string name)} -\/ Sets the treemap layout strategy  
\item {\ttfamily obj.\-Set\-Layout\-Strategy\-To\-Box ()} -\/ Sets the treemap layout strategy  
\item {\ttfamily obj.\-Set\-Layout\-Strategy\-To\-Slice\-And\-Dice ()} -\/ Sets the treemap layout strategy  
\item {\ttfamily obj.\-Set\-Layout\-Strategy\-To\-Squarify ()} -\/ Sets the treemap layout strategy  
\item {\ttfamily obj.\-Set\-Font\-Size\-Range (int max\-Size, int min\-Size, int delta)} -\/ The sizes of the fonts used for labeling.  
\item {\ttfamily obj.\-Get\-Font\-Size\-Range (int range\mbox{[}3\mbox{]})} -\/ The sizes of the fonts used for labeling.  
\end{DoxyItemize}\hypertarget{vtkviews_vtktreeringview}{}\section{vtk\-Tree\-Ring\-View}\label{vtkviews_vtktreeringview}
Section\-: \hyperlink{sec_vtkviews}{Visualization Toolkit View Classes} \hypertarget{vtkwidgets_vtkxyplotwidget_Usage}{}\subsection{Usage}\label{vtkwidgets_vtkxyplotwidget_Usage}
Accepts a graph and a hierarchy -\/ currently a tree -\/ and provides a hierarchy-\/aware display. Currently, this means displaying the hierarchy using a tree ring layout, then rendering the graph vertices as leaves of the tree with curved graph edges between leaves.

.S\-E\-E A\-L\-S\-O vtk\-Graph\-Layout\-View

.S\-E\-C\-T\-I\-O\-N Thanks Thanks to Jason Shepherd for implementing this class

To create an instance of class vtk\-Tree\-Ring\-View, simply invoke its constructor as follows \begin{DoxyVerb}  obj = vtkTreeRingView
\end{DoxyVerb}
 \hypertarget{vtkwidgets_vtkxyplotwidget_Methods}{}\subsection{Methods}\label{vtkwidgets_vtkxyplotwidget_Methods}
The class vtk\-Tree\-Ring\-View has several methods that can be used. They are listed below. Note that the documentation is translated automatically from the V\-T\-K sources, and may not be completely intelligible. When in doubt, consult the V\-T\-K website. In the methods listed below, {\ttfamily obj} is an instance of the vtk\-Tree\-Ring\-View class. 
\begin{DoxyItemize}
\item {\ttfamily string = obj.\-Get\-Class\-Name ()}  
\item {\ttfamily int = obj.\-Is\-A (string name)}  
\item {\ttfamily vtk\-Tree\-Ring\-View = obj.\-New\-Instance ()}  
\item {\ttfamily vtk\-Tree\-Ring\-View = obj.\-Safe\-Down\-Cast (vtk\-Object o)}  
\item {\ttfamily obj.\-Set\-Root\-Angles (double start, double end)} -\/ Set the root angles for laying out the hierarchy.  
\item {\ttfamily obj.\-Set\-Root\-At\-Center (bool value)} -\/ Sets whether the root is at the center or around the outside.  
\item {\ttfamily bool = obj.\-Get\-Root\-At\-Center ()} -\/ Sets whether the root is at the center or around the outside.  
\item {\ttfamily obj.\-Root\-At\-Center\-On ()} -\/ Sets whether the root is at the center or around the outside.  
\item {\ttfamily obj.\-Root\-At\-Center\-Off ()} -\/ Sets whether the root is at the center or around the outside.  
\item {\ttfamily obj.\-Set\-Layer\-Thickness (double thickness)} -\/ Set the thickness of each layer.  
\item {\ttfamily double = obj.\-Get\-Layer\-Thickness ()} -\/ Set the thickness of each layer.  
\item {\ttfamily obj.\-Set\-Interior\-Radius (double thickness)} -\/ Set the interior radius of the tree (i.\-e. the size of the \char`\"{}hole\char`\"{} in the center).  
\item {\ttfamily double = obj.\-Get\-Interior\-Radius ()} -\/ Set the interior radius of the tree (i.\-e. the size of the \char`\"{}hole\char`\"{} in the center).  
\item {\ttfamily obj.\-Set\-Interior\-Log\-Spacing\-Value (double thickness)} -\/ Set the log spacing factor for the invisible interior tree used for routing edges of the overlaid graph.  
\item {\ttfamily double = obj.\-Get\-Interior\-Log\-Spacing\-Value ()} -\/ Set the log spacing factor for the invisible interior tree used for routing edges of the overlaid graph.  
\end{DoxyItemize}\hypertarget{vtkviews_vtkview}{}\section{vtk\-View}\label{vtkviews_vtkview}
Section\-: \hyperlink{sec_vtkviews}{Visualization Toolkit View Classes} \hypertarget{vtkwidgets_vtkxyplotwidget_Usage}{}\subsection{Usage}\label{vtkwidgets_vtkxyplotwidget_Usage}
vtk\-View is the superclass for views. A view is generally an area of an application's canvas devoted to displaying one or more V\-T\-K data objects. Associated representations (subclasses of vtk\-Data\-Representation) are responsible for converting the data into a displayable format. These representations are then added to the view.

For views which display only one data object at a time you may set a data object or pipeline connection directly on the view itself (e.\-g. vtk\-Graph\-Layout\-View, vtk\-Landscape\-View, vtk\-Tree\-Map\-View). The view will internally create a vtk\-Data\-Representation for the data.

A view has the concept of linked selection. If the same data is displayed in multiple views, their selections may be linked by setting the same vtk\-Annotation\-Link on their representations (see vtk\-Data\-Representation).

To create an instance of class vtk\-View, simply invoke its constructor as follows \begin{DoxyVerb}  obj = vtkView
\end{DoxyVerb}
 \hypertarget{vtkwidgets_vtkxyplotwidget_Methods}{}\subsection{Methods}\label{vtkwidgets_vtkxyplotwidget_Methods}
The class vtk\-View has several methods that can be used. They are listed below. Note that the documentation is translated automatically from the V\-T\-K sources, and may not be completely intelligible. When in doubt, consult the V\-T\-K website. In the methods listed below, {\ttfamily obj} is an instance of the vtk\-View class. 
\begin{DoxyItemize}
\item {\ttfamily string = obj.\-Get\-Class\-Name ()}  
\item {\ttfamily int = obj.\-Is\-A (string name)}  
\item {\ttfamily vtk\-View = obj.\-New\-Instance ()}  
\item {\ttfamily vtk\-View = obj.\-Safe\-Down\-Cast (vtk\-Object o)}  
\item {\ttfamily obj.\-Add\-Representation (vtk\-Data\-Representation rep)} -\/ Adds the representation to the view.  
\item {\ttfamily obj.\-Set\-Representation (vtk\-Data\-Representation rep)} -\/ Set the representation to the view.  
\item {\ttfamily vtk\-Data\-Representation = obj.\-Add\-Representation\-From\-Input\-Connection (vtk\-Algorithm\-Output conn)} -\/ Convenience method which creates a simple representation with the connection and adds it to the view. Returns the representation internally created. N\-O\-T\-E\-: The returned representation pointer is not reference-\/counted, so you M\-U\-S\-T call Register() on the representation if you want to keep a reference to it.  
\item {\ttfamily vtk\-Data\-Representation = obj.\-Set\-Representation\-From\-Input\-Connection (vtk\-Algorithm\-Output conn)} -\/ Convenience method which sets the representation with the connection and adds it to the view. Returns the representation internally created. N\-O\-T\-E\-: The returned representation pointer is not reference-\/counted, so you M\-U\-S\-T call Register() on the representation if you want to keep a reference to it.  
\item {\ttfamily vtk\-Data\-Representation = obj.\-Add\-Representation\-From\-Input (vtk\-Data\-Object input)} -\/ Convenience method which creates a simple representation with the specified input and adds it to the view. N\-O\-T\-E\-: The returned representation pointer is not reference-\/counted, so you M\-U\-S\-T call Register() on the representation if you want to keep a reference to it.  
\item {\ttfamily vtk\-Data\-Representation = obj.\-Set\-Representation\-From\-Input (vtk\-Data\-Object input)} -\/ Convenience method which sets the representation to the specified input and adds it to the view. N\-O\-T\-E\-: The returned representation pointer is not reference-\/counted, so you M\-U\-S\-T call Register() on the representation if you want to keep a reference to it.  
\item {\ttfamily obj.\-Remove\-Representation (vtk\-Data\-Representation rep)} -\/ Removes the representation from the view.  
\item {\ttfamily obj.\-Remove\-Representation (vtk\-Algorithm\-Output rep)} -\/ Removes any representation with this connection from the view.  
\item {\ttfamily obj.\-Remove\-All\-Representations ()} -\/ Removes all representations from the view.  
\item {\ttfamily int = obj.\-Get\-Number\-Of\-Representations ()} -\/ Returns the number of representations from first port(0) in this view.  
\item {\ttfamily vtk\-Data\-Representation = obj.\-Get\-Representation (int index)} -\/ The representation at a specified index.  
\item {\ttfamily bool = obj.\-Is\-Representation\-Present (vtk\-Data\-Representation rep)} -\/ Check to see if a representation is present in the view.  
\item {\ttfamily obj.\-Update ()} -\/ Update the view.  
\item {\ttfamily obj.\-Apply\-View\-Theme (vtk\-View\-Theme )} -\/ Meant for use by subclasses and vtk\-Representation subclasses. Call this method to register vtk\-Objects (generally vtk\-Algorithm subclasses) which fire vtk\-Command\-::\-Progress\-Event with the view. The view listens to vtk\-Command\-::\-Progress\-Event and fires View\-Progress\-Event with View\-Progress\-Event\-Call\-Data containing the message and the progress amount. If message is not provided, then the class name for the algorithm is used.  
\item {\ttfamily obj.\-Register\-Progress (vtk\-Object algorithm, string message\-N\-U\-L\-L)} -\/ Meant for use by subclasses and vtk\-Representation subclasses. Call this method to register vtk\-Objects (generally vtk\-Algorithm subclasses) which fire vtk\-Command\-::\-Progress\-Event with the view. The view listens to vtk\-Command\-::\-Progress\-Event and fires View\-Progress\-Event with View\-Progress\-Event\-Call\-Data containing the message and the progress amount. If message is not provided, then the class name for the algorithm is used.  
\item {\ttfamily obj.\-Un\-Register\-Progress (vtk\-Object algorithm)} -\/ Unregister objects previously registered with Register\-Progress.  
\end{DoxyItemize}\hypertarget{vtkviews_vtkviewupdater}{}\section{vtk\-View\-Updater}\label{vtkviews_vtkviewupdater}
Section\-: \hyperlink{sec_vtkviews}{Visualization Toolkit View Classes} \hypertarget{vtkwidgets_vtkxyplotwidget_Usage}{}\subsection{Usage}\label{vtkwidgets_vtkxyplotwidget_Usage}
vtk\-View\-Updater registers with annotation change events for a set of annotation links, and updates all views when an annotation link fires an annotation changed event. This is often needed when multiple views share a selection with vtk\-Annotation\-Link.

To create an instance of class vtk\-View\-Updater, simply invoke its constructor as follows \begin{DoxyVerb}  obj = vtkViewUpdater
\end{DoxyVerb}
 \hypertarget{vtkwidgets_vtkxyplotwidget_Methods}{}\subsection{Methods}\label{vtkwidgets_vtkxyplotwidget_Methods}
The class vtk\-View\-Updater has several methods that can be used. They are listed below. Note that the documentation is translated automatically from the V\-T\-K sources, and may not be completely intelligible. When in doubt, consult the V\-T\-K website. In the methods listed below, {\ttfamily obj} is an instance of the vtk\-View\-Updater class. 
\begin{DoxyItemize}
\item {\ttfamily string = obj.\-Get\-Class\-Name ()}  
\item {\ttfamily int = obj.\-Is\-A (string name)}  
\item {\ttfamily vtk\-View\-Updater = obj.\-New\-Instance ()}  
\item {\ttfamily vtk\-View\-Updater = obj.\-Safe\-Down\-Cast (vtk\-Object o)}  
\item {\ttfamily obj.\-Add\-View (vtk\-View view)}  
\item {\ttfamily obj.\-Add\-Annotation\-Link (vtk\-Annotation\-Link link)}  
\end{DoxyItemize}