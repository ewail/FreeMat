
\begin{DoxyItemize}
\item \hyperlink{vtkgeovis_vtkcompassrepresentation}{vtk\-Compass\-Representation}  
\item \hyperlink{vtkgeovis_vtkcompasswidget}{vtk\-Compass\-Widget}  
\item \hyperlink{vtkgeovis_vtkgeoadaptivearcs}{vtk\-Geo\-Adaptive\-Arcs}  
\item \hyperlink{vtkgeovis_vtkgeoalignedimagerepresentation}{vtk\-Geo\-Aligned\-Image\-Representation}  
\item \hyperlink{vtkgeovis_vtkgeoalignedimagesource}{vtk\-Geo\-Aligned\-Image\-Source}  
\item \hyperlink{vtkgeovis_vtkgeoarcs}{vtk\-Geo\-Arcs}  
\item \hyperlink{vtkgeovis_vtkgeoassigncoordinates}{vtk\-Geo\-Assign\-Coordinates}  
\item \hyperlink{vtkgeovis_vtkgeocamera}{vtk\-Geo\-Camera}  
\item \hyperlink{vtkgeovis_vtkgeofileimagesource}{vtk\-Geo\-File\-Image\-Source}  
\item \hyperlink{vtkgeovis_vtkgeofileterrainsource}{vtk\-Geo\-File\-Terrain\-Source}  
\item \hyperlink{vtkgeovis_vtkgeoglobesource}{vtk\-Geo\-Globe\-Source}  
\item \hyperlink{vtkgeovis_vtkgeograticule}{vtk\-Geo\-Graticule}  
\item \hyperlink{vtkgeovis_vtkgeoimagenode}{vtk\-Geo\-Image\-Node}  
\item \hyperlink{vtkgeovis_vtkgeointeractorstyle}{vtk\-Geo\-Interactor\-Style}  
\item \hyperlink{vtkgeovis_vtkgeoprojection}{vtk\-Geo\-Projection}  
\item \hyperlink{vtkgeovis_vtkgeoprojectionsource}{vtk\-Geo\-Projection\-Source}  
\item \hyperlink{vtkgeovis_vtkgeorandomgraphsource}{vtk\-Geo\-Random\-Graph\-Source}  
\item \hyperlink{vtkgeovis_vtkgeosamplearcs}{vtk\-Geo\-Sample\-Arcs}  
\item \hyperlink{vtkgeovis_vtkgeosource}{vtk\-Geo\-Source}  
\item \hyperlink{vtkgeovis_vtkgeospheretransform}{vtk\-Geo\-Sphere\-Transform}  
\item \hyperlink{vtkgeovis_vtkgeoterrain}{vtk\-Geo\-Terrain}  
\item \hyperlink{vtkgeovis_vtkgeoterrain2d}{vtk\-Geo\-Terrain2\-D}  
\item \hyperlink{vtkgeovis_vtkgeoterrainnode}{vtk\-Geo\-Terrain\-Node}  
\item \hyperlink{vtkgeovis_vtkgeotransform}{vtk\-Geo\-Transform}  
\item \hyperlink{vtkgeovis_vtkgeotreenode}{vtk\-Geo\-Tree\-Node}  
\item \hyperlink{vtkgeovis_vtkgeotreenodecache}{vtk\-Geo\-Tree\-Node\-Cache}  
\item \hyperlink{vtkgeovis_vtkgeoview}{vtk\-Geo\-View}  
\item \hyperlink{vtkgeovis_vtkgeoview2d}{vtk\-Geo\-View2\-D}  
\item \hyperlink{vtkgeovis_vtkglobesource}{vtk\-Globe\-Source}  
\end{DoxyItemize}\hypertarget{vtkgeovis_vtkcompassrepresentation}{}\section{vtk\-Compass\-Representation}\label{vtkgeovis_vtkcompassrepresentation}
Section\-: \hyperlink{sec_vtkgeovis}{Visualization Toolkit Geo Vis Classes} \hypertarget{vtkwidgets_vtkxyplotwidget_Usage}{}\subsection{Usage}\label{vtkwidgets_vtkxyplotwidget_Usage}
This class is used to represent and render a compass.

To create an instance of class vtk\-Compass\-Representation, simply invoke its constructor as follows \begin{DoxyVerb}  obj = vtkCompassRepresentation
\end{DoxyVerb}
 \hypertarget{vtkwidgets_vtkxyplotwidget_Methods}{}\subsection{Methods}\label{vtkwidgets_vtkxyplotwidget_Methods}
The class vtk\-Compass\-Representation has several methods that can be used. They are listed below. Note that the documentation is translated automatically from the V\-T\-K sources, and may not be completely intelligible. When in doubt, consult the V\-T\-K website. In the methods listed below, {\ttfamily obj} is an instance of the vtk\-Compass\-Representation class. 
\begin{DoxyItemize}
\item {\ttfamily string = obj.\-Get\-Class\-Name ()} -\/ Standard methods for the class.  
\item {\ttfamily int = obj.\-Is\-A (string name)} -\/ Standard methods for the class.  
\item {\ttfamily vtk\-Compass\-Representation = obj.\-New\-Instance ()} -\/ Standard methods for the class.  
\item {\ttfamily vtk\-Compass\-Representation = obj.\-Safe\-Down\-Cast (vtk\-Object o)} -\/ Standard methods for the class.  
\item {\ttfamily vtk\-Coordinate = obj.\-Get\-Point1\-Coordinate ()} -\/ Position the first end point of the slider. Note that this point is an instance of vtk\-Coordinate, meaning that Point 1 can be specified in a variety of coordinate systems, and can even be relative to another point. To set the point, you'll want to get the Point1\-Coordinate and then invoke the necessary methods to put it into the correct coordinate system and set the correct initial value.  
\item {\ttfamily vtk\-Coordinate = obj.\-Get\-Point2\-Coordinate ()} -\/ Position the second end point of the slider. Note that this point is an instance of vtk\-Coordinate, meaning that Point 1 can be specified in a variety of coordinate systems, and can even be relative to another point. To set the point, you'll want to get the Point2\-Coordinate and then invoke the necessary methods to put it into the correct coordinate system and set the correct initial value.  
\item {\ttfamily vtk\-Property2\-D = obj.\-Get\-Ring\-Property ()} -\/ Get the slider properties. The properties of the slider when selected and unselected can be manipulated.  
\item {\ttfamily vtk\-Property2\-D = obj.\-Get\-Selected\-Property ()} -\/ Get the selection property. This property is used to modify the appearance of selected objects (e.\-g., the slider).  
\item {\ttfamily vtk\-Text\-Property = obj.\-Get\-Label\-Property ()} -\/ Set/\-Get the properties for the label and title text.  
\item {\ttfamily obj.\-Place\-Widget (double bounds\mbox{[}6\mbox{]})} -\/ Methods to interface with the vtk\-Slider\-Widget. The Place\-Widget() method assumes that the parameter bounds\mbox{[}6\mbox{]} specifies the location in display space where the widget should be placed.  
\item {\ttfamily obj.\-Build\-Representation ()} -\/ Methods to interface with the vtk\-Slider\-Widget. The Place\-Widget() method assumes that the parameter bounds\mbox{[}6\mbox{]} specifies the location in display space where the widget should be placed.  
\item {\ttfamily obj.\-Start\-Widget\-Interaction (double event\-Pos\mbox{[}2\mbox{]})} -\/ Methods to interface with the vtk\-Slider\-Widget. The Place\-Widget() method assumes that the parameter bounds\mbox{[}6\mbox{]} specifies the location in display space where the widget should be placed.  
\item {\ttfamily obj.\-Widget\-Interaction (double event\-Pos\mbox{[}2\mbox{]})} -\/ Methods to interface with the vtk\-Slider\-Widget. The Place\-Widget() method assumes that the parameter bounds\mbox{[}6\mbox{]} specifies the location in display space where the widget should be placed.  
\item {\ttfamily obj.\-Tilt\-Widget\-Interaction (double event\-Pos\mbox{[}2\mbox{]})} -\/ Methods to interface with the vtk\-Slider\-Widget. The Place\-Widget() method assumes that the parameter bounds\mbox{[}6\mbox{]} specifies the location in display space where the widget should be placed.  
\item {\ttfamily obj.\-Distance\-Widget\-Interaction (double event\-Pos\mbox{[}2\mbox{]})} -\/ Methods to interface with the vtk\-Slider\-Widget. The Place\-Widget() method assumes that the parameter bounds\mbox{[}6\mbox{]} specifies the location in display space where the widget should be placed.  
\item {\ttfamily int = obj.\-Compute\-Interaction\-State (int X, int Y, int modify)} -\/ Methods to interface with the vtk\-Slider\-Widget. The Place\-Widget() method assumes that the parameter bounds\mbox{[}6\mbox{]} specifies the location in display space where the widget should be placed.  
\item {\ttfamily obj.\-Highlight (int )} -\/ Methods to interface with the vtk\-Slider\-Widget. The Place\-Widget() method assumes that the parameter bounds\mbox{[}6\mbox{]} specifies the location in display space where the widget should be placed.  
\item {\ttfamily obj.\-Get\-Actors (vtk\-Prop\-Collection )}  
\item {\ttfamily obj.\-Release\-Graphics\-Resources (vtk\-Window )}  
\item {\ttfamily int = obj.\-Render\-Overlay (vtk\-Viewport )}  
\item {\ttfamily int = obj.\-Render\-Opaque\-Geometry (vtk\-Viewport )}  
\item {\ttfamily obj.\-Set\-Heading (double value)}  
\item {\ttfamily double = obj.\-Get\-Heading ()}  
\item {\ttfamily obj.\-Set\-Tilt (double value)}  
\item {\ttfamily double = obj.\-Get\-Tilt ()}  
\item {\ttfamily obj.\-Update\-Tilt (double time)}  
\item {\ttfamily obj.\-End\-Tilt ()}  
\item {\ttfamily obj.\-Set\-Distance (double value)}  
\item {\ttfamily double = obj.\-Get\-Distance ()}  
\item {\ttfamily obj.\-Update\-Distance (double time)}  
\item {\ttfamily obj.\-End\-Distance ()}  
\item {\ttfamily obj.\-Set\-Renderer (vtk\-Renderer ren)}  
\end{DoxyItemize}\hypertarget{vtkgeovis_vtkcompasswidget}{}\section{vtk\-Compass\-Widget}\label{vtkgeovis_vtkcompasswidget}
Section\-: \hyperlink{sec_vtkgeovis}{Visualization Toolkit Geo Vis Classes} \hypertarget{vtkwidgets_vtkxyplotwidget_Usage}{}\subsection{Usage}\label{vtkwidgets_vtkxyplotwidget_Usage}
The vtk\-Compass\-Widget is used to adjust a scalar value in an application. Note that the actual appearance of the widget depends on the specific representation for the widget.

To use this widget, set the widget representation. (the details may vary depending on the particulars of the representation).

.S\-E\-C\-T\-I\-O\-N Event Bindings By default, the widget responds to the following V\-T\-K events (i.\-e., it watches the vtk\-Render\-Window\-Interactor for these events)\-: 
\begin{DoxyPre}
 If the slider bead is selected:
   LeftButtonPressEvent - select slider 
   LeftButtonReleaseEvent - release slider 
   MouseMoveEvent - move slider
 \end{DoxyPre}


Note that the event bindings described above can be changed using this class's vtk\-Widget\-Event\-Translator. This class translates V\-T\-K events into the vtk\-Compass\-Widget's widget events\-: 
\begin{DoxyPre}
   vtkWidgetEvent::Select -- some part of the widget has been selected
   vtkWidgetEvent::EndSelect -- the selection process has completed
   vtkWidgetEvent::Move -- a request for slider motion has been invoked
 \end{DoxyPre}


In turn, when these widget events are processed, the vtk\-Compass\-Widget invokes the following V\-T\-K events on itself (which observers can listen for)\-: 
\begin{DoxyPre}
   vtkCommand::StartInteractionEvent (on vtkWidgetEvent::Select)
   vtkCommand::EndInteractionEvent (on vtkWidgetEvent::EndSelect)
   vtkCommand::InteractionEvent (on vtkWidgetEvent::Move)
 \end{DoxyPre}


To create an instance of class vtk\-Compass\-Widget, simply invoke its constructor as follows \begin{DoxyVerb}  obj = vtkCompassWidget
\end{DoxyVerb}
 \hypertarget{vtkwidgets_vtkxyplotwidget_Methods}{}\subsection{Methods}\label{vtkwidgets_vtkxyplotwidget_Methods}
The class vtk\-Compass\-Widget has several methods that can be used. They are listed below. Note that the documentation is translated automatically from the V\-T\-K sources, and may not be completely intelligible. When in doubt, consult the V\-T\-K website. In the methods listed below, {\ttfamily obj} is an instance of the vtk\-Compass\-Widget class. 
\begin{DoxyItemize}
\item {\ttfamily string = obj.\-Get\-Class\-Name ()} -\/ Standard macros.  
\item {\ttfamily int = obj.\-Is\-A (string name)} -\/ Standard macros.  
\item {\ttfamily vtk\-Compass\-Widget = obj.\-New\-Instance ()} -\/ Standard macros.  
\item {\ttfamily vtk\-Compass\-Widget = obj.\-Safe\-Down\-Cast (vtk\-Object o)} -\/ Standard macros.  
\item {\ttfamily obj.\-Set\-Representation (vtk\-Compass\-Representation r)} -\/ Create the default widget representation if one is not set.  
\item {\ttfamily obj.\-Create\-Default\-Representation ()} -\/ Create the default widget representation if one is not set.  
\item {\ttfamily double = obj.\-Get\-Heading ()} -\/ Get the value for this widget.  
\item {\ttfamily obj.\-Set\-Heading (double v)} -\/ Get the value for this widget.  
\item {\ttfamily double = obj.\-Get\-Tilt ()} -\/ Get the value for this widget.  
\item {\ttfamily obj.\-Set\-Tilt (double t)} -\/ Get the value for this widget.  
\item {\ttfamily double = obj.\-Get\-Distance ()} -\/ Get the value for this widget.  
\item {\ttfamily obj.\-Set\-Distance (double t)} -\/ Get the value for this widget.  
\end{DoxyItemize}\hypertarget{vtkgeovis_vtkgeoadaptivearcs}{}\section{vtk\-Geo\-Adaptive\-Arcs}\label{vtkgeovis_vtkgeoadaptivearcs}
Section\-: \hyperlink{sec_vtkgeovis}{Visualization Toolkit Geo Vis Classes} \hypertarget{vtkwidgets_vtkxyplotwidget_Usage}{}\subsection{Usage}\label{vtkwidgets_vtkxyplotwidget_Usage}
To create an instance of class vtk\-Geo\-Adaptive\-Arcs, simply invoke its constructor as follows \begin{DoxyVerb}  obj = vtkGeoAdaptiveArcs
\end{DoxyVerb}
 \hypertarget{vtkwidgets_vtkxyplotwidget_Methods}{}\subsection{Methods}\label{vtkwidgets_vtkxyplotwidget_Methods}
The class vtk\-Geo\-Adaptive\-Arcs has several methods that can be used. They are listed below. Note that the documentation is translated automatically from the V\-T\-K sources, and may not be completely intelligible. When in doubt, consult the V\-T\-K website. In the methods listed below, {\ttfamily obj} is an instance of the vtk\-Geo\-Adaptive\-Arcs class. 
\begin{DoxyItemize}
\item {\ttfamily string = obj.\-Get\-Class\-Name ()}  
\item {\ttfamily int = obj.\-Is\-A (string name)}  
\item {\ttfamily vtk\-Geo\-Adaptive\-Arcs = obj.\-New\-Instance ()}  
\item {\ttfamily vtk\-Geo\-Adaptive\-Arcs = obj.\-Safe\-Down\-Cast (vtk\-Object o)}  
\item {\ttfamily obj.\-Set\-Globe\-Radius (double )} -\/ The base radius used to determine the earth's surface. Default is the earth's radius in meters. T\-O\-D\-O\-: Change this to take in a vtk\-Geo\-Terrain to get altitude.  
\item {\ttfamily double = obj.\-Get\-Globe\-Radius ()} -\/ The base radius used to determine the earth's surface. Default is the earth's radius in meters. T\-O\-D\-O\-: Change this to take in a vtk\-Geo\-Terrain to get altitude.  
\item {\ttfamily obj.\-Set\-Maximum\-Pixel\-Separation (double )} -\/ The maximum number of pixels between points on the arcs. If two adjacent points are farther than the threshold, the line segment will be subdivided such that each point is separated by at most the threshold.  
\item {\ttfamily double = obj.\-Get\-Maximum\-Pixel\-Separation ()} -\/ The maximum number of pixels between points on the arcs. If two adjacent points are farther than the threshold, the line segment will be subdivided such that each point is separated by at most the threshold.  
\item {\ttfamily obj.\-Set\-Minimum\-Pixel\-Separation (double )} -\/ The minimum number of pixels between points on the arcs. Points closer than the threshold will be skipped until a point farther than the minimum threshold is reached.  
\item {\ttfamily double = obj.\-Get\-Minimum\-Pixel\-Separation ()} -\/ The minimum number of pixels between points on the arcs. Points closer than the threshold will be skipped until a point farther than the minimum threshold is reached.  
\item {\ttfamily obj.\-Set\-Renderer (vtk\-Renderer ren)} -\/ The renderer used to estimate the number of pixels between points.  
\item {\ttfamily vtk\-Renderer = obj.\-Get\-Renderer ()} -\/ The renderer used to estimate the number of pixels between points.  
\item {\ttfamily long = obj.\-Get\-M\-Time ()} -\/ Return the modified time of this object.  
\end{DoxyItemize}\hypertarget{vtkgeovis_vtkgeoalignedimagerepresentation}{}\section{vtk\-Geo\-Aligned\-Image\-Representation}\label{vtkgeovis_vtkgeoalignedimagerepresentation}
Section\-: \hyperlink{sec_vtkgeovis}{Visualization Toolkit Geo Vis Classes} \hypertarget{vtkwidgets_vtkxyplotwidget_Usage}{}\subsection{Usage}\label{vtkwidgets_vtkxyplotwidget_Usage}
vtk\-Geo\-Aligned\-Image\-Representation represents a high resolution image over the globle. It has an associated vtk\-Geo\-Source which is responsible for fetching new data. This class keeps the fetched data in a quad-\/tree structure organized by latitude and longitude.

To create an instance of class vtk\-Geo\-Aligned\-Image\-Representation, simply invoke its constructor as follows \begin{DoxyVerb}  obj = vtkGeoAlignedImageRepresentation
\end{DoxyVerb}
 \hypertarget{vtkwidgets_vtkxyplotwidget_Methods}{}\subsection{Methods}\label{vtkwidgets_vtkxyplotwidget_Methods}
The class vtk\-Geo\-Aligned\-Image\-Representation has several methods that can be used. They are listed below. Note that the documentation is translated automatically from the V\-T\-K sources, and may not be completely intelligible. When in doubt, consult the V\-T\-K website. In the methods listed below, {\ttfamily obj} is an instance of the vtk\-Geo\-Aligned\-Image\-Representation class. 
\begin{DoxyItemize}
\item {\ttfamily string = obj.\-Get\-Class\-Name ()}  
\item {\ttfamily int = obj.\-Is\-A (string name)}  
\item {\ttfamily vtk\-Geo\-Aligned\-Image\-Representation = obj.\-New\-Instance ()}  
\item {\ttfamily vtk\-Geo\-Aligned\-Image\-Representation = obj.\-Safe\-Down\-Cast (vtk\-Object o)}  
\item {\ttfamily vtk\-Geo\-Image\-Node = obj.\-Get\-Best\-Image\-For\-Bounds (double bounds\mbox{[}4\mbox{]})} -\/ Retrieve the most refined image patch that covers the specified latitude and longitude bounds (lat-\/min, lat-\/max, long-\/min, long-\/max).  
\item {\ttfamily vtk\-Geo\-Source = obj.\-Get\-Source ()} -\/ The source for this representation. This must be set first before calling Get\-Best\-Image\-For\-Bounds.  
\item {\ttfamily obj.\-Set\-Source (vtk\-Geo\-Source source)} -\/ The source for this representation. This must be set first before calling Get\-Best\-Image\-For\-Bounds.  
\item {\ttfamily obj.\-Save\-Database (string path)} -\/ Serialize the database to the specified directory. Each image is stored as a .vti file. The Origin and Spacing of the saved image contain (lat-\/min, long-\/min) and (lat-\/max, long-\/max), respectively. Files are named based on their level and id within that level.  
\end{DoxyItemize}\hypertarget{vtkgeovis_vtkgeoalignedimagesource}{}\section{vtk\-Geo\-Aligned\-Image\-Source}\label{vtkgeovis_vtkgeoalignedimagesource}
Section\-: \hyperlink{sec_vtkgeovis}{Visualization Toolkit Geo Vis Classes} \hypertarget{vtkwidgets_vtkxyplotwidget_Usage}{}\subsection{Usage}\label{vtkwidgets_vtkxyplotwidget_Usage}
vtk\-Geo\-Aligned\-Image\-Source uses a high resolution image to generate tiles at multiple resolutions in a hierarchy. It should be used as a source in vtk\-Geo\-Aligned\-Image\-Representation.

To create an instance of class vtk\-Geo\-Aligned\-Image\-Source, simply invoke its constructor as follows \begin{DoxyVerb}  obj = vtkGeoAlignedImageSource
\end{DoxyVerb}
 \hypertarget{vtkwidgets_vtkxyplotwidget_Methods}{}\subsection{Methods}\label{vtkwidgets_vtkxyplotwidget_Methods}
The class vtk\-Geo\-Aligned\-Image\-Source has several methods that can be used. They are listed below. Note that the documentation is translated automatically from the V\-T\-K sources, and may not be completely intelligible. When in doubt, consult the V\-T\-K website. In the methods listed below, {\ttfamily obj} is an instance of the vtk\-Geo\-Aligned\-Image\-Source class. 
\begin{DoxyItemize}
\item {\ttfamily string = obj.\-Get\-Class\-Name ()}  
\item {\ttfamily int = obj.\-Is\-A (string name)}  
\item {\ttfamily vtk\-Geo\-Aligned\-Image\-Source = obj.\-New\-Instance ()}  
\item {\ttfamily vtk\-Geo\-Aligned\-Image\-Source = obj.\-Safe\-Down\-Cast (vtk\-Object o)}  
\item {\ttfamily bool = obj.\-Fetch\-Root (vtk\-Geo\-Tree\-Node node)} -\/ Fetch the root image.  
\item {\ttfamily bool = obj.\-Fetch\-Child (vtk\-Geo\-Tree\-Node parent, int index, vtk\-Geo\-Tree\-Node child)} -\/ Fetch a child image.  
\item {\ttfamily vtk\-Image\-Data = obj.\-Get\-Image ()} -\/ The high-\/resolution image to be used to cover the globe.  
\item {\ttfamily obj.\-Set\-Image (vtk\-Image\-Data image)} -\/ The high-\/resolution image to be used to cover the globe.  
\item {\ttfamily obj.\-Set\-Latitude\-Range (double , double )} -\/ The range of the input hi-\/res image.  
\item {\ttfamily obj.\-Set\-Latitude\-Range (double a\mbox{[}2\mbox{]})} -\/ The range of the input hi-\/res image.  
\item {\ttfamily double = obj. Get\-Latitude\-Range ()} -\/ The range of the input hi-\/res image.  
\item {\ttfamily obj.\-Set\-Longitude\-Range (double , double )} -\/ The range of the input hi-\/res image.  
\item {\ttfamily obj.\-Set\-Longitude\-Range (double a\mbox{[}2\mbox{]})} -\/ The range of the input hi-\/res image.  
\item {\ttfamily double = obj. Get\-Longitude\-Range ()} -\/ The range of the input hi-\/res image.  
\item {\ttfamily obj.\-Set\-Overlap (double )} -\/ The overlap of adjacent tiles.  
\item {\ttfamily double = obj.\-Get\-Overlap\-Min\-Value ()} -\/ The overlap of adjacent tiles.  
\item {\ttfamily double = obj.\-Get\-Overlap\-Max\-Value ()} -\/ The overlap of adjacent tiles.  
\item {\ttfamily double = obj.\-Get\-Overlap ()} -\/ The overlap of adjacent tiles.  
\item {\ttfamily obj.\-Set\-Power\-Of\-Two\-Size (bool )} -\/ Whether to force image sizes to a power of two.  
\item {\ttfamily bool = obj.\-Get\-Power\-Of\-Two\-Size ()} -\/ Whether to force image sizes to a power of two.  
\item {\ttfamily obj.\-Power\-Of\-Two\-Size\-On ()} -\/ Whether to force image sizes to a power of two.  
\item {\ttfamily obj.\-Power\-Of\-Two\-Size\-Off ()} -\/ Whether to force image sizes to a power of two.  
\end{DoxyItemize}\hypertarget{vtkgeovis_vtkgeoarcs}{}\section{vtk\-Geo\-Arcs}\label{vtkgeovis_vtkgeoarcs}
Section\-: \hyperlink{sec_vtkgeovis}{Visualization Toolkit Geo Vis Classes} \hypertarget{vtkwidgets_vtkxyplotwidget_Usage}{}\subsection{Usage}\label{vtkwidgets_vtkxyplotwidget_Usage}
vtk\-Geo\-Arcs produces arcs for each line in the input polydata. This is useful for viewing lines on a sphere (e.\-g. the earth). The arcs may \char`\"{}jump\char`\"{} above the sphere's surface using Explode\-Factor.

To create an instance of class vtk\-Geo\-Arcs, simply invoke its constructor as follows \begin{DoxyVerb}  obj = vtkGeoArcs
\end{DoxyVerb}
 \hypertarget{vtkwidgets_vtkxyplotwidget_Methods}{}\subsection{Methods}\label{vtkwidgets_vtkxyplotwidget_Methods}
The class vtk\-Geo\-Arcs has several methods that can be used. They are listed below. Note that the documentation is translated automatically from the V\-T\-K sources, and may not be completely intelligible. When in doubt, consult the V\-T\-K website. In the methods listed below, {\ttfamily obj} is an instance of the vtk\-Geo\-Arcs class. 
\begin{DoxyItemize}
\item {\ttfamily string = obj.\-Get\-Class\-Name ()}  
\item {\ttfamily int = obj.\-Is\-A (string name)}  
\item {\ttfamily vtk\-Geo\-Arcs = obj.\-New\-Instance ()}  
\item {\ttfamily vtk\-Geo\-Arcs = obj.\-Safe\-Down\-Cast (vtk\-Object o)}  
\item {\ttfamily obj.\-Set\-Globe\-Radius (double )} -\/ The base radius used to determine the earth's surface. Default is the earth's radius in meters.  
\item {\ttfamily double = obj.\-Get\-Globe\-Radius ()} -\/ The base radius used to determine the earth's surface. Default is the earth's radius in meters.  
\item {\ttfamily obj.\-Set\-Explode\-Factor (double )} -\/ Factor on which to \char`\"{}explode\char`\"{} the arcs away from the surface. A value of 0.\-0 keeps the values on the surface. Values larger than 0.\-0 push the arcs away from the surface by a distance proportional to the distance between the points. The default is 0.\-2.  
\item {\ttfamily double = obj.\-Get\-Explode\-Factor ()} -\/ Factor on which to \char`\"{}explode\char`\"{} the arcs away from the surface. A value of 0.\-0 keeps the values on the surface. Values larger than 0.\-0 push the arcs away from the surface by a distance proportional to the distance between the points. The default is 0.\-2.  
\item {\ttfamily obj.\-Set\-Number\-Of\-Subdivisions (int )} -\/ The number of subdivisions in the arc. The default is 20.  
\item {\ttfamily int = obj.\-Get\-Number\-Of\-Subdivisions ()} -\/ The number of subdivisions in the arc. The default is 20.  
\end{DoxyItemize}\hypertarget{vtkgeovis_vtkgeoassigncoordinates}{}\section{vtk\-Geo\-Assign\-Coordinates}\label{vtkgeovis_vtkgeoassigncoordinates}
Section\-: \hyperlink{sec_vtkgeovis}{Visualization Toolkit Geo Vis Classes} \hypertarget{vtkwidgets_vtkxyplotwidget_Usage}{}\subsection{Usage}\label{vtkwidgets_vtkxyplotwidget_Usage}
Givem latitude and longitude arrays, take the values in those arrays and convert them to x,y,z world coordinates. Uses a spherical model of the earth to do the conversion. The position is in meters relative to the center of the earth.

If a transform is given, use the transform to convert latitude and longitude to the world coordinate.

To create an instance of class vtk\-Geo\-Assign\-Coordinates, simply invoke its constructor as follows \begin{DoxyVerb}  obj = vtkGeoAssignCoordinates
\end{DoxyVerb}
 \hypertarget{vtkwidgets_vtkxyplotwidget_Methods}{}\subsection{Methods}\label{vtkwidgets_vtkxyplotwidget_Methods}
The class vtk\-Geo\-Assign\-Coordinates has several methods that can be used. They are listed below. Note that the documentation is translated automatically from the V\-T\-K sources, and may not be completely intelligible. When in doubt, consult the V\-T\-K website. In the methods listed below, {\ttfamily obj} is an instance of the vtk\-Geo\-Assign\-Coordinates class. 
\begin{DoxyItemize}
\item {\ttfamily string = obj.\-Get\-Class\-Name ()}  
\item {\ttfamily int = obj.\-Is\-A (string name)}  
\item {\ttfamily vtk\-Geo\-Assign\-Coordinates = obj.\-New\-Instance ()}  
\item {\ttfamily vtk\-Geo\-Assign\-Coordinates = obj.\-Safe\-Down\-Cast (vtk\-Object o)}  
\item {\ttfamily obj.\-Set\-Longitude\-Array\-Name (string )} -\/ Set the longitude coordinate array name.  
\item {\ttfamily string = obj.\-Get\-Longitude\-Array\-Name ()} -\/ Set the longitude coordinate array name.  
\item {\ttfamily obj.\-Set\-Latitude\-Array\-Name (string )} -\/ Set the latitude coordinate array name.  
\item {\ttfamily string = obj.\-Get\-Latitude\-Array\-Name ()} -\/ Set the latitude coordinate array name.  
\item {\ttfamily obj.\-Set\-Globe\-Radius (double )} -\/ The base radius to use in G\-L\-O\-B\-A\-L mode. Default is the earth's radius.  
\item {\ttfamily double = obj.\-Get\-Globe\-Radius ()} -\/ The base radius to use in G\-L\-O\-B\-A\-L mode. Default is the earth's radius.  
\item {\ttfamily obj.\-Set\-Transform (vtk\-Abstract\-Transform trans)} -\/ The transform to use to convert coordinates of the form (lat, long, 0) to (x, y z). If this is N\-U\-L\-L (the default), use Globe\-Radius to perform a spherical embedding.  
\item {\ttfamily vtk\-Abstract\-Transform = obj.\-Get\-Transform ()} -\/ The transform to use to convert coordinates of the form (lat, long, 0) to (x, y z). If this is N\-U\-L\-L (the default), use Globe\-Radius to perform a spherical embedding.  
\item {\ttfamily obj.\-Set\-Coordinates\-In\-Arrays (bool )} -\/ If on, uses Latitude\-Array\-Name and Longitude\-Array\-Name to move values in data arrays into the points of the data set. Turn off if the lattitude and longitude are already in the points.  
\item {\ttfamily bool = obj.\-Get\-Coordinates\-In\-Arrays ()} -\/ If on, uses Latitude\-Array\-Name and Longitude\-Array\-Name to move values in data arrays into the points of the data set. Turn off if the lattitude and longitude are already in the points.  
\item {\ttfamily obj.\-Coordinates\-In\-Arrays\-On ()} -\/ If on, uses Latitude\-Array\-Name and Longitude\-Array\-Name to move values in data arrays into the points of the data set. Turn off if the lattitude and longitude are already in the points.  
\item {\ttfamily obj.\-Coordinates\-In\-Arrays\-Off ()} -\/ If on, uses Latitude\-Array\-Name and Longitude\-Array\-Name to move values in data arrays into the points of the data set. Turn off if the lattitude and longitude are already in the points.  
\end{DoxyItemize}\hypertarget{vtkgeovis_vtkgeocamera}{}\section{vtk\-Geo\-Camera}\label{vtkgeovis_vtkgeocamera}
Section\-: \hyperlink{sec_vtkgeovis}{Visualization Toolkit Geo Vis Classes} \hypertarget{vtkwidgets_vtkxyplotwidget_Usage}{}\subsection{Usage}\label{vtkwidgets_vtkxyplotwidget_Usage}
I wanted to hide the normal vtk\-Camera A\-P\-I so I did not make this a subclass. The camera is a helper object. You can get a pointer to the camera, but it should be treated like a const.

To create an instance of class vtk\-Geo\-Camera, simply invoke its constructor as follows \begin{DoxyVerb}  obj = vtkGeoCamera
\end{DoxyVerb}
 \hypertarget{vtkwidgets_vtkxyplotwidget_Methods}{}\subsection{Methods}\label{vtkwidgets_vtkxyplotwidget_Methods}
The class vtk\-Geo\-Camera has several methods that can be used. They are listed below. Note that the documentation is translated automatically from the V\-T\-K sources, and may not be completely intelligible. When in doubt, consult the V\-T\-K website. In the methods listed below, {\ttfamily obj} is an instance of the vtk\-Geo\-Camera class. 
\begin{DoxyItemize}
\item {\ttfamily string = obj.\-Get\-Class\-Name ()}  
\item {\ttfamily int = obj.\-Is\-A (string name)}  
\item {\ttfamily vtk\-Geo\-Camera = obj.\-New\-Instance ()}  
\item {\ttfamily vtk\-Geo\-Camera = obj.\-Safe\-Down\-Cast (vtk\-Object o)}  
\item {\ttfamily double = obj. Get\-Position ()} -\/ Get the world position without the origin shift.  
\item {\ttfamily obj.\-Set\-Longitude (double longitude)} -\/ Longitude is in degrees\-: (-\/180-\/$>$180) Relative to absolute coordinates. Rotate Longitude around z axis (earth axis),  
\item {\ttfamily double = obj.\-Get\-Longitude ()} -\/ Longitude is in degrees\-: (-\/180-\/$>$180) Relative to absolute coordinates. Rotate Longitude around z axis (earth axis),  
\item {\ttfamily obj.\-Set\-Latitude (double latitude)} -\/ Latitude is in degrees\-: (-\/90-\/$>$90) Relative to Longitude. Rotate Latitude around x axis by Latitude,  
\item {\ttfamily double = obj.\-Get\-Latitude ()} -\/ Latitude is in degrees\-: (-\/90-\/$>$90) Relative to Longitude. Rotate Latitude around x axis by Latitude,  
\item {\ttfamily obj.\-Set\-Distance (double Distance)} -\/ Distance is in Meters Relative to Longitude and Latitude. above sea level ???? should we make this from center of earth ???? ???? what about equatorial bulge ????  
\item {\ttfamily double = obj.\-Get\-Distance ()} -\/ Distance is in Meters Relative to Longitude and Latitude. above sea level ???? should we make this from center of earth ???? ???? what about equatorial bulge ????  
\item {\ttfamily obj.\-Set\-Heading (double heading)} -\/ Heading is in degrees\-: (-\/180-\/$>$180) Relative to Logitude and Latitude. 0 is north. 90 is east. ???? what is the standard ???? 180 is south. -\/90 is west. Rotate Heading around -\/y axis Center,  
\item {\ttfamily double = obj.\-Get\-Heading ()} -\/ Heading is in degrees\-: (-\/180-\/$>$180) Relative to Logitude and Latitude. 0 is north. 90 is east. ???? what is the standard ???? 180 is south. -\/90 is west. Rotate Heading around -\/y axis Center,  
\item {\ttfamily obj.\-Set\-Tilt (double tilt)} -\/ Tilt is also know as pitch. Tilt is in degrees\-: (0-\/$>$90) Relative to Longitude, Latitude, and Heading. Rotate Tilt around x axis,  
\item {\ttfamily double = obj.\-Get\-Tilt ()} -\/ Tilt is also know as pitch. Tilt is in degrees\-: (0-\/$>$90) Relative to Longitude, Latitude, and Heading. Rotate Tilt around x axis,  
\item {\ttfamily vtk\-Camera = obj.\-Get\-V\-T\-K\-Camera ()} -\/ This vtk camera is updated to match this geo cameras state. It should be treated as a const and should not be modified.  
\item {\ttfamily obj.\-Initialize\-Node\-Analysis (int renderer\-Size\mbox{[}2\mbox{]})} -\/ We precompute some values to speed up update of the terrain. Unfortunately, they have to be manually/explicitely updated when the camera or renderer size changes.  
\item {\ttfamily double = obj.\-Get\-Node\-Coverage (vtk\-Geo\-Terrain\-Node node)} -\/ This method estimates how much of the view is covered by the sphere. Returns a value from 0 to 1.  
\item {\ttfamily bool = obj.\-Get\-Lock\-Heading ()}  
\item {\ttfamily obj.\-Set\-Lock\-Heading (bool )}  
\item {\ttfamily obj.\-Lock\-Heading\-On ()}  
\item {\ttfamily obj.\-Lock\-Heading\-Off ()}  
\item {\ttfamily obj.\-Set\-Origin\-Latitude (double o\-Lat)} -\/ This point is shifted to 0,0,0 to avoid open\-G\-L issues.  
\item {\ttfamily double = obj.\-Get\-Origin\-Latitude ()} -\/ This point is shifted to 0,0,0 to avoid open\-G\-L issues.  
\item {\ttfamily obj.\-Set\-Origin\-Longitude (double o\-Lat)} -\/ This point is shifted to 0,0,0 to avoid open\-G\-L issues.  
\item {\ttfamily double = obj.\-Get\-Origin\-Longitude ()} -\/ This point is shifted to 0,0,0 to avoid open\-G\-L issues.  
\item {\ttfamily double = obj. Get\-Origin ()} -\/ Get the rectilinear cooridinate location of the origin. This is used to shift the terrain points.  
\item {\ttfamily obj.\-Set\-Origin (double ox, double oy, double oz)}  
\end{DoxyItemize}\hypertarget{vtkgeovis_vtkgeofileimagesource}{}\section{vtk\-Geo\-File\-Image\-Source}\label{vtkgeovis_vtkgeofileimagesource}
Section\-: \hyperlink{sec_vtkgeovis}{Visualization Toolkit Geo Vis Classes} \hypertarget{vtkwidgets_vtkxyplotwidget_Usage}{}\subsection{Usage}\label{vtkwidgets_vtkxyplotwidget_Usage}
vtk\-Geo\-File\-Image\-Source is a vtk\-Geo\-Source that fetches .vti images from disk in a directory with a certain naming scheme. You may use vtk\-Geo\-Aligned\-Image\-Representation's Save\-Database method to generate an database of image tiles in this format.

To create an instance of class vtk\-Geo\-File\-Image\-Source, simply invoke its constructor as follows \begin{DoxyVerb}  obj = vtkGeoFileImageSource
\end{DoxyVerb}
 \hypertarget{vtkwidgets_vtkxyplotwidget_Methods}{}\subsection{Methods}\label{vtkwidgets_vtkxyplotwidget_Methods}
The class vtk\-Geo\-File\-Image\-Source has several methods that can be used. They are listed below. Note that the documentation is translated automatically from the V\-T\-K sources, and may not be completely intelligible. When in doubt, consult the V\-T\-K website. In the methods listed below, {\ttfamily obj} is an instance of the vtk\-Geo\-File\-Image\-Source class. 
\begin{DoxyItemize}
\item {\ttfamily string = obj.\-Get\-Class\-Name ()}  
\item {\ttfamily int = obj.\-Is\-A (string name)}  
\item {\ttfamily vtk\-Geo\-File\-Image\-Source = obj.\-New\-Instance ()}  
\item {\ttfamily vtk\-Geo\-File\-Image\-Source = obj.\-Safe\-Down\-Cast (vtk\-Object o)}  
\item {\ttfamily vtk\-Geo\-File\-Image\-Source = obj.()}  
\item {\ttfamily $\sim$vtk\-Geo\-File\-Image\-Source = obj.()}  
\item {\ttfamily bool = obj.\-Fetch\-Root (vtk\-Geo\-Tree\-Node root)} -\/ Fetches the root image representing the whole globe.  
\item {\ttfamily bool = obj.\-Fetch\-Child (vtk\-Geo\-Tree\-Node node, int index, vtk\-Geo\-Tree\-Node child)} -\/ Fetches the child image of a parent from disk.  
\item {\ttfamily obj.\-Set\-Path (string )} -\/ The path the tiled image database.  
\item {\ttfamily string = obj.\-Get\-Path ()} -\/ The path the tiled image database.  
\end{DoxyItemize}\hypertarget{vtkgeovis_vtkgeofileterrainsource}{}\section{vtk\-Geo\-File\-Terrain\-Source}\label{vtkgeovis_vtkgeofileterrainsource}
Section\-: \hyperlink{sec_vtkgeovis}{Visualization Toolkit Geo Vis Classes} \hypertarget{vtkwidgets_vtkxyplotwidget_Usage}{}\subsection{Usage}\label{vtkwidgets_vtkxyplotwidget_Usage}
vtk\-Geo\-File\-Terrain\-Source reads geometry tiles as .vtp files from a directory that follow a certain naming convention containing the level of the patch and the position within that level. Use vtk\-Geo\-Terrain's Save\-Database method to create a database of files in this format.

To create an instance of class vtk\-Geo\-File\-Terrain\-Source, simply invoke its constructor as follows \begin{DoxyVerb}  obj = vtkGeoFileTerrainSource
\end{DoxyVerb}
 \hypertarget{vtkwidgets_vtkxyplotwidget_Methods}{}\subsection{Methods}\label{vtkwidgets_vtkxyplotwidget_Methods}
The class vtk\-Geo\-File\-Terrain\-Source has several methods that can be used. They are listed below. Note that the documentation is translated automatically from the V\-T\-K sources, and may not be completely intelligible. When in doubt, consult the V\-T\-K website. In the methods listed below, {\ttfamily obj} is an instance of the vtk\-Geo\-File\-Terrain\-Source class. 
\begin{DoxyItemize}
\item {\ttfamily string = obj.\-Get\-Class\-Name ()}  
\item {\ttfamily int = obj.\-Is\-A (string name)}  
\item {\ttfamily vtk\-Geo\-File\-Terrain\-Source = obj.\-New\-Instance ()}  
\item {\ttfamily vtk\-Geo\-File\-Terrain\-Source = obj.\-Safe\-Down\-Cast (vtk\-Object o)}  
\item {\ttfamily vtk\-Geo\-File\-Terrain\-Source = obj.()}  
\item {\ttfamily $\sim$vtk\-Geo\-File\-Terrain\-Source = obj.()}  
\item {\ttfamily bool = obj.\-Fetch\-Root (vtk\-Geo\-Tree\-Node root)} -\/ Retrieve the root geometry representing the entire globe.  
\item {\ttfamily bool = obj.\-Fetch\-Child (vtk\-Geo\-Tree\-Node node, int index, vtk\-Geo\-Tree\-Node child)}  
\item {\ttfamily obj.\-Set\-Path (string )} -\/ The path the tiled geometry database.  
\item {\ttfamily string = obj.\-Get\-Path ()} -\/ The path the tiled geometry database.  
\end{DoxyItemize}\hypertarget{vtkgeovis_vtkgeoglobesource}{}\section{vtk\-Geo\-Globe\-Source}\label{vtkgeovis_vtkgeoglobesource}
Section\-: \hyperlink{sec_vtkgeovis}{Visualization Toolkit Geo Vis Classes} \hypertarget{vtkwidgets_vtkxyplotwidget_Usage}{}\subsection{Usage}\label{vtkwidgets_vtkxyplotwidget_Usage}
vtk\-Geo\-Globe\-Source is a 3\-D vtk\-Geo\-Source suitable for use in vtk\-Geo\-Terrain. It uses the vtk\-Globe\-Source filter to produce terrain patches.

To create an instance of class vtk\-Geo\-Globe\-Source, simply invoke its constructor as follows \begin{DoxyVerb}  obj = vtkGeoGlobeSource
\end{DoxyVerb}
 \hypertarget{vtkwidgets_vtkxyplotwidget_Methods}{}\subsection{Methods}\label{vtkwidgets_vtkxyplotwidget_Methods}
The class vtk\-Geo\-Globe\-Source has several methods that can be used. They are listed below. Note that the documentation is translated automatically from the V\-T\-K sources, and may not be completely intelligible. When in doubt, consult the V\-T\-K website. In the methods listed below, {\ttfamily obj} is an instance of the vtk\-Geo\-Globe\-Source class. 
\begin{DoxyItemize}
\item {\ttfamily string = obj.\-Get\-Class\-Name ()}  
\item {\ttfamily int = obj.\-Is\-A (string name)}  
\item {\ttfamily vtk\-Geo\-Globe\-Source = obj.\-New\-Instance ()}  
\item {\ttfamily vtk\-Geo\-Globe\-Source = obj.\-Safe\-Down\-Cast (vtk\-Object o)}  
\item {\ttfamily bool = obj.\-Fetch\-Root (vtk\-Geo\-Tree\-Node root)} -\/ Fetches a low-\/resolution sphere for the entire globe.  
\item {\ttfamily bool = obj.\-Fetch\-Child (vtk\-Geo\-Tree\-Node node, int index, vtk\-Geo\-Tree\-Node child)} -\/ Fetches a refined geometry patch, a section of a sphere.  
\end{DoxyItemize}\hypertarget{vtkgeovis_vtkgeograticule}{}\section{vtk\-Geo\-Graticule}\label{vtkgeovis_vtkgeograticule}
Section\-: \hyperlink{sec_vtkgeovis}{Visualization Toolkit Geo Vis Classes} \hypertarget{vtkwidgets_vtkxyplotwidget_Usage}{}\subsection{Usage}\label{vtkwidgets_vtkxyplotwidget_Usage}
This filter generates polydata to illustrate the distortions introduced by a map projection. The level parameter specifies the number of lines to be drawn. Poles are treated differently than other regions; hence the use of a Level parameter instead of a Number\-Of\-Lines parameter. The latitude and longitude are specified as half-\/open intervals with units of degrees. By default the latitude bounds are \mbox{[}-\/90,90\mbox{[} and the longitude bounds are \mbox{[}0,180\mbox{[}.

To create an instance of class vtk\-Geo\-Graticule, simply invoke its constructor as follows \begin{DoxyVerb}  obj = vtkGeoGraticule
\end{DoxyVerb}
 \hypertarget{vtkwidgets_vtkxyplotwidget_Methods}{}\subsection{Methods}\label{vtkwidgets_vtkxyplotwidget_Methods}
The class vtk\-Geo\-Graticule has several methods that can be used. They are listed below. Note that the documentation is translated automatically from the V\-T\-K sources, and may not be completely intelligible. When in doubt, consult the V\-T\-K website. In the methods listed below, {\ttfamily obj} is an instance of the vtk\-Geo\-Graticule class. 
\begin{DoxyItemize}
\item {\ttfamily string = obj.\-Get\-Class\-Name ()}  
\item {\ttfamily int = obj.\-Is\-A (string name)}  
\item {\ttfamily vtk\-Geo\-Graticule = obj.\-New\-Instance ()}  
\item {\ttfamily vtk\-Geo\-Graticule = obj.\-Safe\-Down\-Cast (vtk\-Object o)}  
\item {\ttfamily obj.\-Set\-Latitude\-Bounds (double , double )} -\/ The latitude bounds of the graticule.  
\item {\ttfamily obj.\-Set\-Latitude\-Bounds (double a\mbox{[}2\mbox{]})} -\/ The latitude bounds of the graticule.  
\item {\ttfamily double = obj. Get\-Latitude\-Bounds ()} -\/ The latitude bounds of the graticule.  
\item {\ttfamily obj.\-Set\-Longitude\-Bounds (double , double )} -\/ The longitude bounds of the graticule.  
\item {\ttfamily obj.\-Set\-Longitude\-Bounds (double a\mbox{[}2\mbox{]})} -\/ The longitude bounds of the graticule.  
\item {\ttfamily double = obj. Get\-Longitude\-Bounds ()} -\/ The longitude bounds of the graticule.  
\item {\ttfamily obj.\-Set\-Latitude\-Level (int )} -\/ The frequency level of latitude lines.  
\item {\ttfamily int = obj.\-Get\-Latitude\-Level\-Min\-Value ()} -\/ The frequency level of latitude lines.  
\item {\ttfamily int = obj.\-Get\-Latitude\-Level\-Max\-Value ()} -\/ The frequency level of latitude lines.  
\item {\ttfamily int = obj.\-Get\-Latitude\-Level ()} -\/ The frequency level of latitude lines.  
\item {\ttfamily obj.\-Set\-Longitude\-Level (int )} -\/ The frequency level of longitude lines.  
\item {\ttfamily int = obj.\-Get\-Longitude\-Level\-Min\-Value ()} -\/ The frequency level of longitude lines.  
\item {\ttfamily int = obj.\-Get\-Longitude\-Level\-Max\-Value ()} -\/ The frequency level of longitude lines.  
\item {\ttfamily int = obj.\-Get\-Longitude\-Level ()} -\/ The frequency level of longitude lines.  
\item {\ttfamily obj.\-Set\-Geometry\-Type (int )} -\/ Set//get the type(s) of cells that will be output by the filter. By default, polylines are output. You may also request quadrilaterals. This is a bit vector of Geometry\-Type enums.  
\item {\ttfamily int = obj.\-Get\-Geometry\-Type ()} -\/ Set//get the type(s) of cells that will be output by the filter. By default, polylines are output. You may also request quadrilaterals. This is a bit vector of Geometry\-Type enums.  
\end{DoxyItemize}\hypertarget{vtkgeovis_vtkgeoimagenode}{}\section{vtk\-Geo\-Image\-Node}\label{vtkgeovis_vtkgeoimagenode}
Section\-: \hyperlink{sec_vtkgeovis}{Visualization Toolkit Geo Vis Classes} \hypertarget{vtkwidgets_vtkxyplotwidget_Usage}{}\subsection{Usage}\label{vtkwidgets_vtkxyplotwidget_Usage}
vtk\-Geo\-Image\-Node contains an image tile in a multi-\/resolution image tree, along with metadata about that image's extents.

To create an instance of class vtk\-Geo\-Image\-Node, simply invoke its constructor as follows \begin{DoxyVerb}  obj = vtkGeoImageNode
\end{DoxyVerb}
 \hypertarget{vtkwidgets_vtkxyplotwidget_Methods}{}\subsection{Methods}\label{vtkwidgets_vtkxyplotwidget_Methods}
The class vtk\-Geo\-Image\-Node has several methods that can be used. They are listed below. Note that the documentation is translated automatically from the V\-T\-K sources, and may not be completely intelligible. When in doubt, consult the V\-T\-K website. In the methods listed below, {\ttfamily obj} is an instance of the vtk\-Geo\-Image\-Node class. 
\begin{DoxyItemize}
\item {\ttfamily string = obj.\-Get\-Class\-Name ()}  
\item {\ttfamily int = obj.\-Is\-A (string name)}  
\item {\ttfamily vtk\-Geo\-Image\-Node = obj.\-New\-Instance ()}  
\item {\ttfamily vtk\-Geo\-Image\-Node = obj.\-Safe\-Down\-Cast (vtk\-Object o)}  
\item {\ttfamily vtk\-Geo\-Image\-Node = obj.\-Get\-Child (int idx)}  
\item {\ttfamily vtk\-Geo\-Image\-Node = obj.\-Get\-Parent ()}  
\item {\ttfamily vtk\-Image\-Data = obj.\-Get\-Image ()} -\/ Get the image tile.  
\item {\ttfamily obj.\-Set\-Image (vtk\-Image\-Data image)} -\/ Get the image tile.  
\item {\ttfamily vtk\-Texture = obj.\-Get\-Texture ()} -\/ Get the image tile.  
\item {\ttfamily obj.\-Set\-Texture (vtk\-Texture texture)} -\/ Get the image tile.  
\item {\ttfamily obj.\-Crop\-Image\-For\-Tile (vtk\-Image\-Data image, double image\-Lon\-Lat\-Ext, string prefix)} -\/ This crops the image as small as possible while still covering the patch. The Longitude Latitude range may get bigger to reflect the actual size of the image. If prefix is specified, writes the tile to that location.  
\item {\ttfamily obj.\-Load\-An\-Image (string prefix)} -\/ This loads the image from a tile database at the specified location.  
\item {\ttfamily obj.\-Shallow\-Copy (vtk\-Geo\-Tree\-Node src)} -\/ Shallow and Deep copy.  
\item {\ttfamily obj.\-Deep\-Copy (vtk\-Geo\-Tree\-Node src)} -\/ Shallow and Deep copy.  
\item {\ttfamily bool = obj.\-Has\-Data ()}  
\item {\ttfamily obj.\-Delete\-Data ()} -\/ Deletes the data associated with the node to make this an \char`\"{}empty\char`\"{} node. This is performed when the node has been unused for a certain amount of time.  
\end{DoxyItemize}\hypertarget{vtkgeovis_vtkgeointeractorstyle}{}\section{vtk\-Geo\-Interactor\-Style}\label{vtkgeovis_vtkgeointeractorstyle}
Section\-: \hyperlink{sec_vtkgeovis}{Visualization Toolkit Geo Vis Classes} \hypertarget{vtkwidgets_vtkxyplotwidget_Usage}{}\subsection{Usage}\label{vtkwidgets_vtkxyplotwidget_Usage}
vtk\-Geo\-Interactor\-Style contains interaction capabilities for a geographic view including orbit, zoom, and tilt. It also includes a compass widget for changing view parameters.

To create an instance of class vtk\-Geo\-Interactor\-Style, simply invoke its constructor as follows \begin{DoxyVerb}  obj = vtkGeoInteractorStyle
\end{DoxyVerb}
 \hypertarget{vtkwidgets_vtkxyplotwidget_Methods}{}\subsection{Methods}\label{vtkwidgets_vtkxyplotwidget_Methods}
The class vtk\-Geo\-Interactor\-Style has several methods that can be used. They are listed below. Note that the documentation is translated automatically from the V\-T\-K sources, and may not be completely intelligible. When in doubt, consult the V\-T\-K website. In the methods listed below, {\ttfamily obj} is an instance of the vtk\-Geo\-Interactor\-Style class. 
\begin{DoxyItemize}
\item {\ttfamily string = obj.\-Get\-Class\-Name ()}  
\item {\ttfamily int = obj.\-Is\-A (string name)}  
\item {\ttfamily vtk\-Geo\-Interactor\-Style = obj.\-New\-Instance ()}  
\item {\ttfamily vtk\-Geo\-Interactor\-Style = obj.\-Safe\-Down\-Cast (vtk\-Object o)}  
\item {\ttfamily obj.\-On\-Enter ()} -\/ Event bindings  
\item {\ttfamily obj.\-On\-Leave ()} -\/ Event bindings  
\item {\ttfamily obj.\-On\-Mouse\-Move ()} -\/ Event bindings  
\item {\ttfamily obj.\-On\-Left\-Button\-Up ()} -\/ Event bindings  
\item {\ttfamily obj.\-On\-Middle\-Button\-Up ()} -\/ Event bindings  
\item {\ttfamily obj.\-On\-Right\-Button\-Up ()} -\/ Event bindings  
\item {\ttfamily obj.\-On\-Left\-Button\-Down ()} -\/ Event bindings  
\item {\ttfamily obj.\-On\-Middle\-Button\-Down ()} -\/ Event bindings  
\item {\ttfamily obj.\-On\-Right\-Button\-Down ()} -\/ Event bindings  
\item {\ttfamily obj.\-On\-Char ()} -\/ Event bindings  
\item {\ttfamily obj.\-Rubber\-Band\-Zoom ()}  
\item {\ttfamily obj.\-Pan ()}  
\item {\ttfamily obj.\-Dolly ()}  
\item {\ttfamily obj.\-Redraw\-Rectangle ()}  
\item {\ttfamily obj.\-Start\-State (int newstate)}  
\item {\ttfamily vtk\-Geo\-Camera = obj.\-Get\-Geo\-Camera ()}  
\item {\ttfamily obj.\-Reset\-Camera ()} -\/ This can be used to set the camera to the standard view of the earth.  
\item {\ttfamily obj.\-Widget\-Interaction (vtk\-Object caller)}  
\item {\ttfamily obj.\-Set\-Interactor (vtk\-Render\-Window\-Interactor interactor)} -\/ Set/\-Get the Interactor wrapper being controlled by this object. (Satisfy superclass A\-P\-I.)  
\item {\ttfamily int = obj.\-Get\-Ray\-Intersection (double origin\mbox{[}3\mbox{]}, double direction\mbox{[}3\mbox{]}, double intersection\mbox{[}3\mbox{]})}  
\item {\ttfamily obj.\-Set\-Current\-Renderer (vtk\-Renderer )} -\/ Override to make the renderer use this camera subclass  
\item {\ttfamily bool = obj.\-Get\-Lock\-Heading ()}  
\item {\ttfamily obj.\-Set\-Lock\-Heading (bool )}  
\item {\ttfamily obj.\-Lock\-Heading\-On ()}  
\item {\ttfamily obj.\-Lock\-Heading\-Off ()}  
\item {\ttfamily obj.\-Reset\-Camera\-Clipping\-Range ()}  
\end{DoxyItemize}\hypertarget{vtkgeovis_vtkgeoprojection}{}\section{vtk\-Geo\-Projection}\label{vtkgeovis_vtkgeoprojection}
Section\-: \hyperlink{sec_vtkgeovis}{Visualization Toolkit Geo Vis Classes} \hypertarget{vtkwidgets_vtkxyplotwidget_Usage}{}\subsection{Usage}\label{vtkwidgets_vtkxyplotwidget_Usage}
This class uses the P\-R\-O\-J.\-4 library to represent geographic coordinate projections.

To create an instance of class vtk\-Geo\-Projection, simply invoke its constructor as follows \begin{DoxyVerb}  obj = vtkGeoProjection
\end{DoxyVerb}
 \hypertarget{vtkwidgets_vtkxyplotwidget_Methods}{}\subsection{Methods}\label{vtkwidgets_vtkxyplotwidget_Methods}
The class vtk\-Geo\-Projection has several methods that can be used. They are listed below. Note that the documentation is translated automatically from the V\-T\-K sources, and may not be completely intelligible. When in doubt, consult the V\-T\-K website. In the methods listed below, {\ttfamily obj} is an instance of the vtk\-Geo\-Projection class. 
\begin{DoxyItemize}
\item {\ttfamily string = obj.\-Get\-Class\-Name ()}  
\item {\ttfamily int = obj.\-Is\-A (string name)}  
\item {\ttfamily vtk\-Geo\-Projection = obj.\-New\-Instance ()}  
\item {\ttfamily vtk\-Geo\-Projection = obj.\-Safe\-Down\-Cast (vtk\-Object o)}  
\item {\ttfamily obj.\-Set\-Name (string )} -\/ Set/get the short name describing the projection you wish to use. This defaults to \char`\"{}rpoly\char`\"{} for no reason other than I like it. To get a list of valid values, use the Get\-Number\-Of\-Projections() and Get\-Projection\-Name(int) static methods.  
\item {\ttfamily string = obj.\-Get\-Name ()} -\/ Set/get the short name describing the projection you wish to use. This defaults to \char`\"{}rpoly\char`\"{} for no reason other than I like it. To get a list of valid values, use the Get\-Number\-Of\-Projections() and Get\-Projection\-Name(int) static methods.  
\item {\ttfamily int = obj.\-Get\-Index ()} -\/ Return the index of the current projection's type in the list of all projection types. On error, this will return -\/1. On success, it returns a number in \mbox{[}0,Get\-Number\-Of\-Projections()\mbox{[}.  
\item {\ttfamily string = obj.\-Get\-Description ()} -\/ Get the description of a projection. This will return N\-U\-L\-L if the projection name is invalid.  
\item {\ttfamily obj.\-Set\-Central\-Meridian (double )} -\/ Set/get the longitude which corresponds to the central meridian of the projection. This defaults to 0, the Greenwich Meridian.  
\item {\ttfamily double = obj.\-Get\-Central\-Meridian ()} -\/ Set/get the longitude which corresponds to the central meridian of the projection. This defaults to 0, the Greenwich Meridian.  
\end{DoxyItemize}\hypertarget{vtkgeovis_vtkgeoprojectionsource}{}\section{vtk\-Geo\-Projection\-Source}\label{vtkgeovis_vtkgeoprojectionsource}
Section\-: \hyperlink{sec_vtkgeovis}{Visualization Toolkit Geo Vis Classes} \hypertarget{vtkwidgets_vtkxyplotwidget_Usage}{}\subsection{Usage}\label{vtkwidgets_vtkxyplotwidget_Usage}
vtk\-Geo\-Projection\-Source is a vtk\-Geo\-Source suitable for use in vtk\-Terrain2\-D. This source uses the libproj4 library to produce geometry patches at multiple resolutions. Each patch covers a specific region in projected space.

To create an instance of class vtk\-Geo\-Projection\-Source, simply invoke its constructor as follows \begin{DoxyVerb}  obj = vtkGeoProjectionSource
\end{DoxyVerb}
 \hypertarget{vtkwidgets_vtkxyplotwidget_Methods}{}\subsection{Methods}\label{vtkwidgets_vtkxyplotwidget_Methods}
The class vtk\-Geo\-Projection\-Source has several methods that can be used. They are listed below. Note that the documentation is translated automatically from the V\-T\-K sources, and may not be completely intelligible. When in doubt, consult the V\-T\-K website. In the methods listed below, {\ttfamily obj} is an instance of the vtk\-Geo\-Projection\-Source class. 
\begin{DoxyItemize}
\item {\ttfamily string = obj.\-Get\-Class\-Name ()}  
\item {\ttfamily int = obj.\-Is\-A (string name)}  
\item {\ttfamily vtk\-Geo\-Projection\-Source = obj.\-New\-Instance ()}  
\item {\ttfamily vtk\-Geo\-Projection\-Source = obj.\-Safe\-Down\-Cast (vtk\-Object o)}  
\item {\ttfamily vtk\-Geo\-Projection\-Source = obj.()}  
\item {\ttfamily $\sim$vtk\-Geo\-Projection\-Source = obj.()}  
\item {\ttfamily bool = obj.\-Fetch\-Root (vtk\-Geo\-Tree\-Node root)} -\/ Blocking methods for sources with low latency.  
\item {\ttfamily bool = obj.\-Fetch\-Child (vtk\-Geo\-Tree\-Node node, int index, vtk\-Geo\-Tree\-Node child)} -\/ Blocking methods for sources with low latency.  
\item {\ttfamily int = obj.\-Get\-Projection ()} -\/ The projection I\-D defining the projection. Initial value is 0.  
\item {\ttfamily obj.\-Set\-Projection (int projection)} -\/ The projection I\-D defining the projection. Initial value is 0.  
\item {\ttfamily int = obj.\-Get\-Min\-Cells\-Per\-Node ()} -\/ The minimum number of cells per node.  
\item {\ttfamily obj.\-Set\-Min\-Cells\-Per\-Node (int )} -\/ The minimum number of cells per node.  
\item {\ttfamily vtk\-Abstract\-Transform = obj.\-Get\-Transform ()} -\/ Return the projection transformation used by this 2\-D terrain.  
\end{DoxyItemize}\hypertarget{vtkgeovis_vtkgeorandomgraphsource}{}\section{vtk\-Geo\-Random\-Graph\-Source}\label{vtkgeovis_vtkgeorandomgraphsource}
Section\-: \hyperlink{sec_vtkgeovis}{Visualization Toolkit Geo Vis Classes} \hypertarget{vtkwidgets_vtkxyplotwidget_Usage}{}\subsection{Usage}\label{vtkwidgets_vtkxyplotwidget_Usage}
Generates a graph with a specified number of vertices, with the density of edges specified by either an exact number of edges or the probability of an edge. You may additionally specify whether to begin with a random tree (which enforces graph connectivity).

The filter also adds random vertex attributes called latitude and longitude. The latitude is distributed uniformly from -\/90 to 90, while longitude is distributed uniformly from -\/180 to 180.

To create an instance of class vtk\-Geo\-Random\-Graph\-Source, simply invoke its constructor as follows \begin{DoxyVerb}  obj = vtkGeoRandomGraphSource
\end{DoxyVerb}
 \hypertarget{vtkwidgets_vtkxyplotwidget_Methods}{}\subsection{Methods}\label{vtkwidgets_vtkxyplotwidget_Methods}
The class vtk\-Geo\-Random\-Graph\-Source has several methods that can be used. They are listed below. Note that the documentation is translated automatically from the V\-T\-K sources, and may not be completely intelligible. When in doubt, consult the V\-T\-K website. In the methods listed below, {\ttfamily obj} is an instance of the vtk\-Geo\-Random\-Graph\-Source class. 
\begin{DoxyItemize}
\item {\ttfamily string = obj.\-Get\-Class\-Name ()}  
\item {\ttfamily int = obj.\-Is\-A (string name)}  
\item {\ttfamily vtk\-Geo\-Random\-Graph\-Source = obj.\-New\-Instance ()}  
\item {\ttfamily vtk\-Geo\-Random\-Graph\-Source = obj.\-Safe\-Down\-Cast (vtk\-Object o)}  
\end{DoxyItemize}\hypertarget{vtkgeovis_vtkgeosamplearcs}{}\section{vtk\-Geo\-Sample\-Arcs}\label{vtkgeovis_vtkgeosamplearcs}
Section\-: \hyperlink{sec_vtkgeovis}{Visualization Toolkit Geo Vis Classes} \hypertarget{vtkwidgets_vtkxyplotwidget_Usage}{}\subsection{Usage}\label{vtkwidgets_vtkxyplotwidget_Usage}
vtk\-Geo\-Sample\-Arcs refines lines in the input polygonal data so that the distance between adjacent points is no more than a threshold distance. Points are interpolated along the surface of the globe. This is useful in order to keep lines such as political boundaries from intersecting the globe and becoming invisible.

To create an instance of class vtk\-Geo\-Sample\-Arcs, simply invoke its constructor as follows \begin{DoxyVerb}  obj = vtkGeoSampleArcs
\end{DoxyVerb}
 \hypertarget{vtkwidgets_vtkxyplotwidget_Methods}{}\subsection{Methods}\label{vtkwidgets_vtkxyplotwidget_Methods}
The class vtk\-Geo\-Sample\-Arcs has several methods that can be used. They are listed below. Note that the documentation is translated automatically from the V\-T\-K sources, and may not be completely intelligible. When in doubt, consult the V\-T\-K website. In the methods listed below, {\ttfamily obj} is an instance of the vtk\-Geo\-Sample\-Arcs class. 
\begin{DoxyItemize}
\item {\ttfamily string = obj.\-Get\-Class\-Name ()}  
\item {\ttfamily int = obj.\-Is\-A (string name)}  
\item {\ttfamily vtk\-Geo\-Sample\-Arcs = obj.\-New\-Instance ()}  
\item {\ttfamily vtk\-Geo\-Sample\-Arcs = obj.\-Safe\-Down\-Cast (vtk\-Object o)}  
\item {\ttfamily obj.\-Set\-Globe\-Radius (double )} -\/ The base radius used to determine the earth's surface. Default is the earth's radius in meters. T\-O\-D\-O\-: Change this to take in a vtk\-Geo\-Terrain to get altitude.  
\item {\ttfamily double = obj.\-Get\-Globe\-Radius ()} -\/ The base radius used to determine the earth's surface. Default is the earth's radius in meters. T\-O\-D\-O\-: Change this to take in a vtk\-Geo\-Terrain to get altitude.  
\item {\ttfamily obj.\-Set\-Maximum\-Distance\-Meters (double )} -\/ The maximum distance, in meters, between adjacent points.  
\item {\ttfamily double = obj.\-Get\-Maximum\-Distance\-Meters ()} -\/ The maximum distance, in meters, between adjacent points.  
\item {\ttfamily obj.\-Set\-Input\-Coordinate\-System (int )} -\/ The input coordinate system. R\-E\-C\-T\-A\-N\-G\-U\-L\-A\-R is x,y,z meters relative the the earth center. S\-P\-H\-E\-R\-I\-C\-A\-L is longitude,latitude,altitude.  
\item {\ttfamily int = obj.\-Get\-Input\-Coordinate\-System ()} -\/ The input coordinate system. R\-E\-C\-T\-A\-N\-G\-U\-L\-A\-R is x,y,z meters relative the the earth center. S\-P\-H\-E\-R\-I\-C\-A\-L is longitude,latitude,altitude.  
\item {\ttfamily obj.\-Set\-Input\-Coordinate\-System\-To\-Rectangular ()} -\/ The input coordinate system. R\-E\-C\-T\-A\-N\-G\-U\-L\-A\-R is x,y,z meters relative the the earth center. S\-P\-H\-E\-R\-I\-C\-A\-L is longitude,latitude,altitude.  
\item {\ttfamily obj.\-Set\-Input\-Coordinate\-System\-To\-Spherical ()} -\/ The desired output coordinate system. R\-E\-C\-T\-A\-N\-G\-U\-L\-A\-R is x,y,z meters relative the the earth center. S\-P\-H\-E\-R\-I\-C\-A\-L is longitude,latitude,altitude.  
\item {\ttfamily obj.\-Set\-Output\-Coordinate\-System (int )} -\/ The desired output coordinate system. R\-E\-C\-T\-A\-N\-G\-U\-L\-A\-R is x,y,z meters relative the the earth center. S\-P\-H\-E\-R\-I\-C\-A\-L is longitude,latitude,altitude.  
\item {\ttfamily int = obj.\-Get\-Output\-Coordinate\-System ()} -\/ The desired output coordinate system. R\-E\-C\-T\-A\-N\-G\-U\-L\-A\-R is x,y,z meters relative the the earth center. S\-P\-H\-E\-R\-I\-C\-A\-L is longitude,latitude,altitude.  
\item {\ttfamily obj.\-Set\-Output\-Coordinate\-System\-To\-Rectangular ()} -\/ The desired output coordinate system. R\-E\-C\-T\-A\-N\-G\-U\-L\-A\-R is x,y,z meters relative the the earth center. S\-P\-H\-E\-R\-I\-C\-A\-L is longitude,latitude,altitude.  
\item {\ttfamily obj.\-Set\-Output\-Coordinate\-System\-To\-Spherical ()}  
\end{DoxyItemize}\hypertarget{vtkgeovis_vtkgeosource}{}\section{vtk\-Geo\-Source}\label{vtkgeovis_vtkgeosource}
Section\-: \hyperlink{sec_vtkgeovis}{Visualization Toolkit Geo Vis Classes} \hypertarget{vtkwidgets_vtkxyplotwidget_Usage}{}\subsection{Usage}\label{vtkwidgets_vtkxyplotwidget_Usage}
vtk\-Geo\-Source is an abstract superclass for all multi-\/resolution data sources shown in a geographic view like vtk\-Geo\-View or vtk\-Geo\-View2\-D. vtk\-Geo\-Source subclasses need to implement the Fetch\-Root() method, which fills a vtk\-Geo\-Tree\-Node with the low-\/res data at the root, and Fetch\-Child(), which produces a refinement of a parent node. Other geovis classes such as vtk\-Geo\-Terrain, vtk\-Geo\-Terrain2\-D, and vtk\-Geo\-Aligned\-Image\-Source use a vtk\-Geo\-Source subclass to build their geometry or image caches which are stored in trees. The source itself does not maintain the tree, but simply provides a mechanism for generating refined tree nodes.

Sources are multi-\/threaded. Each source may have one or more worker threads associated with it, which this superclass manages. It is essential that the Fetch\-Child() method is thread-\/safe, since it may be called from multiple workers simultaneously.

To create an instance of class vtk\-Geo\-Source, simply invoke its constructor as follows \begin{DoxyVerb}  obj = vtkGeoSource
\end{DoxyVerb}
 \hypertarget{vtkwidgets_vtkxyplotwidget_Methods}{}\subsection{Methods}\label{vtkwidgets_vtkxyplotwidget_Methods}
The class vtk\-Geo\-Source has several methods that can be used. They are listed below. Note that the documentation is translated automatically from the V\-T\-K sources, and may not be completely intelligible. When in doubt, consult the V\-T\-K website. In the methods listed below, {\ttfamily obj} is an instance of the vtk\-Geo\-Source class. 
\begin{DoxyItemize}
\item {\ttfamily string = obj.\-Get\-Class\-Name ()}  
\item {\ttfamily int = obj.\-Is\-A (string name)}  
\item {\ttfamily vtk\-Geo\-Source = obj.\-New\-Instance ()}  
\item {\ttfamily vtk\-Geo\-Source = obj.\-Safe\-Down\-Cast (vtk\-Object o)}  
\item {\ttfamily vtk\-Geo\-Source = obj.()}  
\item {\ttfamily $\sim$vtk\-Geo\-Source = obj.()}  
\item {\ttfamily bool = obj.\-Fetch\-Root (vtk\-Geo\-Tree\-Node root)} -\/ Blocking access methods to be implemented in subclasses.  
\item {\ttfamily bool = obj.\-Fetch\-Child (vtk\-Geo\-Tree\-Node node, int index, vtk\-Geo\-Tree\-Node child)} -\/ Blocking access methods to be implemented in subclasses.  
\item {\ttfamily obj.\-Request\-Children (vtk\-Geo\-Tree\-Node node)} -\/ Non-\/blocking methods for to use from the main application. After calling Request\-Children() for a certain node, Get\-Requested\-Nodes() will after a certain period of time return a non-\/null pointer to a collection of four vtk\-Geo\-Tree\-Node objects, which are the four children of the requested node. The collection is reference counted, so you need to eventually call Delete() on the returned collection pointer (if it is non-\/null).  
\item {\ttfamily vtk\-Collection = obj.\-Get\-Requested\-Nodes (vtk\-Geo\-Tree\-Node node)} -\/ Non-\/blocking methods for to use from the main application. After calling Request\-Children() for a certain node, Get\-Requested\-Nodes() will after a certain period of time return a non-\/null pointer to a collection of four vtk\-Geo\-Tree\-Node objects, which are the four children of the requested node. The collection is reference counted, so you need to eventually call Delete() on the returned collection pointer (if it is non-\/null).  
\item {\ttfamily obj.\-Initialize (int num\-Threads)} -\/ Spawn worker threads.  
\item {\ttfamily obj.\-Shut\-Down ()} -\/ Shut down the source. This terminates the thread and releases memory.  
\item {\ttfamily obj.\-Worker\-Thread ()}  
\item {\ttfamily vtk\-Abstract\-Transform = obj.\-Get\-Transform ()}  
\end{DoxyItemize}\hypertarget{vtkgeovis_vtkgeospheretransform}{}\section{vtk\-Geo\-Sphere\-Transform}\label{vtkgeovis_vtkgeospheretransform}
Section\-: \hyperlink{sec_vtkgeovis}{Visualization Toolkit Geo Vis Classes} \hypertarget{vtkwidgets_vtkxyplotwidget_Usage}{}\subsection{Usage}\label{vtkwidgets_vtkxyplotwidget_Usage}
the cartesian coordinate system is the following (if Base\-Altitude is 0),
\begin{DoxyItemize}
\item the origin is at the center of the earth
\item the x axis goes from the origin to (longtitude=-\/90,latitude=0), intersection of equator and the meridian passing just east of Galapagos Islands
\item the y axis goes from the origin to the intersection of Greenwitch meridian and equator (longitude=0,latitude=0)
\item the z axis goes from the origin to the Geographic North Pole (latitude=90)
\item therefore the frame is right-\/handed.
\end{DoxyItemize}

To create an instance of class vtk\-Geo\-Sphere\-Transform, simply invoke its constructor as follows \begin{DoxyVerb}  obj = vtkGeoSphereTransform
\end{DoxyVerb}
 \hypertarget{vtkwidgets_vtkxyplotwidget_Methods}{}\subsection{Methods}\label{vtkwidgets_vtkxyplotwidget_Methods}
The class vtk\-Geo\-Sphere\-Transform has several methods that can be used. They are listed below. Note that the documentation is translated automatically from the V\-T\-K sources, and may not be completely intelligible. When in doubt, consult the V\-T\-K website. In the methods listed below, {\ttfamily obj} is an instance of the vtk\-Geo\-Sphere\-Transform class. 
\begin{DoxyItemize}
\item {\ttfamily string = obj.\-Get\-Class\-Name ()}  
\item {\ttfamily int = obj.\-Is\-A (string name)}  
\item {\ttfamily vtk\-Geo\-Sphere\-Transform = obj.\-New\-Instance ()}  
\item {\ttfamily vtk\-Geo\-Sphere\-Transform = obj.\-Safe\-Down\-Cast (vtk\-Object o)}  
\item {\ttfamily obj.\-Inverse ()} -\/ Invert the transformation.  
\item {\ttfamily obj.\-Internal\-Transform\-Point (float in\mbox{[}3\mbox{]}, float out\mbox{[}3\mbox{]})} -\/ This will calculate the transformation without calling Update. Meant for use only within other V\-T\-K classes.  
\item {\ttfamily obj.\-Internal\-Transform\-Point (double in\mbox{[}3\mbox{]}, double out\mbox{[}3\mbox{]})} -\/ This will calculate the transformation without calling Update. Meant for use only within other V\-T\-K classes.  
\item {\ttfamily vtk\-Abstract\-Transform = obj.\-Make\-Transform ()} -\/ Make another transform of the same type.  
\item {\ttfamily obj.\-Set\-To\-Rectangular (bool )} -\/ If on, this transform converts (long,lat,alt) triples to (x,y,z) as an offset from the center of the earth. Alt, x, y, and z are all be in meters. If off, the tranform works in the reverse direction. Initial value is on.  
\item {\ttfamily bool = obj.\-Get\-To\-Rectangular ()} -\/ If on, this transform converts (long,lat,alt) triples to (x,y,z) as an offset from the center of the earth. Alt, x, y, and z are all be in meters. If off, the tranform works in the reverse direction. Initial value is on.  
\item {\ttfamily obj.\-To\-Rectangular\-On ()} -\/ If on, this transform converts (long,lat,alt) triples to (x,y,z) as an offset from the center of the earth. Alt, x, y, and z are all be in meters. If off, the tranform works in the reverse direction. Initial value is on.  
\item {\ttfamily obj.\-To\-Rectangular\-Off ()} -\/ If on, this transform converts (long,lat,alt) triples to (x,y,z) as an offset from the center of the earth. Alt, x, y, and z are all be in meters. If off, the tranform works in the reverse direction. Initial value is on.  
\item {\ttfamily obj.\-Set\-Base\-Altitude (double )} -\/ The base altitude to transform coordinates to. This can be useful for transforming lines just above the earth's surface. Default is 0.  
\item {\ttfamily double = obj.\-Get\-Base\-Altitude ()} -\/ The base altitude to transform coordinates to. This can be useful for transforming lines just above the earth's surface. Default is 0.  
\end{DoxyItemize}\hypertarget{vtkgeovis_vtkgeoterrain}{}\section{vtk\-Geo\-Terrain}\label{vtkgeovis_vtkgeoterrain}
Section\-: \hyperlink{sec_vtkgeovis}{Visualization Toolkit Geo Vis Classes} \hypertarget{vtkwidgets_vtkxyplotwidget_Usage}{}\subsection{Usage}\label{vtkwidgets_vtkxyplotwidget_Usage}
vtk\-Geo\-Terrain contains a multi-\/resolution tree of geometry representing the globe. It uses a vtk\-Geo\-Source subclass to generate the terrain, such as vtk\-Geo\-Globe\-Source. This source must be set before using the terrain in a vtk\-Geo\-View. The terrain also contains an Add\-Actors() method which will update the set of actors representing the globe given the current camera position.

To create an instance of class vtk\-Geo\-Terrain, simply invoke its constructor as follows \begin{DoxyVerb}  obj = vtkGeoTerrain
\end{DoxyVerb}
 \hypertarget{vtkwidgets_vtkxyplotwidget_Methods}{}\subsection{Methods}\label{vtkwidgets_vtkxyplotwidget_Methods}
The class vtk\-Geo\-Terrain has several methods that can be used. They are listed below. Note that the documentation is translated automatically from the V\-T\-K sources, and may not be completely intelligible. When in doubt, consult the V\-T\-K website. In the methods listed below, {\ttfamily obj} is an instance of the vtk\-Geo\-Terrain class. 
\begin{DoxyItemize}
\item {\ttfamily string = obj.\-Get\-Class\-Name ()}  
\item {\ttfamily int = obj.\-Is\-A (string name)}  
\item {\ttfamily vtk\-Geo\-Terrain = obj.\-New\-Instance ()}  
\item {\ttfamily vtk\-Geo\-Terrain = obj.\-Safe\-Down\-Cast (vtk\-Object o)}  
\item {\ttfamily vtk\-Geo\-Source = obj.\-Get\-Source ()} -\/ The source used to obtain geometry patches.  
\item {\ttfamily obj.\-Set\-Source (vtk\-Geo\-Source source)} -\/ The source used to obtain geometry patches.  
\item {\ttfamily obj.\-Save\-Database (string path, int depth)} -\/ Save the set of patches up to a given maximum depth.  
\item {\ttfamily obj.\-Add\-Actors (vtk\-Renderer ren, vtk\-Assembly assembly, vtk\-Collection image\-Reps)} -\/ Update the actors in an assembly used to render the globe. ren is the current renderer, and image\-Reps holds the collection of vtk\-Geo\-Aligned\-Image\-Representations that will be blended together to form the image on the globe.  
\item {\ttfamily obj.\-Set\-Origin (double , double , double )} -\/ The world-\/coordinate origin offset used to eliminate precision errors when zoomed in to a particular region of the globe.  
\item {\ttfamily obj.\-Set\-Origin (double a\mbox{[}3\mbox{]})} -\/ The world-\/coordinate origin offset used to eliminate precision errors when zoomed in to a particular region of the globe.  
\item {\ttfamily double = obj. Get\-Origin ()} -\/ The world-\/coordinate origin offset used to eliminate precision errors when zoomed in to a particular region of the globe.  
\item {\ttfamily obj.\-Set\-Max\-Level (int )} -\/ The maximum level of the terrain tree.  
\item {\ttfamily int = obj.\-Get\-Max\-Level\-Min\-Value ()} -\/ The maximum level of the terrain tree.  
\item {\ttfamily int = obj.\-Get\-Max\-Level\-Max\-Value ()} -\/ The maximum level of the terrain tree.  
\item {\ttfamily int = obj.\-Get\-Max\-Level ()} -\/ The maximum level of the terrain tree.  
\end{DoxyItemize}\hypertarget{vtkgeovis_vtkgeoterrain2d}{}\section{vtk\-Geo\-Terrain2\-D}\label{vtkgeovis_vtkgeoterrain2d}
Section\-: \hyperlink{sec_vtkgeovis}{Visualization Toolkit Geo Vis Classes} \hypertarget{vtkwidgets_vtkxyplotwidget_Usage}{}\subsection{Usage}\label{vtkwidgets_vtkxyplotwidget_Usage}
vtk\-Geo\-Terrain2\-D contains a multi-\/resolution tree of geometry representing the globe. It uses a vtk\-Geo\-Source subclass to generate the terrain, such as vtk\-Geo\-Projection\-Source. This source must be set before using the terrain in a vtk\-Geo\-View2\-D. The terrain also contains an Add\-Actors() method which updates the set of actors representing the globe given the current camera position.

To create an instance of class vtk\-Geo\-Terrain2\-D, simply invoke its constructor as follows \begin{DoxyVerb}  obj = vtkGeoTerrain2D
\end{DoxyVerb}
 \hypertarget{vtkwidgets_vtkxyplotwidget_Methods}{}\subsection{Methods}\label{vtkwidgets_vtkxyplotwidget_Methods}
The class vtk\-Geo\-Terrain2\-D has several methods that can be used. They are listed below. Note that the documentation is translated automatically from the V\-T\-K sources, and may not be completely intelligible. When in doubt, consult the V\-T\-K website. In the methods listed below, {\ttfamily obj} is an instance of the vtk\-Geo\-Terrain2\-D class. 
\begin{DoxyItemize}
\item {\ttfamily string = obj.\-Get\-Class\-Name ()}  
\item {\ttfamily int = obj.\-Is\-A (string name)}  
\item {\ttfamily vtk\-Geo\-Terrain2\-D = obj.\-New\-Instance ()}  
\item {\ttfamily vtk\-Geo\-Terrain2\-D = obj.\-Safe\-Down\-Cast (vtk\-Object o)}  
\item {\ttfamily obj.\-Set\-Texture\-Tolerance (double )} -\/ The maximum size of a single texel in pixels. Images will be refined if a texel becomes larger than the tolerance.  
\item {\ttfamily double = obj.\-Get\-Texture\-Tolerance ()} -\/ The maximum size of a single texel in pixels. Images will be refined if a texel becomes larger than the tolerance.  
\item {\ttfamily obj.\-Set\-Location\-Tolerance (double )} -\/ The maximum allowed deviation of geometry in pixels. Geometry will be refined if the deviation is larger than the tolerance.  
\item {\ttfamily double = obj.\-Get\-Location\-Tolerance ()} -\/ The maximum allowed deviation of geometry in pixels. Geometry will be refined if the deviation is larger than the tolerance.  
\item {\ttfamily vtk\-Abstract\-Transform = obj.\-Get\-Transform ()} -\/ Return the projection transformation used by this 2\-D terrain.  
\end{DoxyItemize}\hypertarget{vtkgeovis_vtkgeoterrainnode}{}\section{vtk\-Geo\-Terrain\-Node}\label{vtkgeovis_vtkgeoterrainnode}
Section\-: \hyperlink{sec_vtkgeovis}{Visualization Toolkit Geo Vis Classes} \hypertarget{vtkwidgets_vtkxyplotwidget_Usage}{}\subsection{Usage}\label{vtkwidgets_vtkxyplotwidget_Usage}
To create an instance of class vtk\-Geo\-Terrain\-Node, simply invoke its constructor as follows \begin{DoxyVerb}  obj = vtkGeoTerrainNode
\end{DoxyVerb}
 \hypertarget{vtkwidgets_vtkxyplotwidget_Methods}{}\subsection{Methods}\label{vtkwidgets_vtkxyplotwidget_Methods}
The class vtk\-Geo\-Terrain\-Node has several methods that can be used. They are listed below. Note that the documentation is translated automatically from the V\-T\-K sources, and may not be completely intelligible. When in doubt, consult the V\-T\-K website. In the methods listed below, {\ttfamily obj} is an instance of the vtk\-Geo\-Terrain\-Node class. 
\begin{DoxyItemize}
\item {\ttfamily string = obj.\-Get\-Class\-Name ()}  
\item {\ttfamily int = obj.\-Is\-A (string name)}  
\item {\ttfamily vtk\-Geo\-Terrain\-Node = obj.\-New\-Instance ()}  
\item {\ttfamily vtk\-Geo\-Terrain\-Node = obj.\-Safe\-Down\-Cast (vtk\-Object o)}  
\item {\ttfamily vtk\-Geo\-Terrain\-Node = obj.\-Get\-Child (int idx)}  
\item {\ttfamily vtk\-Geo\-Terrain\-Node = obj.\-Get\-Parent ()}  
\item {\ttfamily double = obj.\-Get\-Altitude (double longitude, double latitude)}  
\item {\ttfamily vtk\-Poly\-Data = obj.\-Get\-Model ()} -\/ Get the terrrain model. The user has to copy the terrain into this object.  
\item {\ttfamily obj.\-Set\-Model (vtk\-Poly\-Data model)} -\/ Get the terrrain model. The user has to copy the terrain into this object.  
\item {\ttfamily obj.\-Update\-Bounding\-Sphere ()} -\/ Bounding sphere is precomputed for faster updates of terrain.  
\item {\ttfamily double = obj.\-Get\-Bounding\-Sphere\-Radius ()} -\/ Bounding sphere is precomputed for faster updates of terrain.  
\item {\ttfamily double = obj. Get\-Bounding\-Sphere\-Center ()} -\/ Bounding sphere is precomputed for faster updates of terrain.  
\item {\ttfamily double = obj. Get\-Corner\-Normal00 ()}  
\item {\ttfamily double = obj. Get\-Corner\-Normal01 ()}  
\item {\ttfamily double = obj. Get\-Corner\-Normal10 ()}  
\item {\ttfamily double = obj. Get\-Corner\-Normal11 ()}  
\item {\ttfamily double = obj. Get\-Projection\-Bounds ()} -\/ For 2\-D projections, store the bounds of the node in projected space to quickly determine if a node is offscreen.  
\item {\ttfamily obj.\-Set\-Projection\-Bounds (double , double , double , double )} -\/ For 2\-D projections, store the bounds of the node in projected space to quickly determine if a node is offscreen.  
\item {\ttfamily obj.\-Set\-Projection\-Bounds (double a\mbox{[}4\mbox{]})} -\/ For 2\-D projections, store the bounds of the node in projected space to quickly determine if a node is offscreen.  
\item {\ttfamily int = obj.\-Get\-Graticule\-Level ()} -\/ For 2\-D projections, store the granularity of the graticule in this node.  
\item {\ttfamily obj.\-Set\-Graticule\-Level (int )} -\/ For 2\-D projections, store the granularity of the graticule in this node.  
\item {\ttfamily double = obj.\-Get\-Error ()} -\/ For 2\-D projections, store the maximum deviation of line segment centers from the actual projection value.  
\item {\ttfamily obj.\-Set\-Error (double )} -\/ For 2\-D projections, store the maximum deviation of line segment centers from the actual projection value.  
\item {\ttfamily float = obj.\-Get\-Coverage ()} -\/ For 2\-D projections, store the maximum deviation of line segment centers from the actual projection value.  
\item {\ttfamily obj.\-Set\-Coverage (float )} -\/ For 2\-D projections, store the maximum deviation of line segment centers from the actual projection value.  
\item {\ttfamily obj.\-Shallow\-Copy (vtk\-Geo\-Tree\-Node src)} -\/ Shallow and Deep copy.  
\item {\ttfamily obj.\-Deep\-Copy (vtk\-Geo\-Tree\-Node src)} -\/ Shallow and Deep copy.  
\item {\ttfamily bool = obj.\-Has\-Data ()} -\/ Returns whether this node has valid data associated with it, or if it is an \char`\"{}empty\char`\"{} node.  
\item {\ttfamily obj.\-Delete\-Data ()} -\/ Deletes the data associated with the node to make this an \char`\"{}empty\char`\"{} node. This is performed when the node has been unused for a certain amount of time.  
\end{DoxyItemize}\hypertarget{vtkgeovis_vtkgeotransform}{}\section{vtk\-Geo\-Transform}\label{vtkgeovis_vtkgeotransform}
Section\-: \hyperlink{sec_vtkgeovis}{Visualization Toolkit Geo Vis Classes} \hypertarget{vtkwidgets_vtkxyplotwidget_Usage}{}\subsection{Usage}\label{vtkwidgets_vtkxyplotwidget_Usage}
This class takes two geographic projections and transforms point coordinates between them.

To create an instance of class vtk\-Geo\-Transform, simply invoke its constructor as follows \begin{DoxyVerb}  obj = vtkGeoTransform
\end{DoxyVerb}
 \hypertarget{vtkwidgets_vtkxyplotwidget_Methods}{}\subsection{Methods}\label{vtkwidgets_vtkxyplotwidget_Methods}
The class vtk\-Geo\-Transform has several methods that can be used. They are listed below. Note that the documentation is translated automatically from the V\-T\-K sources, and may not be completely intelligible. When in doubt, consult the V\-T\-K website. In the methods listed below, {\ttfamily obj} is an instance of the vtk\-Geo\-Transform class. 
\begin{DoxyItemize}
\item {\ttfamily string = obj.\-Get\-Class\-Name ()}  
\item {\ttfamily int = obj.\-Is\-A (string name)}  
\item {\ttfamily vtk\-Geo\-Transform = obj.\-New\-Instance ()}  
\item {\ttfamily vtk\-Geo\-Transform = obj.\-Safe\-Down\-Cast (vtk\-Object o)}  
\item {\ttfamily obj.\-Set\-Source\-Projection (vtk\-Geo\-Projection source)} -\/ The source geographic projection.  
\item {\ttfamily vtk\-Geo\-Projection = obj.\-Get\-Source\-Projection ()} -\/ The source geographic projection.  
\item {\ttfamily obj.\-Set\-Destination\-Projection (vtk\-Geo\-Projection dest)} -\/ The target geographic projection.  
\item {\ttfamily vtk\-Geo\-Projection = obj.\-Get\-Destination\-Projection ()} -\/ The target geographic projection.  
\item {\ttfamily obj.\-Transform\-Points (vtk\-Points src, vtk\-Points dst)} -\/ Transform many points at once.  
\item {\ttfamily obj.\-Inverse ()} -\/ Invert the transformation.  
\item {\ttfamily obj.\-Internal\-Transform\-Point (float in\mbox{[}3\mbox{]}, float out\mbox{[}3\mbox{]})} -\/ This will calculate the transformation without calling Update. Meant for use only within other V\-T\-K classes.  
\item {\ttfamily obj.\-Internal\-Transform\-Point (double in\mbox{[}3\mbox{]}, double out\mbox{[}3\mbox{]})} -\/ This will calculate the transformation without calling Update. Meant for use only within other V\-T\-K classes.  
\item {\ttfamily vtk\-Abstract\-Transform = obj.\-Make\-Transform ()} -\/ Make another transform of the same type.  
\end{DoxyItemize}\hypertarget{vtkgeovis_vtkgeotreenode}{}\section{vtk\-Geo\-Tree\-Node}\label{vtkgeovis_vtkgeotreenode}
Section\-: \hyperlink{sec_vtkgeovis}{Visualization Toolkit Geo Vis Classes} \hypertarget{vtkwidgets_vtkxyplotwidget_Usage}{}\subsection{Usage}\label{vtkwidgets_vtkxyplotwidget_Usage}
A self-\/referential data structure for storing geometry or imagery for the geospatial views. The data is organized in a quadtree. Each node contains a pointer to its parent and owns references to its four child nodes. The I\-D of each node is unique in its level, and encodes the path from the root node in its bits.

To create an instance of class vtk\-Geo\-Tree\-Node, simply invoke its constructor as follows \begin{DoxyVerb}  obj = vtkGeoTreeNode
\end{DoxyVerb}
 \hypertarget{vtkwidgets_vtkxyplotwidget_Methods}{}\subsection{Methods}\label{vtkwidgets_vtkxyplotwidget_Methods}
The class vtk\-Geo\-Tree\-Node has several methods that can be used. They are listed below. Note that the documentation is translated automatically from the V\-T\-K sources, and may not be completely intelligible. When in doubt, consult the V\-T\-K website. In the methods listed below, {\ttfamily obj} is an instance of the vtk\-Geo\-Tree\-Node class. 
\begin{DoxyItemize}
\item {\ttfamily string = obj.\-Get\-Class\-Name ()}  
\item {\ttfamily int = obj.\-Is\-A (string name)}  
\item {\ttfamily vtk\-Geo\-Tree\-Node = obj.\-New\-Instance ()}  
\item {\ttfamily vtk\-Geo\-Tree\-Node = obj.\-Safe\-Down\-Cast (vtk\-Object o)}  
\item {\ttfamily obj.\-Set\-Id (long )} -\/ The id uniquely specified this node. For this implementation I am going to store the branch path in the bits.  
\item {\ttfamily long = obj.\-Get\-Id ()} -\/ The id uniquely specified this node. For this implementation I am going to store the branch path in the bits.  
\item {\ttfamily obj.\-Set\-Level (int )}  
\item {\ttfamily int = obj.\-Get\-Level ()}  
\item {\ttfamily obj.\-Set\-Longitude\-Range (double , double )} -\/ Longitude and latitude range of the terrain model.  
\item {\ttfamily obj.\-Set\-Longitude\-Range (double a\mbox{[}2\mbox{]})} -\/ Longitude and latitude range of the terrain model.  
\item {\ttfamily double = obj. Get\-Longitude\-Range ()} -\/ Longitude and latitude range of the terrain model.  
\item {\ttfamily obj.\-Set\-Latitude\-Range (double , double )} -\/ Longitude and latitude range of the terrain model.  
\item {\ttfamily obj.\-Set\-Latitude\-Range (double a\mbox{[}2\mbox{]})} -\/ Longitude and latitude range of the terrain model.  
\item {\ttfamily double = obj. Get\-Latitude\-Range ()} -\/ Longitude and latitude range of the terrain model.  
\item {\ttfamily obj.\-Set\-Child (vtk\-Geo\-Tree\-Node node, int idx)} -\/ Get a child of this node. If one is set, then they all should set. No not mix subclasses.  
\item {\ttfamily obj.\-Set\-Parent (vtk\-Geo\-Tree\-Node node)} -\/ Manage links to older and newer tree nodes. These are used to periodically delete unused patches.  
\item {\ttfamily obj.\-Set\-Older (vtk\-Geo\-Tree\-Node node)} -\/ Manage links to older and newer tree nodes. These are used to periodically delete unused patches.  
\item {\ttfamily vtk\-Geo\-Tree\-Node = obj.\-Get\-Older ()} -\/ Manage links to older and newer tree nodes. These are used to periodically delete unused patches.  
\item {\ttfamily obj.\-Set\-Newer (vtk\-Geo\-Tree\-Node node)} -\/ Manage links to older and newer tree nodes. These are used to periodically delete unused patches.  
\item {\ttfamily vtk\-Geo\-Tree\-Node = obj.\-Get\-Newer ()} -\/ Returns whether this node has valid data associated with it, or if it is an \char`\"{}empty\char`\"{} node.  
\item {\ttfamily bool = obj.\-Has\-Data ()} -\/ Deletes the data associated with the node to make this an \char`\"{}empty\char`\"{} node. This is performed when the node has been unused for a certain amount of time.  
\item {\ttfamily obj.\-Delete\-Data ()}  
\item {\ttfamily int = obj.\-Get\-Which\-Child\-Are\-You ()}  
\item {\ttfamily bool = obj.\-Is\-Descendant\-Of (vtk\-Geo\-Tree\-Node elder)} -\/ This method returns true if this node descends from the elder node. The descision is made from the node ids, so the nodes do not have to be in the same tree!  
\item {\ttfamily int = obj.\-Create\-Children ()}  
\item {\ttfamily vtk\-Geo\-Tree\-Node = obj.\-Get\-Child\-Tree\-Node (int idx)} -\/ Get the parent as a vtk\-Geo\-Tree\-Node. Subclasses also implement Get\-Parent() which returns the parent as the appropriate subclass type.  
\item {\ttfamily vtk\-Geo\-Tree\-Node = obj.\-Get\-Parent\-Tree\-Node ()} -\/ Shallow and Deep copy. Deep copy performs a shallow copy of the Child nodes.  
\item {\ttfamily obj.\-Shallow\-Copy (vtk\-Geo\-Tree\-Node src)} -\/ Shallow and Deep copy. Deep copy performs a shallow copy of the Child nodes.  
\item {\ttfamily obj.\-Deep\-Copy (vtk\-Geo\-Tree\-Node src)} -\/ Shallow and Deep copy. Deep copy performs a shallow copy of the Child nodes.  
\end{DoxyItemize}\hypertarget{vtkgeovis_vtkgeotreenodecache}{}\section{vtk\-Geo\-Tree\-Node\-Cache}\label{vtkgeovis_vtkgeotreenodecache}
Section\-: \hyperlink{sec_vtkgeovis}{Visualization Toolkit Geo Vis Classes} \hypertarget{vtkwidgets_vtkxyplotwidget_Usage}{}\subsection{Usage}\label{vtkwidgets_vtkxyplotwidget_Usage}
vtk\-Geo\-Tree\-Node\-Cache keeps track of a linked list of vtk\-Geo\-Tree\-Nodes, and has operations to move nodes to the front of the list and to delete data from the least used nodes. This is used to recover memory from nodes that store data that hasn't been used in a while.

To create an instance of class vtk\-Geo\-Tree\-Node\-Cache, simply invoke its constructor as follows \begin{DoxyVerb}  obj = vtkGeoTreeNodeCache
\end{DoxyVerb}
 \hypertarget{vtkwidgets_vtkxyplotwidget_Methods}{}\subsection{Methods}\label{vtkwidgets_vtkxyplotwidget_Methods}
The class vtk\-Geo\-Tree\-Node\-Cache has several methods that can be used. They are listed below. Note that the documentation is translated automatically from the V\-T\-K sources, and may not be completely intelligible. When in doubt, consult the V\-T\-K website. In the methods listed below, {\ttfamily obj} is an instance of the vtk\-Geo\-Tree\-Node\-Cache class. 
\begin{DoxyItemize}
\item {\ttfamily string = obj.\-Get\-Class\-Name ()}  
\item {\ttfamily int = obj.\-Is\-A (string name)}  
\item {\ttfamily vtk\-Geo\-Tree\-Node\-Cache = obj.\-New\-Instance ()}  
\item {\ttfamily vtk\-Geo\-Tree\-Node\-Cache = obj.\-Safe\-Down\-Cast (vtk\-Object o)}  
\item {\ttfamily obj.\-Set\-Cache\-Maximum\-Limit (int )} -\/ The size of the cache of geospatial nodes. When the size reaches this limit, the list of non-\/empty nodes will be shortened to Cache\-Minimum\-Limit.  
\item {\ttfamily int = obj.\-Get\-Cache\-Maximum\-Limit ()} -\/ The size of the cache of geospatial nodes. When the size reaches this limit, the list of non-\/empty nodes will be shortened to Cache\-Minimum\-Limit.  
\item {\ttfamily obj.\-Set\-Cache\-Minimum\-Limit (int )} -\/ The cache is reduced to this size when the maximum limit is reached.  
\item {\ttfamily int = obj.\-Get\-Cache\-Minimum\-Limit ()} -\/ The cache is reduced to this size when the maximum limit is reached.  
\item {\ttfamily obj.\-Send\-To\-Front (vtk\-Geo\-Tree\-Node node)} -\/ Send a node to the front of the list. Perform this whenever a node is accessed, so that the most recently accessed nodes' data are not deleted.  
\item {\ttfamily obj.\-Remove\-Node (vtk\-Geo\-Tree\-Node node)} -\/ Remove the node from the list.  
\item {\ttfamily int = obj.\-Get\-Size ()} -\/ The current size of the list.  
\end{DoxyItemize}\hypertarget{vtkgeovis_vtkgeoview}{}\section{vtk\-Geo\-View}\label{vtkgeovis_vtkgeoview}
Section\-: \hyperlink{sec_vtkgeovis}{Visualization Toolkit Geo Vis Classes} \hypertarget{vtkwidgets_vtkxyplotwidget_Usage}{}\subsection{Usage}\label{vtkwidgets_vtkxyplotwidget_Usage}
vtk\-Geo\-View is a 3\-D globe view. The globe may contain a multi-\/resolution geometry source (vtk\-Geo\-Terrain), multiple multi-\/resolution image sources (vtk\-Geo\-Aligned\-Image\-Representation), as well as other representations such as vtk\-Rendered\-Graph\-Representation. At a minimum, the view must have a terrain and one image representation. The view uses vtk\-Geo\-Interactor\-Style to orbit, zoom, and tilt the view, and contains a vtk\-Compass\-Widget for manipulating the camera.

Each terrain or image representation contains a vtk\-Geo\-Source subclass which generates geometry or imagery at multiple resolutions. As the camera position changes, the terrain and/or image representations may ask its vtk\-Geo\-Source to refine the geometry. This refinement is performed on a separate thread, and the data is added to the view when it becomes available.

To create an instance of class vtk\-Geo\-View, simply invoke its constructor as follows \begin{DoxyVerb}  obj = vtkGeoView
\end{DoxyVerb}
 \hypertarget{vtkwidgets_vtkxyplotwidget_Methods}{}\subsection{Methods}\label{vtkwidgets_vtkxyplotwidget_Methods}
The class vtk\-Geo\-View has several methods that can be used. They are listed below. Note that the documentation is translated automatically from the V\-T\-K sources, and may not be completely intelligible. When in doubt, consult the V\-T\-K website. In the methods listed below, {\ttfamily obj} is an instance of the vtk\-Geo\-View class. 
\begin{DoxyItemize}
\item {\ttfamily string = obj.\-Get\-Class\-Name ()}  
\item {\ttfamily int = obj.\-Is\-A (string name)}  
\item {\ttfamily vtk\-Geo\-View = obj.\-New\-Instance ()}  
\item {\ttfamily vtk\-Geo\-View = obj.\-Safe\-Down\-Cast (vtk\-Object o)}  
\item {\ttfamily vtk\-Geo\-Aligned\-Image\-Representation = obj.\-Add\-Default\-Image\-Representation (vtk\-Image\-Data image)} -\/ Adds an image representation with a simple terrain model using the image in the specified file as the globe terrain.  
\item {\ttfamily obj.\-Prepare\-For\-Rendering ()}  
\item {\ttfamily obj.\-Build\-Low\-Res\-Earth (double origin\mbox{[}3\mbox{]})} -\/ Rebuild low-\/res earth source; call after (re)setting origin.  
\item {\ttfamily obj.\-Set\-Lock\-Heading (bool lock)} -\/ Whether the view locks the heading when panning. Default is off.  
\item {\ttfamily bool = obj.\-Get\-Lock\-Heading ()} -\/ Whether the view locks the heading when panning. Default is off.  
\item {\ttfamily obj.\-Lock\-Heading\-On ()} -\/ Whether the view locks the heading when panning. Default is off.  
\item {\ttfamily obj.\-Lock\-Heading\-Off ()} -\/ Whether the view locks the heading when panning. Default is off.  
\item {\ttfamily vtk\-Geo\-Interactor\-Style = obj.\-Get\-Geo\-Interactor\-Style ()} -\/ Convenience method for obtaining the internal interactor style.  
\item {\ttfamily obj.\-Set\-Geo\-Interactor\-Style (vtk\-Geo\-Interactor\-Style style)} -\/ Method to change the interactor style.  
\item {\ttfamily obj.\-Set\-Terrain (vtk\-Geo\-Terrain terrain)} -\/ The terrain (geometry) model for this earth view.  
\item {\ttfamily vtk\-Geo\-Terrain = obj.\-Get\-Terrain ()} -\/ The terrain (geometry) model for this earth view.  
\item {\ttfamily obj.\-Render ()} -\/ Update and render the view.  
\end{DoxyItemize}\hypertarget{vtkgeovis_vtkgeoview2d}{}\section{vtk\-Geo\-View2\-D}\label{vtkgeovis_vtkgeoview2d}
Section\-: \hyperlink{sec_vtkgeovis}{Visualization Toolkit Geo Vis Classes} \hypertarget{vtkwidgets_vtkxyplotwidget_Usage}{}\subsection{Usage}\label{vtkwidgets_vtkxyplotwidget_Usage}
vtk\-Geo\-View is a 2\-D globe view. The globe may contain a multi-\/resolution geometry source (vtk\-Geo\-Terrain2\-D), multiple multi-\/resolution image sources (vtk\-Geo\-Aligned\-Image\-Representation), as well as other representations such as vtk\-Geo\-Graph\-Representation2\-D. At a minimum, the view must have a terrain and one image representation. By default, you may select in the view with the left mouse button, pan with the middle button, and zoom with the right mouse button or scroll wheel.

Each terrain or image representation contains a vtk\-Geo\-Source subclass which generates geometry or imagery at multiple resolutions. As the camera position changes, the terrain and/or image representations may ask its vtk\-Geo\-Source to refine the geometry. This refinement is performed on a separate thread, and the data is added to the view when it becomes available.

To create an instance of class vtk\-Geo\-View2\-D, simply invoke its constructor as follows \begin{DoxyVerb}  obj = vtkGeoView2D
\end{DoxyVerb}
 \hypertarget{vtkwidgets_vtkxyplotwidget_Methods}{}\subsection{Methods}\label{vtkwidgets_vtkxyplotwidget_Methods}
The class vtk\-Geo\-View2\-D has several methods that can be used. They are listed below. Note that the documentation is translated automatically from the V\-T\-K sources, and may not be completely intelligible. When in doubt, consult the V\-T\-K website. In the methods listed below, {\ttfamily obj} is an instance of the vtk\-Geo\-View2\-D class. 
\begin{DoxyItemize}
\item {\ttfamily string = obj.\-Get\-Class\-Name ()}  
\item {\ttfamily int = obj.\-Is\-A (string name)}  
\item {\ttfamily vtk\-Geo\-View2\-D = obj.\-New\-Instance ()}  
\item {\ttfamily vtk\-Geo\-View2\-D = obj.\-Safe\-Down\-Cast (vtk\-Object o)}  
\item {\ttfamily vtk\-Geo\-View2\-D = obj.()}  
\item {\ttfamily $\sim$vtk\-Geo\-View2\-D = obj.()}  
\item {\ttfamily vtk\-Geo\-Terrain2\-D = obj.\-Get\-Surface ()}  
\item {\ttfamily obj.\-Set\-Surface (vtk\-Geo\-Terrain2\-D surf)}  
\item {\ttfamily vtk\-Abstract\-Transform = obj.\-Get\-Transform ()} -\/ Returns the transform associated with the surface.  
\item {\ttfamily obj.\-Apply\-View\-Theme (vtk\-View\-Theme theme)} -\/ Apply the view theme to this view.  
\item {\ttfamily obj.\-Render ()} -\/ Update and render the view.  
\end{DoxyItemize}\hypertarget{vtkgeovis_vtkglobesource}{}\section{vtk\-Globe\-Source}\label{vtkgeovis_vtkglobesource}
Section\-: \hyperlink{sec_vtkgeovis}{Visualization Toolkit Geo Vis Classes} \hypertarget{vtkwidgets_vtkxyplotwidget_Usage}{}\subsection{Usage}\label{vtkwidgets_vtkxyplotwidget_Usage}
vtk\-Globe\-Source will generate any \char`\"{}rectangular\char`\"{} patch of the globe given its Longitude-\/\-Latitude extent. It adds two point scalar arrays Longitude and Latitude to the output. These arrays can be transformed to generate texture coordinates for any texture map. This source is imperfect near the poles as implmented. It should really reduce the longitude resolution as the triangles become slivers.

To create an instance of class vtk\-Globe\-Source, simply invoke its constructor as follows \begin{DoxyVerb}  obj = vtkGlobeSource
\end{DoxyVerb}
 \hypertarget{vtkwidgets_vtkxyplotwidget_Methods}{}\subsection{Methods}\label{vtkwidgets_vtkxyplotwidget_Methods}
The class vtk\-Globe\-Source has several methods that can be used. They are listed below. Note that the documentation is translated automatically from the V\-T\-K sources, and may not be completely intelligible. When in doubt, consult the V\-T\-K website. In the methods listed below, {\ttfamily obj} is an instance of the vtk\-Globe\-Source class. 
\begin{DoxyItemize}
\item {\ttfamily string = obj.\-Get\-Class\-Name ()}  
\item {\ttfamily int = obj.\-Is\-A (string name)}  
\item {\ttfamily vtk\-Globe\-Source = obj.\-New\-Instance ()}  
\item {\ttfamily vtk\-Globe\-Source = obj.\-Safe\-Down\-Cast (vtk\-Object o)}  
\item {\ttfamily obj.\-Set\-Origin (double , double , double )}  
\item {\ttfamily obj.\-Set\-Origin (double a\mbox{[}3\mbox{]})}  
\item {\ttfamily obj.\-Set\-Start\-Longitude (double )} -\/ Longitude Latitude clamps.  
\item {\ttfamily double = obj.\-Get\-Start\-Longitude\-Min\-Value ()} -\/ Longitude Latitude clamps.  
\item {\ttfamily double = obj.\-Get\-Start\-Longitude\-Max\-Value ()} -\/ Longitude Latitude clamps.  
\item {\ttfamily obj.\-Set\-End\-Longitude (double )} -\/ Longitude Latitude clamps.  
\item {\ttfamily double = obj.\-Get\-End\-Longitude\-Min\-Value ()} -\/ Longitude Latitude clamps.  
\item {\ttfamily double = obj.\-Get\-End\-Longitude\-Max\-Value ()} -\/ Longitude Latitude clamps.  
\item {\ttfamily obj.\-Set\-Start\-Latitude (double )} -\/ Longitude Latitude clamps.  
\item {\ttfamily double = obj.\-Get\-Start\-Latitude\-Min\-Value ()} -\/ Longitude Latitude clamps.  
\item {\ttfamily double = obj.\-Get\-Start\-Latitude\-Max\-Value ()} -\/ Longitude Latitude clamps.  
\item {\ttfamily obj.\-Set\-End\-Latitude (double )} -\/ Longitude Latitude clamps.  
\item {\ttfamily double = obj.\-Get\-End\-Latitude\-Min\-Value ()} -\/ Longitude Latitude clamps.  
\item {\ttfamily double = obj.\-Get\-End\-Latitude\-Max\-Value ()} -\/ Longitude Latitude clamps.  
\item {\ttfamily obj.\-Set\-Longitude\-Resolution (int )} -\/ Set the number of points in the longitude direction (ranging from Start\-Longitude to End\-Longitude).  
\item {\ttfamily int = obj.\-Get\-Longitude\-Resolution\-Min\-Value ()} -\/ Set the number of points in the longitude direction (ranging from Start\-Longitude to End\-Longitude).  
\item {\ttfamily int = obj.\-Get\-Longitude\-Resolution\-Max\-Value ()} -\/ Set the number of points in the longitude direction (ranging from Start\-Longitude to End\-Longitude).  
\item {\ttfamily int = obj.\-Get\-Longitude\-Resolution ()} -\/ Set the number of points in the longitude direction (ranging from Start\-Longitude to End\-Longitude).  
\item {\ttfamily obj.\-Set\-Latitude\-Resolution (int )} -\/ Set the number of points in the latitude direction (ranging from Start\-Latitude to End\-Latitude).  
\item {\ttfamily int = obj.\-Get\-Latitude\-Resolution\-Min\-Value ()} -\/ Set the number of points in the latitude direction (ranging from Start\-Latitude to End\-Latitude).  
\item {\ttfamily int = obj.\-Get\-Latitude\-Resolution\-Max\-Value ()} -\/ Set the number of points in the latitude direction (ranging from Start\-Latitude to End\-Latitude).  
\item {\ttfamily int = obj.\-Get\-Latitude\-Resolution ()} -\/ Set the number of points in the latitude direction (ranging from Start\-Latitude to End\-Latitude).  
\item {\ttfamily obj.\-Set\-Radius (double )} -\/ Set radius of sphere. Default is 6356750.\-0  
\item {\ttfamily double = obj.\-Get\-Radius\-Min\-Value ()} -\/ Set radius of sphere. Default is 6356750.\-0  
\item {\ttfamily double = obj.\-Get\-Radius\-Max\-Value ()} -\/ Set radius of sphere. Default is 6356750.\-0  
\item {\ttfamily double = obj.\-Get\-Radius ()} -\/ Set radius of sphere. Default is 6356750.\-0  
\item {\ttfamily obj.\-Set\-Curtain\-Height (double )}  
\item {\ttfamily double = obj.\-Get\-Curtain\-Height\-Min\-Value ()}  
\item {\ttfamily double = obj.\-Get\-Curtain\-Height\-Max\-Value ()}  
\item {\ttfamily double = obj.\-Get\-Curtain\-Height ()}  
\item {\ttfamily obj.\-Set\-Quadrilateral\-Tessellation (int )} -\/ Cause the sphere to be tessellated with edges along the latitude and longitude lines. If off, triangles are generated at non-\/polar regions, which results in edges that are not parallel to latitude and longitude lines. If on, quadrilaterals are generated everywhere except at the poles. This can be useful for generating a wireframe sphere with natural latitude and longitude lines.  
\item {\ttfamily int = obj.\-Get\-Quadrilateral\-Tessellation ()} -\/ Cause the sphere to be tessellated with edges along the latitude and longitude lines. If off, triangles are generated at non-\/polar regions, which results in edges that are not parallel to latitude and longitude lines. If on, quadrilaterals are generated everywhere except at the poles. This can be useful for generating a wireframe sphere with natural latitude and longitude lines.  
\item {\ttfamily obj.\-Quadrilateral\-Tessellation\-On ()} -\/ Cause the sphere to be tessellated with edges along the latitude and longitude lines. If off, triangles are generated at non-\/polar regions, which results in edges that are not parallel to latitude and longitude lines. If on, quadrilaterals are generated everywhere except at the poles. This can be useful for generating a wireframe sphere with natural latitude and longitude lines.  
\item {\ttfamily obj.\-Quadrilateral\-Tessellation\-Off ()} -\/ Cause the sphere to be tessellated with edges along the latitude and longitude lines. If off, triangles are generated at non-\/polar regions, which results in edges that are not parallel to latitude and longitude lines. If on, quadrilaterals are generated everywhere except at the poles. This can be useful for generating a wireframe sphere with natural latitude and longitude lines.  
\end{DoxyItemize}