
\begin{DoxyItemize}
\item \hyperlink{vtkinfovis_vtkaddmembershiparray}{vtk\-Add\-Membership\-Array}  
\item \hyperlink{vtkinfovis_vtkadjacencymatrixtoedgetable}{vtk\-Adjacency\-Matrix\-To\-Edge\-Table}  
\item \hyperlink{vtkinfovis_vtkappendpoints}{vtk\-Append\-Points}  
\item \hyperlink{vtkinfovis_vtkapplycolors}{vtk\-Apply\-Colors}  
\item \hyperlink{vtkinfovis_vtkapplyicons}{vtk\-Apply\-Icons}  
\item \hyperlink{vtkinfovis_vtkarcparalleledgestrategy}{vtk\-Arc\-Parallel\-Edge\-Strategy}  
\item \hyperlink{vtkinfovis_vtkarealayout}{vtk\-Area\-Layout}  
\item \hyperlink{vtkinfovis_vtkarealayoutstrategy}{vtk\-Area\-Layout\-Strategy}  
\item \hyperlink{vtkinfovis_vtkarraynorm}{vtk\-Array\-Norm}  
\item \hyperlink{vtkinfovis_vtkassigncoordinates}{vtk\-Assign\-Coordinates}  
\item \hyperlink{vtkinfovis_vtkassigncoordinateslayoutstrategy}{vtk\-Assign\-Coordinates\-Layout\-Strategy}  
\item \hyperlink{vtkinfovis_vtkattributeclustering2dlayoutstrategy}{vtk\-Attribute\-Clustering2\-D\-Layout\-Strategy}  
\item \hyperlink{vtkinfovis_vtkbivariatelineartablethreshold}{vtk\-Bivariate\-Linear\-Table\-Threshold}  
\item \hyperlink{vtkinfovis_vtkbivariatestatisticsalgorithm}{vtk\-Bivariate\-Statistics\-Algorithm}  
\item \hyperlink{vtkinfovis_vtkboxlayoutstrategy}{vtk\-Box\-Layout\-Strategy}  
\item \hyperlink{vtkinfovis_vtkchacographreader}{vtk\-Chaco\-Graph\-Reader}  
\item \hyperlink{vtkinfovis_vtkcircularlayoutstrategy}{vtk\-Circular\-Layout\-Strategy}  
\item \hyperlink{vtkinfovis_vtkclustering2dlayoutstrategy}{vtk\-Clustering2\-D\-Layout\-Strategy}  
\item \hyperlink{vtkinfovis_vtkcollapsegraph}{vtk\-Collapse\-Graph}  
\item \hyperlink{vtkinfovis_vtkcollapseverticesbyarray}{vtk\-Collapse\-Vertices\-By\-Array}  
\item \hyperlink{vtkinfovis_vtkcommunity2dlayoutstrategy}{vtk\-Community2\-D\-Layout\-Strategy}  
\item \hyperlink{vtkinfovis_vtkcomputehistogram2doutliers}{vtk\-Compute\-Histogram2\-D\-Outliers}  
\item \hyperlink{vtkinfovis_vtkconelayoutstrategy}{vtk\-Cone\-Layout\-Strategy}  
\item \hyperlink{vtkinfovis_vtkconstrained2dlayoutstrategy}{vtk\-Constrained2\-D\-Layout\-Strategy}  
\item \hyperlink{vtkinfovis_vtkcontingencystatistics}{vtk\-Contingency\-Statistics}  
\item \hyperlink{vtkinfovis_vtkcorrelativestatistics}{vtk\-Correlative\-Statistics}  
\item \hyperlink{vtkinfovis_vtkcosmictreelayoutstrategy}{vtk\-Cosmic\-Tree\-Layout\-Strategy}  
\item \hyperlink{vtkinfovis_vtkdataobjecttotable}{vtk\-Data\-Object\-To\-Table}  
\item \hyperlink{vtkinfovis_vtkdelimitedtextreader}{vtk\-Delimited\-Text\-Reader}  
\item \hyperlink{vtkinfovis_vtkdescriptivestatistics}{vtk\-Descriptive\-Statistics}  
\item \hyperlink{vtkinfovis_vtkdotproductsimilarity}{vtk\-Dot\-Product\-Similarity}  
\item \hyperlink{vtkinfovis_vtkedgecenters}{vtk\-Edge\-Centers}  
\item \hyperlink{vtkinfovis_vtkedgelayout}{vtk\-Edge\-Layout}  
\item \hyperlink{vtkinfovis_vtkedgelayoutstrategy}{vtk\-Edge\-Layout\-Strategy}  
\item \hyperlink{vtkinfovis_vtkexpandselectedgraph}{vtk\-Expand\-Selected\-Graph}  
\item \hyperlink{vtkinfovis_vtkextracthistogram2d}{vtk\-Extract\-Histogram2\-D}  
\item \hyperlink{vtkinfovis_vtkextractselectedgraph}{vtk\-Extract\-Selected\-Graph}  
\item \hyperlink{vtkinfovis_vtkfast2dlayoutstrategy}{vtk\-Fast2\-D\-Layout\-Strategy}  
\item \hyperlink{vtkinfovis_vtkfixedwidthtextreader}{vtk\-Fixed\-Width\-Text\-Reader}  
\item \hyperlink{vtkinfovis_vtkforcedirectedlayoutstrategy}{vtk\-Force\-Directed\-Layout\-Strategy}  
\item \hyperlink{vtkinfovis_vtkgenerateindexarray}{vtk\-Generate\-Index\-Array}  
\item \hyperlink{vtkinfovis_vtkgeoedgestrategy}{vtk\-Geo\-Edge\-Strategy}  
\item \hyperlink{vtkinfovis_vtkgeomath}{vtk\-Geo\-Math}  
\item \hyperlink{vtkinfovis_vtkgraphhierarchicalbundle}{vtk\-Graph\-Hierarchical\-Bundle}  
\item \hyperlink{vtkinfovis_vtkgraphhierarchicalbundleedges}{vtk\-Graph\-Hierarchical\-Bundle\-Edges}  
\item \hyperlink{vtkinfovis_vtkgraphlayout}{vtk\-Graph\-Layout}  
\item \hyperlink{vtkinfovis_vtkgraphlayoutstrategy}{vtk\-Graph\-Layout\-Strategy}  
\item \hyperlink{vtkinfovis_vtkgroupleafvertices}{vtk\-Group\-Leaf\-Vertices}  
\item \hyperlink{vtkinfovis_vtkisireader}{vtk\-I\-S\-I\-Reader}  
\item \hyperlink{vtkinfovis_vtkkmeansdistancefunctor}{vtk\-K\-Means\-Distance\-Functor}  
\item \hyperlink{vtkinfovis_vtkkmeansdistancefunctorcalculator}{vtk\-K\-Means\-Distance\-Functor\-Calculator}  
\item \hyperlink{vtkinfovis_vtkkmeansstatistics}{vtk\-K\-Means\-Statistics}  
\item \hyperlink{vtkinfovis_vtkmatricizearray}{vtk\-Matricize\-Array}  
\item \hyperlink{vtkinfovis_vtkmergecolumns}{vtk\-Merge\-Columns}  
\item \hyperlink{vtkinfovis_vtkmergegraphs}{vtk\-Merge\-Graphs}  
\item \hyperlink{vtkinfovis_vtkmergetables}{vtk\-Merge\-Tables}  
\item \hyperlink{vtkinfovis_vtkmulticorrelativestatistics}{vtk\-Multi\-Correlative\-Statistics}  
\item \hyperlink{vtkinfovis_vtkmutablegraphhelper}{vtk\-Mutable\-Graph\-Helper}  
\item \hyperlink{vtkinfovis_vtknetworkhierarchy}{vtk\-Network\-Hierarchy}  
\item \hyperlink{vtkinfovis_vtkorderstatistics}{vtk\-Order\-Statistics}  
\item \hyperlink{vtkinfovis_vtkpairwiseextracthistogram2d}{vtk\-Pairwise\-Extract\-Histogram2\-D}  
\item \hyperlink{vtkinfovis_vtkpassarrays}{vtk\-Pass\-Arrays}  
\item \hyperlink{vtkinfovis_vtkpassthrough}{vtk\-Pass\-Through}  
\item \hyperlink{vtkinfovis_vtkpassthroughedgestrategy}{vtk\-Pass\-Through\-Edge\-Strategy}  
\item \hyperlink{vtkinfovis_vtkpassthroughlayoutstrategy}{vtk\-Pass\-Through\-Layout\-Strategy}  
\item \hyperlink{vtkinfovis_vtkpbivariatelineartablethreshold}{vtk\-P\-Bivariate\-Linear\-Table\-Threshold}  
\item \hyperlink{vtkinfovis_vtkpcastatistics}{vtk\-P\-C\-A\-Statistics}  
\item \hyperlink{vtkinfovis_vtkpcomputehistogram2doutliers}{vtk\-P\-Compute\-Histogram2\-D\-Outliers}  
\item \hyperlink{vtkinfovis_vtkpcontingencystatistics}{vtk\-P\-Contingency\-Statistics}  
\item \hyperlink{vtkinfovis_vtkpcorrelativestatistics}{vtk\-P\-Correlative\-Statistics}  
\item \hyperlink{vtkinfovis_vtkpdescriptivestatistics}{vtk\-P\-Descriptive\-Statistics}  
\item \hyperlink{vtkinfovis_vtkperturbcoincidentvertices}{vtk\-Perturb\-Coincident\-Vertices}  
\item \hyperlink{vtkinfovis_vtkpextracthistogram2d}{vtk\-P\-Extract\-Histogram2\-D}  
\item \hyperlink{vtkinfovis_vtkpkmeansstatistics}{vtk\-P\-K\-Means\-Statistics}  
\item \hyperlink{vtkinfovis_vtkpmulticorrelativestatistics}{vtk\-P\-Multi\-Correlative\-Statistics}  
\item \hyperlink{vtkinfovis_vtkppairwiseextracthistogram2d}{vtk\-P\-Pairwise\-Extract\-Histogram2\-D}  
\item \hyperlink{vtkinfovis_vtkppcastatistics}{vtk\-P\-P\-C\-A\-Statistics}  
\item \hyperlink{vtkinfovis_vtkprunetreefilter}{vtk\-Prune\-Tree\-Filter}  
\item \hyperlink{vtkinfovis_vtkrandomgraphsource}{vtk\-Random\-Graph\-Source}  
\item \hyperlink{vtkinfovis_vtkrandomlayoutstrategy}{vtk\-Random\-Layout\-Strategy}  
\item \hyperlink{vtkinfovis_vtkremovehiddendata}{vtk\-Remove\-Hidden\-Data}  
\item \hyperlink{vtkinfovis_vtkremoveisolatedvertices}{vtk\-Remove\-Isolated\-Vertices}  
\item \hyperlink{vtkinfovis_vtkrisreader}{vtk\-R\-I\-S\-Reader}  
\item \hyperlink{vtkinfovis_vtkscurvespline}{vtk\-S\-Curve\-Spline}  
\item \hyperlink{vtkinfovis_vtksimple2dlayoutstrategy}{vtk\-Simple2\-D\-Layout\-Strategy}  
\item \hyperlink{vtkinfovis_vtksimple3dcirclesstrategy}{vtk\-Simple3\-D\-Circles\-Strategy}  
\item \hyperlink{vtkinfovis_vtksliceanddicelayoutstrategy}{vtk\-Slice\-And\-Dice\-Layout\-Strategy}  
\item \hyperlink{vtkinfovis_vtkspantreelayoutstrategy}{vtk\-Span\-Tree\-Layout\-Strategy}  
\item \hyperlink{vtkinfovis_vtksparsearraytotable}{vtk\-Sparse\-Array\-To\-Table}  
\item \hyperlink{vtkinfovis_vtksplinegraphedges}{vtk\-Spline\-Graph\-Edges}  
\item \hyperlink{vtkinfovis_vtksplitcolumncomponents}{vtk\-Split\-Column\-Components}  
\item \hyperlink{vtkinfovis_vtksqldatabasegraphsource}{vtk\-S\-Q\-L\-Database\-Graph\-Source}  
\item \hyperlink{vtkinfovis_vtksqldatabasetablesource}{vtk\-S\-Q\-L\-Database\-Table\-Source}  
\item \hyperlink{vtkinfovis_vtksqlgraphreader}{vtk\-S\-Q\-L\-Graph\-Reader}  
\item \hyperlink{vtkinfovis_vtksquarifylayoutstrategy}{vtk\-Squarify\-Layout\-Strategy}  
\item \hyperlink{vtkinfovis_vtkstackedtreelayoutstrategy}{vtk\-Stacked\-Tree\-Layout\-Strategy}  
\item \hyperlink{vtkinfovis_vtkstatisticsalgorithm}{vtk\-Statistics\-Algorithm}  
\item \hyperlink{vtkinfovis_vtkstrahlermetric}{vtk\-Strahler\-Metric}  
\item \hyperlink{vtkinfovis_vtkstreamgraph}{vtk\-Stream\-Graph}  
\item \hyperlink{vtkinfovis_vtkstringtocategory}{vtk\-String\-To\-Category}  
\item \hyperlink{vtkinfovis_vtkstringtonumeric}{vtk\-String\-To\-Numeric}  
\item \hyperlink{vtkinfovis_vtkstringtotimepoint}{vtk\-String\-To\-Time\-Point}  
\item \hyperlink{vtkinfovis_vtktabletoarray}{vtk\-Table\-To\-Array}  
\item \hyperlink{vtkinfovis_vtktabletograph}{vtk\-Table\-To\-Graph}  
\item \hyperlink{vtkinfovis_vtktabletosparsearray}{vtk\-Table\-To\-Sparse\-Array}  
\item \hyperlink{vtkinfovis_vtktabletotreefilter}{vtk\-Table\-To\-Tree\-Filter}  
\item \hyperlink{vtkinfovis_vtkthresholdtable}{vtk\-Threshold\-Table}  
\item \hyperlink{vtkinfovis_vtktimepointtostring}{vtk\-Time\-Point\-To\-String}  
\item \hyperlink{vtkinfovis_vtktransferattributes}{vtk\-Transfer\-Attributes}  
\item \hyperlink{vtkinfovis_vtktreefieldaggregator}{vtk\-Tree\-Field\-Aggregator}  
\item \hyperlink{vtkinfovis_vtktreelayoutstrategy}{vtk\-Tree\-Layout\-Strategy}  
\item \hyperlink{vtkinfovis_vtktreelevelsfilter}{vtk\-Tree\-Levels\-Filter}  
\item \hyperlink{vtkinfovis_vtktreemaplayout}{vtk\-Tree\-Map\-Layout}  
\item \hyperlink{vtkinfovis_vtktreemaplayoutstrategy}{vtk\-Tree\-Map\-Layout\-Strategy}  
\item \hyperlink{vtkinfovis_vtktreemaptopolydata}{vtk\-Tree\-Map\-To\-Poly\-Data}  
\item \hyperlink{vtkinfovis_vtktreeorbitlayoutstrategy}{vtk\-Tree\-Orbit\-Layout\-Strategy}  
\item \hyperlink{vtkinfovis_vtktreeringtopolydata}{vtk\-Tree\-Ring\-To\-Poly\-Data}  
\item \hyperlink{vtkinfovis_vtktulipreader}{vtk\-Tulip\-Reader}  
\item \hyperlink{vtkinfovis_vtkunivariatestatisticsalgorithm}{vtk\-Univariate\-Statistics\-Algorithm}  
\item \hyperlink{vtkinfovis_vtkvertexdegree}{vtk\-Vertex\-Degree}  
\item \hyperlink{vtkinfovis_vtkxmltreereader}{vtk\-X\-M\-L\-Tree\-Reader}  
\end{DoxyItemize}\hypertarget{vtkinfovis_vtkaddmembershiparray}{}\section{vtk\-Add\-Membership\-Array}\label{vtkinfovis_vtkaddmembershiparray}
Section\-: \hyperlink{sec_vtkinfovis}{Visualization Toolkit Infovis Classes} \hypertarget{vtkwidgets_vtkxyplotwidget_Usage}{}\subsection{Usage}\label{vtkwidgets_vtkxyplotwidget_Usage}
This filter takes an input selection, vtk\-Data\-Set\-Attribute information, and data object and adds a bit array to the output vtk\-Data\-Set\-Attributes indicating whether each index was selected or not.

To create an instance of class vtk\-Add\-Membership\-Array, simply invoke its constructor as follows \begin{DoxyVerb}  obj = vtkAddMembershipArray
\end{DoxyVerb}
 \hypertarget{vtkwidgets_vtkxyplotwidget_Methods}{}\subsection{Methods}\label{vtkwidgets_vtkxyplotwidget_Methods}
The class vtk\-Add\-Membership\-Array has several methods that can be used. They are listed below. Note that the documentation is translated automatically from the V\-T\-K sources, and may not be completely intelligible. When in doubt, consult the V\-T\-K website. In the methods listed below, {\ttfamily obj} is an instance of the vtk\-Add\-Membership\-Array class. 
\begin{DoxyItemize}
\item {\ttfamily string = obj.\-Get\-Class\-Name ()}  
\item {\ttfamily int = obj.\-Is\-A (string name)}  
\item {\ttfamily vtk\-Add\-Membership\-Array = obj.\-New\-Instance ()}  
\item {\ttfamily vtk\-Add\-Membership\-Array = obj.\-Safe\-Down\-Cast (vtk\-Object o)}  
\item {\ttfamily int = obj.\-Get\-Field\-Type ()} -\/ The field type to add the membership array to.  
\item {\ttfamily obj.\-Set\-Field\-Type (int )} -\/ The field type to add the membership array to.  
\item {\ttfamily int = obj.\-Get\-Field\-Type\-Min\-Value ()} -\/ The field type to add the membership array to.  
\item {\ttfamily int = obj.\-Get\-Field\-Type\-Max\-Value ()} -\/ The field type to add the membership array to.  
\item {\ttfamily obj.\-Set\-Output\-Array\-Name (string )} -\/ The name of the array added to the output vtk\-Data\-Set\-Attributes indicating membership. Defaults to \char`\"{}membership\char`\"{}.  
\item {\ttfamily string = obj.\-Get\-Output\-Array\-Name ()} -\/ The name of the array added to the output vtk\-Data\-Set\-Attributes indicating membership. Defaults to \char`\"{}membership\char`\"{}.  
\item {\ttfamily obj.\-Set\-Input\-Array\-Name (string )}  
\item {\ttfamily string = obj.\-Get\-Input\-Array\-Name ()}  
\item {\ttfamily obj.\-Set\-Input\-Values (vtk\-Abstract\-Array )}  
\item {\ttfamily vtk\-Abstract\-Array = obj.\-Get\-Input\-Values ()}  
\end{DoxyItemize}\hypertarget{vtkinfovis_vtkadjacencymatrixtoedgetable}{}\section{vtk\-Adjacency\-Matrix\-To\-Edge\-Table}\label{vtkinfovis_vtkadjacencymatrixtoedgetable}
Section\-: \hyperlink{sec_vtkinfovis}{Visualization Toolkit Infovis Classes} \hypertarget{vtkwidgets_vtkxyplotwidget_Usage}{}\subsection{Usage}\label{vtkwidgets_vtkxyplotwidget_Usage}
Treats a dense 2-\/way array of doubles as an adacency matrix and converts it into a vtk\-Table suitable for use as an edge table with vtk\-Table\-To\-Graph.

To create an instance of class vtk\-Adjacency\-Matrix\-To\-Edge\-Table, simply invoke its constructor as follows \begin{DoxyVerb}  obj = vtkAdjacencyMatrixToEdgeTable
\end{DoxyVerb}
 \hypertarget{vtkwidgets_vtkxyplotwidget_Methods}{}\subsection{Methods}\label{vtkwidgets_vtkxyplotwidget_Methods}
The class vtk\-Adjacency\-Matrix\-To\-Edge\-Table has several methods that can be used. They are listed below. Note that the documentation is translated automatically from the V\-T\-K sources, and may not be completely intelligible. When in doubt, consult the V\-T\-K website. In the methods listed below, {\ttfamily obj} is an instance of the vtk\-Adjacency\-Matrix\-To\-Edge\-Table class. 
\begin{DoxyItemize}
\item {\ttfamily string = obj.\-Get\-Class\-Name ()}  
\item {\ttfamily int = obj.\-Is\-A (string name)}  
\item {\ttfamily vtk\-Adjacency\-Matrix\-To\-Edge\-Table = obj.\-New\-Instance ()}  
\item {\ttfamily vtk\-Adjacency\-Matrix\-To\-Edge\-Table = obj.\-Safe\-Down\-Cast (vtk\-Object o)}  
\item {\ttfamily vtk\-Id\-Type = obj.\-Get\-Source\-Dimension ()} -\/ Specifies whether rows or columns become the \char`\"{}source\char`\"{} in the output edge table. 0 = rows, 1 = columns. Default\-: 0  
\item {\ttfamily obj.\-Set\-Source\-Dimension (vtk\-Id\-Type )} -\/ Specifies whether rows or columns become the \char`\"{}source\char`\"{} in the output edge table. 0 = rows, 1 = columns. Default\-: 0  
\item {\ttfamily string = obj.\-Get\-Value\-Array\-Name ()} -\/ Controls the name of the output table column that contains edge weights. Default\-: \char`\"{}value\char`\"{}  
\item {\ttfamily obj.\-Set\-Value\-Array\-Name (string )} -\/ Controls the name of the output table column that contains edge weights. Default\-: \char`\"{}value\char`\"{}  
\item {\ttfamily vtk\-Id\-Type = obj.\-Get\-Minimum\-Count ()} -\/ Specifies the minimum number of adjacent edges to include for each source vertex. Default\-: 0  
\item {\ttfamily obj.\-Set\-Minimum\-Count (vtk\-Id\-Type )} -\/ Specifies the minimum number of adjacent edges to include for each source vertex. Default\-: 0  
\item {\ttfamily double = obj.\-Get\-Minimum\-Threshold ()} -\/ Specifies a minimum threshold that an edge weight must exceed to be included in the output. Default\-: 0.\-5  
\item {\ttfamily obj.\-Set\-Minimum\-Threshold (double )} -\/ Specifies a minimum threshold that an edge weight must exceed to be included in the output. Default\-: 0.\-5  
\end{DoxyItemize}\hypertarget{vtkinfovis_vtkappendpoints}{}\section{vtk\-Append\-Points}\label{vtkinfovis_vtkappendpoints}
Section\-: \hyperlink{sec_vtkinfovis}{Visualization Toolkit Infovis Classes} \hypertarget{vtkwidgets_vtkxyplotwidget_Usage}{}\subsection{Usage}\label{vtkwidgets_vtkxyplotwidget_Usage}
vtk\-Append\-Points is a filter that appends the points and assoicated data of one or more polygonal (vtk\-Poly\-Data) datasets. This filter can optionally add a new array marking the input index that the point came from.

To create an instance of class vtk\-Append\-Points, simply invoke its constructor as follows \begin{DoxyVerb}  obj = vtkAppendPoints
\end{DoxyVerb}
 \hypertarget{vtkwidgets_vtkxyplotwidget_Methods}{}\subsection{Methods}\label{vtkwidgets_vtkxyplotwidget_Methods}
The class vtk\-Append\-Points has several methods that can be used. They are listed below. Note that the documentation is translated automatically from the V\-T\-K sources, and may not be completely intelligible. When in doubt, consult the V\-T\-K website. In the methods listed below, {\ttfamily obj} is an instance of the vtk\-Append\-Points class. 
\begin{DoxyItemize}
\item {\ttfamily string = obj.\-Get\-Class\-Name ()}  
\item {\ttfamily int = obj.\-Is\-A (string name)}  
\item {\ttfamily vtk\-Append\-Points = obj.\-New\-Instance ()}  
\item {\ttfamily vtk\-Append\-Points = obj.\-Safe\-Down\-Cast (vtk\-Object o)}  
\item {\ttfamily obj.\-Set\-Input\-Id\-Array\-Name (string )} -\/ Sets the output array name to fill with the input connection index for each point. This provides a way to trace a point back to a particular input. If this is N\-U\-L\-L (the default), the array is not generated.  
\item {\ttfamily string = obj.\-Get\-Input\-Id\-Array\-Name ()} -\/ Sets the output array name to fill with the input connection index for each point. This provides a way to trace a point back to a particular input. If this is N\-U\-L\-L (the default), the array is not generated.  
\end{DoxyItemize}\hypertarget{vtkinfovis_vtkapplycolors}{}\section{vtk\-Apply\-Colors}\label{vtkinfovis_vtkapplycolors}
Section\-: \hyperlink{sec_vtkinfovis}{Visualization Toolkit Infovis Classes} \hypertarget{vtkwidgets_vtkxyplotwidget_Usage}{}\subsection{Usage}\label{vtkwidgets_vtkxyplotwidget_Usage}
vtk\-Apply\-Colors performs a coloring of the dataset using default colors, lookup tables, annotations, and/or a selection. The output is a four-\/component vtk\-Unsigned\-Char\-Array containing R\-G\-B\-A tuples for each element in the dataset. The first input is the dataset to be colored, which may be a vtk\-Table, vtk\-Graph subclass, or vtk\-Data\-Set subclass. The A\-P\-I of this algorithm refers to \char`\"{}points\char`\"{} and \char`\"{}cells\char`\"{}. For vtk\-Graph, the \char`\"{}points\char`\"{} refer to the graph vertices and \char`\"{}cells\char`\"{} refer to graph edges. For vtk\-Table, \char`\"{}points\char`\"{} refer to table rows. For vtk\-Data\-Set subclasses, the meaning is obvious.

The second (optional) input is a vtk\-Annotation\-Layers object, which stores a list of annotation layers, with each layer holding a list of vtk\-Annotation objects. The annotation specifies a subset of data along with other properties, including color. For annotations with color properties, this algorithm will use the color to color elements, using a \char`\"{}top one wins\char`\"{} strategy.

The third (optional) input is a vtk\-Selection object, meant for specifying the current selection. You can control the color of the selection.

The algorithm takes two input arrays, specified with Set\-Input\-Array\-To\-Process(0, 0, 0, vtk\-Data\-Object\-::\-F\-I\-E\-L\-D\-\_\-\-A\-S\-S\-O\-C\-I\-A\-T\-I\-O\-N\-\_\-\-P\-O\-I\-N\-T\-S, name) and Set\-Input\-Array\-To\-Process(1, 0, 0, vtk\-Data\-Object\-::\-F\-I\-E\-L\-D\-\_\-\-A\-S\-S\-O\-C\-I\-A\-T\-I\-O\-N\-\_\-\-C\-E\-L\-L\-S, name). These set the point and cell data arrays to use to color the data with the associated lookup table. For vtk\-Graph, vtk\-Table inputs, you would use F\-I\-E\-L\-D\-\_\-\-A\-S\-S\-O\-C\-I\-A\-T\-I\-O\-N\-\_\-\-V\-E\-R\-T\-I\-C\-E\-S, F\-I\-E\-L\-D\-\_\-\-A\-S\-S\-O\-C\-I\-A\-T\-I\-O\-N\-\_\-\-E\-D\-G\-E\-S, or F\-I\-E\-L\-D\-\_\-\-A\-S\-S\-O\-C\-I\-A\-T\-I\-O\-N\-\_\-\-R\-O\-W\-S as appropriate.

To use the color array generated here, you should do the following\-:

mapper-\/$>$Set\-Scalar\-Mode\-To\-Use\-Cell\-Field\-Data(); mapper-\/$>$Select\-Color\-Array(\char`\"{}vtk\-Apply\-Colors color\char`\"{}); mapper-\/$>$Set\-Scalar\-Visibility(true);

Colors are assigned with the following priorities\-: 
\begin{DoxyEnumerate}
\item If an item is part of the selection, it is colored with that color. 
\item Otherwise, if the item is part of an annotation, it is colored with the color of the final (top) annotation in the set of layers. 
\item Otherwise, if the lookup table is used, it is colored using the lookup table color for the data value of the element. 
\item Otherwise it will be colored with the default color. 
\end{DoxyEnumerate}

Note\-: The opacity of an unselected item is defined by the multiplication of default opacity, lookup table opacity, and annotation opacity, where opacity is taken as a number from 0 to 1. So items will never be more opaque than any of these three opacities. Selected items are always given the selection opacity directly.

To create an instance of class vtk\-Apply\-Colors, simply invoke its constructor as follows \begin{DoxyVerb}  obj = vtkApplyColors
\end{DoxyVerb}
 \hypertarget{vtkwidgets_vtkxyplotwidget_Methods}{}\subsection{Methods}\label{vtkwidgets_vtkxyplotwidget_Methods}
The class vtk\-Apply\-Colors has several methods that can be used. They are listed below. Note that the documentation is translated automatically from the V\-T\-K sources, and may not be completely intelligible. When in doubt, consult the V\-T\-K website. In the methods listed below, {\ttfamily obj} is an instance of the vtk\-Apply\-Colors class. 
\begin{DoxyItemize}
\item {\ttfamily string = obj.\-Get\-Class\-Name ()}  
\item {\ttfamily int = obj.\-Is\-A (string name)}  
\item {\ttfamily vtk\-Apply\-Colors = obj.\-New\-Instance ()}  
\item {\ttfamily vtk\-Apply\-Colors = obj.\-Safe\-Down\-Cast (vtk\-Object o)}  
\item {\ttfamily obj.\-Set\-Point\-Lookup\-Table (vtk\-Scalars\-To\-Colors lut)} -\/ The lookup table to use for point colors. This is only used if input array 0 is set and Use\-Point\-Lookup\-Table is on.  
\item {\ttfamily vtk\-Scalars\-To\-Colors = obj.\-Get\-Point\-Lookup\-Table ()} -\/ The lookup table to use for point colors. This is only used if input array 0 is set and Use\-Point\-Lookup\-Table is on.  
\item {\ttfamily obj.\-Set\-Use\-Point\-Lookup\-Table (bool )} -\/ If on, uses the point lookup table to set the colors of unannotated, unselected elements of the data.  
\item {\ttfamily bool = obj.\-Get\-Use\-Point\-Lookup\-Table ()} -\/ If on, uses the point lookup table to set the colors of unannotated, unselected elements of the data.  
\item {\ttfamily obj.\-Use\-Point\-Lookup\-Table\-On ()} -\/ If on, uses the point lookup table to set the colors of unannotated, unselected elements of the data.  
\item {\ttfamily obj.\-Use\-Point\-Lookup\-Table\-Off ()} -\/ If on, uses the point lookup table to set the colors of unannotated, unselected elements of the data.  
\item {\ttfamily obj.\-Set\-Scale\-Point\-Lookup\-Table (bool )} -\/ If on, uses the range of the data to scale the lookup table range. Otherwise, uses the range defined in the lookup table.  
\item {\ttfamily bool = obj.\-Get\-Scale\-Point\-Lookup\-Table ()} -\/ If on, uses the range of the data to scale the lookup table range. Otherwise, uses the range defined in the lookup table.  
\item {\ttfamily obj.\-Scale\-Point\-Lookup\-Table\-On ()} -\/ If on, uses the range of the data to scale the lookup table range. Otherwise, uses the range defined in the lookup table.  
\item {\ttfamily obj.\-Scale\-Point\-Lookup\-Table\-Off ()} -\/ If on, uses the range of the data to scale the lookup table range. Otherwise, uses the range defined in the lookup table.  
\item {\ttfamily obj.\-Set\-Default\-Point\-Color (double , double , double )} -\/ The default point color for all unannotated, unselected elements of the data. This is used if Use\-Point\-Lookup\-Table is off.  
\item {\ttfamily obj.\-Set\-Default\-Point\-Color (double a\mbox{[}3\mbox{]})} -\/ The default point color for all unannotated, unselected elements of the data. This is used if Use\-Point\-Lookup\-Table is off.  
\item {\ttfamily double = obj. Get\-Default\-Point\-Color ()} -\/ The default point color for all unannotated, unselected elements of the data. This is used if Use\-Point\-Lookup\-Table is off.  
\item {\ttfamily obj.\-Set\-Default\-Point\-Opacity (double )} -\/ The default point opacity for all unannotated, unselected elements of the data. This is used if Use\-Point\-Lookup\-Table is off.  
\item {\ttfamily double = obj.\-Get\-Default\-Point\-Opacity ()} -\/ The default point opacity for all unannotated, unselected elements of the data. This is used if Use\-Point\-Lookup\-Table is off.  
\item {\ttfamily obj.\-Set\-Selected\-Point\-Color (double , double , double )} -\/ The point color for all selected elements of the data. This is used if the selection input is available.  
\item {\ttfamily obj.\-Set\-Selected\-Point\-Color (double a\mbox{[}3\mbox{]})} -\/ The point color for all selected elements of the data. This is used if the selection input is available.  
\item {\ttfamily double = obj. Get\-Selected\-Point\-Color ()} -\/ The point color for all selected elements of the data. This is used if the selection input is available.  
\item {\ttfamily obj.\-Set\-Selected\-Point\-Opacity (double )} -\/ The point opacity for all selected elements of the data. This is used if the selection input is available.  
\item {\ttfamily double = obj.\-Get\-Selected\-Point\-Opacity ()} -\/ The point opacity for all selected elements of the data. This is used if the selection input is available.  
\item {\ttfamily obj.\-Set\-Point\-Color\-Output\-Array\-Name (string )} -\/ The output array name for the point color R\-G\-B\-A array. Default is \char`\"{}vtk\-Apply\-Colors color\char`\"{}.  
\item {\ttfamily string = obj.\-Get\-Point\-Color\-Output\-Array\-Name ()} -\/ The output array name for the point color R\-G\-B\-A array. Default is \char`\"{}vtk\-Apply\-Colors color\char`\"{}.  
\item {\ttfamily obj.\-Set\-Cell\-Lookup\-Table (vtk\-Scalars\-To\-Colors lut)} -\/ The lookup table to use for cell colors. This is only used if input array 1 is set and Use\-Cell\-Lookup\-Table is on.  
\item {\ttfamily vtk\-Scalars\-To\-Colors = obj.\-Get\-Cell\-Lookup\-Table ()} -\/ The lookup table to use for cell colors. This is only used if input array 1 is set and Use\-Cell\-Lookup\-Table is on.  
\item {\ttfamily obj.\-Set\-Use\-Cell\-Lookup\-Table (bool )} -\/ If on, uses the cell lookup table to set the colors of unannotated, unselected elements of the data.  
\item {\ttfamily bool = obj.\-Get\-Use\-Cell\-Lookup\-Table ()} -\/ If on, uses the cell lookup table to set the colors of unannotated, unselected elements of the data.  
\item {\ttfamily obj.\-Use\-Cell\-Lookup\-Table\-On ()} -\/ If on, uses the cell lookup table to set the colors of unannotated, unselected elements of the data.  
\item {\ttfamily obj.\-Use\-Cell\-Lookup\-Table\-Off ()} -\/ If on, uses the cell lookup table to set the colors of unannotated, unselected elements of the data.  
\item {\ttfamily obj.\-Set\-Scale\-Cell\-Lookup\-Table (bool )} -\/ If on, uses the range of the data to scale the lookup table range. Otherwise, uses the range defined in the lookup table.  
\item {\ttfamily bool = obj.\-Get\-Scale\-Cell\-Lookup\-Table ()} -\/ If on, uses the range of the data to scale the lookup table range. Otherwise, uses the range defined in the lookup table.  
\item {\ttfamily obj.\-Scale\-Cell\-Lookup\-Table\-On ()} -\/ If on, uses the range of the data to scale the lookup table range. Otherwise, uses the range defined in the lookup table.  
\item {\ttfamily obj.\-Scale\-Cell\-Lookup\-Table\-Off ()} -\/ If on, uses the range of the data to scale the lookup table range. Otherwise, uses the range defined in the lookup table.  
\item {\ttfamily obj.\-Set\-Default\-Cell\-Color (double , double , double )} -\/ The default cell color for all unannotated, unselected elements of the data. This is used if Use\-Cell\-Lookup\-Table is off.  
\item {\ttfamily obj.\-Set\-Default\-Cell\-Color (double a\mbox{[}3\mbox{]})} -\/ The default cell color for all unannotated, unselected elements of the data. This is used if Use\-Cell\-Lookup\-Table is off.  
\item {\ttfamily double = obj. Get\-Default\-Cell\-Color ()} -\/ The default cell color for all unannotated, unselected elements of the data. This is used if Use\-Cell\-Lookup\-Table is off.  
\item {\ttfamily obj.\-Set\-Default\-Cell\-Opacity (double )} -\/ The default cell opacity for all unannotated, unselected elements of the data. This is used if Use\-Cell\-Lookup\-Table is off.  
\item {\ttfamily double = obj.\-Get\-Default\-Cell\-Opacity ()} -\/ The default cell opacity for all unannotated, unselected elements of the data. This is used if Use\-Cell\-Lookup\-Table is off.  
\item {\ttfamily obj.\-Set\-Selected\-Cell\-Color (double , double , double )} -\/ The cell color for all selected elements of the data. This is used if the selection input is available.  
\item {\ttfamily obj.\-Set\-Selected\-Cell\-Color (double a\mbox{[}3\mbox{]})} -\/ The cell color for all selected elements of the data. This is used if the selection input is available.  
\item {\ttfamily double = obj. Get\-Selected\-Cell\-Color ()} -\/ The cell color for all selected elements of the data. This is used if the selection input is available.  
\item {\ttfamily obj.\-Set\-Selected\-Cell\-Opacity (double )} -\/ The cell opacity for all selected elements of the data. This is used if the selection input is available.  
\item {\ttfamily double = obj.\-Get\-Selected\-Cell\-Opacity ()} -\/ The cell opacity for all selected elements of the data. This is used if the selection input is available.  
\item {\ttfamily obj.\-Set\-Cell\-Color\-Output\-Array\-Name (string )} -\/ The output array name for the cell color R\-G\-B\-A array. Default is \char`\"{}vtk\-Apply\-Colors color\char`\"{}.  
\item {\ttfamily string = obj.\-Get\-Cell\-Color\-Output\-Array\-Name ()} -\/ The output array name for the cell color R\-G\-B\-A array. Default is \char`\"{}vtk\-Apply\-Colors color\char`\"{}.  
\item {\ttfamily obj.\-Set\-Use\-Current\-Annotation\-Color (bool )} -\/ Use the annotation to color the current annotation (i.\-e. the current selection). Otherwise use the selection color attributes of this filter.  
\item {\ttfamily bool = obj.\-Get\-Use\-Current\-Annotation\-Color ()} -\/ Use the annotation to color the current annotation (i.\-e. the current selection). Otherwise use the selection color attributes of this filter.  
\item {\ttfamily obj.\-Use\-Current\-Annotation\-Color\-On ()} -\/ Use the annotation to color the current annotation (i.\-e. the current selection). Otherwise use the selection color attributes of this filter.  
\item {\ttfamily obj.\-Use\-Current\-Annotation\-Color\-Off ()} -\/ Use the annotation to color the current annotation (i.\-e. the current selection). Otherwise use the selection color attributes of this filter.  
\end{DoxyItemize}\hypertarget{vtkinfovis_vtkapplyicons}{}\section{vtk\-Apply\-Icons}\label{vtkinfovis_vtkapplyicons}
Section\-: \hyperlink{sec_vtkinfovis}{Visualization Toolkit Infovis Classes} \hypertarget{vtkwidgets_vtkxyplotwidget_Usage}{}\subsection{Usage}\label{vtkwidgets_vtkxyplotwidget_Usage}
vtk\-Apply\-Icons performs a iconing of the dataset using default icons, lookup tables, annotations, and/or a selection. The output is a vtk\-Int\-Array containing the icon index for each element in the dataset. The first input is the dataset to be iconed, which may be a vtk\-Table, vtk\-Graph subclass, or vtk\-Data\-Set subclass.

The second (optional) input is a vtk\-Annotation\-Layers object, which stores a list of annotation layers, with each layer holding a list of vtk\-Annotation objects. The annotation specifies a subset of data along with other properties, including icon. For annotations with icon properties, this algorithm will use the icon index of annotated elements, using a \char`\"{}top one wins\char`\"{} strategy.

The third (optional) input is a vtk\-Selection object, meant for specifying the current selection. You can control the icon of the selection, or whether there is a set of selected icons at a particular offset in the icon sheet.

The algorithm takes an input array, specified with Set\-Input\-Array\-To\-Process(0, 0, 0, vtk\-Data\-Object\-::\-F\-I\-E\-L\-D\-\_\-\-A\-S\-S\-O\-C\-I\-A\-T\-I\-O\-N\-\_\-\-P\-O\-I\-N\-T\-S, name) This sets data arrays to use to icon the data with the associated lookup table. For vtk\-Graph and vtk\-Table inputs, you would use F\-I\-E\-L\-D\-\_\-\-A\-S\-S\-O\-C\-I\-A\-T\-I\-O\-N\-\_\-\-V\-E\-R\-T\-I\-C\-E\-S, F\-I\-E\-L\-D\-\_\-\-A\-S\-S\-O\-C\-I\-A\-T\-I\-O\-N\-\_\-\-E\-D\-G\-E\-S, or F\-I\-E\-L\-D\-\_\-\-A\-S\-S\-O\-C\-I\-A\-T\-I\-O\-N\-\_\-\-R\-O\-W\-S as appropriate. The icon array will be added to the same set of attributes that the input array came from. If there is no input array, the icon array will be applied to the attributes associated with the Attribute\-Type parameter.

Icons are assigned with the following priorities\-: 
\begin{DoxyEnumerate}
\item If an item is part of the selection, it is glyphed with that icon. 
\item Otherwise, if the item is part of an annotation, it is glyphed with the icon of the final (top) annotation in the set of layers. 
\item Otherwise, if a lookup table is used, it is glyphed using the lookup table icon for the data value of the element. 
\item Otherwise it will be glyphed with the default icon. 
\end{DoxyEnumerate}

To create an instance of class vtk\-Apply\-Icons, simply invoke its constructor as follows \begin{DoxyVerb}  obj = vtkApplyIcons
\end{DoxyVerb}
 \hypertarget{vtkwidgets_vtkxyplotwidget_Methods}{}\subsection{Methods}\label{vtkwidgets_vtkxyplotwidget_Methods}
The class vtk\-Apply\-Icons has several methods that can be used. They are listed below. Note that the documentation is translated automatically from the V\-T\-K sources, and may not be completely intelligible. When in doubt, consult the V\-T\-K website. In the methods listed below, {\ttfamily obj} is an instance of the vtk\-Apply\-Icons class. 
\begin{DoxyItemize}
\item {\ttfamily string = obj.\-Get\-Class\-Name ()}  
\item {\ttfamily int = obj.\-Is\-A (string name)}  
\item {\ttfamily vtk\-Apply\-Icons = obj.\-New\-Instance ()}  
\item {\ttfamily vtk\-Apply\-Icons = obj.\-Safe\-Down\-Cast (vtk\-Object o)}  
\item {\ttfamily obj.\-Set\-Icon\-Type (double v, int icon)} -\/ Edits the lookup table to use for point icons. This is only used if input array 0 is set and Use\-Point\-Lookup\-Table is on.  
\item {\ttfamily obj.\-Set\-Icon\-Type (string v, int icon)} -\/ Edits the lookup table to use for point icons. This is only used if input array 0 is set and Use\-Point\-Lookup\-Table is on.  
\item {\ttfamily obj.\-Clear\-All\-Icon\-Types ()} -\/ Edits the lookup table to use for point icons. This is only used if input array 0 is set and Use\-Point\-Lookup\-Table is on.  
\item {\ttfamily obj.\-Set\-Use\-Lookup\-Table (bool )} -\/ If on, uses the point lookup table to set the colors of unannotated, unselected elements of the data.  
\item {\ttfamily bool = obj.\-Get\-Use\-Lookup\-Table ()} -\/ If on, uses the point lookup table to set the colors of unannotated, unselected elements of the data.  
\item {\ttfamily obj.\-Use\-Lookup\-Table\-On ()} -\/ If on, uses the point lookup table to set the colors of unannotated, unselected elements of the data.  
\item {\ttfamily obj.\-Use\-Lookup\-Table\-Off ()} -\/ If on, uses the point lookup table to set the colors of unannotated, unselected elements of the data.  
\item {\ttfamily obj.\-Set\-Default\-Icon (int )} -\/ The default point icon for all unannotated, unselected elements of the data. This is used if Use\-Point\-Lookup\-Table is off.  
\item {\ttfamily int = obj.\-Get\-Default\-Icon ()} -\/ The default point icon for all unannotated, unselected elements of the data. This is used if Use\-Point\-Lookup\-Table is off.  
\item {\ttfamily obj.\-Set\-Selected\-Icon (int )} -\/ The point icon for all selected elements of the data. This is used if the annotation input has a current selection.  
\item {\ttfamily int = obj.\-Get\-Selected\-Icon ()} -\/ The point icon for all selected elements of the data. This is used if the annotation input has a current selection.  
\item {\ttfamily obj.\-Set\-Icon\-Output\-Array\-Name (string )} -\/ The output array name for the point icon index array. Default is \char`\"{}vtk\-Apply\-Icons icon\char`\"{}.  
\item {\ttfamily string = obj.\-Get\-Icon\-Output\-Array\-Name ()} -\/ The output array name for the point icon index array. Default is \char`\"{}vtk\-Apply\-Icons icon\char`\"{}.  
\item {\ttfamily obj.\-Set\-Selection\-Mode (int )} -\/ Changes the behavior of the icon to use for selected items. 
\begin{DoxyItemize}
\item S\-E\-L\-E\-C\-T\-E\-D\-\_\-\-I\-C\-O\-N uses Selected\-Icon as the icon for all selected elements. 
\item S\-E\-L\-E\-C\-T\-E\-D\-\_\-\-O\-F\-F\-S\-E\-T uses Selected\-Icon as an offset to add to all selected elements. 
\item A\-N\-N\-O\-T\-A\-T\-I\-O\-N\-\_\-\-I\-C\-O\-N uses the I\-C\-O\-N\-\_\-\-I\-N\-D\-E\-X() property of the current annotation. 
\item I\-G\-N\-O\-R\-E\-\_\-\-S\-E\-L\-E\-C\-T\-I\-O\-N does not change the icon based on the current selection. 
\end{DoxyItemize}The default is I\-G\-N\-O\-R\-E\-\_\-\-S\-E\-L\-E\-C\-T\-I\-O\-N.  
\item {\ttfamily int = obj.\-Get\-Selection\-Mode ()} -\/ Changes the behavior of the icon to use for selected items. 
\begin{DoxyItemize}
\item S\-E\-L\-E\-C\-T\-E\-D\-\_\-\-I\-C\-O\-N uses Selected\-Icon as the icon for all selected elements. 
\item S\-E\-L\-E\-C\-T\-E\-D\-\_\-\-O\-F\-F\-S\-E\-T uses Selected\-Icon as an offset to add to all selected elements. 
\item A\-N\-N\-O\-T\-A\-T\-I\-O\-N\-\_\-\-I\-C\-O\-N uses the I\-C\-O\-N\-\_\-\-I\-N\-D\-E\-X() property of the current annotation. 
\item I\-G\-N\-O\-R\-E\-\_\-\-S\-E\-L\-E\-C\-T\-I\-O\-N does not change the icon based on the current selection. 
\end{DoxyItemize}The default is I\-G\-N\-O\-R\-E\-\_\-\-S\-E\-L\-E\-C\-T\-I\-O\-N.  
\item {\ttfamily obj.\-Set\-Selection\-Mode\-To\-Selected\-Icon ()} -\/ Changes the behavior of the icon to use for selected items. 
\begin{DoxyItemize}
\item S\-E\-L\-E\-C\-T\-E\-D\-\_\-\-I\-C\-O\-N uses Selected\-Icon as the icon for all selected elements. 
\item S\-E\-L\-E\-C\-T\-E\-D\-\_\-\-O\-F\-F\-S\-E\-T uses Selected\-Icon as an offset to add to all selected elements. 
\item A\-N\-N\-O\-T\-A\-T\-I\-O\-N\-\_\-\-I\-C\-O\-N uses the I\-C\-O\-N\-\_\-\-I\-N\-D\-E\-X() property of the current annotation. 
\item I\-G\-N\-O\-R\-E\-\_\-\-S\-E\-L\-E\-C\-T\-I\-O\-N does not change the icon based on the current selection. 
\end{DoxyItemize}The default is I\-G\-N\-O\-R\-E\-\_\-\-S\-E\-L\-E\-C\-T\-I\-O\-N.  
\item {\ttfamily obj.\-Set\-Selection\-Mode\-To\-Selected\-Offset ()} -\/ Changes the behavior of the icon to use for selected items. 
\begin{DoxyItemize}
\item S\-E\-L\-E\-C\-T\-E\-D\-\_\-\-I\-C\-O\-N uses Selected\-Icon as the icon for all selected elements. 
\item S\-E\-L\-E\-C\-T\-E\-D\-\_\-\-O\-F\-F\-S\-E\-T uses Selected\-Icon as an offset to add to all selected elements. 
\item A\-N\-N\-O\-T\-A\-T\-I\-O\-N\-\_\-\-I\-C\-O\-N uses the I\-C\-O\-N\-\_\-\-I\-N\-D\-E\-X() property of the current annotation. 
\item I\-G\-N\-O\-R\-E\-\_\-\-S\-E\-L\-E\-C\-T\-I\-O\-N does not change the icon based on the current selection. 
\end{DoxyItemize}The default is I\-G\-N\-O\-R\-E\-\_\-\-S\-E\-L\-E\-C\-T\-I\-O\-N.  
\item {\ttfamily obj.\-Set\-Selection\-Mode\-To\-Annotation\-Icon ()} -\/ Changes the behavior of the icon to use for selected items. 
\begin{DoxyItemize}
\item S\-E\-L\-E\-C\-T\-E\-D\-\_\-\-I\-C\-O\-N uses Selected\-Icon as the icon for all selected elements. 
\item S\-E\-L\-E\-C\-T\-E\-D\-\_\-\-O\-F\-F\-S\-E\-T uses Selected\-Icon as an offset to add to all selected elements. 
\item A\-N\-N\-O\-T\-A\-T\-I\-O\-N\-\_\-\-I\-C\-O\-N uses the I\-C\-O\-N\-\_\-\-I\-N\-D\-E\-X() property of the current annotation. 
\item I\-G\-N\-O\-R\-E\-\_\-\-S\-E\-L\-E\-C\-T\-I\-O\-N does not change the icon based on the current selection. 
\end{DoxyItemize}The default is I\-G\-N\-O\-R\-E\-\_\-\-S\-E\-L\-E\-C\-T\-I\-O\-N.  
\item {\ttfamily obj.\-Set\-Selection\-Mode\-To\-Ignore\-Selection ()} -\/ The attribute type to append the icon array to, used only if the input array is not specified or does not exist. This is set to one of the Attribute\-Types enum in vtk\-Data\-Object (e.\-g. P\-O\-I\-N\-T, C\-E\-L\-L, V\-E\-R\-T\-E\-X E\-D\-G\-E, F\-I\-E\-L\-D).  
\item {\ttfamily obj.\-Set\-Attribute\-Type (int )} -\/ The attribute type to append the icon array to, used only if the input array is not specified or does not exist. This is set to one of the Attribute\-Types enum in vtk\-Data\-Object (e.\-g. P\-O\-I\-N\-T, C\-E\-L\-L, V\-E\-R\-T\-E\-X E\-D\-G\-E, F\-I\-E\-L\-D).  
\item {\ttfamily int = obj.\-Get\-Attribute\-Type ()} -\/ The attribute type to append the icon array to, used only if the input array is not specified or does not exist. This is set to one of the Attribute\-Types enum in vtk\-Data\-Object (e.\-g. P\-O\-I\-N\-T, C\-E\-L\-L, V\-E\-R\-T\-E\-X E\-D\-G\-E, F\-I\-E\-L\-D).  
\end{DoxyItemize}\hypertarget{vtkinfovis_vtkarcparalleledgestrategy}{}\section{vtk\-Arc\-Parallel\-Edge\-Strategy}\label{vtkinfovis_vtkarcparalleledgestrategy}
Section\-: \hyperlink{sec_vtkinfovis}{Visualization Toolkit Infovis Classes} \hypertarget{vtkwidgets_vtkxyplotwidget_Usage}{}\subsection{Usage}\label{vtkwidgets_vtkxyplotwidget_Usage}
Parallel edges are drawn as arcs, and self-\/loops are drawn as ovals. When only one edge connects two vertices it is drawn as a straight line.

To create an instance of class vtk\-Arc\-Parallel\-Edge\-Strategy, simply invoke its constructor as follows \begin{DoxyVerb}  obj = vtkArcParallelEdgeStrategy
\end{DoxyVerb}
 \hypertarget{vtkwidgets_vtkxyplotwidget_Methods}{}\subsection{Methods}\label{vtkwidgets_vtkxyplotwidget_Methods}
The class vtk\-Arc\-Parallel\-Edge\-Strategy has several methods that can be used. They are listed below. Note that the documentation is translated automatically from the V\-T\-K sources, and may not be completely intelligible. When in doubt, consult the V\-T\-K website. In the methods listed below, {\ttfamily obj} is an instance of the vtk\-Arc\-Parallel\-Edge\-Strategy class. 
\begin{DoxyItemize}
\item {\ttfamily string = obj.\-Get\-Class\-Name ()}  
\item {\ttfamily int = obj.\-Is\-A (string name)}  
\item {\ttfamily vtk\-Arc\-Parallel\-Edge\-Strategy = obj.\-New\-Instance ()}  
\item {\ttfamily vtk\-Arc\-Parallel\-Edge\-Strategy = obj.\-Safe\-Down\-Cast (vtk\-Object o)}  
\item {\ttfamily obj.\-Layout ()} -\/ This is the layout method where the graph that was set in Set\-Graph() is laid out.  
\item {\ttfamily int = obj.\-Get\-Number\-Of\-Subdivisions ()} -\/ Get/\-Set the number of subdivisions on each edge.  
\item {\ttfamily obj.\-Set\-Number\-Of\-Subdivisions (int )} -\/ Get/\-Set the number of subdivisions on each edge.  
\end{DoxyItemize}\hypertarget{vtkinfovis_vtkarealayout}{}\section{vtk\-Area\-Layout}\label{vtkinfovis_vtkarealayout}
Section\-: \hyperlink{sec_vtkinfovis}{Visualization Toolkit Infovis Classes} \hypertarget{vtkwidgets_vtkxyplotwidget_Usage}{}\subsection{Usage}\label{vtkwidgets_vtkxyplotwidget_Usage}
vtk\-Area\-Layout assigns sector regions to each vertex in the tree, creating a tree ring. The data is added as a data array with four components per tuple representing the location and size of the sector using the format (Start\-Angle, End\-Angle, inner\-Radius, outer\-Radius).

This algorithm relies on a helper class to perform the actual layout. This helper class is a subclass of vtk\-Area\-Layout\-Strategy.

.S\-E\-C\-T\-I\-O\-N Thanks Thanks to Jason Shepherd from Sandia National Laboratories for help developing this class.

To create an instance of class vtk\-Area\-Layout, simply invoke its constructor as follows \begin{DoxyVerb}  obj = vtkAreaLayout
\end{DoxyVerb}
 \hypertarget{vtkwidgets_vtkxyplotwidget_Methods}{}\subsection{Methods}\label{vtkwidgets_vtkxyplotwidget_Methods}
The class vtk\-Area\-Layout has several methods that can be used. They are listed below. Note that the documentation is translated automatically from the V\-T\-K sources, and may not be completely intelligible. When in doubt, consult the V\-T\-K website. In the methods listed below, {\ttfamily obj} is an instance of the vtk\-Area\-Layout class. 
\begin{DoxyItemize}
\item {\ttfamily string = obj.\-Get\-Class\-Name ()}  
\item {\ttfamily int = obj.\-Is\-A (string name)}  
\item {\ttfamily vtk\-Area\-Layout = obj.\-New\-Instance ()}  
\item {\ttfamily vtk\-Area\-Layout = obj.\-Safe\-Down\-Cast (vtk\-Object o)}  
\item {\ttfamily obj.\-Set\-Size\-Array\-Name (string name)} -\/ The name for the array created for the area for each vertex. The rectangles are stored in a quadruple float array (start\-Angle, end\-Angle, inner\-Radius, outer\-Radius). For rectangular layouts, this is (minx, maxx, miny, maxy).  
\item {\ttfamily string = obj.\-Get\-Area\-Array\-Name ()} -\/ The name for the array created for the area for each vertex. The rectangles are stored in a quadruple float array (start\-Angle, end\-Angle, inner\-Radius, outer\-Radius). For rectangular layouts, this is (minx, maxx, miny, maxy).  
\item {\ttfamily obj.\-Set\-Area\-Array\-Name (string )} -\/ The name for the array created for the area for each vertex. The rectangles are stored in a quadruple float array (start\-Angle, end\-Angle, inner\-Radius, outer\-Radius). For rectangular layouts, this is (minx, maxx, miny, maxy).  
\item {\ttfamily bool = obj.\-Get\-Edge\-Routing\-Points ()} -\/ Whether to output a second output tree with vertex locations appropriate for routing bundled edges. Default is on.  
\item {\ttfamily obj.\-Set\-Edge\-Routing\-Points (bool )} -\/ Whether to output a second output tree with vertex locations appropriate for routing bundled edges. Default is on.  
\item {\ttfamily obj.\-Edge\-Routing\-Points\-On ()} -\/ Whether to output a second output tree with vertex locations appropriate for routing bundled edges. Default is on.  
\item {\ttfamily obj.\-Edge\-Routing\-Points\-Off ()} -\/ Whether to output a second output tree with vertex locations appropriate for routing bundled edges. Default is on.  
\item {\ttfamily vtk\-Area\-Layout\-Strategy = obj.\-Get\-Layout\-Strategy ()} -\/ The strategy to use when laying out the tree map.  
\item {\ttfamily obj.\-Set\-Layout\-Strategy (vtk\-Area\-Layout\-Strategy strategy)} -\/ The strategy to use when laying out the tree map.  
\item {\ttfamily long = obj.\-Get\-M\-Time ()} -\/ Get the modification time of the layout algorithm.  
\item {\ttfamily vtk\-Id\-Type = obj.\-Find\-Vertex (float pnt\mbox{[}2\mbox{]})} -\/ Get the vertex whose area contains the point, or return -\/1 if no vertex area covers the point.  
\item {\ttfamily obj.\-Get\-Bounding\-Area (vtk\-Id\-Type id, float sinfo)} -\/ The bounding area information for a certain vertex id.  
\end{DoxyItemize}\hypertarget{vtkinfovis_vtkarealayoutstrategy}{}\section{vtk\-Area\-Layout\-Strategy}\label{vtkinfovis_vtkarealayoutstrategy}
Section\-: \hyperlink{sec_vtkinfovis}{Visualization Toolkit Infovis Classes} \hypertarget{vtkwidgets_vtkxyplotwidget_Usage}{}\subsection{Usage}\label{vtkwidgets_vtkxyplotwidget_Usage}
All subclasses of this class perform a area layout on a tree. This involves assigning a region to each vertex in the tree, and placing that information in a data array with four components per tuple representing (inner\-Radius, outer\-Radius, start\-Angle, end\-Angle).

Instances of subclasses of this class may be assigned as the layout strategy to vtk\-Area\-Layout

.S\-E\-C\-T\-I\-O\-N Thanks Thanks to Jason Shepherd from Sandia National Laboratories for help developing this class.

To create an instance of class vtk\-Area\-Layout\-Strategy, simply invoke its constructor as follows \begin{DoxyVerb}  obj = vtkAreaLayoutStrategy
\end{DoxyVerb}
 \hypertarget{vtkwidgets_vtkxyplotwidget_Methods}{}\subsection{Methods}\label{vtkwidgets_vtkxyplotwidget_Methods}
The class vtk\-Area\-Layout\-Strategy has several methods that can be used. They are listed below. Note that the documentation is translated automatically from the V\-T\-K sources, and may not be completely intelligible. When in doubt, consult the V\-T\-K website. In the methods listed below, {\ttfamily obj} is an instance of the vtk\-Area\-Layout\-Strategy class. 
\begin{DoxyItemize}
\item {\ttfamily string = obj.\-Get\-Class\-Name ()}  
\item {\ttfamily int = obj.\-Is\-A (string name)}  
\item {\ttfamily vtk\-Area\-Layout\-Strategy = obj.\-New\-Instance ()}  
\item {\ttfamily vtk\-Area\-Layout\-Strategy = obj.\-Safe\-Down\-Cast (vtk\-Object o)}  
\item {\ttfamily obj.\-Layout (vtk\-Tree input\-Tree, vtk\-Data\-Array area\-Array, vtk\-Data\-Array size\-Array)} -\/ Perform the layout of the input tree, and store the sector bounds of each vertex as a tuple in a data array. For radial layout, this is (inner\-Radius, outer\-Radius, start\-Angle, end\-Angle). For rectangular layout, this is (xmin, xmax, ymin, ymax).

The size\-Array may be N\-U\-L\-L, or may contain the desired size of each vertex in the tree.  
\item {\ttfamily obj.\-Layout\-Edge\-Points (vtk\-Tree input\-Tree, vtk\-Data\-Array area\-Array, vtk\-Data\-Array size\-Array, vtk\-Tree edge\-Layout\-Tree)}  
\item {\ttfamily vtk\-Id\-Type = obj.\-Find\-Vertex (vtk\-Tree tree, vtk\-Data\-Array array, float pnt\mbox{[}2\mbox{]})} -\/ Returns the vertex id that contains pnt (or -\/1 if no one contains it)  
\item {\ttfamily obj.\-Set\-Shrink\-Percentage (double )}  
\item {\ttfamily double = obj.\-Get\-Shrink\-Percentage\-Min\-Value ()}  
\item {\ttfamily double = obj.\-Get\-Shrink\-Percentage\-Max\-Value ()}  
\item {\ttfamily double = obj.\-Get\-Shrink\-Percentage ()}  
\end{DoxyItemize}\hypertarget{vtkinfovis_vtkarraynorm}{}\section{vtk\-Array\-Norm}\label{vtkinfovis_vtkarraynorm}
Section\-: \hyperlink{sec_vtkinfovis}{Visualization Toolkit Infovis Classes} \hypertarget{vtkwidgets_vtkxyplotwidget_Usage}{}\subsection{Usage}\label{vtkwidgets_vtkxyplotwidget_Usage}
Given an input matrix (vtk\-Typed\-Array$<$double$>$), computes the L-\/norm for each vector along either dimension, storing the results in a dense output vector (1\-D vtk\-Dense\-Array$<$double$>$). The caller may optionally request the inverse norm as output (useful for subsequent normalization), and may limit the computation to a \char`\"{}window\char`\"{} of vector elements, to avoid data copying.

.S\-E\-C\-T\-I\-O\-N Thanks Developed by Timothy M. Shead (\href{mailto:tshead@sandia.gov}{\tt tshead@sandia.\-gov}) at Sandia National Laboratories.

To create an instance of class vtk\-Array\-Norm, simply invoke its constructor as follows \begin{DoxyVerb}  obj = vtkArrayNorm
\end{DoxyVerb}
 \hypertarget{vtkwidgets_vtkxyplotwidget_Methods}{}\subsection{Methods}\label{vtkwidgets_vtkxyplotwidget_Methods}
The class vtk\-Array\-Norm has several methods that can be used. They are listed below. Note that the documentation is translated automatically from the V\-T\-K sources, and may not be completely intelligible. When in doubt, consult the V\-T\-K website. In the methods listed below, {\ttfamily obj} is an instance of the vtk\-Array\-Norm class. 
\begin{DoxyItemize}
\item {\ttfamily string = obj.\-Get\-Class\-Name ()}  
\item {\ttfamily int = obj.\-Is\-A (string name)}  
\item {\ttfamily vtk\-Array\-Norm = obj.\-New\-Instance ()}  
\item {\ttfamily vtk\-Array\-Norm = obj.\-Safe\-Down\-Cast (vtk\-Object o)}  
\item {\ttfamily int = obj.\-Get\-Dimension ()} -\/ Controls the dimension along which norms will be computed. For input matrices, For input matrices, use \char`\"{}0\char`\"{} (rows) or \char`\"{}1\char`\"{} (columns). Default\-: 0  
\item {\ttfamily obj.\-Set\-Dimension (int )} -\/ Controls the dimension along which norms will be computed. For input matrices, For input matrices, use \char`\"{}0\char`\"{} (rows) or \char`\"{}1\char`\"{} (columns). Default\-: 0  
\item {\ttfamily int = obj.\-Get\-L ()} -\/ Controls the L-\/value. Default\-: 2  
\item {\ttfamily obj.\-Set\-L (int value)} -\/ Controls the L-\/value. Default\-: 2  
\item {\ttfamily obj.\-Set\-Invert (int )} -\/ Controls whether to invert output values. Default\-: false  
\item {\ttfamily int = obj.\-Get\-Invert ()} -\/ Controls whether to invert output values. Default\-: false  
\end{DoxyItemize}\hypertarget{vtkinfovis_vtkassigncoordinates}{}\section{vtk\-Assign\-Coordinates}\label{vtkinfovis_vtkassigncoordinates}
Section\-: \hyperlink{sec_vtkinfovis}{Visualization Toolkit Infovis Classes} \hypertarget{vtkwidgets_vtkxyplotwidget_Usage}{}\subsection{Usage}\label{vtkwidgets_vtkxyplotwidget_Usage}
Given two(or three) arrays take the values in those arrays and simply assign them to the coordinates of the vertices. Yes you could do this with the array calculator, but your mom wears army boots so we're not going to.

To create an instance of class vtk\-Assign\-Coordinates, simply invoke its constructor as follows \begin{DoxyVerb}  obj = vtkAssignCoordinates
\end{DoxyVerb}
 \hypertarget{vtkwidgets_vtkxyplotwidget_Methods}{}\subsection{Methods}\label{vtkwidgets_vtkxyplotwidget_Methods}
The class vtk\-Assign\-Coordinates has several methods that can be used. They are listed below. Note that the documentation is translated automatically from the V\-T\-K sources, and may not be completely intelligible. When in doubt, consult the V\-T\-K website. In the methods listed below, {\ttfamily obj} is an instance of the vtk\-Assign\-Coordinates class. 
\begin{DoxyItemize}
\item {\ttfamily string = obj.\-Get\-Class\-Name ()}  
\item {\ttfamily int = obj.\-Is\-A (string name)}  
\item {\ttfamily vtk\-Assign\-Coordinates = obj.\-New\-Instance ()}  
\item {\ttfamily vtk\-Assign\-Coordinates = obj.\-Safe\-Down\-Cast (vtk\-Object o)}  
\item {\ttfamily obj.\-Set\-X\-Coord\-Array\-Name (string )} -\/ Set the x coordinate array name.  
\item {\ttfamily string = obj.\-Get\-X\-Coord\-Array\-Name ()} -\/ Set the x coordinate array name.  
\item {\ttfamily obj.\-Set\-Y\-Coord\-Array\-Name (string )} -\/ Set the y coordinate array name.  
\item {\ttfamily string = obj.\-Get\-Y\-Coord\-Array\-Name ()} -\/ Set the y coordinate array name.  
\item {\ttfamily obj.\-Set\-Z\-Coord\-Array\-Name (string )} -\/ Set the z coordinate array name.  
\item {\ttfamily string = obj.\-Get\-Z\-Coord\-Array\-Name ()} -\/ Set the z coordinate array name.  
\item {\ttfamily obj.\-Set\-Jitter (bool )} -\/ Set if you want a random jitter  
\end{DoxyItemize}\hypertarget{vtkinfovis_vtkassigncoordinateslayoutstrategy}{}\section{vtk\-Assign\-Coordinates\-Layout\-Strategy}\label{vtkinfovis_vtkassigncoordinateslayoutstrategy}
Section\-: \hyperlink{sec_vtkinfovis}{Visualization Toolkit Infovis Classes} \hypertarget{vtkwidgets_vtkxyplotwidget_Usage}{}\subsection{Usage}\label{vtkwidgets_vtkxyplotwidget_Usage}
Uses vtk\-Assign\-Coordinates to use values from arrays as the x, y, and z coordinates.

To create an instance of class vtk\-Assign\-Coordinates\-Layout\-Strategy, simply invoke its constructor as follows \begin{DoxyVerb}  obj = vtkAssignCoordinatesLayoutStrategy
\end{DoxyVerb}
 \hypertarget{vtkwidgets_vtkxyplotwidget_Methods}{}\subsection{Methods}\label{vtkwidgets_vtkxyplotwidget_Methods}
The class vtk\-Assign\-Coordinates\-Layout\-Strategy has several methods that can be used. They are listed below. Note that the documentation is translated automatically from the V\-T\-K sources, and may not be completely intelligible. When in doubt, consult the V\-T\-K website. In the methods listed below, {\ttfamily obj} is an instance of the vtk\-Assign\-Coordinates\-Layout\-Strategy class. 
\begin{DoxyItemize}
\item {\ttfamily string = obj.\-Get\-Class\-Name ()}  
\item {\ttfamily int = obj.\-Is\-A (string name)}  
\item {\ttfamily vtk\-Assign\-Coordinates\-Layout\-Strategy = obj.\-New\-Instance ()}  
\item {\ttfamily vtk\-Assign\-Coordinates\-Layout\-Strategy = obj.\-Safe\-Down\-Cast (vtk\-Object o)}  
\item {\ttfamily obj.\-Set\-X\-Coord\-Array\-Name (string name)} -\/ The array to use for the x coordinate values.  
\item {\ttfamily string = obj.\-Get\-X\-Coord\-Array\-Name ()} -\/ The array to use for the x coordinate values.  
\item {\ttfamily obj.\-Set\-Y\-Coord\-Array\-Name (string name)} -\/ The array to use for the y coordinate values.  
\item {\ttfamily string = obj.\-Get\-Y\-Coord\-Array\-Name ()} -\/ The array to use for the y coordinate values.  
\item {\ttfamily obj.\-Set\-Z\-Coord\-Array\-Name (string name)} -\/ The array to use for the z coordinate values.  
\item {\ttfamily string = obj.\-Get\-Z\-Coord\-Array\-Name ()} -\/ The array to use for the z coordinate values.  
\item {\ttfamily obj.\-Layout ()} -\/ Perform the random layout.  
\end{DoxyItemize}\hypertarget{vtkinfovis_vtkattributeclustering2dlayoutstrategy}{}\section{vtk\-Attribute\-Clustering2\-D\-Layout\-Strategy}\label{vtkinfovis_vtkattributeclustering2dlayoutstrategy}
Section\-: \hyperlink{sec_vtkinfovis}{Visualization Toolkit Infovis Classes} \hypertarget{vtkwidgets_vtkxyplotwidget_Usage}{}\subsection{Usage}\label{vtkwidgets_vtkxyplotwidget_Usage}
This class is a density grid based force directed layout strategy. Also please note that 'fast' is relative to quite slow. \-:) The layout running time is O(V+\-E) with an extremely high constant. .S\-E\-C\-T\-I\-O\-N Thanks Thanks to Godzilla for not eating my computer so that this class could be written.

To create an instance of class vtk\-Attribute\-Clustering2\-D\-Layout\-Strategy, simply invoke its constructor as follows \begin{DoxyVerb}  obj = vtkAttributeClustering2DLayoutStrategy
\end{DoxyVerb}
 \hypertarget{vtkwidgets_vtkxyplotwidget_Methods}{}\subsection{Methods}\label{vtkwidgets_vtkxyplotwidget_Methods}
The class vtk\-Attribute\-Clustering2\-D\-Layout\-Strategy has several methods that can be used. They are listed below. Note that the documentation is translated automatically from the V\-T\-K sources, and may not be completely intelligible. When in doubt, consult the V\-T\-K website. In the methods listed below, {\ttfamily obj} is an instance of the vtk\-Attribute\-Clustering2\-D\-Layout\-Strategy class. 
\begin{DoxyItemize}
\item {\ttfamily string = obj.\-Get\-Class\-Name ()}  
\item {\ttfamily int = obj.\-Is\-A (string name)}  
\item {\ttfamily vtk\-Attribute\-Clustering2\-D\-Layout\-Strategy = obj.\-New\-Instance ()}  
\item {\ttfamily vtk\-Attribute\-Clustering2\-D\-Layout\-Strategy = obj.\-Safe\-Down\-Cast (vtk\-Object o)}  
\item {\ttfamily string = obj.\-Get\-Vertex\-Attribute ()} -\/ The name of the array on the vertices, whose values will be used for determining clusters.  
\item {\ttfamily obj.\-Set\-Vertex\-Attribute (string )} -\/ The name of the array on the vertices, whose values will be used for determining clusters.  
\item {\ttfamily obj.\-Set\-Random\-Seed (int )} -\/ Seed the random number generator used to jitter point positions. This has a significant effect on their final positions when the layout is complete.  
\item {\ttfamily int = obj.\-Get\-Random\-Seed\-Min\-Value ()} -\/ Seed the random number generator used to jitter point positions. This has a significant effect on their final positions when the layout is complete.  
\item {\ttfamily int = obj.\-Get\-Random\-Seed\-Max\-Value ()} -\/ Seed the random number generator used to jitter point positions. This has a significant effect on their final positions when the layout is complete.  
\item {\ttfamily int = obj.\-Get\-Random\-Seed ()} -\/ Seed the random number generator used to jitter point positions. This has a significant effect on their final positions when the layout is complete.  
\item {\ttfamily obj.\-Set\-Max\-Number\-Of\-Iterations (int )} -\/ Set/\-Get the maximum number of iterations to be used. The higher this number, the more iterations through the algorithm is possible, and thus, the more the graph gets modified. The default is '100' for no particular reason Note\-: The strong recommendation is that you do not change this parameter. \-:)  
\item {\ttfamily int = obj.\-Get\-Max\-Number\-Of\-Iterations\-Min\-Value ()} -\/ Set/\-Get the maximum number of iterations to be used. The higher this number, the more iterations through the algorithm is possible, and thus, the more the graph gets modified. The default is '100' for no particular reason Note\-: The strong recommendation is that you do not change this parameter. \-:)  
\item {\ttfamily int = obj.\-Get\-Max\-Number\-Of\-Iterations\-Max\-Value ()} -\/ Set/\-Get the maximum number of iterations to be used. The higher this number, the more iterations through the algorithm is possible, and thus, the more the graph gets modified. The default is '100' for no particular reason Note\-: The strong recommendation is that you do not change this parameter. \-:)  
\item {\ttfamily int = obj.\-Get\-Max\-Number\-Of\-Iterations ()} -\/ Set/\-Get the maximum number of iterations to be used. The higher this number, the more iterations through the algorithm is possible, and thus, the more the graph gets modified. The default is '100' for no particular reason Note\-: The strong recommendation is that you do not change this parameter. \-:)  
\item {\ttfamily obj.\-Set\-Iterations\-Per\-Layout (int )} -\/ Set/\-Get the number of iterations per layout. The only use for this ivar is for the application to do visualizations of the layout before it's complete. The default is '100' to match the default 'Max\-Number\-Of\-Iterations' Note\-: Changing this parameter is just fine \-:)  
\item {\ttfamily int = obj.\-Get\-Iterations\-Per\-Layout\-Min\-Value ()} -\/ Set/\-Get the number of iterations per layout. The only use for this ivar is for the application to do visualizations of the layout before it's complete. The default is '100' to match the default 'Max\-Number\-Of\-Iterations' Note\-: Changing this parameter is just fine \-:)  
\item {\ttfamily int = obj.\-Get\-Iterations\-Per\-Layout\-Max\-Value ()} -\/ Set/\-Get the number of iterations per layout. The only use for this ivar is for the application to do visualizations of the layout before it's complete. The default is '100' to match the default 'Max\-Number\-Of\-Iterations' Note\-: Changing this parameter is just fine \-:)  
\item {\ttfamily int = obj.\-Get\-Iterations\-Per\-Layout ()} -\/ Set/\-Get the number of iterations per layout. The only use for this ivar is for the application to do visualizations of the layout before it's complete. The default is '100' to match the default 'Max\-Number\-Of\-Iterations' Note\-: Changing this parameter is just fine \-:)  
\item {\ttfamily obj.\-Set\-Initial\-Temperature (float )} -\/ Set the initial temperature. The temperature default is '5' for no particular reason Note\-: The strong recommendation is that you do not change this parameter. \-:)  
\item {\ttfamily float = obj.\-Get\-Initial\-Temperature\-Min\-Value ()} -\/ Set the initial temperature. The temperature default is '5' for no particular reason Note\-: The strong recommendation is that you do not change this parameter. \-:)  
\item {\ttfamily float = obj.\-Get\-Initial\-Temperature\-Max\-Value ()} -\/ Set the initial temperature. The temperature default is '5' for no particular reason Note\-: The strong recommendation is that you do not change this parameter. \-:)  
\item {\ttfamily float = obj.\-Get\-Initial\-Temperature ()} -\/ Set the initial temperature. The temperature default is '5' for no particular reason Note\-: The strong recommendation is that you do not change this parameter. \-:)  
\item {\ttfamily obj.\-Set\-Cool\-Down\-Rate (double )} -\/ Set/\-Get the Cool-\/down rate. The higher this number is, the longer it will take to \char`\"{}cool-\/down\char`\"{}, and thus, the more the graph will be modified. The default is '10' for no particular reason. Note\-: The strong recommendation is that you do not change this parameter. \-:)  
\item {\ttfamily double = obj.\-Get\-Cool\-Down\-Rate\-Min\-Value ()} -\/ Set/\-Get the Cool-\/down rate. The higher this number is, the longer it will take to \char`\"{}cool-\/down\char`\"{}, and thus, the more the graph will be modified. The default is '10' for no particular reason. Note\-: The strong recommendation is that you do not change this parameter. \-:)  
\item {\ttfamily double = obj.\-Get\-Cool\-Down\-Rate\-Max\-Value ()} -\/ Set/\-Get the Cool-\/down rate. The higher this number is, the longer it will take to \char`\"{}cool-\/down\char`\"{}, and thus, the more the graph will be modified. The default is '10' for no particular reason. Note\-: The strong recommendation is that you do not change this parameter. \-:)  
\item {\ttfamily double = obj.\-Get\-Cool\-Down\-Rate ()} -\/ Set/\-Get the Cool-\/down rate. The higher this number is, the longer it will take to \char`\"{}cool-\/down\char`\"{}, and thus, the more the graph will be modified. The default is '10' for no particular reason. Note\-: The strong recommendation is that you do not change this parameter. \-:)  
\item {\ttfamily obj.\-Set\-Rest\-Distance (float )} -\/ Manually set the resting distance. Otherwise the distance is computed automatically.  
\item {\ttfamily float = obj.\-Get\-Rest\-Distance ()} -\/ Manually set the resting distance. Otherwise the distance is computed automatically.  
\item {\ttfamily obj.\-Initialize ()} -\/ This strategy sets up some data structures for faster processing of each Layout() call  
\item {\ttfamily obj.\-Layout ()} -\/ This is the layout method where the graph that was set in Set\-Graph() is laid out. The method can either entirely layout the graph or iteratively lay out the graph. If you have an iterative layout please implement the Is\-Layout\-Complete() method.  
\item {\ttfamily int = obj.\-Is\-Layout\-Complete ()}  
\end{DoxyItemize}\hypertarget{vtkinfovis_vtkbivariatelineartablethreshold}{}\section{vtk\-Bivariate\-Linear\-Table\-Threshold}\label{vtkinfovis_vtkbivariatelineartablethreshold}
Section\-: \hyperlink{sec_vtkinfovis}{Visualization Toolkit Infovis Classes} \hypertarget{vtkwidgets_vtkxyplotwidget_Usage}{}\subsection{Usage}\label{vtkwidgets_vtkxyplotwidget_Usage}
Class for filtering the rows of a two numeric columns of a vtk\-Table. The columns are treated as the two variables of a line. This filter will then iterate through the rows of the table determining if X,Y values pairs are above/below/between/near one or more lines.

The \char`\"{}between\char`\"{} mode checks to see if a row is contained within the convex hull of all of the specified lines. The \char`\"{}near\char`\"{} mode checks if a row is within a distance threshold two one of the specified lines. This class is used in conjunction with various plotting classes, so it is useful to rescale the X,Y axes to a particular range of values. Distance comparisons can be performed in the scaled space by setting the Custom\-Ranges ivar and enabling Use\-Normalized\-Distance.

To create an instance of class vtk\-Bivariate\-Linear\-Table\-Threshold, simply invoke its constructor as follows \begin{DoxyVerb}  obj = vtkBivariateLinearTableThreshold
\end{DoxyVerb}
 \hypertarget{vtkwidgets_vtkxyplotwidget_Methods}{}\subsection{Methods}\label{vtkwidgets_vtkxyplotwidget_Methods}
The class vtk\-Bivariate\-Linear\-Table\-Threshold has several methods that can be used. They are listed below. Note that the documentation is translated automatically from the V\-T\-K sources, and may not be completely intelligible. When in doubt, consult the V\-T\-K website. In the methods listed below, {\ttfamily obj} is an instance of the vtk\-Bivariate\-Linear\-Table\-Threshold class. 
\begin{DoxyItemize}
\item {\ttfamily string = obj.\-Get\-Class\-Name ()}  
\item {\ttfamily int = obj.\-Is\-A (string name)}  
\item {\ttfamily vtk\-Bivariate\-Linear\-Table\-Threshold = obj.\-New\-Instance ()}  
\item {\ttfamily vtk\-Bivariate\-Linear\-Table\-Threshold = obj.\-Safe\-Down\-Cast (vtk\-Object o)}  
\item {\ttfamily obj.\-Set\-Inclusive (int )} -\/ Include the line in the threshold. Essentially whether the threshold operation uses $>$ versus $>$=.  
\item {\ttfamily int = obj.\-Get\-Inclusive ()} -\/ Include the line in the threshold. Essentially whether the threshold operation uses $>$ versus $>$=.  
\item {\ttfamily obj.\-Add\-Column\-To\-Threshold (vtk\-Id\-Type column, vtk\-Id\-Type component)} -\/ Add a numeric column to the pair of columns to be thresholded. Call twice.  
\item {\ttfamily int = obj.\-Get\-Number\-Of\-Columns\-To\-Threshold ()} -\/ Return how many columns have been added. Hopefully 2.  
\item {\ttfamily obj.\-Clear\-Columns\-To\-Threshold ()} -\/ Reset the columns to be thresholded.  
\item {\ttfamily vtk\-Id\-Type\-Array = obj.\-Get\-Selected\-Row\-Ids (int selection)} -\/ Get the output as a table of row ids.  
\item {\ttfamily obj.\-Initialize ()} -\/ Reset the columns to threshold, column ranges, etc.  
\item {\ttfamily obj.\-Add\-Line\-Equation (double p1, double p2)} -\/ Add a line for thresholding from two x,y points.  
\item {\ttfamily obj.\-Add\-Line\-Equation (double p, double slope)} -\/ Add a line for thresholding in point-\/slope form.  
\item {\ttfamily obj.\-Add\-Line\-Equation (double a, double b, double c)} -\/ Add a line for thresholding in implicit form (ax + by + c = 0)  
\item {\ttfamily obj.\-Clear\-Line\-Equations ()} -\/ Reset the list of line equations.  
\item {\ttfamily int = obj.\-Get\-Linear\-Threshold\-Type ()} -\/ Set the threshold type. Above\-: find all rows that are above the specified lines. Below\-: find all rows that are below the specified lines. Near\-: find all rows that are near the specified lines. Between\-: find all rows that are between the specified lines.  
\item {\ttfamily obj.\-Set\-Linear\-Threshold\-Type (int )} -\/ Set the threshold type. Above\-: find all rows that are above the specified lines. Below\-: find all rows that are below the specified lines. Near\-: find all rows that are near the specified lines. Between\-: find all rows that are between the specified lines.  
\item {\ttfamily obj.\-Set\-Linear\-Threshold\-Type\-To\-Above ()} -\/ Set the threshold type. Above\-: find all rows that are above the specified lines. Below\-: find all rows that are below the specified lines. Near\-: find all rows that are near the specified lines. Between\-: find all rows that are between the specified lines.  
\item {\ttfamily obj.\-Set\-Linear\-Threshold\-Type\-To\-Below ()} -\/ Set the threshold type. Above\-: find all rows that are above the specified lines. Below\-: find all rows that are below the specified lines. Near\-: find all rows that are near the specified lines. Between\-: find all rows that are between the specified lines.  
\item {\ttfamily obj.\-Set\-Linear\-Threshold\-Type\-To\-Near ()} -\/ Set the threshold type. Above\-: find all rows that are above the specified lines. Below\-: find all rows that are below the specified lines. Near\-: find all rows that are near the specified lines. Between\-: find all rows that are between the specified lines.  
\item {\ttfamily obj.\-Set\-Linear\-Threshold\-Type\-To\-Between ()} -\/ Manually access the maximum/minimum x,y values. This is used in conjunction with Use\-Normalized\-Distance when determining if a row passes the threshold.  
\item {\ttfamily obj.\-Set\-Column\-Ranges (double , double )} -\/ Manually access the maximum/minimum x,y values. This is used in conjunction with Use\-Normalized\-Distance when determining if a row passes the threshold.  
\item {\ttfamily obj.\-Set\-Column\-Ranges (double a\mbox{[}2\mbox{]})} -\/ Manually access the maximum/minimum x,y values. This is used in conjunction with Use\-Normalized\-Distance when determining if a row passes the threshold.  
\item {\ttfamily double = obj. Get\-Column\-Ranges ()} -\/ Manually access the maximum/minimum x,y values. This is used in conjunction with Use\-Normalized\-Distance when determining if a row passes the threshold.  
\item {\ttfamily obj.\-Set\-Distance\-Threshold (double )} -\/ The Cartesian distance within which a point will pass the near threshold.  
\item {\ttfamily double = obj.\-Get\-Distance\-Threshold ()} -\/ The Cartesian distance within which a point will pass the near threshold.  
\item {\ttfamily obj.\-Set\-Use\-Normalized\-Distance (int )} -\/ Renormalize the space of the data such that the X and Y axes are \char`\"{}square\char`\"{} over the specified Column\-Ranges. This essentially scales the data space so that Column\-Ranges\mbox{[}1\mbox{]}-\/\-Column\-Ranges\mbox{[}0\mbox{]} = 1.\-0 and Column\-Ranges\mbox{[}3\mbox{]}-\/\-Column\-Ranges\mbox{[}2\mbox{]} = 1.\-0. Used for scatter plot distance calculations. Be sure to set Distance\-Threshold accordingly, when used.  
\item {\ttfamily int = obj.\-Get\-Use\-Normalized\-Distance ()} -\/ Renormalize the space of the data such that the X and Y axes are \char`\"{}square\char`\"{} over the specified Column\-Ranges. This essentially scales the data space so that Column\-Ranges\mbox{[}1\mbox{]}-\/\-Column\-Ranges\mbox{[}0\mbox{]} = 1.\-0 and Column\-Ranges\mbox{[}3\mbox{]}-\/\-Column\-Ranges\mbox{[}2\mbox{]} = 1.\-0. Used for scatter plot distance calculations. Be sure to set Distance\-Threshold accordingly, when used.  
\item {\ttfamily obj.\-Use\-Normalized\-Distance\-On ()} -\/ Renormalize the space of the data such that the X and Y axes are \char`\"{}square\char`\"{} over the specified Column\-Ranges. This essentially scales the data space so that Column\-Ranges\mbox{[}1\mbox{]}-\/\-Column\-Ranges\mbox{[}0\mbox{]} = 1.\-0 and Column\-Ranges\mbox{[}3\mbox{]}-\/\-Column\-Ranges\mbox{[}2\mbox{]} = 1.\-0. Used for scatter plot distance calculations. Be sure to set Distance\-Threshold accordingly, when used.  
\item {\ttfamily obj.\-Use\-Normalized\-Distance\-Off ()} -\/ Renormalize the space of the data such that the X and Y axes are \char`\"{}square\char`\"{} over the specified Column\-Ranges. This essentially scales the data space so that Column\-Ranges\mbox{[}1\mbox{]}-\/\-Column\-Ranges\mbox{[}0\mbox{]} = 1.\-0 and Column\-Ranges\mbox{[}3\mbox{]}-\/\-Column\-Ranges\mbox{[}2\mbox{]} = 1.\-0. Used for scatter plot distance calculations. Be sure to set Distance\-Threshold accordingly, when used.  
\end{DoxyItemize}\hypertarget{vtkinfovis_vtkbivariatestatisticsalgorithm}{}\section{vtk\-Bivariate\-Statistics\-Algorithm}\label{vtkinfovis_vtkbivariatestatisticsalgorithm}
Section\-: \hyperlink{sec_vtkinfovis}{Visualization Toolkit Infovis Classes} \hypertarget{vtkwidgets_vtkxyplotwidget_Usage}{}\subsection{Usage}\label{vtkwidgets_vtkxyplotwidget_Usage}
This class specializes statistics algorithms to the bivariate case, where a number of pairs of columns of interest can be selected in the input data set. This is done by the means of the following functions\-:

Reset\-Columns() -\/ reset the list of columns of interest. Add/\-Remove\-Colum( nam\-Col\-X, nam\-Col\-Y ) -\/ try to add/remove column pair ( nam\-Col\-X, nam\-Coly ) to/from the list. Set\-Column\-Status ( nam\-Col, status ) -\/ mostly for U\-I wrapping purposes, try to add/remove (depending on status) nam\-Col from a list of buffered columns, from which all possible pairs are generated. The verb \char`\"{}try\char`\"{} is used in the sense that neither attempting to repeat an existing entry nor to remove a non-\/existent entry will work.

.S\-E\-C\-T\-I\-O\-N Thanks Thanks to Philippe Pebay and David Thompson from Sandia National Laboratories for implementing this class.

To create an instance of class vtk\-Bivariate\-Statistics\-Algorithm, simply invoke its constructor as follows \begin{DoxyVerb}  obj = vtkBivariateStatisticsAlgorithm
\end{DoxyVerb}
 \hypertarget{vtkwidgets_vtkxyplotwidget_Methods}{}\subsection{Methods}\label{vtkwidgets_vtkxyplotwidget_Methods}
The class vtk\-Bivariate\-Statistics\-Algorithm has several methods that can be used. They are listed below. Note that the documentation is translated automatically from the V\-T\-K sources, and may not be completely intelligible. When in doubt, consult the V\-T\-K website. In the methods listed below, {\ttfamily obj} is an instance of the vtk\-Bivariate\-Statistics\-Algorithm class. 
\begin{DoxyItemize}
\item {\ttfamily string = obj.\-Get\-Class\-Name ()}  
\item {\ttfamily int = obj.\-Is\-A (string name)}  
\item {\ttfamily vtk\-Bivariate\-Statistics\-Algorithm = obj.\-New\-Instance ()}  
\item {\ttfamily vtk\-Bivariate\-Statistics\-Algorithm = obj.\-Safe\-Down\-Cast (vtk\-Object o)}  
\item {\ttfamily obj.\-Add\-Column\-Pair (string nam\-Col\-X, string nam\-Col\-Y)} -\/ Convenience method to create a request with a single column name pair ({\ttfamily nam\-Col\-X}, {\ttfamily nam\-Col\-Y}) in a single call; this is the preferred method to select columns pairs, ensuring selection consistency (a pair of columns per request).

Unlike Set\-Column\-Status(), you need not call Request\-Selected\-Columns() after Add\-Column\-Pair().

Warning\-: {\ttfamily nam\-Col\-X} and {\ttfamily nam\-Col\-Y} are only checked for their validity as strings; no check is made that either are valid column names.  
\item {\ttfamily int = obj.\-Request\-Selected\-Columns ()} -\/ Use the current column status values to produce a new request for statistics to be produced when Request\-Data() is called. Unlike the superclass implementation, this version adds a new request for every possible pairing of the selected columns instead of a single request containing all the columns.  
\end{DoxyItemize}\hypertarget{vtkinfovis_vtkboxlayoutstrategy}{}\section{vtk\-Box\-Layout\-Strategy}\label{vtkinfovis_vtkboxlayoutstrategy}
Section\-: \hyperlink{sec_vtkinfovis}{Visualization Toolkit Infovis Classes} \hypertarget{vtkwidgets_vtkxyplotwidget_Usage}{}\subsection{Usage}\label{vtkwidgets_vtkxyplotwidget_Usage}
vtk\-Box\-Layout\-Strategy recursively partitions the space for children vertices in a tree-\/map into square regions (or regions very close to a square).

.S\-E\-C\-T\-I\-O\-N Thanks Thanks to Brian Wylie from Sandia National Laboratories for creating this class.

To create an instance of class vtk\-Box\-Layout\-Strategy, simply invoke its constructor as follows \begin{DoxyVerb}  obj = vtkBoxLayoutStrategy
\end{DoxyVerb}
 \hypertarget{vtkwidgets_vtkxyplotwidget_Methods}{}\subsection{Methods}\label{vtkwidgets_vtkxyplotwidget_Methods}
The class vtk\-Box\-Layout\-Strategy has several methods that can be used. They are listed below. Note that the documentation is translated automatically from the V\-T\-K sources, and may not be completely intelligible. When in doubt, consult the V\-T\-K website. In the methods listed below, {\ttfamily obj} is an instance of the vtk\-Box\-Layout\-Strategy class. 
\begin{DoxyItemize}
\item {\ttfamily string = obj.\-Get\-Class\-Name ()}  
\item {\ttfamily int = obj.\-Is\-A (string name)}  
\item {\ttfamily vtk\-Box\-Layout\-Strategy = obj.\-New\-Instance ()}  
\item {\ttfamily vtk\-Box\-Layout\-Strategy = obj.\-Safe\-Down\-Cast (vtk\-Object o)}  
\item {\ttfamily obj.\-Layout (vtk\-Tree input\-Tree, vtk\-Data\-Array coords\-Array, vtk\-Data\-Array size\-Array)} -\/ Perform the layout of a tree and place the results as 4-\/tuples in coords\-Array (Xmin, Xmax, Ymin, Ymax).  
\end{DoxyItemize}\hypertarget{vtkinfovis_vtkchacographreader}{}\section{vtk\-Chaco\-Graph\-Reader}\label{vtkinfovis_vtkchacographreader}
Section\-: \hyperlink{sec_vtkinfovis}{Visualization Toolkit Infovis Classes} \hypertarget{vtkwidgets_vtkxyplotwidget_Usage}{}\subsection{Usage}\label{vtkwidgets_vtkxyplotwidget_Usage}
vtk\-Chaco\-Graph\-Reader reads in files in the Chaco format into a vtk\-Graph. An example is the following {\ttfamily  10 13 2 6 10 1 3 2 4 8 3 5 4 6 10 1 5 7 6 8 3 7 9 8 10 1 5 9 } The first line specifies the number of vertices and edges in the graph. Each additional line contains the vertices adjacent to a particular vertex. In this example, vertex 1 is adjacent to 2, 6 and 10, vertex 2 is adjacent to 1 and 3, etc. Since Chaco ids start at 1 and V\-T\-K ids start at 0, the vertex ids in the vtk\-Graph will be 1 less than the Chaco ids.

To create an instance of class vtk\-Chaco\-Graph\-Reader, simply invoke its constructor as follows \begin{DoxyVerb}  obj = vtkChacoGraphReader
\end{DoxyVerb}
 \hypertarget{vtkwidgets_vtkxyplotwidget_Methods}{}\subsection{Methods}\label{vtkwidgets_vtkxyplotwidget_Methods}
The class vtk\-Chaco\-Graph\-Reader has several methods that can be used. They are listed below. Note that the documentation is translated automatically from the V\-T\-K sources, and may not be completely intelligible. When in doubt, consult the V\-T\-K website. In the methods listed below, {\ttfamily obj} is an instance of the vtk\-Chaco\-Graph\-Reader class. 
\begin{DoxyItemize}
\item {\ttfamily string = obj.\-Get\-Class\-Name ()}  
\item {\ttfamily int = obj.\-Is\-A (string name)}  
\item {\ttfamily vtk\-Chaco\-Graph\-Reader = obj.\-New\-Instance ()}  
\item {\ttfamily vtk\-Chaco\-Graph\-Reader = obj.\-Safe\-Down\-Cast (vtk\-Object o)}  
\item {\ttfamily string = obj.\-Get\-File\-Name ()} -\/ The Chaco file name.  
\item {\ttfamily obj.\-Set\-File\-Name (string )} -\/ The Chaco file name.  
\end{DoxyItemize}\hypertarget{vtkinfovis_vtkcircularlayoutstrategy}{}\section{vtk\-Circular\-Layout\-Strategy}\label{vtkinfovis_vtkcircularlayoutstrategy}
Section\-: \hyperlink{sec_vtkinfovis}{Visualization Toolkit Infovis Classes} \hypertarget{vtkwidgets_vtkxyplotwidget_Usage}{}\subsection{Usage}\label{vtkwidgets_vtkxyplotwidget_Usage}
Assigns points to the vertices around a circle with unit radius.

To create an instance of class vtk\-Circular\-Layout\-Strategy, simply invoke its constructor as follows \begin{DoxyVerb}  obj = vtkCircularLayoutStrategy
\end{DoxyVerb}
 \hypertarget{vtkwidgets_vtkxyplotwidget_Methods}{}\subsection{Methods}\label{vtkwidgets_vtkxyplotwidget_Methods}
The class vtk\-Circular\-Layout\-Strategy has several methods that can be used. They are listed below. Note that the documentation is translated automatically from the V\-T\-K sources, and may not be completely intelligible. When in doubt, consult the V\-T\-K website. In the methods listed below, {\ttfamily obj} is an instance of the vtk\-Circular\-Layout\-Strategy class. 
\begin{DoxyItemize}
\item {\ttfamily string = obj.\-Get\-Class\-Name ()}  
\item {\ttfamily int = obj.\-Is\-A (string name)}  
\item {\ttfamily vtk\-Circular\-Layout\-Strategy = obj.\-New\-Instance ()}  
\item {\ttfamily vtk\-Circular\-Layout\-Strategy = obj.\-Safe\-Down\-Cast (vtk\-Object o)}  
\item {\ttfamily obj.\-Layout ()} -\/ Perform the layout.  
\end{DoxyItemize}\hypertarget{vtkinfovis_vtkclustering2dlayoutstrategy}{}\section{vtk\-Clustering2\-D\-Layout\-Strategy}\label{vtkinfovis_vtkclustering2dlayoutstrategy}
Section\-: \hyperlink{sec_vtkinfovis}{Visualization Toolkit Infovis Classes} \hypertarget{vtkwidgets_vtkxyplotwidget_Usage}{}\subsection{Usage}\label{vtkwidgets_vtkxyplotwidget_Usage}
This class is a density grid based force directed layout strategy. Also please note that 'fast' is relative to quite slow. \-:) The layout running time is O(V+\-E) with an extremely high constant. .S\-E\-C\-T\-I\-O\-N Thanks Thanks to Godzilla for not eating my computer so that this class could be written.

To create an instance of class vtk\-Clustering2\-D\-Layout\-Strategy, simply invoke its constructor as follows \begin{DoxyVerb}  obj = vtkClustering2DLayoutStrategy
\end{DoxyVerb}
 \hypertarget{vtkwidgets_vtkxyplotwidget_Methods}{}\subsection{Methods}\label{vtkwidgets_vtkxyplotwidget_Methods}
The class vtk\-Clustering2\-D\-Layout\-Strategy has several methods that can be used. They are listed below. Note that the documentation is translated automatically from the V\-T\-K sources, and may not be completely intelligible. When in doubt, consult the V\-T\-K website. In the methods listed below, {\ttfamily obj} is an instance of the vtk\-Clustering2\-D\-Layout\-Strategy class. 
\begin{DoxyItemize}
\item {\ttfamily string = obj.\-Get\-Class\-Name ()}  
\item {\ttfamily int = obj.\-Is\-A (string name)}  
\item {\ttfamily vtk\-Clustering2\-D\-Layout\-Strategy = obj.\-New\-Instance ()}  
\item {\ttfamily vtk\-Clustering2\-D\-Layout\-Strategy = obj.\-Safe\-Down\-Cast (vtk\-Object o)}  
\item {\ttfamily obj.\-Set\-Random\-Seed (int )} -\/ Seed the random number generator used to jitter point positions. This has a significant effect on their final positions when the layout is complete.  
\item {\ttfamily int = obj.\-Get\-Random\-Seed\-Min\-Value ()} -\/ Seed the random number generator used to jitter point positions. This has a significant effect on their final positions when the layout is complete.  
\item {\ttfamily int = obj.\-Get\-Random\-Seed\-Max\-Value ()} -\/ Seed the random number generator used to jitter point positions. This has a significant effect on their final positions when the layout is complete.  
\item {\ttfamily int = obj.\-Get\-Random\-Seed ()} -\/ Seed the random number generator used to jitter point positions. This has a significant effect on their final positions when the layout is complete.  
\item {\ttfamily obj.\-Set\-Max\-Number\-Of\-Iterations (int )} -\/ Set/\-Get the maximum number of iterations to be used. The higher this number, the more iterations through the algorithm is possible, and thus, the more the graph gets modified. The default is '100' for no particular reason Note\-: The strong recommendation is that you do not change this parameter. \-:)  
\item {\ttfamily int = obj.\-Get\-Max\-Number\-Of\-Iterations\-Min\-Value ()} -\/ Set/\-Get the maximum number of iterations to be used. The higher this number, the more iterations through the algorithm is possible, and thus, the more the graph gets modified. The default is '100' for no particular reason Note\-: The strong recommendation is that you do not change this parameter. \-:)  
\item {\ttfamily int = obj.\-Get\-Max\-Number\-Of\-Iterations\-Max\-Value ()} -\/ Set/\-Get the maximum number of iterations to be used. The higher this number, the more iterations through the algorithm is possible, and thus, the more the graph gets modified. The default is '100' for no particular reason Note\-: The strong recommendation is that you do not change this parameter. \-:)  
\item {\ttfamily int = obj.\-Get\-Max\-Number\-Of\-Iterations ()} -\/ Set/\-Get the maximum number of iterations to be used. The higher this number, the more iterations through the algorithm is possible, and thus, the more the graph gets modified. The default is '100' for no particular reason Note\-: The strong recommendation is that you do not change this parameter. \-:)  
\item {\ttfamily obj.\-Set\-Iterations\-Per\-Layout (int )} -\/ Set/\-Get the number of iterations per layout. The only use for this ivar is for the application to do visualizations of the layout before it's complete. The default is '100' to match the default 'Max\-Number\-Of\-Iterations' Note\-: Changing this parameter is just fine \-:)  
\item {\ttfamily int = obj.\-Get\-Iterations\-Per\-Layout\-Min\-Value ()} -\/ Set/\-Get the number of iterations per layout. The only use for this ivar is for the application to do visualizations of the layout before it's complete. The default is '100' to match the default 'Max\-Number\-Of\-Iterations' Note\-: Changing this parameter is just fine \-:)  
\item {\ttfamily int = obj.\-Get\-Iterations\-Per\-Layout\-Max\-Value ()} -\/ Set/\-Get the number of iterations per layout. The only use for this ivar is for the application to do visualizations of the layout before it's complete. The default is '100' to match the default 'Max\-Number\-Of\-Iterations' Note\-: Changing this parameter is just fine \-:)  
\item {\ttfamily int = obj.\-Get\-Iterations\-Per\-Layout ()} -\/ Set/\-Get the number of iterations per layout. The only use for this ivar is for the application to do visualizations of the layout before it's complete. The default is '100' to match the default 'Max\-Number\-Of\-Iterations' Note\-: Changing this parameter is just fine \-:)  
\item {\ttfamily obj.\-Set\-Initial\-Temperature (float )} -\/ Set the initial temperature. The temperature default is '5' for no particular reason Note\-: The strong recommendation is that you do not change this parameter. \-:)  
\item {\ttfamily float = obj.\-Get\-Initial\-Temperature\-Min\-Value ()} -\/ Set the initial temperature. The temperature default is '5' for no particular reason Note\-: The strong recommendation is that you do not change this parameter. \-:)  
\item {\ttfamily float = obj.\-Get\-Initial\-Temperature\-Max\-Value ()} -\/ Set the initial temperature. The temperature default is '5' for no particular reason Note\-: The strong recommendation is that you do not change this parameter. \-:)  
\item {\ttfamily float = obj.\-Get\-Initial\-Temperature ()} -\/ Set the initial temperature. The temperature default is '5' for no particular reason Note\-: The strong recommendation is that you do not change this parameter. \-:)  
\item {\ttfamily obj.\-Set\-Cool\-Down\-Rate (double )} -\/ Set/\-Get the Cool-\/down rate. The higher this number is, the longer it will take to \char`\"{}cool-\/down\char`\"{}, and thus, the more the graph will be modified. The default is '10' for no particular reason. Note\-: The strong recommendation is that you do not change this parameter. \-:)  
\item {\ttfamily double = obj.\-Get\-Cool\-Down\-Rate\-Min\-Value ()} -\/ Set/\-Get the Cool-\/down rate. The higher this number is, the longer it will take to \char`\"{}cool-\/down\char`\"{}, and thus, the more the graph will be modified. The default is '10' for no particular reason. Note\-: The strong recommendation is that you do not change this parameter. \-:)  
\item {\ttfamily double = obj.\-Get\-Cool\-Down\-Rate\-Max\-Value ()} -\/ Set/\-Get the Cool-\/down rate. The higher this number is, the longer it will take to \char`\"{}cool-\/down\char`\"{}, and thus, the more the graph will be modified. The default is '10' for no particular reason. Note\-: The strong recommendation is that you do not change this parameter. \-:)  
\item {\ttfamily double = obj.\-Get\-Cool\-Down\-Rate ()} -\/ Set/\-Get the Cool-\/down rate. The higher this number is, the longer it will take to \char`\"{}cool-\/down\char`\"{}, and thus, the more the graph will be modified. The default is '10' for no particular reason. Note\-: The strong recommendation is that you do not change this parameter. \-:)  
\item {\ttfamily obj.\-Set\-Rest\-Distance (float )} -\/ Manually set the resting distance. Otherwise the distance is computed automatically.  
\item {\ttfamily float = obj.\-Get\-Rest\-Distance ()} -\/ Manually set the resting distance. Otherwise the distance is computed automatically.  
\item {\ttfamily obj.\-Initialize ()} -\/ This strategy sets up some data structures for faster processing of each Layout() call  
\item {\ttfamily obj.\-Layout ()} -\/ This is the layout method where the graph that was set in Set\-Graph() is laid out. The method can either entirely layout the graph or iteratively lay out the graph. If you have an iterative layout please implement the Is\-Layout\-Complete() method.  
\item {\ttfamily int = obj.\-Is\-Layout\-Complete ()}  
\end{DoxyItemize}\hypertarget{vtkinfovis_vtkcollapsegraph}{}\section{vtk\-Collapse\-Graph}\label{vtkinfovis_vtkcollapsegraph}
Section\-: \hyperlink{sec_vtkinfovis}{Visualization Toolkit Infovis Classes} \hypertarget{vtkwidgets_vtkxyplotwidget_Usage}{}\subsection{Usage}\label{vtkwidgets_vtkxyplotwidget_Usage}
vtk\-Collapse\-Graph \char`\"{}collapses\char`\"{} vertices onto their neighbors, while maintaining connectivity. Two inputs are required -\/ a graph (directed or undirected), and a vertex selection that can be converted to indices.

Conceptually, each of the vertices specified in the input selection expands, \char`\"{}swallowing\char`\"{} adacent vertices. Edges to-\/or-\/from the \char`\"{}swallowed\char`\"{} vertices become edges to-\/or-\/from the expanding vertices, maintaining the overall graph connectivity.

In the case of directed graphs, expanding vertices only swallow vertices that are connected via out edges. This rule provides intuitive behavior when working with trees, so that \char`\"{}child\char`\"{} vertices collapse into their parents when the parents are part of the input selection.

Input port 0\-: graph Input port 1\-: selection

To create an instance of class vtk\-Collapse\-Graph, simply invoke its constructor as follows \begin{DoxyVerb}  obj = vtkCollapseGraph
\end{DoxyVerb}
 \hypertarget{vtkwidgets_vtkxyplotwidget_Methods}{}\subsection{Methods}\label{vtkwidgets_vtkxyplotwidget_Methods}
The class vtk\-Collapse\-Graph has several methods that can be used. They are listed below. Note that the documentation is translated automatically from the V\-T\-K sources, and may not be completely intelligible. When in doubt, consult the V\-T\-K website. In the methods listed below, {\ttfamily obj} is an instance of the vtk\-Collapse\-Graph class. 
\begin{DoxyItemize}
\item {\ttfamily string = obj.\-Get\-Class\-Name ()}  
\item {\ttfamily int = obj.\-Is\-A (string name)}  
\item {\ttfamily vtk\-Collapse\-Graph = obj.\-New\-Instance ()}  
\item {\ttfamily vtk\-Collapse\-Graph = obj.\-Safe\-Down\-Cast (vtk\-Object o)}  
\item {\ttfamily obj.\-Set\-Graph\-Connection (vtk\-Algorithm\-Output )}  
\item {\ttfamily obj.\-Set\-Selection\-Connection (vtk\-Algorithm\-Output )}  
\end{DoxyItemize}\hypertarget{vtkinfovis_vtkcollapseverticesbyarray}{}\section{vtk\-Collapse\-Vertices\-By\-Array}\label{vtkinfovis_vtkcollapseverticesbyarray}
Section\-: \hyperlink{sec_vtkinfovis}{Visualization Toolkit Infovis Classes} \hypertarget{vtkwidgets_vtkxyplotwidget_Usage}{}\subsection{Usage}\label{vtkwidgets_vtkxyplotwidget_Usage}
vtk\-Collapse\-Vertices\-By\-Array is a class which collapses the graph using a vertex array as the key. So if the graph has vertices sharing common traits then this class combines all these vertices into one. This class does not perform aggregation on vertex data but allow to do so for edge data. Users can choose one or more edge data arrays for aggregation using Add\-Aggregate\-Edge\-Array function.

.S\-E\-C\-T\-I\-O\-N Thanks

To create an instance of class vtk\-Collapse\-Vertices\-By\-Array, simply invoke its constructor as follows \begin{DoxyVerb}  obj = vtkCollapseVerticesByArray
\end{DoxyVerb}
 \hypertarget{vtkwidgets_vtkxyplotwidget_Methods}{}\subsection{Methods}\label{vtkwidgets_vtkxyplotwidget_Methods}
The class vtk\-Collapse\-Vertices\-By\-Array has several methods that can be used. They are listed below. Note that the documentation is translated automatically from the V\-T\-K sources, and may not be completely intelligible. When in doubt, consult the V\-T\-K website. In the methods listed below, {\ttfamily obj} is an instance of the vtk\-Collapse\-Vertices\-By\-Array class. 
\begin{DoxyItemize}
\item {\ttfamily string = obj.\-Get\-Class\-Name ()}  
\item {\ttfamily int = obj.\-Is\-A (string name)}  
\item {\ttfamily vtk\-Collapse\-Vertices\-By\-Array = obj.\-New\-Instance ()}  
\item {\ttfamily vtk\-Collapse\-Vertices\-By\-Array = obj.\-Safe\-Down\-Cast (vtk\-Object o)}  
\item {\ttfamily bool = obj.\-Get\-Allow\-Self\-Loops ()} -\/ Boolean to allow self loops during collapse.  
\item {\ttfamily obj.\-Set\-Allow\-Self\-Loops (bool )} -\/ Boolean to allow self loops during collapse.  
\item {\ttfamily obj.\-Allow\-Self\-Loops\-On ()} -\/ Boolean to allow self loops during collapse.  
\item {\ttfamily obj.\-Allow\-Self\-Loops\-Off ()} -\/ Boolean to allow self loops during collapse.  
\item {\ttfamily obj.\-Add\-Aggregate\-Edge\-Array (string arr\-Name)} -\/ Add arrays on which aggregation of data is allowed. Default if replaced by the last value.  
\item {\ttfamily obj.\-Clear\-Aggregate\-Edge\-Array ()} -\/ Clear the list of arrays on which aggregation was set to allow.  
\item {\ttfamily string = obj.\-Get\-Vertex\-Array ()} -\/ Set the array using which perform the collapse.  
\item {\ttfamily obj.\-Set\-Vertex\-Array (string )} -\/ Set the array using which perform the collapse.  
\end{DoxyItemize}\hypertarget{vtkinfovis_vtkcommunity2dlayoutstrategy}{}\section{vtk\-Community2\-D\-Layout\-Strategy}\label{vtkinfovis_vtkcommunity2dlayoutstrategy}
Section\-: \hyperlink{sec_vtkinfovis}{Visualization Toolkit Infovis Classes} \hypertarget{vtkwidgets_vtkxyplotwidget_Usage}{}\subsection{Usage}\label{vtkwidgets_vtkxyplotwidget_Usage}
This class is a density grid based force directed layout strategy. Also please note that 'fast' is relative to quite slow. \-:) The layout running time is O(V+\-E) with an extremely high constant. .S\-E\-C\-T\-I\-O\-N Thanks Thanks to Godzilla for not eating my computer so that this class could be written.

To create an instance of class vtk\-Community2\-D\-Layout\-Strategy, simply invoke its constructor as follows \begin{DoxyVerb}  obj = vtkCommunity2DLayoutStrategy
\end{DoxyVerb}
 \hypertarget{vtkwidgets_vtkxyplotwidget_Methods}{}\subsection{Methods}\label{vtkwidgets_vtkxyplotwidget_Methods}
The class vtk\-Community2\-D\-Layout\-Strategy has several methods that can be used. They are listed below. Note that the documentation is translated automatically from the V\-T\-K sources, and may not be completely intelligible. When in doubt, consult the V\-T\-K website. In the methods listed below, {\ttfamily obj} is an instance of the vtk\-Community2\-D\-Layout\-Strategy class. 
\begin{DoxyItemize}
\item {\ttfamily string = obj.\-Get\-Class\-Name ()}  
\item {\ttfamily int = obj.\-Is\-A (string name)}  
\item {\ttfamily vtk\-Community2\-D\-Layout\-Strategy = obj.\-New\-Instance ()}  
\item {\ttfamily vtk\-Community2\-D\-Layout\-Strategy = obj.\-Safe\-Down\-Cast (vtk\-Object o)}  
\item {\ttfamily obj.\-Set\-Random\-Seed (int )} -\/ Seed the random number generator used to jitter point positions. This has a significant effect on their final positions when the layout is complete.  
\item {\ttfamily int = obj.\-Get\-Random\-Seed\-Min\-Value ()} -\/ Seed the random number generator used to jitter point positions. This has a significant effect on their final positions when the layout is complete.  
\item {\ttfamily int = obj.\-Get\-Random\-Seed\-Max\-Value ()} -\/ Seed the random number generator used to jitter point positions. This has a significant effect on their final positions when the layout is complete.  
\item {\ttfamily int = obj.\-Get\-Random\-Seed ()} -\/ Seed the random number generator used to jitter point positions. This has a significant effect on their final positions when the layout is complete.  
\item {\ttfamily obj.\-Set\-Max\-Number\-Of\-Iterations (int )} -\/ Set/\-Get the maximum number of iterations to be used. The higher this number, the more iterations through the algorithm is possible, and thus, the more the graph gets modified. The default is '100' for no particular reason Note\-: The strong recommendation is that you do not change this parameter. \-:)  
\item {\ttfamily int = obj.\-Get\-Max\-Number\-Of\-Iterations\-Min\-Value ()} -\/ Set/\-Get the maximum number of iterations to be used. The higher this number, the more iterations through the algorithm is possible, and thus, the more the graph gets modified. The default is '100' for no particular reason Note\-: The strong recommendation is that you do not change this parameter. \-:)  
\item {\ttfamily int = obj.\-Get\-Max\-Number\-Of\-Iterations\-Max\-Value ()} -\/ Set/\-Get the maximum number of iterations to be used. The higher this number, the more iterations through the algorithm is possible, and thus, the more the graph gets modified. The default is '100' for no particular reason Note\-: The strong recommendation is that you do not change this parameter. \-:)  
\item {\ttfamily int = obj.\-Get\-Max\-Number\-Of\-Iterations ()} -\/ Set/\-Get the maximum number of iterations to be used. The higher this number, the more iterations through the algorithm is possible, and thus, the more the graph gets modified. The default is '100' for no particular reason Note\-: The strong recommendation is that you do not change this parameter. \-:)  
\item {\ttfamily obj.\-Set\-Iterations\-Per\-Layout (int )} -\/ Set/\-Get the number of iterations per layout. The only use for this ivar is for the application to do visualizations of the layout before it's complete. The default is '100' to match the default 'Max\-Number\-Of\-Iterations' Note\-: Changing this parameter is just fine \-:)  
\item {\ttfamily int = obj.\-Get\-Iterations\-Per\-Layout\-Min\-Value ()} -\/ Set/\-Get the number of iterations per layout. The only use for this ivar is for the application to do visualizations of the layout before it's complete. The default is '100' to match the default 'Max\-Number\-Of\-Iterations' Note\-: Changing this parameter is just fine \-:)  
\item {\ttfamily int = obj.\-Get\-Iterations\-Per\-Layout\-Max\-Value ()} -\/ Set/\-Get the number of iterations per layout. The only use for this ivar is for the application to do visualizations of the layout before it's complete. The default is '100' to match the default 'Max\-Number\-Of\-Iterations' Note\-: Changing this parameter is just fine \-:)  
\item {\ttfamily int = obj.\-Get\-Iterations\-Per\-Layout ()} -\/ Set/\-Get the number of iterations per layout. The only use for this ivar is for the application to do visualizations of the layout before it's complete. The default is '100' to match the default 'Max\-Number\-Of\-Iterations' Note\-: Changing this parameter is just fine \-:)  
\item {\ttfamily obj.\-Set\-Initial\-Temperature (float )} -\/ Set the initial temperature. The temperature default is '5' for no particular reason Note\-: The strong recommendation is that you do not change this parameter. \-:)  
\item {\ttfamily float = obj.\-Get\-Initial\-Temperature\-Min\-Value ()} -\/ Set the initial temperature. The temperature default is '5' for no particular reason Note\-: The strong recommendation is that you do not change this parameter. \-:)  
\item {\ttfamily float = obj.\-Get\-Initial\-Temperature\-Max\-Value ()} -\/ Set the initial temperature. The temperature default is '5' for no particular reason Note\-: The strong recommendation is that you do not change this parameter. \-:)  
\item {\ttfamily float = obj.\-Get\-Initial\-Temperature ()} -\/ Set the initial temperature. The temperature default is '5' for no particular reason Note\-: The strong recommendation is that you do not change this parameter. \-:)  
\item {\ttfamily obj.\-Set\-Cool\-Down\-Rate (double )} -\/ Set/\-Get the Cool-\/down rate. The higher this number is, the longer it will take to \char`\"{}cool-\/down\char`\"{}, and thus, the more the graph will be modified. The default is '10' for no particular reason. Note\-: The strong recommendation is that you do not change this parameter. \-:)  
\item {\ttfamily double = obj.\-Get\-Cool\-Down\-Rate\-Min\-Value ()} -\/ Set/\-Get the Cool-\/down rate. The higher this number is, the longer it will take to \char`\"{}cool-\/down\char`\"{}, and thus, the more the graph will be modified. The default is '10' for no particular reason. Note\-: The strong recommendation is that you do not change this parameter. \-:)  
\item {\ttfamily double = obj.\-Get\-Cool\-Down\-Rate\-Max\-Value ()} -\/ Set/\-Get the Cool-\/down rate. The higher this number is, the longer it will take to \char`\"{}cool-\/down\char`\"{}, and thus, the more the graph will be modified. The default is '10' for no particular reason. Note\-: The strong recommendation is that you do not change this parameter. \-:)  
\item {\ttfamily double = obj.\-Get\-Cool\-Down\-Rate ()} -\/ Set/\-Get the Cool-\/down rate. The higher this number is, the longer it will take to \char`\"{}cool-\/down\char`\"{}, and thus, the more the graph will be modified. The default is '10' for no particular reason. Note\-: The strong recommendation is that you do not change this parameter. \-:)  
\item {\ttfamily obj.\-Set\-Rest\-Distance (float )} -\/ Manually set the resting distance. Otherwise the distance is computed automatically.  
\item {\ttfamily float = obj.\-Get\-Rest\-Distance ()} -\/ Manually set the resting distance. Otherwise the distance is computed automatically.  
\item {\ttfamily obj.\-Initialize ()} -\/ This strategy sets up some data structures for faster processing of each Layout() call  
\item {\ttfamily obj.\-Layout ()} -\/ This is the layout method where the graph that was set in Set\-Graph() is laid out. The method can either entirely layout the graph or iteratively lay out the graph. If you have an iterative layout please implement the Is\-Layout\-Complete() method.  
\item {\ttfamily int = obj.\-Is\-Layout\-Complete ()} -\/ Get/\-Set the community array name  
\item {\ttfamily string = obj.\-Get\-Community\-Array\-Name ()} -\/ Get/\-Set the community array name  
\item {\ttfamily obj.\-Set\-Community\-Array\-Name (string )} -\/ Get/\-Set the community array name  
\item {\ttfamily obj.\-Set\-Community\-Strength (float )} -\/ Set the community 'strength'. The default is '1' which means vertices in the same community will be placed close together, values closer to .1 (minimum) will mean a layout closer to traditional force directed.  
\item {\ttfamily float = obj.\-Get\-Community\-Strength\-Min\-Value ()} -\/ Set the community 'strength'. The default is '1' which means vertices in the same community will be placed close together, values closer to .1 (minimum) will mean a layout closer to traditional force directed.  
\item {\ttfamily float = obj.\-Get\-Community\-Strength\-Max\-Value ()} -\/ Set the community 'strength'. The default is '1' which means vertices in the same community will be placed close together, values closer to .1 (minimum) will mean a layout closer to traditional force directed.  
\item {\ttfamily float = obj.\-Get\-Community\-Strength ()} -\/ Set the community 'strength'. The default is '1' which means vertices in the same community will be placed close together, values closer to .1 (minimum) will mean a layout closer to traditional force directed.  
\end{DoxyItemize}\hypertarget{vtkinfovis_vtkcomputehistogram2doutliers}{}\section{vtk\-Compute\-Histogram2\-D\-Outliers}\label{vtkinfovis_vtkcomputehistogram2doutliers}
Section\-: \hyperlink{sec_vtkinfovis}{Visualization Toolkit Infovis Classes} \hypertarget{vtkwidgets_vtkxyplotwidget_Usage}{}\subsection{Usage}\label{vtkwidgets_vtkxyplotwidget_Usage}
This class takes a table and one or more vtk\-Image\-Data histograms as input and computes the outliers in that data. In general it does so by identifying histogram bins that are removed by a median (salt and pepper) filter and below a threshold. This threshold is automatically identified to retrieve a number of outliers close to a user-\/determined value. This value is set by calling Set\-Preferred\-Number\-Of\-Outliers(int).

The image data input can come either as a multiple vtk\-Image\-Data via the repeatable I\-N\-P\-U\-T\-\_\-\-H\-I\-S\-T\-O\-G\-R\-A\-M\-\_\-\-I\-M\-A\-G\-E\-\_\-\-D\-A\-T\-A port, or as a single vtk\-Multi\-Block\-Data\-Set containing vtk\-Image\-Data objects as blocks. One or the other must be set, not both (or neither).

The output can be retrieved as a set of row ids in a vtk\-Selection or as a vtk\-Table containing the actual outlier row data.

To create an instance of class vtk\-Compute\-Histogram2\-D\-Outliers, simply invoke its constructor as follows \begin{DoxyVerb}  obj = vtkComputeHistogram2DOutliers
\end{DoxyVerb}
 \hypertarget{vtkwidgets_vtkxyplotwidget_Methods}{}\subsection{Methods}\label{vtkwidgets_vtkxyplotwidget_Methods}
The class vtk\-Compute\-Histogram2\-D\-Outliers has several methods that can be used. They are listed below. Note that the documentation is translated automatically from the V\-T\-K sources, and may not be completely intelligible. When in doubt, consult the V\-T\-K website. In the methods listed below, {\ttfamily obj} is an instance of the vtk\-Compute\-Histogram2\-D\-Outliers class. 
\begin{DoxyItemize}
\item {\ttfamily string = obj.\-Get\-Class\-Name ()}  
\item {\ttfamily int = obj.\-Is\-A (string name)}  
\item {\ttfamily vtk\-Compute\-Histogram2\-D\-Outliers = obj.\-New\-Instance ()}  
\item {\ttfamily vtk\-Compute\-Histogram2\-D\-Outliers = obj.\-Safe\-Down\-Cast (vtk\-Object o)}  
\item {\ttfamily obj.\-Set\-Preferred\-Number\-Of\-Outliers (int )}  
\item {\ttfamily int = obj.\-Get\-Preferred\-Number\-Of\-Outliers ()}  
\item {\ttfamily vtk\-Table = obj.\-Get\-Output\-Table ()}  
\item {\ttfamily obj.\-Set\-Input\-Table\-Connection (vtk\-Algorithm\-Output cxn)} -\/ Set the input histogram data as a (repeatable) vtk\-Image\-Data  
\item {\ttfamily obj.\-Set\-Input\-Histogram\-Image\-Data\-Connection (vtk\-Algorithm\-Output cxn)} -\/ Set the input histogram data as a vtk\-Multi\-Block\-Data set containing multiple vtk\-Image\-Data objects.  
\item {\ttfamily obj.\-Set\-Input\-Histogram\-Multi\-Block\-Connection (vtk\-Algorithm\-Output cxn)}  
\end{DoxyItemize}\hypertarget{vtkinfovis_vtkconelayoutstrategy}{}\section{vtk\-Cone\-Layout\-Strategy}\label{vtkinfovis_vtkconelayoutstrategy}
Section\-: \hyperlink{sec_vtkinfovis}{Visualization Toolkit Infovis Classes} \hypertarget{vtkwidgets_vtkxyplotwidget_Usage}{}\subsection{Usage}\label{vtkwidgets_vtkxyplotwidget_Usage}
vtk\-Cone\-Layout\-Strategy positions the nodes of a tree(forest) in 3\-D space based on the cone-\/tree approach first described by Robertson, Mackinlay and Card in Proc. C\-H\-I'91. This implementation incorporates refinements to the layout developed by Carriere and Kazman, and by Auber.

The input graph must be a forest (i.\-e. a set of trees, or a single tree); in the case of a forest, the input will be converted to a single tree by introducing a new root node, and connecting each root in the input forest to the meta-\/root. The tree is then laid out, after which the meta-\/root is removed.

The cones are positioned so that children lie in planes parallel to the X-\/\-Y plane, with the axis of cones parallel to Z, and with Z coordinate increasing with distance of nodes from the root.

.S\-E\-C\-T\-I\-O\-N Thanks Thanks to David Duke from the University of Leeds for providing this implementation.

To create an instance of class vtk\-Cone\-Layout\-Strategy, simply invoke its constructor as follows \begin{DoxyVerb}  obj = vtkConeLayoutStrategy
\end{DoxyVerb}
 \hypertarget{vtkwidgets_vtkxyplotwidget_Methods}{}\subsection{Methods}\label{vtkwidgets_vtkxyplotwidget_Methods}
The class vtk\-Cone\-Layout\-Strategy has several methods that can be used. They are listed below. Note that the documentation is translated automatically from the V\-T\-K sources, and may not be completely intelligible. When in doubt, consult the V\-T\-K website. In the methods listed below, {\ttfamily obj} is an instance of the vtk\-Cone\-Layout\-Strategy class. 
\begin{DoxyItemize}
\item {\ttfamily string = obj.\-Get\-Class\-Name ()}  
\item {\ttfamily int = obj.\-Is\-A (string name)}  
\item {\ttfamily vtk\-Cone\-Layout\-Strategy = obj.\-New\-Instance ()}  
\item {\ttfamily vtk\-Cone\-Layout\-Strategy = obj.\-Safe\-Down\-Cast (vtk\-Object o)}  
\item {\ttfamily obj.\-Set\-Compactness (float )} -\/ Determine the compactness, the ratio between the average width of a cone in the tree, and the height of the cone. The default setting is 0.\-75 which (empirically) seems reasonable, but this will need adapting depending on the data.  
\item {\ttfamily float = obj.\-Get\-Compactness ()} -\/ Determine the compactness, the ratio between the average width of a cone in the tree, and the height of the cone. The default setting is 0.\-75 which (empirically) seems reasonable, but this will need adapting depending on the data.  
\item {\ttfamily obj.\-Set\-Compression (int )} -\/ Determine if layout should be compressed, i.\-e. the layout puts children closer together, possibly allowing sub-\/trees to overlap. This is useful if the tree is actually the spanning tree of a graph. For \char`\"{}real\char`\"{} trees, non-\/compressed layout is best, and is the default.  
\item {\ttfamily int = obj.\-Get\-Compression ()} -\/ Determine if layout should be compressed, i.\-e. the layout puts children closer together, possibly allowing sub-\/trees to overlap. This is useful if the tree is actually the spanning tree of a graph. For \char`\"{}real\char`\"{} trees, non-\/compressed layout is best, and is the default.  
\item {\ttfamily obj.\-Compression\-On ()} -\/ Determine if layout should be compressed, i.\-e. the layout puts children closer together, possibly allowing sub-\/trees to overlap. This is useful if the tree is actually the spanning tree of a graph. For \char`\"{}real\char`\"{} trees, non-\/compressed layout is best, and is the default.  
\item {\ttfamily obj.\-Compression\-Off ()} -\/ Determine if layout should be compressed, i.\-e. the layout puts children closer together, possibly allowing sub-\/trees to overlap. This is useful if the tree is actually the spanning tree of a graph. For \char`\"{}real\char`\"{} trees, non-\/compressed layout is best, and is the default.  
\item {\ttfamily obj.\-Set\-Spacing (float )} -\/ Set the spacing parameter that affects space between layers of the tree. If compression is on, Spacing is the actual distance between layers. If compression is off, actual distance also includes a factor of the compactness and maximum cone radius.  
\item {\ttfamily float = obj.\-Get\-Spacing ()} -\/ Set the spacing parameter that affects space between layers of the tree. If compression is on, Spacing is the actual distance between layers. If compression is off, actual distance also includes a factor of the compactness and maximum cone radius.  
\item {\ttfamily obj.\-Layout ()} -\/ Perform the layout.  
\end{DoxyItemize}\hypertarget{vtkinfovis_vtkconstrained2dlayoutstrategy}{}\section{vtk\-Constrained2\-D\-Layout\-Strategy}\label{vtkinfovis_vtkconstrained2dlayoutstrategy}
Section\-: \hyperlink{sec_vtkinfovis}{Visualization Toolkit Infovis Classes} \hypertarget{vtkwidgets_vtkxyplotwidget_Usage}{}\subsection{Usage}\label{vtkwidgets_vtkxyplotwidget_Usage}
This class is a density grid based force directed layout strategy. Also please note that 'fast' is relative to quite slow. \-:) The layout running time is O(V+\-E) with an extremely high constant. .S\-E\-C\-T\-I\-O\-N Thanks We would like to thank Mothra for distracting Godzilla while we wrote this class.

To create an instance of class vtk\-Constrained2\-D\-Layout\-Strategy, simply invoke its constructor as follows \begin{DoxyVerb}  obj = vtkConstrained2DLayoutStrategy
\end{DoxyVerb}
 \hypertarget{vtkwidgets_vtkxyplotwidget_Methods}{}\subsection{Methods}\label{vtkwidgets_vtkxyplotwidget_Methods}
The class vtk\-Constrained2\-D\-Layout\-Strategy has several methods that can be used. They are listed below. Note that the documentation is translated automatically from the V\-T\-K sources, and may not be completely intelligible. When in doubt, consult the V\-T\-K website. In the methods listed below, {\ttfamily obj} is an instance of the vtk\-Constrained2\-D\-Layout\-Strategy class. 
\begin{DoxyItemize}
\item {\ttfamily string = obj.\-Get\-Class\-Name ()}  
\item {\ttfamily int = obj.\-Is\-A (string name)}  
\item {\ttfamily vtk\-Constrained2\-D\-Layout\-Strategy = obj.\-New\-Instance ()}  
\item {\ttfamily vtk\-Constrained2\-D\-Layout\-Strategy = obj.\-Safe\-Down\-Cast (vtk\-Object o)}  
\item {\ttfamily obj.\-Set\-Random\-Seed (int )} -\/ Seed the random number generator used to jitter point positions. This has a significant effect on their final positions when the layout is complete.  
\item {\ttfamily int = obj.\-Get\-Random\-Seed\-Min\-Value ()} -\/ Seed the random number generator used to jitter point positions. This has a significant effect on their final positions when the layout is complete.  
\item {\ttfamily int = obj.\-Get\-Random\-Seed\-Max\-Value ()} -\/ Seed the random number generator used to jitter point positions. This has a significant effect on their final positions when the layout is complete.  
\item {\ttfamily int = obj.\-Get\-Random\-Seed ()} -\/ Seed the random number generator used to jitter point positions. This has a significant effect on their final positions when the layout is complete.  
\item {\ttfamily obj.\-Set\-Max\-Number\-Of\-Iterations (int )} -\/ Set/\-Get the maximum number of iterations to be used. The higher this number, the more iterations through the algorithm is possible, and thus, the more the graph gets modified. The default is '100' for no particular reason Note\-: The strong recommendation is that you do not change this parameter. \-:)  
\item {\ttfamily int = obj.\-Get\-Max\-Number\-Of\-Iterations\-Min\-Value ()} -\/ Set/\-Get the maximum number of iterations to be used. The higher this number, the more iterations through the algorithm is possible, and thus, the more the graph gets modified. The default is '100' for no particular reason Note\-: The strong recommendation is that you do not change this parameter. \-:)  
\item {\ttfamily int = obj.\-Get\-Max\-Number\-Of\-Iterations\-Max\-Value ()} -\/ Set/\-Get the maximum number of iterations to be used. The higher this number, the more iterations through the algorithm is possible, and thus, the more the graph gets modified. The default is '100' for no particular reason Note\-: The strong recommendation is that you do not change this parameter. \-:)  
\item {\ttfamily int = obj.\-Get\-Max\-Number\-Of\-Iterations ()} -\/ Set/\-Get the maximum number of iterations to be used. The higher this number, the more iterations through the algorithm is possible, and thus, the more the graph gets modified. The default is '100' for no particular reason Note\-: The strong recommendation is that you do not change this parameter. \-:)  
\item {\ttfamily obj.\-Set\-Iterations\-Per\-Layout (int )} -\/ Set/\-Get the number of iterations per layout. The only use for this ivar is for the application to do visualizations of the layout before it's complete. The default is '100' to match the default 'Max\-Number\-Of\-Iterations' Note\-: Changing this parameter is just fine \-:)  
\item {\ttfamily int = obj.\-Get\-Iterations\-Per\-Layout\-Min\-Value ()} -\/ Set/\-Get the number of iterations per layout. The only use for this ivar is for the application to do visualizations of the layout before it's complete. The default is '100' to match the default 'Max\-Number\-Of\-Iterations' Note\-: Changing this parameter is just fine \-:)  
\item {\ttfamily int = obj.\-Get\-Iterations\-Per\-Layout\-Max\-Value ()} -\/ Set/\-Get the number of iterations per layout. The only use for this ivar is for the application to do visualizations of the layout before it's complete. The default is '100' to match the default 'Max\-Number\-Of\-Iterations' Note\-: Changing this parameter is just fine \-:)  
\item {\ttfamily int = obj.\-Get\-Iterations\-Per\-Layout ()} -\/ Set/\-Get the number of iterations per layout. The only use for this ivar is for the application to do visualizations of the layout before it's complete. The default is '100' to match the default 'Max\-Number\-Of\-Iterations' Note\-: Changing this parameter is just fine \-:)  
\item {\ttfamily obj.\-Set\-Initial\-Temperature (float )} -\/ Set the initial temperature. The temperature default is '5' for no particular reason Note\-: The strong recommendation is that you do not change this parameter. \-:)  
\item {\ttfamily float = obj.\-Get\-Initial\-Temperature\-Min\-Value ()} -\/ Set the initial temperature. The temperature default is '5' for no particular reason Note\-: The strong recommendation is that you do not change this parameter. \-:)  
\item {\ttfamily float = obj.\-Get\-Initial\-Temperature\-Max\-Value ()} -\/ Set the initial temperature. The temperature default is '5' for no particular reason Note\-: The strong recommendation is that you do not change this parameter. \-:)  
\item {\ttfamily float = obj.\-Get\-Initial\-Temperature ()} -\/ Set the initial temperature. The temperature default is '5' for no particular reason Note\-: The strong recommendation is that you do not change this parameter. \-:)  
\item {\ttfamily obj.\-Set\-Cool\-Down\-Rate (double )} -\/ Set/\-Get the Cool-\/down rate. The higher this number is, the longer it will take to \char`\"{}cool-\/down\char`\"{}, and thus, the more the graph will be modified. The default is '10' for no particular reason. Note\-: The strong recommendation is that you do not change this parameter. \-:)  
\item {\ttfamily double = obj.\-Get\-Cool\-Down\-Rate\-Min\-Value ()} -\/ Set/\-Get the Cool-\/down rate. The higher this number is, the longer it will take to \char`\"{}cool-\/down\char`\"{}, and thus, the more the graph will be modified. The default is '10' for no particular reason. Note\-: The strong recommendation is that you do not change this parameter. \-:)  
\item {\ttfamily double = obj.\-Get\-Cool\-Down\-Rate\-Max\-Value ()} -\/ Set/\-Get the Cool-\/down rate. The higher this number is, the longer it will take to \char`\"{}cool-\/down\char`\"{}, and thus, the more the graph will be modified. The default is '10' for no particular reason. Note\-: The strong recommendation is that you do not change this parameter. \-:)  
\item {\ttfamily double = obj.\-Get\-Cool\-Down\-Rate ()} -\/ Set/\-Get the Cool-\/down rate. The higher this number is, the longer it will take to \char`\"{}cool-\/down\char`\"{}, and thus, the more the graph will be modified. The default is '10' for no particular reason. Note\-: The strong recommendation is that you do not change this parameter. \-:)  
\item {\ttfamily obj.\-Set\-Rest\-Distance (float )} -\/ Manually set the resting distance. Otherwise the distance is computed automatically.  
\item {\ttfamily float = obj.\-Get\-Rest\-Distance ()} -\/ Manually set the resting distance. Otherwise the distance is computed automatically.  
\item {\ttfamily obj.\-Initialize ()} -\/ This strategy sets up some data structures for faster processing of each Layout() call  
\item {\ttfamily obj.\-Layout ()} -\/ This is the layout method where the graph that was set in Set\-Graph() is laid out. The method can either entirely layout the graph or iteratively lay out the graph. If you have an iterative layout please implement the Is\-Layout\-Complete() method.  
\item {\ttfamily int = obj.\-Is\-Layout\-Complete ()} -\/ Set/\-Get the input constraint array name. If no input array name is set then the name 'constraint' is used.  
\item {\ttfamily obj.\-Set\-Input\-Array\-Name (string )} -\/ Set/\-Get the input constraint array name. If no input array name is set then the name 'constraint' is used.  
\item {\ttfamily string = obj.\-Get\-Input\-Array\-Name ()} -\/ Set/\-Get the input constraint array name. If no input array name is set then the name 'constraint' is used.  
\end{DoxyItemize}\hypertarget{vtkinfovis_vtkcontingencystatistics}{}\section{vtk\-Contingency\-Statistics}\label{vtkinfovis_vtkcontingencystatistics}
Section\-: \hyperlink{sec_vtkinfovis}{Visualization Toolkit Infovis Classes} \hypertarget{vtkwidgets_vtkxyplotwidget_Usage}{}\subsection{Usage}\label{vtkwidgets_vtkxyplotwidget_Usage}
Given a pair of columns of interest, this class provides the following functionalities, depending on the execution mode it is executed in\-: Learn\-: calculate contigency tables and corresponding discrete bivariate probability distribution. Assess\-: given two columns of interest with the same number of entries as input in port I\-N\-P\-U\-T\-\_\-\-D\-A\-T\-A, and a corresponding bivariate probability distribution,

.S\-E\-C\-T\-I\-O\-N Thanks Thanks to Philippe Pebay and David Thompson from Sandia National Laboratories for implementing this class.

To create an instance of class vtk\-Contingency\-Statistics, simply invoke its constructor as follows \begin{DoxyVerb}  obj = vtkContingencyStatistics
\end{DoxyVerb}
 \hypertarget{vtkwidgets_vtkxyplotwidget_Methods}{}\subsection{Methods}\label{vtkwidgets_vtkxyplotwidget_Methods}
The class vtk\-Contingency\-Statistics has several methods that can be used. They are listed below. Note that the documentation is translated automatically from the V\-T\-K sources, and may not be completely intelligible. When in doubt, consult the V\-T\-K website. In the methods listed below, {\ttfamily obj} is an instance of the vtk\-Contingency\-Statistics class. 
\begin{DoxyItemize}
\item {\ttfamily string = obj.\-Get\-Class\-Name ()}  
\item {\ttfamily int = obj.\-Is\-A (string name)}  
\item {\ttfamily vtk\-Contingency\-Statistics = obj.\-New\-Instance ()}  
\item {\ttfamily vtk\-Contingency\-Statistics = obj.\-Safe\-Down\-Cast (vtk\-Object o)}  
\item {\ttfamily obj.\-Aggregate (vtk\-Data\-Object\-Collection , vtk\-Data\-Object )} -\/ Given a collection of models, calculate aggregate model N\-B\-: not implemented  
\end{DoxyItemize}\hypertarget{vtkinfovis_vtkcorrelativestatistics}{}\section{vtk\-Correlative\-Statistics}\label{vtkinfovis_vtkcorrelativestatistics}
Section\-: \hyperlink{sec_vtkinfovis}{Visualization Toolkit Infovis Classes} \hypertarget{vtkwidgets_vtkxyplotwidget_Usage}{}\subsection{Usage}\label{vtkwidgets_vtkxyplotwidget_Usage}
Given a selection of pairs of columns of interest, this class provides the following functionalities, depending on the execution mode it is executed in\-: Learn\-: calculate means, unbiased variance and covariance estimators of column pairs, and corresponding linear regressions and linear correlation coefficient. More precisely, Learn calculates the sums; if {\ttfamily finalize} is set to true (default), the final statistics are calculated with the function Calculate\-From\-Sums. Otherwise, only raw sums are output; this option is made for efficient parallel calculations. Note that Calculate\-From\-Sums is a static function, so that it can be used directly with no need to instantiate a vtk\-Correlative\-Statistics object. Assess\-: given two data vectors X and Y with the same number of entries as input in port I\-N\-P\-U\-T\-\_\-\-D\-A\-T\-A, and reference means, variances, and covariance, along with an acceptable threshold t$>$1, assess all pairs of values of (X,Y) whose relative P\-D\-F (assuming a bivariate Gaussian model) is below t.

.S\-E\-C\-T\-I\-O\-N Thanks Thanks to Philippe Pebay and David Thompson from Sandia National Laboratories for implementing this class.

To create an instance of class vtk\-Correlative\-Statistics, simply invoke its constructor as follows \begin{DoxyVerb}  obj = vtkCorrelativeStatistics
\end{DoxyVerb}
 \hypertarget{vtkwidgets_vtkxyplotwidget_Methods}{}\subsection{Methods}\label{vtkwidgets_vtkxyplotwidget_Methods}
The class vtk\-Correlative\-Statistics has several methods that can be used. They are listed below. Note that the documentation is translated automatically from the V\-T\-K sources, and may not be completely intelligible. When in doubt, consult the V\-T\-K website. In the methods listed below, {\ttfamily obj} is an instance of the vtk\-Correlative\-Statistics class. 
\begin{DoxyItemize}
\item {\ttfamily string = obj.\-Get\-Class\-Name ()}  
\item {\ttfamily int = obj.\-Is\-A (string name)}  
\item {\ttfamily vtk\-Correlative\-Statistics = obj.\-New\-Instance ()}  
\item {\ttfamily vtk\-Correlative\-Statistics = obj.\-Safe\-Down\-Cast (vtk\-Object o)}  
\item {\ttfamily obj.\-Aggregate (vtk\-Data\-Object\-Collection , vtk\-Data\-Object )} -\/ Given a collection of models, calculate aggregate model  
\end{DoxyItemize}\hypertarget{vtkinfovis_vtkcosmictreelayoutstrategy}{}\section{vtk\-Cosmic\-Tree\-Layout\-Strategy}\label{vtkinfovis_vtkcosmictreelayoutstrategy}
Section\-: \hyperlink{sec_vtkinfovis}{Visualization Toolkit Infovis Classes} \hypertarget{vtkwidgets_vtkxyplotwidget_Usage}{}\subsection{Usage}\label{vtkwidgets_vtkxyplotwidget_Usage}
This layout strategy takes an input tree and places all the children of a node into a containing circle. The placement is such that each child placed can be represented with a circle tangent to the containing circle and (usually) 2 other children. The interior of the circle is left empty so that graph edges drawn on top of the tree will not obfuscate the tree. However, when one child is much larger than all the others, it may encroach on the center of the containing circle; that's O\-K, because it's large enough not to be obscured by edges drawn atop it.

.S\-E\-C\-T\-I\-O\-N Thanks Thanks to the galaxy and David Thompson hierarchically nested inside it for inspiring this layout strategy.

To create an instance of class vtk\-Cosmic\-Tree\-Layout\-Strategy, simply invoke its constructor as follows \begin{DoxyVerb}  obj = vtkCosmicTreeLayoutStrategy
\end{DoxyVerb}
 \hypertarget{vtkwidgets_vtkxyplotwidget_Methods}{}\subsection{Methods}\label{vtkwidgets_vtkxyplotwidget_Methods}
The class vtk\-Cosmic\-Tree\-Layout\-Strategy has several methods that can be used. They are listed below. Note that the documentation is translated automatically from the V\-T\-K sources, and may not be completely intelligible. When in doubt, consult the V\-T\-K website. In the methods listed below, {\ttfamily obj} is an instance of the vtk\-Cosmic\-Tree\-Layout\-Strategy class. 
\begin{DoxyItemize}
\item {\ttfamily string = obj.\-Get\-Class\-Name ()}  
\item {\ttfamily int = obj.\-Is\-A (string name)}  
\item {\ttfamily vtk\-Cosmic\-Tree\-Layout\-Strategy = obj.\-New\-Instance ()}  
\item {\ttfamily vtk\-Cosmic\-Tree\-Layout\-Strategy = obj.\-Safe\-Down\-Cast (vtk\-Object o)}  
\item {\ttfamily obj.\-Layout ()} -\/ Perform the layout.  
\item {\ttfamily obj.\-Set\-Size\-Leaf\-Nodes\-Only (int )} -\/ Should node size specifications be obeyed at leaf nodes only or (with scaling as required to meet constraints) at every node in the tree? This defaults to true, so that leaf nodes are scaled according to the size specification provided, and the parent node sizes are calculated by the algorithm.  
\item {\ttfamily int = obj.\-Get\-Size\-Leaf\-Nodes\-Only ()} -\/ Should node size specifications be obeyed at leaf nodes only or (with scaling as required to meet constraints) at every node in the tree? This defaults to true, so that leaf nodes are scaled according to the size specification provided, and the parent node sizes are calculated by the algorithm.  
\item {\ttfamily obj.\-Size\-Leaf\-Nodes\-Only\-On ()} -\/ Should node size specifications be obeyed at leaf nodes only or (with scaling as required to meet constraints) at every node in the tree? This defaults to true, so that leaf nodes are scaled according to the size specification provided, and the parent node sizes are calculated by the algorithm.  
\item {\ttfamily obj.\-Size\-Leaf\-Nodes\-Only\-Off ()} -\/ Should node size specifications be obeyed at leaf nodes only or (with scaling as required to meet constraints) at every node in the tree? This defaults to true, so that leaf nodes are scaled according to the size specification provided, and the parent node sizes are calculated by the algorithm.  
\item {\ttfamily obj.\-Set\-Layout\-Depth (int )} -\/ How many levels of the tree should be laid out? For large trees, you may wish to set the root and maximum depth in order to retrieve the layout for the visible portion of the tree. When this value is zero or negative, all nodes below and including the Layout\-Root will be presented. This defaults to 0.  
\item {\ttfamily int = obj.\-Get\-Layout\-Depth ()} -\/ How many levels of the tree should be laid out? For large trees, you may wish to set the root and maximum depth in order to retrieve the layout for the visible portion of the tree. When this value is zero or negative, all nodes below and including the Layout\-Root will be presented. This defaults to 0.  
\item {\ttfamily obj.\-Set\-Layout\-Root (vtk\-Id\-Type )} -\/ What is the top-\/most tree node to lay out? This node will become the largest containing circle in the layout. Use this in combination with Set\-Layout\-Depth to retrieve the layout of a subtree of interest for rendering. Setting Layout\-Root to a negative number signals that the root node of the tree should be used as the root node of the layout. This defaults to -\/1.  
\item {\ttfamily vtk\-Id\-Type = obj.\-Get\-Layout\-Root ()} -\/ What is the top-\/most tree node to lay out? This node will become the largest containing circle in the layout. Use this in combination with Set\-Layout\-Depth to retrieve the layout of a subtree of interest for rendering. Setting Layout\-Root to a negative number signals that the root node of the tree should be used as the root node of the layout. This defaults to -\/1.  
\item {\ttfamily obj.\-Set\-Node\-Size\-Array\-Name (string )} -\/ Set the array to be used for sizing nodes. If this is set to an empty string or N\-U\-L\-L (the default), then all leaf nodes (or all nodes, when Size\-Leaf\-Nodes\-Only is false) will be assigned a unit size.  
\item {\ttfamily string = obj.\-Get\-Node\-Size\-Array\-Name ()} -\/ Set the array to be used for sizing nodes. If this is set to an empty string or N\-U\-L\-L (the default), then all leaf nodes (or all nodes, when Size\-Leaf\-Nodes\-Only is false) will be assigned a unit size.  
\end{DoxyItemize}\hypertarget{vtkinfovis_vtkdataobjecttotable}{}\section{vtk\-Data\-Object\-To\-Table}\label{vtkinfovis_vtkdataobjecttotable}
Section\-: \hyperlink{sec_vtkinfovis}{Visualization Toolkit Infovis Classes} \hypertarget{vtkwidgets_vtkxyplotwidget_Usage}{}\subsection{Usage}\label{vtkwidgets_vtkxyplotwidget_Usage}
This filter is used to extract either the field, cell or point data of any data object as a table.

To create an instance of class vtk\-Data\-Object\-To\-Table, simply invoke its constructor as follows \begin{DoxyVerb}  obj = vtkDataObjectToTable
\end{DoxyVerb}
 \hypertarget{vtkwidgets_vtkxyplotwidget_Methods}{}\subsection{Methods}\label{vtkwidgets_vtkxyplotwidget_Methods}
The class vtk\-Data\-Object\-To\-Table has several methods that can be used. They are listed below. Note that the documentation is translated automatically from the V\-T\-K sources, and may not be completely intelligible. When in doubt, consult the V\-T\-K website. In the methods listed below, {\ttfamily obj} is an instance of the vtk\-Data\-Object\-To\-Table class. 
\begin{DoxyItemize}
\item {\ttfamily string = obj.\-Get\-Class\-Name ()}  
\item {\ttfamily int = obj.\-Is\-A (string name)}  
\item {\ttfamily vtk\-Data\-Object\-To\-Table = obj.\-New\-Instance ()}  
\item {\ttfamily vtk\-Data\-Object\-To\-Table = obj.\-Safe\-Down\-Cast (vtk\-Object o)}  
\item {\ttfamily int = obj.\-Get\-Field\-Type ()} -\/ The field type to copy into the output table. Should be one of F\-I\-E\-L\-D\-\_\-\-D\-A\-T\-A, P\-O\-I\-N\-T\-\_\-\-D\-A\-T\-A, C\-E\-L\-L\-\_\-\-D\-A\-T\-A, V\-E\-R\-T\-E\-X\-\_\-\-D\-A\-T\-A, E\-D\-G\-E\-\_\-\-D\-A\-T\-A.  
\item {\ttfamily obj.\-Set\-Field\-Type (int )} -\/ The field type to copy into the output table. Should be one of F\-I\-E\-L\-D\-\_\-\-D\-A\-T\-A, P\-O\-I\-N\-T\-\_\-\-D\-A\-T\-A, C\-E\-L\-L\-\_\-\-D\-A\-T\-A, V\-E\-R\-T\-E\-X\-\_\-\-D\-A\-T\-A, E\-D\-G\-E\-\_\-\-D\-A\-T\-A.  
\item {\ttfamily int = obj.\-Get\-Field\-Type\-Min\-Value ()} -\/ The field type to copy into the output table. Should be one of F\-I\-E\-L\-D\-\_\-\-D\-A\-T\-A, P\-O\-I\-N\-T\-\_\-\-D\-A\-T\-A, C\-E\-L\-L\-\_\-\-D\-A\-T\-A, V\-E\-R\-T\-E\-X\-\_\-\-D\-A\-T\-A, E\-D\-G\-E\-\_\-\-D\-A\-T\-A.  
\item {\ttfamily int = obj.\-Get\-Field\-Type\-Max\-Value ()} -\/ The field type to copy into the output table. Should be one of F\-I\-E\-L\-D\-\_\-\-D\-A\-T\-A, P\-O\-I\-N\-T\-\_\-\-D\-A\-T\-A, C\-E\-L\-L\-\_\-\-D\-A\-T\-A, V\-E\-R\-T\-E\-X\-\_\-\-D\-A\-T\-A, E\-D\-G\-E\-\_\-\-D\-A\-T\-A.  
\end{DoxyItemize}\hypertarget{vtkinfovis_vtkdelimitedtextreader}{}\section{vtk\-Delimited\-Text\-Reader}\label{vtkinfovis_vtkdelimitedtextreader}
Section\-: \hyperlink{sec_vtkinfovis}{Visualization Toolkit Infovis Classes} \hypertarget{vtkwidgets_vtkxyplotwidget_Usage}{}\subsection{Usage}\label{vtkwidgets_vtkxyplotwidget_Usage}
vtk\-Delimited\-Text\-Reader is an interface for pulling in data from a flat, delimited ascii or unicode text file (delimiter can be any character).

The behavior of the reader with respect to ascii or unicode input is controlled by the Set\-Unicode\-Character\-Set() method. By default (without calling Set\-Unicode\-Character\-Set()), the reader will expect to read ascii text and will output vtk\-Std\-String columns. Use the Set and Get methods to set delimiters that do not contain U\-T\-F8 in the name when operating the reader in default ascii mode. If the Set\-Unicode\-Character\-Set() method is called, the reader will output vtk\-Unicode\-String columns in the output table. In addition, it is necessary to use the Set and Get methods that contain U\-T\-F8 in the name to specify delimiters when operating in unicode mode.

This class emits Progress\-Event for every 100 lines it reads.

.S\-E\-C\-T\-I\-O\-N Thanks Thanks to Andy Wilson, Brian Wylie, Tim Shead, and Thomas Otahal from Sandia National Laboratories for implementing this class.

To create an instance of class vtk\-Delimited\-Text\-Reader, simply invoke its constructor as follows \begin{DoxyVerb}  obj = vtkDelimitedTextReader
\end{DoxyVerb}
 \hypertarget{vtkwidgets_vtkxyplotwidget_Methods}{}\subsection{Methods}\label{vtkwidgets_vtkxyplotwidget_Methods}
The class vtk\-Delimited\-Text\-Reader has several methods that can be used. They are listed below. Note that the documentation is translated automatically from the V\-T\-K sources, and may not be completely intelligible. When in doubt, consult the V\-T\-K website. In the methods listed below, {\ttfamily obj} is an instance of the vtk\-Delimited\-Text\-Reader class. 
\begin{DoxyItemize}
\item {\ttfamily string = obj.\-Get\-Class\-Name ()}  
\item {\ttfamily int = obj.\-Is\-A (string name)}  
\item {\ttfamily vtk\-Delimited\-Text\-Reader = obj.\-New\-Instance ()}  
\item {\ttfamily vtk\-Delimited\-Text\-Reader = obj.\-Safe\-Down\-Cast (vtk\-Object o)}  
\item {\ttfamily string = obj.\-Get\-File\-Name ()}  
\item {\ttfamily obj.\-Set\-File\-Name (string )}  
\item {\ttfamily string = obj.\-Get\-Unicode\-Character\-Set ()} -\/ Specifies the character set used in the input file. Valid character set names will be drawn from the list maintained by the Internet Assigned Name Authority at

\href{http://www.iana.org/assignments/character-sets}{\tt http\-://www.\-iana.\-org/assignments/character-\/sets}

Where multiple aliases are provided for a character set, the preferred M\-I\-M\-E name will be used. vtk\-Unicode\-Delimited\-Text\-Reader currently supports \char`\"{}\-U\-S-\/\-A\-S\-C\-I\-I\char`\"{}, \char`\"{}\-U\-T\-F-\/8\char`\"{}, \char`\"{}\-U\-T\-F-\/16\char`\"{}, \char`\"{}\-U\-T\-F-\/16\-B\-E\char`\"{}, and \char`\"{}\-U\-T\-F-\/16\-L\-E\char`\"{} character sets.  
\item {\ttfamily obj.\-Set\-Unicode\-Character\-Set (string )} -\/ Specifies the character set used in the input file. Valid character set names will be drawn from the list maintained by the Internet Assigned Name Authority at

\href{http://www.iana.org/assignments/character-sets}{\tt http\-://www.\-iana.\-org/assignments/character-\/sets}

Where multiple aliases are provided for a character set, the preferred M\-I\-M\-E name will be used. vtk\-Unicode\-Delimited\-Text\-Reader currently supports \char`\"{}\-U\-S-\/\-A\-S\-C\-I\-I\char`\"{}, \char`\"{}\-U\-T\-F-\/8\char`\"{}, \char`\"{}\-U\-T\-F-\/16\char`\"{}, \char`\"{}\-U\-T\-F-\/16\-B\-E\char`\"{}, and \char`\"{}\-U\-T\-F-\/16\-L\-E\char`\"{} character sets.  
\item {\ttfamily obj.\-Set\-U\-T\-F8\-Record\-Delimiters (string delimiters)} -\/ Specify the character(s) that will be used to separate records. The order of characters in the string does not matter. Defaults to \char`\"{}\textbackslash{}r\textbackslash{}n\char`\"{}.  
\item {\ttfamily string = obj.\-Get\-U\-T\-F8\-Record\-Delimiters ()} -\/ Specify the character(s) that will be used to separate records. The order of characters in the string does not matter. Defaults to \char`\"{}\textbackslash{}r\textbackslash{}n\char`\"{}.  
\item {\ttfamily obj.\-Set\-Field\-Delimiter\-Characters (string )} -\/ Specify the character(s) that will be used to separate fields. For example, set this to \char`\"{},\char`\"{} for a comma-\/separated value file. Set it to \char`\"{}.\-:;\char`\"{} for a file where columns can be separated by a period, colon or semicolon. The order of the characters in the string does not matter. Defaults to a comma.  
\item {\ttfamily string = obj.\-Get\-Field\-Delimiter\-Characters ()} -\/ Specify the character(s) that will be used to separate fields. For example, set this to \char`\"{},\char`\"{} for a comma-\/separated value file. Set it to \char`\"{}.\-:;\char`\"{} for a file where columns can be separated by a period, colon or semicolon. The order of the characters in the string does not matter. Defaults to a comma.  
\item {\ttfamily obj.\-Set\-U\-T\-F8\-Field\-Delimiters (string delimiters)}  
\item {\ttfamily string = obj.\-Get\-U\-T\-F8\-Field\-Delimiters ()}  
\item {\ttfamily char = obj.\-Get\-String\-Delimiter ()} -\/ Get/set the character that will begin and end strings. Microsoft Excel, for example, will export the following format\-:

\char`\"{}\-First Field\char`\"{},\char`\"{}\-Second Field\char`\"{},\char`\"{}\-Field, With, Commas\char`\"{},\char`\"{}\-Fourth Field\char`\"{}

The third field has a comma in it. By using a string delimiter, this will be correctly read. The delimiter defaults to '\char`\"{}'.  
\item {\ttfamily obj.\-Set\-String\-Delimiter (char )} -\/ Get/set the character that will begin and end strings. Microsoft Excel, for example, will export the following format\-:

\char`\"{}\-First Field\char`\"{},\char`\"{}\-Second Field\char`\"{},\char`\"{}\-Field, With, Commas\char`\"{},\char`\"{}\-Fourth Field\char`\"{}

The third field has a comma in it. By using a string delimiter, this will be correctly read. The delimiter defaults to '\char`\"{}'.  
\item {\ttfamily obj.\-Set\-U\-T\-F8\-String\-Delimiters (string delimiters)}  
\item {\ttfamily string = obj.\-Get\-U\-T\-F8\-String\-Delimiters ()}  
\item {\ttfamily obj.\-Set\-Use\-String\-Delimiter (bool )} -\/ Set/get whether to use the string delimiter. Defaults to on.  
\item {\ttfamily bool = obj.\-Get\-Use\-String\-Delimiter ()} -\/ Set/get whether to use the string delimiter. Defaults to on.  
\item {\ttfamily obj.\-Use\-String\-Delimiter\-On ()} -\/ Set/get whether to use the string delimiter. Defaults to on.  
\item {\ttfamily obj.\-Use\-String\-Delimiter\-Off ()} -\/ Set/get whether to use the string delimiter. Defaults to on.  
\item {\ttfamily bool = obj.\-Get\-Have\-Headers ()} -\/ Set/get whether to treat the first line of the file as headers.  
\item {\ttfamily obj.\-Set\-Have\-Headers (bool )} -\/ Set/get whether to treat the first line of the file as headers.  
\item {\ttfamily obj.\-Set\-Merge\-Consecutive\-Delimiters (bool )} -\/ Set/get whether to merge successive delimiters. Use this if (for example) your fields are separated by spaces but you don't know exactly how many.  
\item {\ttfamily bool = obj.\-Get\-Merge\-Consecutive\-Delimiters ()} -\/ Set/get whether to merge successive delimiters. Use this if (for example) your fields are separated by spaces but you don't know exactly how many.  
\item {\ttfamily obj.\-Merge\-Consecutive\-Delimiters\-On ()} -\/ Set/get whether to merge successive delimiters. Use this if (for example) your fields are separated by spaces but you don't know exactly how many.  
\item {\ttfamily obj.\-Merge\-Consecutive\-Delimiters\-Off ()} -\/ Set/get whether to merge successive delimiters. Use this if (for example) your fields are separated by spaces but you don't know exactly how many.  
\item {\ttfamily vtk\-Id\-Type = obj.\-Get\-Max\-Records ()} -\/ Specifies the maximum number of records to read from the file. Limiting the number of records to read is useful for previewing the contents of a file.  
\item {\ttfamily obj.\-Set\-Max\-Records (vtk\-Id\-Type )} -\/ Specifies the maximum number of records to read from the file. Limiting the number of records to read is useful for previewing the contents of a file.  
\item {\ttfamily obj.\-Set\-Detect\-Numeric\-Columns (bool )} -\/ When set to true, the reader will detect numeric columns and create vtk\-Double\-Array or vtk\-Int\-Array for those instead of vtk\-String\-Array. Default is off.  
\item {\ttfamily bool = obj.\-Get\-Detect\-Numeric\-Columns ()} -\/ When set to true, the reader will detect numeric columns and create vtk\-Double\-Array or vtk\-Int\-Array for those instead of vtk\-String\-Array. Default is off.  
\item {\ttfamily obj.\-Detect\-Numeric\-Columns\-On ()} -\/ When set to true, the reader will detect numeric columns and create vtk\-Double\-Array or vtk\-Int\-Array for those instead of vtk\-String\-Array. Default is off.  
\item {\ttfamily obj.\-Detect\-Numeric\-Columns\-Off ()} -\/ When set to true, the reader will detect numeric columns and create vtk\-Double\-Array or vtk\-Int\-Array for those instead of vtk\-String\-Array. Default is off.  
\item {\ttfamily obj.\-Set\-Pedigree\-Id\-Array\-Name (string )} -\/ The name of the array for generating or assigning pedigree ids (default \char`\"{}id\char`\"{}).  
\item {\ttfamily string = obj.\-Get\-Pedigree\-Id\-Array\-Name ()} -\/ The name of the array for generating or assigning pedigree ids (default \char`\"{}id\char`\"{}).  
\item {\ttfamily obj.\-Set\-Generate\-Pedigree\-Ids (bool )} -\/ If on (default), generates pedigree ids automatically. If off, assign one of the arrays to be the pedigree id.  
\item {\ttfamily bool = obj.\-Get\-Generate\-Pedigree\-Ids ()} -\/ If on (default), generates pedigree ids automatically. If off, assign one of the arrays to be the pedigree id.  
\item {\ttfamily obj.\-Generate\-Pedigree\-Ids\-On ()} -\/ If on (default), generates pedigree ids automatically. If off, assign one of the arrays to be the pedigree id.  
\item {\ttfamily obj.\-Generate\-Pedigree\-Ids\-Off ()} -\/ If on (default), generates pedigree ids automatically. If off, assign one of the arrays to be the pedigree id.  
\item {\ttfamily obj.\-Set\-Output\-Pedigree\-Ids (bool )} -\/ If on, assigns pedigree ids to output. Defaults to off.  
\item {\ttfamily bool = obj.\-Get\-Output\-Pedigree\-Ids ()} -\/ If on, assigns pedigree ids to output. Defaults to off.  
\item {\ttfamily obj.\-Output\-Pedigree\-Ids\-On ()} -\/ If on, assigns pedigree ids to output. Defaults to off.  
\item {\ttfamily obj.\-Output\-Pedigree\-Ids\-Off ()} -\/ If on, assigns pedigree ids to output. Defaults to off.  
\item {\ttfamily vtk\-Std\-String = obj.\-Get\-Last\-Error ()} -\/ Returns a human-\/readable description of the most recent error, if any. Otherwise, returns an empty string. Note that the result is only valid after calling Update().  
\end{DoxyItemize}\hypertarget{vtkinfovis_vtkdescriptivestatistics}{}\section{vtk\-Descriptive\-Statistics}\label{vtkinfovis_vtkdescriptivestatistics}
Section\-: \hyperlink{sec_vtkinfovis}{Visualization Toolkit Infovis Classes} \hypertarget{vtkwidgets_vtkxyplotwidget_Usage}{}\subsection{Usage}\label{vtkwidgets_vtkxyplotwidget_Usage}
Given a selection of columns of interest in an input data table, this class provides the following functionalities, depending on the execution mode it is executed in\-: Learn\-: calculate extremal values, arithmetic mean, unbiased variance estimator, skewness estimator, and both sample and G2 estimation of the kurtosis excess. More precisely, Learn calculates the sums; if {\ttfamily finalize} is set to true (default), the final statistics are calculated with Calculate\-From\-Sums. Otherwise, only raw sums are output; this option is made for efficient parallel calculations. Note that Calculate\-From\-Sums is a static function, so that it can be used directly with no need to instantiate a vtk\-Descriptive\-Statistics object. Assess\-: given an input data set in port I\-N\-P\-U\-T\-\_\-\-D\-A\-T\-A, and a reference value x along with an acceptable deviation d$>$0, assess all entries in the data set which are outside of \mbox{[}x-\/d,x+d\mbox{]}.

.S\-E\-C\-T\-I\-O\-N Thanks Thanks to Philippe Pebay and David Thompson from Sandia National Laboratories for implementing this class.

To create an instance of class vtk\-Descriptive\-Statistics, simply invoke its constructor as follows \begin{DoxyVerb}  obj = vtkDescriptiveStatistics
\end{DoxyVerb}
 \hypertarget{vtkwidgets_vtkxyplotwidget_Methods}{}\subsection{Methods}\label{vtkwidgets_vtkxyplotwidget_Methods}
The class vtk\-Descriptive\-Statistics has several methods that can be used. They are listed below. Note that the documentation is translated automatically from the V\-T\-K sources, and may not be completely intelligible. When in doubt, consult the V\-T\-K website. In the methods listed below, {\ttfamily obj} is an instance of the vtk\-Descriptive\-Statistics class. 
\begin{DoxyItemize}
\item {\ttfamily string = obj.\-Get\-Class\-Name ()}  
\item {\ttfamily int = obj.\-Is\-A (string name)}  
\item {\ttfamily vtk\-Descriptive\-Statistics = obj.\-New\-Instance ()}  
\item {\ttfamily vtk\-Descriptive\-Statistics = obj.\-Safe\-Down\-Cast (vtk\-Object o)}  
\item {\ttfamily obj.\-Set\-Unbiased\-Variance (int )} -\/ Set/get whether the unbiased estimator for the variance should be used, or if the population variance will be calculated. The default is that the unbiased estimator will be used.  
\item {\ttfamily int = obj.\-Get\-Unbiased\-Variance ()} -\/ Set/get whether the unbiased estimator for the variance should be used, or if the population variance will be calculated. The default is that the unbiased estimator will be used.  
\item {\ttfamily obj.\-Unbiased\-Variance\-On ()} -\/ Set/get whether the unbiased estimator for the variance should be used, or if the population variance will be calculated. The default is that the unbiased estimator will be used.  
\item {\ttfamily obj.\-Unbiased\-Variance\-Off ()} -\/ Set/get whether the unbiased estimator for the variance should be used, or if the population variance will be calculated. The default is that the unbiased estimator will be used.  
\item {\ttfamily obj.\-Set\-Signed\-Deviations (int )} -\/ Set/get whether the deviations returned should be signed, or should only have their magnitude reported. The default is that signed deviations will be computed.  
\item {\ttfamily int = obj.\-Get\-Signed\-Deviations ()} -\/ Set/get whether the deviations returned should be signed, or should only have their magnitude reported. The default is that signed deviations will be computed.  
\item {\ttfamily obj.\-Signed\-Deviations\-On ()} -\/ Set/get whether the deviations returned should be signed, or should only have their magnitude reported. The default is that signed deviations will be computed.  
\item {\ttfamily obj.\-Signed\-Deviations\-Off ()} -\/ Set/get whether the deviations returned should be signed, or should only have their magnitude reported. The default is that signed deviations will be computed.  
\item {\ttfamily obj.\-Set\-Nominal\-Parameter (string name)} -\/ A convenience method (in particular for U\-I wrapping) to set the name of the column that contains the nominal value for the Assess option.  
\item {\ttfamily obj.\-Set\-Deviation\-Parameter (string name)} -\/ A convenience method (in particular for U\-I wrapping) to set the name of the column that contains the deviation for the Assess option.  
\item {\ttfamily obj.\-Aggregate (vtk\-Data\-Object\-Collection , vtk\-Data\-Object )} -\/ Given a collection of models, calculate aggregate model  
\end{DoxyItemize}\hypertarget{vtkinfovis_vtkdotproductsimilarity}{}\section{vtk\-Dot\-Product\-Similarity}\label{vtkinfovis_vtkdotproductsimilarity}
Section\-: \hyperlink{sec_vtkinfovis}{Visualization Toolkit Infovis Classes} \hypertarget{vtkwidgets_vtkxyplotwidget_Usage}{}\subsection{Usage}\label{vtkwidgets_vtkxyplotwidget_Usage}
Treats matrices as collections of vectors and computes dot-\/product similarity metrics between vectors.

The results are returned as an edge-\/table that lists the index of each vector and their computed similarity. The output edge-\/table is typically used with vtk\-Table\-To\-Graph to create a similarity graph.

This filter can be used with one or two input matrices. If you provide a single matrix as input, every vector in the matrix is compared with every other vector. If you provide two matrices, every vector in the first matrix is compared with every vector in the second matrix.

Note that this filter {\itshape only} computes the dot-\/product between each pair of vectors; if you want to compute the cosine of the angles between vectors, you will need to normalize the inputs yourself.

Inputs\-: Input port 0\-: (required) A vtk\-Dense\-Array$<$double$>$ with two dimensions (a matrix). Input port 1\-: (optional) A vtk\-Dense\-Array$<$double$>$ with two dimensions (a matrix).

Outputs\-: Output port 0\-: A vtk\-Table containing \char`\"{}source\char`\"{}, \char`\"{}target\char`\"{}, and \char`\"{}similarity\char`\"{} columns.

To create an instance of class vtk\-Dot\-Product\-Similarity, simply invoke its constructor as follows \begin{DoxyVerb}  obj = vtkDotProductSimilarity
\end{DoxyVerb}
 \hypertarget{vtkwidgets_vtkxyplotwidget_Methods}{}\subsection{Methods}\label{vtkwidgets_vtkxyplotwidget_Methods}
The class vtk\-Dot\-Product\-Similarity has several methods that can be used. They are listed below. Note that the documentation is translated automatically from the V\-T\-K sources, and may not be completely intelligible. When in doubt, consult the V\-T\-K website. In the methods listed below, {\ttfamily obj} is an instance of the vtk\-Dot\-Product\-Similarity class. 
\begin{DoxyItemize}
\item {\ttfamily string = obj.\-Get\-Class\-Name ()}  
\item {\ttfamily int = obj.\-Is\-A (string name)}  
\item {\ttfamily vtk\-Dot\-Product\-Similarity = obj.\-New\-Instance ()}  
\item {\ttfamily vtk\-Dot\-Product\-Similarity = obj.\-Safe\-Down\-Cast (vtk\-Object o)}  
\item {\ttfamily vtk\-Id\-Type = obj.\-Get\-Vector\-Dimension ()} -\/ Controls whether to compute similarities for row-\/vectors or column-\/vectors. 0 = rows, 1 = columns.  
\item {\ttfamily obj.\-Set\-Vector\-Dimension (vtk\-Id\-Type )} -\/ Controls whether to compute similarities for row-\/vectors or column-\/vectors. 0 = rows, 1 = columns.  
\item {\ttfamily int = obj.\-Get\-Upper\-Diagonal ()} -\/ When computing similarities for a single input matrix, controls whether the results will include the upper diagonal of the similarity matrix. Default\-: true.  
\item {\ttfamily obj.\-Set\-Upper\-Diagonal (int )} -\/ When computing similarities for a single input matrix, controls whether the results will include the upper diagonal of the similarity matrix. Default\-: true.  
\item {\ttfamily int = obj.\-Get\-Diagonal ()} -\/ When computing similarities for a single input matrix, controls whether the results will include the diagonal of the similarity matrix. Default\-: false.  
\item {\ttfamily obj.\-Set\-Diagonal (int )} -\/ When computing similarities for a single input matrix, controls whether the results will include the diagonal of the similarity matrix. Default\-: false.  
\item {\ttfamily int = obj.\-Get\-Lower\-Diagonal ()} -\/ When computing similarities for a single input matrix, controls whether the results will include the lower diagonal of the similarity matrix. Default\-: false.  
\item {\ttfamily obj.\-Set\-Lower\-Diagonal (int )} -\/ When computing similarities for a single input matrix, controls whether the results will include the lower diagonal of the similarity matrix. Default\-: false.  
\item {\ttfamily int = obj.\-Get\-First\-Second ()} -\/ When computing similarities for two input matrices, controls whether the results will include comparisons from the first matrix to the second matrix.  
\item {\ttfamily obj.\-Set\-First\-Second (int )} -\/ When computing similarities for two input matrices, controls whether the results will include comparisons from the first matrix to the second matrix.  
\item {\ttfamily int = obj.\-Get\-Second\-First ()} -\/ When computing similarities for two input matrices, controls whether the results will include comparisons from the second matrix to the first matrix.  
\item {\ttfamily obj.\-Set\-Second\-First (int )} -\/ When computing similarities for two input matrices, controls whether the results will include comparisons from the second matrix to the first matrix.  
\item {\ttfamily double = obj.\-Get\-Minimum\-Threshold ()} -\/ Specifies a minimum threshold that a similarity must exceed to be included in the output.  
\item {\ttfamily obj.\-Set\-Minimum\-Threshold (double )} -\/ Specifies a minimum threshold that a similarity must exceed to be included in the output.  
\item {\ttfamily vtk\-Id\-Type = obj.\-Get\-Minimum\-Count ()} -\/ Specifies a minimum number of edges to include for each vector.  
\item {\ttfamily obj.\-Set\-Minimum\-Count (vtk\-Id\-Type )} -\/ Specifies a minimum number of edges to include for each vector.  
\item {\ttfamily vtk\-Id\-Type = obj.\-Get\-Maximum\-Count ()} -\/ Specifies a maximum number of edges to include for each vector.  
\item {\ttfamily obj.\-Set\-Maximum\-Count (vtk\-Id\-Type )} -\/ Specifies a maximum number of edges to include for each vector.  
\end{DoxyItemize}\hypertarget{vtkinfovis_vtkedgecenters}{}\section{vtk\-Edge\-Centers}\label{vtkinfovis_vtkedgecenters}
Section\-: \hyperlink{sec_vtkinfovis}{Visualization Toolkit Infovis Classes} \hypertarget{vtkwidgets_vtkxyplotwidget_Usage}{}\subsection{Usage}\label{vtkwidgets_vtkxyplotwidget_Usage}
vtk\-Edge\-Centers is a filter that takes as input any graph and generates on output points at the center of the cells in the dataset. These points can be used for placing glyphs (vtk\-Glyph3\-D) or labeling (vtk\-Labeled\-Data\-Mapper). (The center is the parametric center of the cell, not necessarily the geometric or bounding box center.) The edge attributes will be associated with the points on output.

To create an instance of class vtk\-Edge\-Centers, simply invoke its constructor as follows \begin{DoxyVerb}  obj = vtkEdgeCenters
\end{DoxyVerb}
 \hypertarget{vtkwidgets_vtkxyplotwidget_Methods}{}\subsection{Methods}\label{vtkwidgets_vtkxyplotwidget_Methods}
The class vtk\-Edge\-Centers has several methods that can be used. They are listed below. Note that the documentation is translated automatically from the V\-T\-K sources, and may not be completely intelligible. When in doubt, consult the V\-T\-K website. In the methods listed below, {\ttfamily obj} is an instance of the vtk\-Edge\-Centers class. 
\begin{DoxyItemize}
\item {\ttfamily string = obj.\-Get\-Class\-Name ()}  
\item {\ttfamily int = obj.\-Is\-A (string name)}  
\item {\ttfamily vtk\-Edge\-Centers = obj.\-New\-Instance ()}  
\item {\ttfamily vtk\-Edge\-Centers = obj.\-Safe\-Down\-Cast (vtk\-Object o)}  
\item {\ttfamily obj.\-Set\-Vertex\-Cells (int )} -\/ Enable/disable the generation of vertex cells.  
\item {\ttfamily int = obj.\-Get\-Vertex\-Cells ()} -\/ Enable/disable the generation of vertex cells.  
\item {\ttfamily obj.\-Vertex\-Cells\-On ()} -\/ Enable/disable the generation of vertex cells.  
\item {\ttfamily obj.\-Vertex\-Cells\-Off ()} -\/ Enable/disable the generation of vertex cells.  
\end{DoxyItemize}\hypertarget{vtkinfovis_vtkedgelayout}{}\section{vtk\-Edge\-Layout}\label{vtkinfovis_vtkedgelayout}
Section\-: \hyperlink{sec_vtkinfovis}{Visualization Toolkit Infovis Classes} \hypertarget{vtkwidgets_vtkxyplotwidget_Usage}{}\subsection{Usage}\label{vtkwidgets_vtkxyplotwidget_Usage}
This class is a shell for many edge layout strategies which may be set using the Set\-Layout\-Strategy() function. The layout strategies do the actual work.

To create an instance of class vtk\-Edge\-Layout, simply invoke its constructor as follows \begin{DoxyVerb}  obj = vtkEdgeLayout
\end{DoxyVerb}
 \hypertarget{vtkwidgets_vtkxyplotwidget_Methods}{}\subsection{Methods}\label{vtkwidgets_vtkxyplotwidget_Methods}
The class vtk\-Edge\-Layout has several methods that can be used. They are listed below. Note that the documentation is translated automatically from the V\-T\-K sources, and may not be completely intelligible. When in doubt, consult the V\-T\-K website. In the methods listed below, {\ttfamily obj} is an instance of the vtk\-Edge\-Layout class. 
\begin{DoxyItemize}
\item {\ttfamily string = obj.\-Get\-Class\-Name ()}  
\item {\ttfamily int = obj.\-Is\-A (string name)}  
\item {\ttfamily vtk\-Edge\-Layout = obj.\-New\-Instance ()}  
\item {\ttfamily vtk\-Edge\-Layout = obj.\-Safe\-Down\-Cast (vtk\-Object o)}  
\item {\ttfamily obj.\-Set\-Layout\-Strategy (vtk\-Edge\-Layout\-Strategy strategy)} -\/ The layout strategy to use during graph layout.  
\item {\ttfamily vtk\-Edge\-Layout\-Strategy = obj.\-Get\-Layout\-Strategy ()} -\/ The layout strategy to use during graph layout.  
\item {\ttfamily long = obj.\-Get\-M\-Time ()} -\/ Get the modification time of the layout algorithm.  
\end{DoxyItemize}\hypertarget{vtkinfovis_vtkedgelayoutstrategy}{}\section{vtk\-Edge\-Layout\-Strategy}\label{vtkinfovis_vtkedgelayoutstrategy}
Section\-: \hyperlink{sec_vtkinfovis}{Visualization Toolkit Infovis Classes} \hypertarget{vtkwidgets_vtkxyplotwidget_Usage}{}\subsection{Usage}\label{vtkwidgets_vtkxyplotwidget_Usage}
All edge layouts should subclass from this class. vtk\-Edge\-Layout\-Strategy works as a plug-\/in to the vtk\-Edge\-Layout algorithm.

To create an instance of class vtk\-Edge\-Layout\-Strategy, simply invoke its constructor as follows \begin{DoxyVerb}  obj = vtkEdgeLayoutStrategy
\end{DoxyVerb}
 \hypertarget{vtkwidgets_vtkxyplotwidget_Methods}{}\subsection{Methods}\label{vtkwidgets_vtkxyplotwidget_Methods}
The class vtk\-Edge\-Layout\-Strategy has several methods that can be used. They are listed below. Note that the documentation is translated automatically from the V\-T\-K sources, and may not be completely intelligible. When in doubt, consult the V\-T\-K website. In the methods listed below, {\ttfamily obj} is an instance of the vtk\-Edge\-Layout\-Strategy class. 
\begin{DoxyItemize}
\item {\ttfamily string = obj.\-Get\-Class\-Name ()}  
\item {\ttfamily int = obj.\-Is\-A (string name)}  
\item {\ttfamily vtk\-Edge\-Layout\-Strategy = obj.\-New\-Instance ()}  
\item {\ttfamily vtk\-Edge\-Layout\-Strategy = obj.\-Safe\-Down\-Cast (vtk\-Object o)}  
\item {\ttfamily obj.\-Set\-Graph (vtk\-Graph graph)} -\/ Setting the graph for the layout strategy  
\item {\ttfamily obj.\-Initialize ()} -\/ This method allows the layout strategy to do initialization of data structures or whatever else it might want to do.  
\item {\ttfamily obj.\-Layout ()} -\/ This is the layout method where the graph that was set in Set\-Graph() is laid out.  
\item {\ttfamily obj.\-Set\-Edge\-Weight\-Array\-Name (string )} -\/ Set/\-Get the field to use for the edge weights.  
\item {\ttfamily string = obj.\-Get\-Edge\-Weight\-Array\-Name ()} -\/ Set/\-Get the field to use for the edge weights.  
\end{DoxyItemize}\hypertarget{vtkinfovis_vtkexpandselectedgraph}{}\section{vtk\-Expand\-Selected\-Graph}\label{vtkinfovis_vtkexpandselectedgraph}
Section\-: \hyperlink{sec_vtkinfovis}{Visualization Toolkit Infovis Classes} \hypertarget{vtkwidgets_vtkxyplotwidget_Usage}{}\subsection{Usage}\label{vtkwidgets_vtkxyplotwidget_Usage}
The first input is a vtk\-Selection containing the selected vertices. The second input is a vtk\-Graph. This filter 'grows' the selection set in one of the following ways 1) Set\-B\-F\-S\-Distance controls how many 'hops' the selection is grown from each seed point in the selection set (defaults to 1) 2) Include\-Shortest\-Paths controls whether this filter tries to 'connect' the vertices in the selection set by computing the shortest path between the vertices (if such a path exists) Note\-: Include\-Shortest\-Paths is currently non-\/functional

To create an instance of class vtk\-Expand\-Selected\-Graph, simply invoke its constructor as follows \begin{DoxyVerb}  obj = vtkExpandSelectedGraph
\end{DoxyVerb}
 \hypertarget{vtkwidgets_vtkxyplotwidget_Methods}{}\subsection{Methods}\label{vtkwidgets_vtkxyplotwidget_Methods}
The class vtk\-Expand\-Selected\-Graph has several methods that can be used. They are listed below. Note that the documentation is translated automatically from the V\-T\-K sources, and may not be completely intelligible. When in doubt, consult the V\-T\-K website. In the methods listed below, {\ttfamily obj} is an instance of the vtk\-Expand\-Selected\-Graph class. 
\begin{DoxyItemize}
\item {\ttfamily string = obj.\-Get\-Class\-Name ()}  
\item {\ttfamily int = obj.\-Is\-A (string name)}  
\item {\ttfamily vtk\-Expand\-Selected\-Graph = obj.\-New\-Instance ()}  
\item {\ttfamily vtk\-Expand\-Selected\-Graph = obj.\-Safe\-Down\-Cast (vtk\-Object o)}  
\item {\ttfamily obj.\-Set\-Graph\-Connection (vtk\-Algorithm\-Output in)} -\/ A convenience method for setting the second input (i.\-e. the graph).  
\item {\ttfamily int = obj.\-Fill\-Input\-Port\-Information (int port, vtk\-Information info)} -\/ Specify the first vtk\-Selection input and the second vtk\-Graph input.  
\item {\ttfamily obj.\-Set\-B\-F\-S\-Distance (int )} -\/ Set/\-Get B\-F\-S\-Distance which controls how many 'hops' the selection is grown from each seed point in the selection set (defaults to 1)  
\item {\ttfamily int = obj.\-Get\-B\-F\-S\-Distance ()} -\/ Set/\-Get B\-F\-S\-Distance which controls how many 'hops' the selection is grown from each seed point in the selection set (defaults to 1)  
\item {\ttfamily obj.\-Set\-Include\-Shortest\-Paths (bool )} -\/ Set/\-Get Include\-Shortest\-Paths controls whether this filter tries to 'connect' the vertices in the selection set by computing the shortest path between the vertices (if such a path exists) Note\-: Include\-Shortest\-Paths is currently non-\/functional  
\item {\ttfamily bool = obj.\-Get\-Include\-Shortest\-Paths ()} -\/ Set/\-Get Include\-Shortest\-Paths controls whether this filter tries to 'connect' the vertices in the selection set by computing the shortest path between the vertices (if such a path exists) Note\-: Include\-Shortest\-Paths is currently non-\/functional  
\item {\ttfamily obj.\-Include\-Shortest\-Paths\-On ()} -\/ Set/\-Get Include\-Shortest\-Paths controls whether this filter tries to 'connect' the vertices in the selection set by computing the shortest path between the vertices (if such a path exists) Note\-: Include\-Shortest\-Paths is currently non-\/functional  
\item {\ttfamily obj.\-Include\-Shortest\-Paths\-Off ()} -\/ Set/\-Get Include\-Shortest\-Paths controls whether this filter tries to 'connect' the vertices in the selection set by computing the shortest path between the vertices (if such a path exists) Note\-: Include\-Shortest\-Paths is currently non-\/functional  
\item {\ttfamily obj.\-Set\-Domain (string )} -\/ Set/\-Get the vertex domain to use in the expansion.  
\item {\ttfamily string = obj.\-Get\-Domain ()} -\/ Set/\-Get the vertex domain to use in the expansion.  
\item {\ttfamily obj.\-Set\-Use\-Domain (bool )} -\/ Whether or not to use the domain when deciding to add a vertex to the expansion. Defaults to false.  
\item {\ttfamily bool = obj.\-Get\-Use\-Domain ()} -\/ Whether or not to use the domain when deciding to add a vertex to the expansion. Defaults to false.  
\item {\ttfamily obj.\-Use\-Domain\-On ()} -\/ Whether or not to use the domain when deciding to add a vertex to the expansion. Defaults to false.  
\item {\ttfamily obj.\-Use\-Domain\-Off ()} -\/ Whether or not to use the domain when deciding to add a vertex to the expansion. Defaults to false.  
\end{DoxyItemize}\hypertarget{vtkinfovis_vtkextracthistogram2d}{}\section{vtk\-Extract\-Histogram2\-D}\label{vtkinfovis_vtkextracthistogram2d}
Section\-: \hyperlink{sec_vtkinfovis}{Visualization Toolkit Infovis Classes} \hypertarget{vtkwidgets_vtkxyplotwidget_Usage}{}\subsection{Usage}\label{vtkwidgets_vtkxyplotwidget_Usage}
This class computes a 2\-D histogram between two columns of an input vtk\-Table. Just as with a 1\-D histogram, a 2\-D histogram breaks up the input domain into bins, and each pair of values (row in the table) fits into a single bin and increments a row counter for that bin.

To use this class, set the input with a table and call Add\-Column\-Pair(name\-X,name\-Y), where name\-X and name\-Y are the names of the two columns to be used.

In addition to the number of bins (in X and Y), the domain of the histogram can be customized by toggling the Use\-Custom\-Histogram\-Extents flag and setting the Custom\-Histogram\-Extents variable to the desired value.

To create an instance of class vtk\-Extract\-Histogram2\-D, simply invoke its constructor as follows \begin{DoxyVerb}  obj = vtkExtractHistogram2D
\end{DoxyVerb}
 \hypertarget{vtkwidgets_vtkxyplotwidget_Methods}{}\subsection{Methods}\label{vtkwidgets_vtkxyplotwidget_Methods}
The class vtk\-Extract\-Histogram2\-D has several methods that can be used. They are listed below. Note that the documentation is translated automatically from the V\-T\-K sources, and may not be completely intelligible. When in doubt, consult the V\-T\-K website. In the methods listed below, {\ttfamily obj} is an instance of the vtk\-Extract\-Histogram2\-D class. 
\begin{DoxyItemize}
\item {\ttfamily string = obj.\-Get\-Class\-Name ()}  
\item {\ttfamily int = obj.\-Is\-A (string name)}  
\item {\ttfamily vtk\-Extract\-Histogram2\-D = obj.\-New\-Instance ()}  
\item {\ttfamily vtk\-Extract\-Histogram2\-D = obj.\-Safe\-Down\-Cast (vtk\-Object o)}  
\item {\ttfamily obj.\-Set\-Number\-Of\-Bins (int , int )} -\/ Set/get the number of bins to be used per dimension (x,y)  
\item {\ttfamily obj.\-Set\-Number\-Of\-Bins (int a\mbox{[}2\mbox{]})} -\/ Set/get the number of bins to be used per dimension (x,y)  
\item {\ttfamily int = obj. Get\-Number\-Of\-Bins ()} -\/ Set/get the number of bins to be used per dimension (x,y)  
\item {\ttfamily obj.\-Set\-Components\-To\-Process (int , int )} -\/ Set/get the components of the arrays in the two input columns to be used during histogram computation. Defaults to component 0.  
\item {\ttfamily obj.\-Set\-Components\-To\-Process (int a\mbox{[}2\mbox{]})} -\/ Set/get the components of the arrays in the two input columns to be used during histogram computation. Defaults to component 0.  
\item {\ttfamily int = obj. Get\-Components\-To\-Process ()} -\/ Set/get the components of the arrays in the two input columns to be used during histogram computation. Defaults to component 0.  
\item {\ttfamily obj.\-Set\-Custom\-Histogram\-Extents (double , double , double , double )} -\/ Set/get a custom domain for histogram computation. Use\-Custom\-Histogram\-Extents must be called for these to actually be used.  
\item {\ttfamily obj.\-Set\-Custom\-Histogram\-Extents (double a\mbox{[}4\mbox{]})} -\/ Set/get a custom domain for histogram computation. Use\-Custom\-Histogram\-Extents must be called for these to actually be used.  
\item {\ttfamily double = obj. Get\-Custom\-Histogram\-Extents ()} -\/ Set/get a custom domain for histogram computation. Use\-Custom\-Histogram\-Extents must be called for these to actually be used.  
\item {\ttfamily obj.\-Set\-Use\-Custom\-Histogram\-Extents (int )} -\/ Use the extents in Custom\-Histogram\-Extents when computing the histogram, rather than the simple range of the input columns.  
\item {\ttfamily int = obj.\-Get\-Use\-Custom\-Histogram\-Extents ()} -\/ Use the extents in Custom\-Histogram\-Extents when computing the histogram, rather than the simple range of the input columns.  
\item {\ttfamily obj.\-Use\-Custom\-Histogram\-Extents\-On ()} -\/ Use the extents in Custom\-Histogram\-Extents when computing the histogram, rather than the simple range of the input columns.  
\item {\ttfamily obj.\-Use\-Custom\-Histogram\-Extents\-Off ()} -\/ Use the extents in Custom\-Histogram\-Extents when computing the histogram, rather than the simple range of the input columns.  
\item {\ttfamily obj.\-Set\-Scalar\-Type (int )} -\/ Control the scalar type of the output histogram. If the input is relatively small, you can save space by using a smaller data type. Defaults to unsigned integer.  
\item {\ttfamily obj.\-Set\-Scalar\-Type\-To\-Unsigned\-Int ()} -\/ Control the scalar type of the output histogram. If the input is relatively small, you can save space by using a smaller data type. Defaults to unsigned integer.  
\item {\ttfamily obj.\-Set\-Scalar\-Type\-To\-Unsigned\-Long ()} -\/ Control the scalar type of the output histogram. If the input is relatively small, you can save space by using a smaller data type. Defaults to unsigned integer.  
\item {\ttfamily obj.\-Set\-Scalar\-Type\-To\-Unsigned\-Short ()} -\/ Control the scalar type of the output histogram. If the input is relatively small, you can save space by using a smaller data type. Defaults to unsigned integer.  
\item {\ttfamily obj.\-Set\-Scalar\-Type\-To\-Unsigned\-Char ()} -\/ Control the scalar type of the output histogram. If the input is relatively small, you can save space by using a smaller data type. Defaults to unsigned integer.  
\item {\ttfamily obj.\-Set\-Scalar\-Type\-To\-Float ()} -\/ Control the scalar type of the output histogram. If the input is relatively small, you can save space by using a smaller data type. Defaults to unsigned integer.  
\item {\ttfamily obj.\-Set\-Scalar\-Type\-To\-Double ()} -\/ Control the scalar type of the output histogram. If the input is relatively small, you can save space by using a smaller data type. Defaults to unsigned integer.  
\item {\ttfamily int = obj.\-Get\-Scalar\-Type ()} -\/ Control the scalar type of the output histogram. If the input is relatively small, you can save space by using a smaller data type. Defaults to unsigned integer.  
\item {\ttfamily double = obj.\-Get\-Maximum\-Bin\-Count ()} -\/ Access the count of the histogram bin containing the largest number of input rows.  
\item {\ttfamily int = obj.\-Get\-Bin\-Range (vtk\-Id\-Type bin\-X, vtk\-Id\-Type bin\-Y, double range\mbox{[}4\mbox{]})} -\/ Compute the range of the bin located at position (bin\-X,bin\-Y) in the 2\-D histogram.  
\item {\ttfamily int = obj.\-Get\-Bin\-Range (vtk\-Id\-Type bin, double range\mbox{[}4\mbox{]})} -\/ Get the range of the of the bin located at 1\-D position index bin in the 2\-D histogram array.  
\item {\ttfamily obj.\-Get\-Bin\-Width (double bw\mbox{[}2\mbox{]})} -\/ Get the width of all of the bins. Also stored in the spacing ivar of the histogram image output.  
\item {\ttfamily vtk\-Image\-Data = obj.\-Get\-Output\-Histogram\-Image ()} -\/ Gets the data object at the histogram image output port and casts it to a vtk\-Image\-Data.  
\item {\ttfamily obj.\-Set\-Swap\-Columns (int )}  
\item {\ttfamily int = obj.\-Get\-Swap\-Columns ()}  
\item {\ttfamily obj.\-Swap\-Columns\-On ()}  
\item {\ttfamily obj.\-Swap\-Columns\-Off ()}  
\item {\ttfamily obj.\-Set\-Row\-Mask (vtk\-Data\-Array )} -\/ Get/\-Set an optional mask that can ignore rows of the table  
\item {\ttfamily vtk\-Data\-Array = obj.\-Get\-Row\-Mask ()} -\/ Get/\-Set an optional mask that can ignore rows of the table  
\item {\ttfamily obj.\-Aggregate (vtk\-Data\-Object\-Collection , vtk\-Data\-Object )} -\/ Given a collection of models, calculate aggregate model. Not used.  
\end{DoxyItemize}\hypertarget{vtkinfovis_vtkextractselectedgraph}{}\section{vtk\-Extract\-Selected\-Graph}\label{vtkinfovis_vtkextractselectedgraph}
Section\-: \hyperlink{sec_vtkinfovis}{Visualization Toolkit Infovis Classes} \hypertarget{vtkwidgets_vtkxyplotwidget_Usage}{}\subsection{Usage}\label{vtkwidgets_vtkxyplotwidget_Usage}
The first input is a vtk\-Graph to take a subgraph from. The second input (optional) is a vtk\-Selection containing selected indices. The third input (optional) is a vtk\-Annotations\-Layers whose annotations contain selected specifying selected indices. The vtk\-Selection may have F\-I\-E\-L\-D\-\_\-\-T\-Y\-P\-E set to P\-O\-I\-N\-T\-S (a vertex selection) or C\-E\-L\-L\-S (an edge selection). A vertex selection preserves all edges that connect selected vertices. An edge selection preserves all vertices that are adjacent to at least one selected edge. Alternately, you may indicate that an edge selection should maintain the full set of vertices, by turning Remove\-Isolated\-Vertices off.

To create an instance of class vtk\-Extract\-Selected\-Graph, simply invoke its constructor as follows \begin{DoxyVerb}  obj = vtkExtractSelectedGraph
\end{DoxyVerb}
 \hypertarget{vtkwidgets_vtkxyplotwidget_Methods}{}\subsection{Methods}\label{vtkwidgets_vtkxyplotwidget_Methods}
The class vtk\-Extract\-Selected\-Graph has several methods that can be used. They are listed below. Note that the documentation is translated automatically from the V\-T\-K sources, and may not be completely intelligible. When in doubt, consult the V\-T\-K website. In the methods listed below, {\ttfamily obj} is an instance of the vtk\-Extract\-Selected\-Graph class. 
\begin{DoxyItemize}
\item {\ttfamily string = obj.\-Get\-Class\-Name ()}  
\item {\ttfamily int = obj.\-Is\-A (string name)}  
\item {\ttfamily vtk\-Extract\-Selected\-Graph = obj.\-New\-Instance ()}  
\item {\ttfamily vtk\-Extract\-Selected\-Graph = obj.\-Safe\-Down\-Cast (vtk\-Object o)}  
\item {\ttfamily obj.\-Set\-Selection\-Connection (vtk\-Algorithm\-Output in)} -\/ A convenience method for setting the second input (i.\-e. the selection).  
\item {\ttfamily obj.\-Set\-Annotation\-Layers\-Connection (vtk\-Algorithm\-Output in)} -\/ A convenience method for setting the third input (i.\-e. the annotation layers).  
\item {\ttfamily obj.\-Set\-Remove\-Isolated\-Vertices (bool )} -\/ If set, removes vertices with no adjacent edges in an edge selection. A vertex selection ignores this flag and always returns the full set of selected vertices. Default is on.  
\item {\ttfamily bool = obj.\-Get\-Remove\-Isolated\-Vertices ()} -\/ If set, removes vertices with no adjacent edges in an edge selection. A vertex selection ignores this flag and always returns the full set of selected vertices. Default is on.  
\item {\ttfamily obj.\-Remove\-Isolated\-Vertices\-On ()} -\/ If set, removes vertices with no adjacent edges in an edge selection. A vertex selection ignores this flag and always returns the full set of selected vertices. Default is on.  
\item {\ttfamily obj.\-Remove\-Isolated\-Vertices\-Off ()} -\/ If set, removes vertices with no adjacent edges in an edge selection. A vertex selection ignores this flag and always returns the full set of selected vertices. Default is on.  
\item {\ttfamily int = obj.\-Fill\-Input\-Port\-Information (int port, vtk\-Information info)} -\/ Specify the first vtk\-Graph input and the second vtk\-Selection input.  
\end{DoxyItemize}\hypertarget{vtkinfovis_vtkfast2dlayoutstrategy}{}\section{vtk\-Fast2\-D\-Layout\-Strategy}\label{vtkinfovis_vtkfast2dlayoutstrategy}
Section\-: \hyperlink{sec_vtkinfovis}{Visualization Toolkit Infovis Classes} \hypertarget{vtkwidgets_vtkxyplotwidget_Usage}{}\subsection{Usage}\label{vtkwidgets_vtkxyplotwidget_Usage}
This class is a density grid based force directed layout strategy. Also please note that 'fast' is relative to quite slow. \-:) The layout running time is O(V+\-E) with an extremely high constant. .S\-E\-C\-T\-I\-O\-N Thanks Thanks to Godzilla for not eating my computer so that this class could be written.

To create an instance of class vtk\-Fast2\-D\-Layout\-Strategy, simply invoke its constructor as follows \begin{DoxyVerb}  obj = vtkFast2DLayoutStrategy
\end{DoxyVerb}
 \hypertarget{vtkwidgets_vtkxyplotwidget_Methods}{}\subsection{Methods}\label{vtkwidgets_vtkxyplotwidget_Methods}
The class vtk\-Fast2\-D\-Layout\-Strategy has several methods that can be used. They are listed below. Note that the documentation is translated automatically from the V\-T\-K sources, and may not be completely intelligible. When in doubt, consult the V\-T\-K website. In the methods listed below, {\ttfamily obj} is an instance of the vtk\-Fast2\-D\-Layout\-Strategy class. 
\begin{DoxyItemize}
\item {\ttfamily string = obj.\-Get\-Class\-Name ()}  
\item {\ttfamily int = obj.\-Is\-A (string name)}  
\item {\ttfamily vtk\-Fast2\-D\-Layout\-Strategy = obj.\-New\-Instance ()}  
\item {\ttfamily vtk\-Fast2\-D\-Layout\-Strategy = obj.\-Safe\-Down\-Cast (vtk\-Object o)}  
\item {\ttfamily obj.\-Set\-Random\-Seed (int )} -\/ Seed the random number generator used to jitter point positions. This has a significant effect on their final positions when the layout is complete.  
\item {\ttfamily int = obj.\-Get\-Random\-Seed\-Min\-Value ()} -\/ Seed the random number generator used to jitter point positions. This has a significant effect on their final positions when the layout is complete.  
\item {\ttfamily int = obj.\-Get\-Random\-Seed\-Max\-Value ()} -\/ Seed the random number generator used to jitter point positions. This has a significant effect on their final positions when the layout is complete.  
\item {\ttfamily int = obj.\-Get\-Random\-Seed ()} -\/ Seed the random number generator used to jitter point positions. This has a significant effect on their final positions when the layout is complete.  
\item {\ttfamily obj.\-Set\-Max\-Number\-Of\-Iterations (int )} -\/ Set/\-Get the maximum number of iterations to be used. The higher this number, the more iterations through the algorithm is possible, and thus, the more the graph gets modified. The default is '100' for no particular reason Note\-: The strong recommendation is that you do not change this parameter. \-:)  
\item {\ttfamily int = obj.\-Get\-Max\-Number\-Of\-Iterations\-Min\-Value ()} -\/ Set/\-Get the maximum number of iterations to be used. The higher this number, the more iterations through the algorithm is possible, and thus, the more the graph gets modified. The default is '100' for no particular reason Note\-: The strong recommendation is that you do not change this parameter. \-:)  
\item {\ttfamily int = obj.\-Get\-Max\-Number\-Of\-Iterations\-Max\-Value ()} -\/ Set/\-Get the maximum number of iterations to be used. The higher this number, the more iterations through the algorithm is possible, and thus, the more the graph gets modified. The default is '100' for no particular reason Note\-: The strong recommendation is that you do not change this parameter. \-:)  
\item {\ttfamily int = obj.\-Get\-Max\-Number\-Of\-Iterations ()} -\/ Set/\-Get the maximum number of iterations to be used. The higher this number, the more iterations through the algorithm is possible, and thus, the more the graph gets modified. The default is '100' for no particular reason Note\-: The strong recommendation is that you do not change this parameter. \-:)  
\item {\ttfamily obj.\-Set\-Iterations\-Per\-Layout (int )} -\/ Set/\-Get the number of iterations per layout. The only use for this ivar is for the application to do visualizations of the layout before it's complete. The default is '100' to match the default 'Max\-Number\-Of\-Iterations' Note\-: Changing this parameter is just fine \-:)  
\item {\ttfamily int = obj.\-Get\-Iterations\-Per\-Layout\-Min\-Value ()} -\/ Set/\-Get the number of iterations per layout. The only use for this ivar is for the application to do visualizations of the layout before it's complete. The default is '100' to match the default 'Max\-Number\-Of\-Iterations' Note\-: Changing this parameter is just fine \-:)  
\item {\ttfamily int = obj.\-Get\-Iterations\-Per\-Layout\-Max\-Value ()} -\/ Set/\-Get the number of iterations per layout. The only use for this ivar is for the application to do visualizations of the layout before it's complete. The default is '100' to match the default 'Max\-Number\-Of\-Iterations' Note\-: Changing this parameter is just fine \-:)  
\item {\ttfamily int = obj.\-Get\-Iterations\-Per\-Layout ()} -\/ Set/\-Get the number of iterations per layout. The only use for this ivar is for the application to do visualizations of the layout before it's complete. The default is '100' to match the default 'Max\-Number\-Of\-Iterations' Note\-: Changing this parameter is just fine \-:)  
\item {\ttfamily obj.\-Set\-Initial\-Temperature (float )} -\/ Set the initial temperature. The temperature default is '5' for no particular reason Note\-: The strong recommendation is that you do not change this parameter. \-:)  
\item {\ttfamily float = obj.\-Get\-Initial\-Temperature\-Min\-Value ()} -\/ Set the initial temperature. The temperature default is '5' for no particular reason Note\-: The strong recommendation is that you do not change this parameter. \-:)  
\item {\ttfamily float = obj.\-Get\-Initial\-Temperature\-Max\-Value ()} -\/ Set the initial temperature. The temperature default is '5' for no particular reason Note\-: The strong recommendation is that you do not change this parameter. \-:)  
\item {\ttfamily float = obj.\-Get\-Initial\-Temperature ()} -\/ Set the initial temperature. The temperature default is '5' for no particular reason Note\-: The strong recommendation is that you do not change this parameter. \-:)  
\item {\ttfamily obj.\-Set\-Cool\-Down\-Rate (double )} -\/ Set/\-Get the Cool-\/down rate. The higher this number is, the longer it will take to \char`\"{}cool-\/down\char`\"{}, and thus, the more the graph will be modified. The default is '10' for no particular reason. Note\-: The strong recommendation is that you do not change this parameter. \-:)  
\item {\ttfamily double = obj.\-Get\-Cool\-Down\-Rate\-Min\-Value ()} -\/ Set/\-Get the Cool-\/down rate. The higher this number is, the longer it will take to \char`\"{}cool-\/down\char`\"{}, and thus, the more the graph will be modified. The default is '10' for no particular reason. Note\-: The strong recommendation is that you do not change this parameter. \-:)  
\item {\ttfamily double = obj.\-Get\-Cool\-Down\-Rate\-Max\-Value ()} -\/ Set/\-Get the Cool-\/down rate. The higher this number is, the longer it will take to \char`\"{}cool-\/down\char`\"{}, and thus, the more the graph will be modified. The default is '10' for no particular reason. Note\-: The strong recommendation is that you do not change this parameter. \-:)  
\item {\ttfamily double = obj.\-Get\-Cool\-Down\-Rate ()} -\/ Set/\-Get the Cool-\/down rate. The higher this number is, the longer it will take to \char`\"{}cool-\/down\char`\"{}, and thus, the more the graph will be modified. The default is '10' for no particular reason. Note\-: The strong recommendation is that you do not change this parameter. \-:)  
\item {\ttfamily obj.\-Set\-Rest\-Distance (float )} -\/ Manually set the resting distance. Otherwise the distance is computed automatically.  
\item {\ttfamily float = obj.\-Get\-Rest\-Distance ()} -\/ Manually set the resting distance. Otherwise the distance is computed automatically.  
\item {\ttfamily obj.\-Initialize ()} -\/ This strategy sets up some data structures for faster processing of each Layout() call  
\item {\ttfamily obj.\-Layout ()} -\/ This is the layout method where the graph that was set in Set\-Graph() is laid out. The method can either entirely layout the graph or iteratively lay out the graph. If you have an iterative layout please implement the Is\-Layout\-Complete() method.  
\item {\ttfamily int = obj.\-Is\-Layout\-Complete ()}  
\end{DoxyItemize}\hypertarget{vtkinfovis_vtkfixedwidthtextreader}{}\section{vtk\-Fixed\-Width\-Text\-Reader}\label{vtkinfovis_vtkfixedwidthtextreader}
Section\-: \hyperlink{sec_vtkinfovis}{Visualization Toolkit Infovis Classes} \hypertarget{vtkwidgets_vtkxyplotwidget_Usage}{}\subsection{Usage}\label{vtkwidgets_vtkxyplotwidget_Usage}
vtk\-Fixed\-Width\-Text\-Reader reads in a table from a text file where each column occupies a certain number of characters.

This class emits Progress\-Event for every 100 lines it reads.

To create an instance of class vtk\-Fixed\-Width\-Text\-Reader, simply invoke its constructor as follows \begin{DoxyVerb}  obj = vtkFixedWidthTextReader
\end{DoxyVerb}
 \hypertarget{vtkwidgets_vtkxyplotwidget_Methods}{}\subsection{Methods}\label{vtkwidgets_vtkxyplotwidget_Methods}
The class vtk\-Fixed\-Width\-Text\-Reader has several methods that can be used. They are listed below. Note that the documentation is translated automatically from the V\-T\-K sources, and may not be completely intelligible. When in doubt, consult the V\-T\-K website. In the methods listed below, {\ttfamily obj} is an instance of the vtk\-Fixed\-Width\-Text\-Reader class. 
\begin{DoxyItemize}
\item {\ttfamily string = obj.\-Get\-Class\-Name ()}  
\item {\ttfamily int = obj.\-Is\-A (string name)}  
\item {\ttfamily vtk\-Fixed\-Width\-Text\-Reader = obj.\-New\-Instance ()}  
\item {\ttfamily vtk\-Fixed\-Width\-Text\-Reader = obj.\-Safe\-Down\-Cast (vtk\-Object o)}  
\item {\ttfamily string = obj.\-Get\-File\-Name ()}  
\item {\ttfamily obj.\-Set\-File\-Name (string )}  
\item {\ttfamily obj.\-Set\-Field\-Width (int )} -\/ Set/get the field width  
\item {\ttfamily int = obj.\-Get\-Field\-Width ()} -\/ Set/get the field width  
\item {\ttfamily obj.\-Set\-Strip\-White\-Space (bool )} -\/ If set, this flag will cause the reader to strip whitespace from the beginning and ending of each field. Defaults to off.  
\item {\ttfamily bool = obj.\-Get\-Strip\-White\-Space ()} -\/ If set, this flag will cause the reader to strip whitespace from the beginning and ending of each field. Defaults to off.  
\item {\ttfamily obj.\-Strip\-White\-Space\-On ()} -\/ If set, this flag will cause the reader to strip whitespace from the beginning and ending of each field. Defaults to off.  
\item {\ttfamily obj.\-Strip\-White\-Space\-Off ()} -\/ If set, this flag will cause the reader to strip whitespace from the beginning and ending of each field. Defaults to off.  
\item {\ttfamily bool = obj.\-Get\-Have\-Headers ()} -\/ Set/get whether to treat the first line of the file as headers.  
\item {\ttfamily obj.\-Set\-Have\-Headers (bool )} -\/ Set/get whether to treat the first line of the file as headers.  
\item {\ttfamily obj.\-Have\-Headers\-On ()} -\/ Set/get whether to treat the first line of the file as headers.  
\item {\ttfamily obj.\-Have\-Headers\-Off ()} -\/ Set/get whether to treat the first line of the file as headers.  
\end{DoxyItemize}\hypertarget{vtkinfovis_vtkforcedirectedlayoutstrategy}{}\section{vtk\-Force\-Directed\-Layout\-Strategy}\label{vtkinfovis_vtkforcedirectedlayoutstrategy}
Section\-: \hyperlink{sec_vtkinfovis}{Visualization Toolkit Infovis Classes} \hypertarget{vtkwidgets_vtkxyplotwidget_Usage}{}\subsection{Usage}\label{vtkwidgets_vtkxyplotwidget_Usage}
Lays out a graph in 2\-D or 3\-D using a force-\/directed algorithm. The user may specify whether to layout the graph randomly initially, the bounds, the number of dimensions (2 or 3), and the cool-\/down rate.

.S\-E\-C\-T\-I\-O\-N Thanks Thanks to Brian Wylie for adding functionality for allowing this layout to be incremental.

To create an instance of class vtk\-Force\-Directed\-Layout\-Strategy, simply invoke its constructor as follows \begin{DoxyVerb}  obj = vtkForceDirectedLayoutStrategy
\end{DoxyVerb}
 \hypertarget{vtkwidgets_vtkxyplotwidget_Methods}{}\subsection{Methods}\label{vtkwidgets_vtkxyplotwidget_Methods}
The class vtk\-Force\-Directed\-Layout\-Strategy has several methods that can be used. They are listed below. Note that the documentation is translated automatically from the V\-T\-K sources, and may not be completely intelligible. When in doubt, consult the V\-T\-K website. In the methods listed below, {\ttfamily obj} is an instance of the vtk\-Force\-Directed\-Layout\-Strategy class. 
\begin{DoxyItemize}
\item {\ttfamily string = obj.\-Get\-Class\-Name ()}  
\item {\ttfamily int = obj.\-Is\-A (string name)}  
\item {\ttfamily vtk\-Force\-Directed\-Layout\-Strategy = obj.\-New\-Instance ()}  
\item {\ttfamily vtk\-Force\-Directed\-Layout\-Strategy = obj.\-Safe\-Down\-Cast (vtk\-Object o)}  
\item {\ttfamily obj.\-Set\-Random\-Seed (int )} -\/ Seed the random number generator used to jitter point positions. This has a significant effect on their final positions when the layout is complete.  
\item {\ttfamily int = obj.\-Get\-Random\-Seed\-Min\-Value ()} -\/ Seed the random number generator used to jitter point positions. This has a significant effect on their final positions when the layout is complete.  
\item {\ttfamily int = obj.\-Get\-Random\-Seed\-Max\-Value ()} -\/ Seed the random number generator used to jitter point positions. This has a significant effect on their final positions when the layout is complete.  
\item {\ttfamily int = obj.\-Get\-Random\-Seed ()} -\/ Seed the random number generator used to jitter point positions. This has a significant effect on their final positions when the layout is complete.  
\item {\ttfamily obj.\-Set\-Graph\-Bounds (double , double , double , double , double , double )} -\/ Set / get the region in space in which to place the final graph. The Graph\-Bounds only affects the results if Automatic\-Bounds\-Computation is off.  
\item {\ttfamily obj.\-Set\-Graph\-Bounds (double a\mbox{[}6\mbox{]})} -\/ Set / get the region in space in which to place the final graph. The Graph\-Bounds only affects the results if Automatic\-Bounds\-Computation is off.  
\item {\ttfamily double = obj. Get\-Graph\-Bounds ()} -\/ Set / get the region in space in which to place the final graph. The Graph\-Bounds only affects the results if Automatic\-Bounds\-Computation is off.  
\item {\ttfamily obj.\-Set\-Automatic\-Bounds\-Computation (int )} -\/ Turn on/off automatic graph bounds calculation. If this boolean is off, then the manually specified Graph\-Bounds is used. If on, then the input's bounds us used as the graph bounds.  
\item {\ttfamily int = obj.\-Get\-Automatic\-Bounds\-Computation ()} -\/ Turn on/off automatic graph bounds calculation. If this boolean is off, then the manually specified Graph\-Bounds is used. If on, then the input's bounds us used as the graph bounds.  
\item {\ttfamily obj.\-Automatic\-Bounds\-Computation\-On ()} -\/ Turn on/off automatic graph bounds calculation. If this boolean is off, then the manually specified Graph\-Bounds is used. If on, then the input's bounds us used as the graph bounds.  
\item {\ttfamily obj.\-Automatic\-Bounds\-Computation\-Off ()} -\/ Turn on/off automatic graph bounds calculation. If this boolean is off, then the manually specified Graph\-Bounds is used. If on, then the input's bounds us used as the graph bounds.  
\item {\ttfamily obj.\-Set\-Max\-Number\-Of\-Iterations (int )} -\/ Set/\-Get the maximum number of iterations to be used. The higher this number, the more iterations through the algorithm is possible, and thus, the more the graph gets modified. The default is '50' for no particular reason  
\item {\ttfamily int = obj.\-Get\-Max\-Number\-Of\-Iterations\-Min\-Value ()} -\/ Set/\-Get the maximum number of iterations to be used. The higher this number, the more iterations through the algorithm is possible, and thus, the more the graph gets modified. The default is '50' for no particular reason  
\item {\ttfamily int = obj.\-Get\-Max\-Number\-Of\-Iterations\-Max\-Value ()} -\/ Set/\-Get the maximum number of iterations to be used. The higher this number, the more iterations through the algorithm is possible, and thus, the more the graph gets modified. The default is '50' for no particular reason  
\item {\ttfamily int = obj.\-Get\-Max\-Number\-Of\-Iterations ()} -\/ Set/\-Get the maximum number of iterations to be used. The higher this number, the more iterations through the algorithm is possible, and thus, the more the graph gets modified. The default is '50' for no particular reason  
\item {\ttfamily obj.\-Set\-Iterations\-Per\-Layout (int )} -\/ Set/\-Get the number of iterations per layout. The only use for this ivar is for the application to do visualizations of the layout before it's complete. The default is '50' to match the default 'Max\-Number\-Of\-Iterations'  
\item {\ttfamily int = obj.\-Get\-Iterations\-Per\-Layout\-Min\-Value ()} -\/ Set/\-Get the number of iterations per layout. The only use for this ivar is for the application to do visualizations of the layout before it's complete. The default is '50' to match the default 'Max\-Number\-Of\-Iterations'  
\item {\ttfamily int = obj.\-Get\-Iterations\-Per\-Layout\-Max\-Value ()} -\/ Set/\-Get the number of iterations per layout. The only use for this ivar is for the application to do visualizations of the layout before it's complete. The default is '50' to match the default 'Max\-Number\-Of\-Iterations'  
\item {\ttfamily int = obj.\-Get\-Iterations\-Per\-Layout ()} -\/ Set/\-Get the number of iterations per layout. The only use for this ivar is for the application to do visualizations of the layout before it's complete. The default is '50' to match the default 'Max\-Number\-Of\-Iterations'  
\item {\ttfamily obj.\-Set\-Cool\-Down\-Rate (double )} -\/ Set/\-Get the Cool-\/down rate. The higher this number is, the longer it will take to \char`\"{}cool-\/down\char`\"{}, and thus, the more the graph will be modified.  
\item {\ttfamily double = obj.\-Get\-Cool\-Down\-Rate\-Min\-Value ()} -\/ Set/\-Get the Cool-\/down rate. The higher this number is, the longer it will take to \char`\"{}cool-\/down\char`\"{}, and thus, the more the graph will be modified.  
\item {\ttfamily double = obj.\-Get\-Cool\-Down\-Rate\-Max\-Value ()} -\/ Set/\-Get the Cool-\/down rate. The higher this number is, the longer it will take to \char`\"{}cool-\/down\char`\"{}, and thus, the more the graph will be modified.  
\item {\ttfamily double = obj.\-Get\-Cool\-Down\-Rate ()} -\/ Set/\-Get the Cool-\/down rate. The higher this number is, the longer it will take to \char`\"{}cool-\/down\char`\"{}, and thus, the more the graph will be modified.  
\item {\ttfamily obj.\-Set\-Three\-Dimensional\-Layout (int )} -\/ Turn on/off layout of graph in three dimensions. If off, graph layout occurs in two dimensions. By default, three dimensional layout is off.  
\item {\ttfamily int = obj.\-Get\-Three\-Dimensional\-Layout ()} -\/ Turn on/off layout of graph in three dimensions. If off, graph layout occurs in two dimensions. By default, three dimensional layout is off.  
\item {\ttfamily obj.\-Three\-Dimensional\-Layout\-On ()} -\/ Turn on/off layout of graph in three dimensions. If off, graph layout occurs in two dimensions. By default, three dimensional layout is off.  
\item {\ttfamily obj.\-Three\-Dimensional\-Layout\-Off ()} -\/ Turn on/off layout of graph in three dimensions. If off, graph layout occurs in two dimensions. By default, three dimensional layout is off.  
\item {\ttfamily obj.\-Set\-Random\-Initial\-Points (int )} -\/ Turn on/off use of random positions within the graph bounds as initial points.  
\item {\ttfamily int = obj.\-Get\-Random\-Initial\-Points ()} -\/ Turn on/off use of random positions within the graph bounds as initial points.  
\item {\ttfamily obj.\-Random\-Initial\-Points\-On ()} -\/ Turn on/off use of random positions within the graph bounds as initial points.  
\item {\ttfamily obj.\-Random\-Initial\-Points\-Off ()} -\/ Turn on/off use of random positions within the graph bounds as initial points.  
\item {\ttfamily obj.\-Set\-Initial\-Temperature (float )} -\/ Set the initial temperature. If zero (the default) , the initial temperature will be computed automatically.  
\item {\ttfamily float = obj.\-Get\-Initial\-Temperature\-Min\-Value ()} -\/ Set the initial temperature. If zero (the default) , the initial temperature will be computed automatically.  
\item {\ttfamily float = obj.\-Get\-Initial\-Temperature\-Max\-Value ()} -\/ Set the initial temperature. If zero (the default) , the initial temperature will be computed automatically.  
\item {\ttfamily float = obj.\-Get\-Initial\-Temperature ()} -\/ Set the initial temperature. If zero (the default) , the initial temperature will be computed automatically.  
\item {\ttfamily obj.\-Initialize ()} -\/ This strategy sets up some data structures for faster processing of each Layout() call  
\item {\ttfamily obj.\-Layout ()} -\/ This is the layout method where the graph that was set in Set\-Graph() is laid out. The method can either entirely layout the graph or iteratively lay out the graph. If you have an iterative layout please implement the Is\-Layout\-Complete() method.  
\item {\ttfamily int = obj.\-Is\-Layout\-Complete ()}  
\end{DoxyItemize}\hypertarget{vtkinfovis_vtkgenerateindexarray}{}\section{vtk\-Generate\-Index\-Array}\label{vtkinfovis_vtkgenerateindexarray}
Section\-: \hyperlink{sec_vtkinfovis}{Visualization Toolkit Infovis Classes} \hypertarget{vtkwidgets_vtkxyplotwidget_Usage}{}\subsection{Usage}\label{vtkwidgets_vtkxyplotwidget_Usage}
Generates a new vtk\-Id\-Type\-Array containing zero-\/base indices.

vtk\-Generate\-Index\-Array operates in one of two distinct \char`\"{}modes\char`\"{}. By default, it simply generates an index array containing monotonically-\/increasing integers in the range \mbox{[}0, N), where N is appropriately sized for the field type that will store the results. This mode is useful for generating a unique I\-D field for datasets that have none.

The second \char`\"{}mode\char`\"{} uses an existing array from the input data object as a \char`\"{}reference\char`\"{}. Distinct values from the reference array are sorted in ascending order, and an integer index in the range \mbox{[}0, N) is assigned to each. The resulting map is used to populate the output index array, mapping each value in the reference array to its corresponding index and storing the result in the output array. This mode is especially useful when generating tensors, since it allows us to \char`\"{}map\char`\"{} from an array with arbitrary contents to an index that can be used as tensor coordinates.

To create an instance of class vtk\-Generate\-Index\-Array, simply invoke its constructor as follows \begin{DoxyVerb}  obj = vtkGenerateIndexArray
\end{DoxyVerb}
 \hypertarget{vtkwidgets_vtkxyplotwidget_Methods}{}\subsection{Methods}\label{vtkwidgets_vtkxyplotwidget_Methods}
The class vtk\-Generate\-Index\-Array has several methods that can be used. They are listed below. Note that the documentation is translated automatically from the V\-T\-K sources, and may not be completely intelligible. When in doubt, consult the V\-T\-K website. In the methods listed below, {\ttfamily obj} is an instance of the vtk\-Generate\-Index\-Array class. 
\begin{DoxyItemize}
\item {\ttfamily string = obj.\-Get\-Class\-Name ()}  
\item {\ttfamily int = obj.\-Is\-A (string name)}  
\item {\ttfamily vtk\-Generate\-Index\-Array = obj.\-New\-Instance ()}  
\item {\ttfamily vtk\-Generate\-Index\-Array = obj.\-Safe\-Down\-Cast (vtk\-Object o)}  
\item {\ttfamily obj.\-Set\-Array\-Name (string )} -\/ Control the output index array name. Default\-: \char`\"{}index\char`\"{}.  
\item {\ttfamily string = obj.\-Get\-Array\-Name ()} -\/ Control the output index array name. Default\-: \char`\"{}index\char`\"{}.  
\item {\ttfamily obj.\-Set\-Field\-Type (int )} -\/ Control the location where the index array will be stored.  
\item {\ttfamily int = obj.\-Get\-Field\-Type ()} -\/ Control the location where the index array will be stored.  
\item {\ttfamily obj.\-Set\-Reference\-Array\-Name (string )} -\/ Specifies an optional reference array for index-\/generation.  
\item {\ttfamily string = obj.\-Get\-Reference\-Array\-Name ()} -\/ Specifies an optional reference array for index-\/generation.  
\item {\ttfamily obj.\-Set\-Pedigree\-I\-D (int )} -\/ Specifies whether the index array should be marked as pedigree ids. Default\-: false.  
\item {\ttfamily int = obj.\-Get\-Pedigree\-I\-D ()} -\/ Specifies whether the index array should be marked as pedigree ids. Default\-: false.  
\end{DoxyItemize}\hypertarget{vtkinfovis_vtkgeoedgestrategy}{}\section{vtk\-Geo\-Edge\-Strategy}\label{vtkinfovis_vtkgeoedgestrategy}
Section\-: \hyperlink{sec_vtkinfovis}{Visualization Toolkit Infovis Classes} \hypertarget{vtkwidgets_vtkxyplotwidget_Usage}{}\subsection{Usage}\label{vtkwidgets_vtkxyplotwidget_Usage}
vtk\-Geo\-Edge\-Strategy produces arcs for each edge in the input graph. This is useful for viewing lines on a sphere (e.\-g. the earth). The arcs may \char`\"{}jump\char`\"{} above the sphere's surface using Explode\-Factor.

To create an instance of class vtk\-Geo\-Edge\-Strategy, simply invoke its constructor as follows \begin{DoxyVerb}  obj = vtkGeoEdgeStrategy
\end{DoxyVerb}
 \hypertarget{vtkwidgets_vtkxyplotwidget_Methods}{}\subsection{Methods}\label{vtkwidgets_vtkxyplotwidget_Methods}
The class vtk\-Geo\-Edge\-Strategy has several methods that can be used. They are listed below. Note that the documentation is translated automatically from the V\-T\-K sources, and may not be completely intelligible. When in doubt, consult the V\-T\-K website. In the methods listed below, {\ttfamily obj} is an instance of the vtk\-Geo\-Edge\-Strategy class. 
\begin{DoxyItemize}
\item {\ttfamily string = obj.\-Get\-Class\-Name ()}  
\item {\ttfamily int = obj.\-Is\-A (string name)}  
\item {\ttfamily vtk\-Geo\-Edge\-Strategy = obj.\-New\-Instance ()}  
\item {\ttfamily vtk\-Geo\-Edge\-Strategy = obj.\-Safe\-Down\-Cast (vtk\-Object o)}  
\item {\ttfamily obj.\-Set\-Globe\-Radius (double )} -\/ The base radius used to determine the earth's surface. Default is the earth's radius in meters. T\-O\-D\-O\-: Change this to take in a vtk\-Geo\-Terrain to get altitude.  
\item {\ttfamily double = obj.\-Get\-Globe\-Radius ()} -\/ The base radius used to determine the earth's surface. Default is the earth's radius in meters. T\-O\-D\-O\-: Change this to take in a vtk\-Geo\-Terrain to get altitude.  
\item {\ttfamily obj.\-Set\-Explode\-Factor (double )} -\/ Factor on which to \char`\"{}explode\char`\"{} the arcs away from the surface. A value of 0.\-0 keeps the values on the surface. Values larger than 0.\-0 push the arcs away from the surface by a distance proportional to the distance between the points. The default is 0.\-2.  
\item {\ttfamily double = obj.\-Get\-Explode\-Factor ()} -\/ Factor on which to \char`\"{}explode\char`\"{} the arcs away from the surface. A value of 0.\-0 keeps the values on the surface. Values larger than 0.\-0 push the arcs away from the surface by a distance proportional to the distance between the points. The default is 0.\-2.  
\item {\ttfamily obj.\-Set\-Number\-Of\-Subdivisions (int )} -\/ The number of subdivisions in the arc. The default is 20.  
\item {\ttfamily int = obj.\-Get\-Number\-Of\-Subdivisions ()} -\/ The number of subdivisions in the arc. The default is 20.  
\item {\ttfamily obj.\-Layout ()} -\/ Perform the layout.  
\end{DoxyItemize}\hypertarget{vtkinfovis_vtkgeomath}{}\section{vtk\-Geo\-Math}\label{vtkinfovis_vtkgeomath}
Section\-: \hyperlink{sec_vtkinfovis}{Visualization Toolkit Infovis Classes} \hypertarget{vtkwidgets_vtkxyplotwidget_Usage}{}\subsection{Usage}\label{vtkwidgets_vtkxyplotwidget_Usage}
vtk\-Geo\-Math provides some useful geographic calculations.

To create an instance of class vtk\-Geo\-Math, simply invoke its constructor as follows \begin{DoxyVerb}  obj = vtkGeoMath
\end{DoxyVerb}
 \hypertarget{vtkwidgets_vtkxyplotwidget_Methods}{}\subsection{Methods}\label{vtkwidgets_vtkxyplotwidget_Methods}
The class vtk\-Geo\-Math has several methods that can be used. They are listed below. Note that the documentation is translated automatically from the V\-T\-K sources, and may not be completely intelligible. When in doubt, consult the V\-T\-K website. In the methods listed below, {\ttfamily obj} is an instance of the vtk\-Geo\-Math class. 
\begin{DoxyItemize}
\item {\ttfamily string = obj.\-Get\-Class\-Name ()}  
\item {\ttfamily int = obj.\-Is\-A (string name)}  
\item {\ttfamily vtk\-Geo\-Math = obj.\-New\-Instance ()}  
\item {\ttfamily vtk\-Geo\-Math = obj.\-Safe\-Down\-Cast (vtk\-Object o)}  
\end{DoxyItemize}\hypertarget{vtkinfovis_vtkgraphhierarchicalbundle}{}\section{vtk\-Graph\-Hierarchical\-Bundle}\label{vtkinfovis_vtkgraphhierarchicalbundle}
Section\-: \hyperlink{sec_vtkinfovis}{Visualization Toolkit Infovis Classes} \hypertarget{vtkwidgets_vtkxyplotwidget_Usage}{}\subsection{Usage}\label{vtkwidgets_vtkxyplotwidget_Usage}
This algorithm creates a vtk\-Poly\-Data from a vtk\-Graph. As opposed to vtk\-Graph\-To\-Poly\-Data, which converts each arc into a straight line, each arc is converted to a polyline, following a tree structure. The filter requires both a vtk\-Graph and vtk\-Tree as input. The tree vertices must be a superset of the graph vertices. A common example is when the graph vertices correspond to the leaves of the tree, but the internal vertices of the tree represent groupings of graph vertices. The algorithm matches the vertices using the array \char`\"{}\-Pedigree\-Id\char`\"{}. The user may alternately set the Direct\-Mapping flag to indicate that the two structures must have directly corresponding offsets (i.\-e. node i in the graph must correspond to node i in the tree).

The vtk\-Graph defines the topology of the output vtk\-Poly\-Data (i.\-e. the connections between nodes) while the vtk\-Tree defines the geometry (i.\-e. the location of nodes and arc routes). Thus, the tree must have been assigned vertex locations, but the graph does not need locations, in fact they will be ignored. The edges approximately follow the path from the source to target nodes in the tree. A bundling parameter controls how closely the edges are bundled together along the tree structure.

You may follow this algorithm with vtk\-Spline\-Filter in order to make nicely curved edges.

To create an instance of class vtk\-Graph\-Hierarchical\-Bundle, simply invoke its constructor as follows \begin{DoxyVerb}  obj = vtkGraphHierarchicalBundle
\end{DoxyVerb}
 \hypertarget{vtkwidgets_vtkxyplotwidget_Methods}{}\subsection{Methods}\label{vtkwidgets_vtkxyplotwidget_Methods}
The class vtk\-Graph\-Hierarchical\-Bundle has several methods that can be used. They are listed below. Note that the documentation is translated automatically from the V\-T\-K sources, and may not be completely intelligible. When in doubt, consult the V\-T\-K website. In the methods listed below, {\ttfamily obj} is an instance of the vtk\-Graph\-Hierarchical\-Bundle class. 
\begin{DoxyItemize}
\item {\ttfamily string = obj.\-Get\-Class\-Name ()}  
\item {\ttfamily int = obj.\-Is\-A (string name)}  
\item {\ttfamily vtk\-Graph\-Hierarchical\-Bundle = obj.\-New\-Instance ()}  
\item {\ttfamily vtk\-Graph\-Hierarchical\-Bundle = obj.\-Safe\-Down\-Cast (vtk\-Object o)}  
\item {\ttfamily obj.\-Set\-Bundling\-Strength (double )} -\/ The level of arc bundling in the graph. A strength of 0 creates straight lines, while a strength of 1 forces arcs to pass directly through hierarchy node points. The default value is 0.\-8.  
\item {\ttfamily double = obj.\-Get\-Bundling\-Strength\-Min\-Value ()} -\/ The level of arc bundling in the graph. A strength of 0 creates straight lines, while a strength of 1 forces arcs to pass directly through hierarchy node points. The default value is 0.\-8.  
\item {\ttfamily double = obj.\-Get\-Bundling\-Strength\-Max\-Value ()} -\/ The level of arc bundling in the graph. A strength of 0 creates straight lines, while a strength of 1 forces arcs to pass directly through hierarchy node points. The default value is 0.\-8.  
\item {\ttfamily double = obj.\-Get\-Bundling\-Strength ()} -\/ The level of arc bundling in the graph. A strength of 0 creates straight lines, while a strength of 1 forces arcs to pass directly through hierarchy node points. The default value is 0.\-8.  
\item {\ttfamily obj.\-Set\-Direct\-Mapping (bool )} -\/ If on, uses direct mapping from tree to graph vertices. If off, both the graph and tree must contain Pedigree\-Id arrays which are used to match graph and tree vertices. Default is off.  
\item {\ttfamily bool = obj.\-Get\-Direct\-Mapping ()} -\/ If on, uses direct mapping from tree to graph vertices. If off, both the graph and tree must contain Pedigree\-Id arrays which are used to match graph and tree vertices. Default is off.  
\item {\ttfamily obj.\-Direct\-Mapping\-On ()} -\/ If on, uses direct mapping from tree to graph vertices. If off, both the graph and tree must contain Pedigree\-Id arrays which are used to match graph and tree vertices. Default is off.  
\item {\ttfamily obj.\-Direct\-Mapping\-Off ()} -\/ If on, uses direct mapping from tree to graph vertices. If off, both the graph and tree must contain Pedigree\-Id arrays which are used to match graph and tree vertices. Default is off.  
\item {\ttfamily int = obj.\-Fill\-Input\-Port\-Information (int port, vtk\-Information info)} -\/ Set the input type of the algorithm to vtk\-Graph.  
\end{DoxyItemize}\hypertarget{vtkinfovis_vtkgraphhierarchicalbundleedges}{}\section{vtk\-Graph\-Hierarchical\-Bundle\-Edges}\label{vtkinfovis_vtkgraphhierarchicalbundleedges}
Section\-: \hyperlink{sec_vtkinfovis}{Visualization Toolkit Infovis Classes} \hypertarget{vtkwidgets_vtkxyplotwidget_Usage}{}\subsection{Usage}\label{vtkwidgets_vtkxyplotwidget_Usage}
This algorithm creates a vtk\-Poly\-Data from a vtk\-Graph. As opposed to vtk\-Graph\-To\-Poly\-Data, which converts each arc into a straight line, each arc is converted to a polyline, following a tree structure. The filter requires both a vtk\-Graph and vtk\-Tree as input. The tree vertices must be a superset of the graph vertices. A common example is when the graph vertices correspond to the leaves of the tree, but the internal vertices of the tree represent groupings of graph vertices. The algorithm matches the vertices using the array \char`\"{}\-Pedigree\-Id\char`\"{}. The user may alternately set the Direct\-Mapping flag to indicate that the two structures must have directly corresponding offsets (i.\-e. node i in the graph must correspond to node i in the tree).

The vtk\-Graph defines the topology of the output vtk\-Poly\-Data (i.\-e. the connections between nodes) while the vtk\-Tree defines the geometry (i.\-e. the location of nodes and arc routes). Thus, the tree must have been assigned vertex locations, but the graph does not need locations, in fact they will be ignored. The edges approximately follow the path from the source to target nodes in the tree. A bundling parameter controls how closely the edges are bundled together along the tree structure.

You may follow this algorithm with vtk\-Spline\-Filter in order to make nicely curved edges.

To create an instance of class vtk\-Graph\-Hierarchical\-Bundle\-Edges, simply invoke its constructor as follows \begin{DoxyVerb}  obj = vtkGraphHierarchicalBundleEdges
\end{DoxyVerb}
 \hypertarget{vtkwidgets_vtkxyplotwidget_Methods}{}\subsection{Methods}\label{vtkwidgets_vtkxyplotwidget_Methods}
The class vtk\-Graph\-Hierarchical\-Bundle\-Edges has several methods that can be used. They are listed below. Note that the documentation is translated automatically from the V\-T\-K sources, and may not be completely intelligible. When in doubt, consult the V\-T\-K website. In the methods listed below, {\ttfamily obj} is an instance of the vtk\-Graph\-Hierarchical\-Bundle\-Edges class. 
\begin{DoxyItemize}
\item {\ttfamily string = obj.\-Get\-Class\-Name ()}  
\item {\ttfamily int = obj.\-Is\-A (string name)}  
\item {\ttfamily vtk\-Graph\-Hierarchical\-Bundle\-Edges = obj.\-New\-Instance ()}  
\item {\ttfamily vtk\-Graph\-Hierarchical\-Bundle\-Edges = obj.\-Safe\-Down\-Cast (vtk\-Object o)}  
\item {\ttfamily obj.\-Set\-Bundling\-Strength (double )} -\/ The level of arc bundling in the graph. A strength of 0 creates straight lines, while a strength of 1 forces arcs to pass directly through hierarchy node points. The default value is 0.\-8.  
\item {\ttfamily double = obj.\-Get\-Bundling\-Strength\-Min\-Value ()} -\/ The level of arc bundling in the graph. A strength of 0 creates straight lines, while a strength of 1 forces arcs to pass directly through hierarchy node points. The default value is 0.\-8.  
\item {\ttfamily double = obj.\-Get\-Bundling\-Strength\-Max\-Value ()} -\/ The level of arc bundling in the graph. A strength of 0 creates straight lines, while a strength of 1 forces arcs to pass directly through hierarchy node points. The default value is 0.\-8.  
\item {\ttfamily double = obj.\-Get\-Bundling\-Strength ()} -\/ The level of arc bundling in the graph. A strength of 0 creates straight lines, while a strength of 1 forces arcs to pass directly through hierarchy node points. The default value is 0.\-8.  
\item {\ttfamily obj.\-Set\-Direct\-Mapping (bool )} -\/ If on, uses direct mapping from tree to graph vertices. If off, both the graph and tree must contain Pedigree\-Id arrays which are used to match graph and tree vertices. Default is off.  
\item {\ttfamily bool = obj.\-Get\-Direct\-Mapping ()} -\/ If on, uses direct mapping from tree to graph vertices. If off, both the graph and tree must contain Pedigree\-Id arrays which are used to match graph and tree vertices. Default is off.  
\item {\ttfamily obj.\-Direct\-Mapping\-On ()} -\/ If on, uses direct mapping from tree to graph vertices. If off, both the graph and tree must contain Pedigree\-Id arrays which are used to match graph and tree vertices. Default is off.  
\item {\ttfamily obj.\-Direct\-Mapping\-Off ()} -\/ If on, uses direct mapping from tree to graph vertices. If off, both the graph and tree must contain Pedigree\-Id arrays which are used to match graph and tree vertices. Default is off.  
\item {\ttfamily int = obj.\-Fill\-Input\-Port\-Information (int port, vtk\-Information info)} -\/ Set the input type of the algorithm to vtk\-Graph.  
\end{DoxyItemize}\hypertarget{vtkinfovis_vtkgraphlayout}{}\section{vtk\-Graph\-Layout}\label{vtkinfovis_vtkgraphlayout}
Section\-: \hyperlink{sec_vtkinfovis}{Visualization Toolkit Infovis Classes} \hypertarget{vtkwidgets_vtkxyplotwidget_Usage}{}\subsection{Usage}\label{vtkwidgets_vtkxyplotwidget_Usage}
This class is a shell for many graph layout strategies which may be set using the Set\-Layout\-Strategy() function. The layout strategies do the actual work.

.S\-E\-C\-I\-O\-N Thanks Thanks to Brian Wylie from Sandia National Laboratories for adding incremental layout capabilities.

To create an instance of class vtk\-Graph\-Layout, simply invoke its constructor as follows \begin{DoxyVerb}  obj = vtkGraphLayout
\end{DoxyVerb}
 \hypertarget{vtkwidgets_vtkxyplotwidget_Methods}{}\subsection{Methods}\label{vtkwidgets_vtkxyplotwidget_Methods}
The class vtk\-Graph\-Layout has several methods that can be used. They are listed below. Note that the documentation is translated automatically from the V\-T\-K sources, and may not be completely intelligible. When in doubt, consult the V\-T\-K website. In the methods listed below, {\ttfamily obj} is an instance of the vtk\-Graph\-Layout class. 
\begin{DoxyItemize}
\item {\ttfamily string = obj.\-Get\-Class\-Name ()}  
\item {\ttfamily int = obj.\-Is\-A (string name)}  
\item {\ttfamily vtk\-Graph\-Layout = obj.\-New\-Instance ()}  
\item {\ttfamily vtk\-Graph\-Layout = obj.\-Safe\-Down\-Cast (vtk\-Object o)}  
\item {\ttfamily obj.\-Set\-Layout\-Strategy (vtk\-Graph\-Layout\-Strategy strategy)} -\/ The layout strategy to use during graph layout.  
\item {\ttfamily vtk\-Graph\-Layout\-Strategy = obj.\-Get\-Layout\-Strategy ()} -\/ The layout strategy to use during graph layout.  
\item {\ttfamily int = obj.\-Is\-Layout\-Complete ()} -\/ Ask the layout algorithm if the layout is complete  
\item {\ttfamily long = obj.\-Get\-M\-Time ()} -\/ Get the modification time of the layout algorithm.  
\item {\ttfamily double = obj.\-Get\-Z\-Range ()} -\/ Set the Z\-Range for the output data. If the initial layout is planar (i.\-e. all z coordinates are zero), the coordinates will be evenly spaced from 0.\-0 to Z\-Range. The default is zero, which has no effect.  
\item {\ttfamily obj.\-Set\-Z\-Range (double )} -\/ Set the Z\-Range for the output data. If the initial layout is planar (i.\-e. all z coordinates are zero), the coordinates will be evenly spaced from 0.\-0 to Z\-Range. The default is zero, which has no effect.  
\item {\ttfamily vtk\-Abstract\-Transform = obj.\-Get\-Transform ()} -\/ Transform the graph vertices after the layout.  
\item {\ttfamily obj.\-Set\-Transform (vtk\-Abstract\-Transform t)} -\/ Transform the graph vertices after the layout.  
\item {\ttfamily obj.\-Set\-Use\-Transform (bool )} -\/ Whether to use the specified transform after layout.  
\item {\ttfamily bool = obj.\-Get\-Use\-Transform ()} -\/ Whether to use the specified transform after layout.  
\item {\ttfamily obj.\-Use\-Transform\-On ()} -\/ Whether to use the specified transform after layout.  
\item {\ttfamily obj.\-Use\-Transform\-Off ()} -\/ Whether to use the specified transform after layout.  
\end{DoxyItemize}\hypertarget{vtkinfovis_vtkgraphlayoutstrategy}{}\section{vtk\-Graph\-Layout\-Strategy}\label{vtkinfovis_vtkgraphlayoutstrategy}
Section\-: \hyperlink{sec_vtkinfovis}{Visualization Toolkit Infovis Classes} \hypertarget{vtkwidgets_vtkxyplotwidget_Usage}{}\subsection{Usage}\label{vtkwidgets_vtkxyplotwidget_Usage}
All graph layouts should subclass from this class. vtk\-Graph\-Layout\-Strategy works as a plug-\/in to the vtk\-Graph\-Layout algorithm. The Layout() function should perform some reasonable \char`\"{}chunk\char`\"{} of the layout. This allows the user to be able to see the progress of the layout. Use Is\-Layout\-Complete() to tell the user when there is no more layout to perform.

.S\-E\-C\-T\-I\-O\-N Thanks Thanks to Brian Wylie from Sandia National Laboratories for adding incremental layout capabilities.

To create an instance of class vtk\-Graph\-Layout\-Strategy, simply invoke its constructor as follows \begin{DoxyVerb}  obj = vtkGraphLayoutStrategy
\end{DoxyVerb}
 \hypertarget{vtkwidgets_vtkxyplotwidget_Methods}{}\subsection{Methods}\label{vtkwidgets_vtkxyplotwidget_Methods}
The class vtk\-Graph\-Layout\-Strategy has several methods that can be used. They are listed below. Note that the documentation is translated automatically from the V\-T\-K sources, and may not be completely intelligible. When in doubt, consult the V\-T\-K website. In the methods listed below, {\ttfamily obj} is an instance of the vtk\-Graph\-Layout\-Strategy class. 
\begin{DoxyItemize}
\item {\ttfamily string = obj.\-Get\-Class\-Name ()}  
\item {\ttfamily int = obj.\-Is\-A (string name)}  
\item {\ttfamily vtk\-Graph\-Layout\-Strategy = obj.\-New\-Instance ()}  
\item {\ttfamily vtk\-Graph\-Layout\-Strategy = obj.\-Safe\-Down\-Cast (vtk\-Object o)}  
\item {\ttfamily obj.\-Set\-Graph (vtk\-Graph graph)} -\/ Setting the graph for the layout strategy  
\item {\ttfamily obj.\-Initialize ()} -\/ This method allows the layout strategy to do initialization of data structures or whatever else it might want to do.  
\item {\ttfamily obj.\-Layout ()} -\/ This is the layout method where the graph that was set in Set\-Graph() is laid out. The method can either entirely layout the graph or iteratively lay out the graph. If you have an iterative layout please implement the Is\-Layout\-Complete() method.  
\item {\ttfamily int = obj.\-Is\-Layout\-Complete ()} -\/ Whether to use edge weights in the layout or not.  
\item {\ttfamily obj.\-Set\-Weight\-Edges (bool state)} -\/ Whether to use edge weights in the layout or not.  
\item {\ttfamily bool = obj.\-Get\-Weight\-Edges ()} -\/ Whether to use edge weights in the layout or not.  
\item {\ttfamily obj.\-Set\-Edge\-Weight\-Field (string field)} -\/ Set/\-Get the field to use for the edge weights.  
\item {\ttfamily string = obj.\-Get\-Edge\-Weight\-Field ()} -\/ Set/\-Get the field to use for the edge weights.  
\end{DoxyItemize}\hypertarget{vtkinfovis_vtkgroupleafvertices}{}\section{vtk\-Group\-Leaf\-Vertices}\label{vtkinfovis_vtkgroupleafvertices}
Section\-: \hyperlink{sec_vtkinfovis}{Visualization Toolkit Infovis Classes} \hypertarget{vtkwidgets_vtkxyplotwidget_Usage}{}\subsection{Usage}\label{vtkwidgets_vtkxyplotwidget_Usage}
Use Set\-Input\-Array\-To\-Process(0, ...) to set the array to group on. Currently this array must be a vtk\-String\-Array.

To create an instance of class vtk\-Group\-Leaf\-Vertices, simply invoke its constructor as follows \begin{DoxyVerb}  obj = vtkGroupLeafVertices
\end{DoxyVerb}
 \hypertarget{vtkwidgets_vtkxyplotwidget_Methods}{}\subsection{Methods}\label{vtkwidgets_vtkxyplotwidget_Methods}
The class vtk\-Group\-Leaf\-Vertices has several methods that can be used. They are listed below. Note that the documentation is translated automatically from the V\-T\-K sources, and may not be completely intelligible. When in doubt, consult the V\-T\-K website. In the methods listed below, {\ttfamily obj} is an instance of the vtk\-Group\-Leaf\-Vertices class. 
\begin{DoxyItemize}
\item {\ttfamily string = obj.\-Get\-Class\-Name ()}  
\item {\ttfamily int = obj.\-Is\-A (string name)}  
\item {\ttfamily vtk\-Group\-Leaf\-Vertices = obj.\-New\-Instance ()}  
\item {\ttfamily vtk\-Group\-Leaf\-Vertices = obj.\-Safe\-Down\-Cast (vtk\-Object o)}  
\item {\ttfamily obj.\-Set\-Group\-Domain (string )} -\/ The name of the domain that non-\/leaf vertices will be assigned to. If the input graph already contains vertices in this domain\-:
\begin{DoxyItemize}
\item If the ids for this domain are numeric, starts assignment with max id
\item If the ids for this domain are strings, starts assignment with \char`\"{}group X\char`\"{} where \char`\"{}\-X\char`\"{} is the max id. Default is \char`\"{}group\-\_\-vertex\char`\"{}.  
\end{DoxyItemize}
\item {\ttfamily string = obj.\-Get\-Group\-Domain ()} -\/ The name of the domain that non-\/leaf vertices will be assigned to. If the input graph already contains vertices in this domain\-:
\begin{DoxyItemize}
\item If the ids for this domain are numeric, starts assignment with max id
\item If the ids for this domain are strings, starts assignment with \char`\"{}group X\char`\"{} where \char`\"{}\-X\char`\"{} is the max id. Default is \char`\"{}group\-\_\-vertex\char`\"{}.  
\end{DoxyItemize}
\end{DoxyItemize}\hypertarget{vtkinfovis_vtkisireader}{}\section{vtk\-I\-S\-I\-Reader}\label{vtkinfovis_vtkisireader}
Section\-: \hyperlink{sec_vtkinfovis}{Visualization Toolkit Infovis Classes} \hypertarget{vtkwidgets_vtkxyplotwidget_Usage}{}\subsection{Usage}\label{vtkwidgets_vtkxyplotwidget_Usage}
I\-S\-I is a tagged format for expressing bibliographic citations. Data is structured as a collection of records with each record composed of one-\/to-\/many fields. See

\href{http://isibasic.com/help/helpprn.html#dialog_export_format}{\tt http\-://isibasic.\-com/help/helpprn.\-html\#dialog\-\_\-export\-\_\-format}

for details. vtk\-I\-S\-I\-Reader will convert an I\-S\-I file into a vtk\-Table, with the set of table columns determined dynamically from the contents of the file.

To create an instance of class vtk\-I\-S\-I\-Reader, simply invoke its constructor as follows \begin{DoxyVerb}  obj = vtkISIReader
\end{DoxyVerb}
 \hypertarget{vtkwidgets_vtkxyplotwidget_Methods}{}\subsection{Methods}\label{vtkwidgets_vtkxyplotwidget_Methods}
The class vtk\-I\-S\-I\-Reader has several methods that can be used. They are listed below. Note that the documentation is translated automatically from the V\-T\-K sources, and may not be completely intelligible. When in doubt, consult the V\-T\-K website. In the methods listed below, {\ttfamily obj} is an instance of the vtk\-I\-S\-I\-Reader class. 
\begin{DoxyItemize}
\item {\ttfamily string = obj.\-Get\-Class\-Name ()}  
\item {\ttfamily int = obj.\-Is\-A (string name)}  
\item {\ttfamily vtk\-I\-S\-I\-Reader = obj.\-New\-Instance ()}  
\item {\ttfamily vtk\-I\-S\-I\-Reader = obj.\-Safe\-Down\-Cast (vtk\-Object o)}  
\item {\ttfamily string = obj.\-Get\-File\-Name ()} -\/ Set/get the file to load  
\item {\ttfamily obj.\-Set\-File\-Name (string )} -\/ Set/get the file to load  
\item {\ttfamily string = obj.\-Get\-Delimiter ()} -\/ Set/get the delimiter to be used for concatenating field data (default\-: \char`\"{};\char`\"{})  
\item {\ttfamily obj.\-Set\-Delimiter (string )} -\/ Set/get the delimiter to be used for concatenating field data (default\-: \char`\"{};\char`\"{})  
\item {\ttfamily int = obj.\-Get\-Max\-Records ()} -\/ Set/get the maximum number of records to read from the file (zero = unlimited)  
\item {\ttfamily obj.\-Set\-Max\-Records (int )} -\/ Set/get the maximum number of records to read from the file (zero = unlimited)  
\end{DoxyItemize}\hypertarget{vtkinfovis_vtkkmeansdistancefunctor}{}\section{vtk\-K\-Means\-Distance\-Functor}\label{vtkinfovis_vtkkmeansdistancefunctor}
Section\-: \hyperlink{sec_vtkinfovis}{Visualization Toolkit Infovis Classes} \hypertarget{vtkwidgets_vtkxyplotwidget_Usage}{}\subsection{Usage}\label{vtkwidgets_vtkxyplotwidget_Usage}
This is an abstract class (with a default concrete subclass) that implements algorithms used by the vtk\-K\-Means\-Statistics filter that rely on a distance metric. If you wish to use a non-\/\-Euclidean distance metric (this could include working with strings that do not have a Euclidean distance metric, implementing k-\/mediods, or trying distance metrics in norms other than L2), you should subclass vtk\-K\-Means\-Distance\-Functor.

To create an instance of class vtk\-K\-Means\-Distance\-Functor, simply invoke its constructor as follows \begin{DoxyVerb}  obj = vtkKMeansDistanceFunctor
\end{DoxyVerb}
 \hypertarget{vtkwidgets_vtkxyplotwidget_Methods}{}\subsection{Methods}\label{vtkwidgets_vtkxyplotwidget_Methods}
The class vtk\-K\-Means\-Distance\-Functor has several methods that can be used. They are listed below. Note that the documentation is translated automatically from the V\-T\-K sources, and may not be completely intelligible. When in doubt, consult the V\-T\-K website. In the methods listed below, {\ttfamily obj} is an instance of the vtk\-K\-Means\-Distance\-Functor class. 
\begin{DoxyItemize}
\item {\ttfamily string = obj.\-Get\-Class\-Name ()}  
\item {\ttfamily int = obj.\-Is\-A (string name)}  
\item {\ttfamily vtk\-K\-Means\-Distance\-Functor = obj.\-New\-Instance ()}  
\item {\ttfamily vtk\-K\-Means\-Distance\-Functor = obj.\-Safe\-Down\-Cast (vtk\-Object o)}  
\item {\ttfamily vtk\-Variant\-Array = obj.\-Get\-Empty\-Tuple (vtk\-Id\-Type dimension)} -\/ Return an empty tuple. These values are used as cluster center coordinates when no initial cluster centers are specified.  
\item {\ttfamily obj.\-Pairwise\-Update (vtk\-Table cluster\-Centers, vtk\-Id\-Type row, vtk\-Variant\-Array data, vtk\-Id\-Type data\-Cardinality, vtk\-Id\-Type total\-Cardinality)} -\/ This is called once per observation per run per iteration in order to assign the observation to its nearest cluster center after the distance functor has been evaluated for all the cluster centers.

The distance functor is responsible for incrementally updating the cluster centers to account for the assignment.  
\item {\ttfamily obj.\-Perturb\-Element (vtk\-Table , vtk\-Table , vtk\-Id\-Type , vtk\-Id\-Type , vtk\-Id\-Type , double )} -\/ When a cluster center (1) has no observations that are closer to it than other cluster centers or (2) has exactly the same coordinates as another cluster center, its coordinates should be perturbed. This function should perform that perturbation.

Since perturbation relies on a distance metric, this function is the responsibility of the distance functor.  
\item {\ttfamily vtk\-Abstract\-Array = obj.\-Create\-Coordinate\-Array ()} -\/ Return a vtk\-Abstract\-Array capable of holding cluster center coordinates. This is used by vtk\-P\-K\-Means\-Statistics to hold cluster center coordinates sent to (received from) other processes.  
\item {\ttfamily int = obj.\-Get\-Data\-Type ()} -\/ Return the data type used to store cluster center coordinates.  
\end{DoxyItemize}\hypertarget{vtkinfovis_vtkkmeansdistancefunctorcalculator}{}\section{vtk\-K\-Means\-Distance\-Functor\-Calculator}\label{vtkinfovis_vtkkmeansdistancefunctorcalculator}
Section\-: \hyperlink{sec_vtkinfovis}{Visualization Toolkit Infovis Classes} \hypertarget{vtkwidgets_vtkxyplotwidget_Usage}{}\subsection{Usage}\label{vtkwidgets_vtkxyplotwidget_Usage}
This is a subclass of the default k-\/means distance functor that allows the user to specify a distance function as a string. The provided expression is evaluated whenever the parenthesis operator is invoked but this is much slower than the default distance calculation.

User-\/specified distance expressions should be written in terms of two vector variables named \char`\"{}x\char`\"{} and \char`\"{}y\char`\"{}. The length of the vectors will be determined by the k-\/means request and all columns of interest in the request must contain values that may be converted to a floating point representation. (Strings and vtk\-Object pointers are not allowed.) An example distance expression is \char`\"{}sqrt( (x0-\/y0)$^\wedge$2 + (x1-\/y1)$^\wedge$2 )\char`\"{} which computes Euclidian distance in a plane defined by the first 2 coordinates of the vectors specified.

To create an instance of class vtk\-K\-Means\-Distance\-Functor\-Calculator, simply invoke its constructor as follows \begin{DoxyVerb}  obj = vtkKMeansDistanceFunctorCalculator
\end{DoxyVerb}
 \hypertarget{vtkwidgets_vtkxyplotwidget_Methods}{}\subsection{Methods}\label{vtkwidgets_vtkxyplotwidget_Methods}
The class vtk\-K\-Means\-Distance\-Functor\-Calculator has several methods that can be used. They are listed below. Note that the documentation is translated automatically from the V\-T\-K sources, and may not be completely intelligible. When in doubt, consult the V\-T\-K website. In the methods listed below, {\ttfamily obj} is an instance of the vtk\-K\-Means\-Distance\-Functor\-Calculator class. 
\begin{DoxyItemize}
\item {\ttfamily string = obj.\-Get\-Class\-Name ()}  
\item {\ttfamily int = obj.\-Is\-A (string name)}  
\item {\ttfamily vtk\-K\-Means\-Distance\-Functor\-Calculator = obj.\-New\-Instance ()}  
\item {\ttfamily vtk\-K\-Means\-Distance\-Functor\-Calculator = obj.\-Safe\-Down\-Cast (vtk\-Object o)}  
\item {\ttfamily obj.\-Set\-Distance\-Expression (string )} -\/ Set/get the distance function expression.  
\item {\ttfamily string = obj.\-Get\-Distance\-Expression ()} -\/ Set/get the distance function expression.  
\item {\ttfamily obj.\-Set\-Function\-Parser (vtk\-Function\-Parser )} -\/ Set/get the string containing an expression which evaluates to the distance metric used for k-\/means computation. The scalar variables \char`\"{}x0\char`\"{}, \char`\"{}x1\char`\"{}, ... \char`\"{}xn\char`\"{} and \char`\"{}y0\char`\"{}, \char`\"{}y1\char`\"{}, ..., \char`\"{}yn\char`\"{} refer to the coordinates involved in the computation.  
\item {\ttfamily vtk\-Function\-Parser = obj.\-Get\-Function\-Parser ()} -\/ Set/get the string containing an expression which evaluates to the distance metric used for k-\/means computation. The scalar variables \char`\"{}x0\char`\"{}, \char`\"{}x1\char`\"{}, ... \char`\"{}xn\char`\"{} and \char`\"{}y0\char`\"{}, \char`\"{}y1\char`\"{}, ..., \char`\"{}yn\char`\"{} refer to the coordinates involved in the computation.  
\end{DoxyItemize}\hypertarget{vtkinfovis_vtkkmeansstatistics}{}\section{vtk\-K\-Means\-Statistics}\label{vtkinfovis_vtkkmeansstatistics}
Section\-: \hyperlink{sec_vtkinfovis}{Visualization Toolkit Infovis Classes} \hypertarget{vtkwidgets_vtkxyplotwidget_Usage}{}\subsection{Usage}\label{vtkwidgets_vtkxyplotwidget_Usage}
This class takes as input an optional vtk\-Table on port L\-E\-A\-R\-N\-\_\-\-P\-A\-R\-A\-M\-E\-T\-E\-R\-S specifying initial set(s) of cluster values of the following form\-: 
\begin{DoxyPre}
           K     | Col1            |  ...    | ColN    
      -----------+-----------------+---------+---------------
           M     |clustCoord(1, 1) |  ...    | clustCoord(1, N)
           M     |clustCoord(2, 1) |  ...    | clustCoord(2, N)
           .     |       .         |   .     |        .
           .     |       .         |   .     |        .
           .     |       .         |   .     |        .
           M     |clustCoord(M, 1) |  ...    | clustCoord(M, N)
           L     |clustCoord(1, 1) |  ...    | clustCoord(1, N)
           L     |clustCoord(2, 1) |  ...    | clustCoord(2, N)
           .     |       .         |   .     |        .
           .     |       .         |   .     |        .
           .     |       .         |   .     |        .
           L     |clustCoord(L, 1) |  ...    | clustCoord(L, N)
 \end{DoxyPre}


Because the desired value of K is often not known in advance and the results of the algorithm are dependent on the initial cluster centers, we provide a mechanism for the user to test multiple runs or sets of cluster centers within a single call to the Learn phase. The first column of the table identifies the number of clusters K in the particular run (the entries in this column should be of type vtk\-Id\-Type), while the remaining columns are a subset of the columns contained in the table on port I\-N\-P\-U\-T\-\_\-\-D\-A\-T\-A. We require that all user specified clusters be of the same dimension N and consequently, that the L\-E\-A\-R\-N\-\_\-\-P\-A\-R\-A\-M\-E\-T\-E\-R\-S table have N+1 columns. Due to this restriction, only one request can be processed for each call to the Learn phase and subsequent requests are silently ignored. Note that, if the first column of the L\-E\-A\-R\-N\-\_\-\-P\-A\-R\-A\-M\-E\-T\-E\-R\-S table is not of type vtk\-Id\-Type, then the table will be ignored and a single run will be performed using the first Default\-Number\-Of\-Clusters input data observations as initial cluster centers.

When the user does not supply an initial set of clusters, then the first Default\-Number\-Of\-Clusters input data observations are used as initial cluster centers and a single run is performed.

This class provides the following functionalities, depending on the mode it is executed in\-: Learn\-: calculates new cluster centers for each run. The output metadata on port O\-U\-T\-P\-U\-T\-\_\-\-M\-O\-D\-E\-L is a multiblock dataset containing at a minimum one vtk\-Table with columns specifying the following for each run\-: the run I\-D, number of clusters, number of iterations required for convergence, total error associated with the cluster (sum of squared Euclidean distance from each observation to its nearest cluster center), the cardinality of the cluster, and the new cluster coordinates.

$\ast$\-Derive\-: An additional vtk\-Table is stored in the multiblock dataset output on port O\-U\-T\-P\-U\-T\-\_\-\-M\-O\-D\-E\-L. This table contains columns that store for each run\-: the run\-I\-D, number of clusters, total error for all clusters in the run, local rank, and global rank. The local rank is computed by comparing squared Euclidean errors of all runs with the same number of clusters. The global rank is computed analagously across all runs.

Assess\-: This requires a multiblock dataset (as computed from Learn and Derive) on input port I\-N\-P\-U\-T\-\_\-\-M\-O\-D\-E\-L and tabular data on input port I\-N\-P\-U\-T\-\_\-\-D\-A\-T\-A that contains column names matching those of the tables on input port I\-N\-P\-U\-T\-\_\-\-M\-O\-D\-E\-L. The assess mode reports the closest cluster center and associated squared Euclidean distance of each observation in port I\-N\-P\-U\-T\-\_\-\-D\-A\-T\-A's table to the cluster centers for each run in the multiblock dataset provided on port I\-N\-P\-U\-T\-\_\-\-M\-O\-D\-E\-L.

The code can handle a wide variety of data types as it operates on vtk\-Abstract\-Arrays and is not limited to vtk\-Data\-Arrays. A default distance functor that computes the sum of the squares of the Euclidean distance between two objects is provided (vtk\-K\-Means\-Distance\-Functor). The default distance functor can be overridden to use alternative distance metrics.

.S\-E\-C\-T\-I\-O\-N Thanks Thanks to Janine Bennett, David Thompson, and Philippe Pebay of Sandia National Laboratories for implementing this class.

To create an instance of class vtk\-K\-Means\-Statistics, simply invoke its constructor as follows \begin{DoxyVerb}  obj = vtkKMeansStatistics
\end{DoxyVerb}
 \hypertarget{vtkwidgets_vtkxyplotwidget_Methods}{}\subsection{Methods}\label{vtkwidgets_vtkxyplotwidget_Methods}
The class vtk\-K\-Means\-Statistics has several methods that can be used. They are listed below. Note that the documentation is translated automatically from the V\-T\-K sources, and may not be completely intelligible. When in doubt, consult the V\-T\-K website. In the methods listed below, {\ttfamily obj} is an instance of the vtk\-K\-Means\-Statistics class. 
\begin{DoxyItemize}
\item {\ttfamily string = obj.\-Get\-Class\-Name ()}  
\item {\ttfamily int = obj.\-Is\-A (string name)}  
\item {\ttfamily vtk\-K\-Means\-Statistics = obj.\-New\-Instance ()}  
\item {\ttfamily vtk\-K\-Means\-Statistics = obj.\-Safe\-Down\-Cast (vtk\-Object o)}  
\item {\ttfamily obj.\-Set\-Distance\-Functor (vtk\-K\-Means\-Distance\-Functor )} -\/ Set the Distance\-Functor.  
\item {\ttfamily vtk\-K\-Means\-Distance\-Functor = obj.\-Get\-Distance\-Functor ()} -\/ Set the Distance\-Functor.  
\item {\ttfamily obj.\-Set\-Default\-Number\-Of\-Clusters (int )} -\/ Set/get the {\itshape Default\-Number\-Of\-Clusters}, used when no initial cluster coordinates are specified.  
\item {\ttfamily int = obj.\-Get\-Default\-Number\-Of\-Clusters ()} -\/ Set/get the {\itshape Default\-Number\-Of\-Clusters}, used when no initial cluster coordinates are specified.  
\item {\ttfamily obj.\-Set\-K\-Values\-Array\-Name (string )} -\/ Set/get the K\-Values\-Array\-Name.  
\item {\ttfamily string = obj.\-Get\-K\-Values\-Array\-Name ()} -\/ Set/get the K\-Values\-Array\-Name.  
\item {\ttfamily obj.\-Set\-Max\-Num\-Iterations (int )} -\/ Set/get the Max\-Num\-Iterations used to terminate iterations on cluster center coordinates when the relative tolerance can not be met.  
\item {\ttfamily int = obj.\-Get\-Max\-Num\-Iterations ()} -\/ Set/get the Max\-Num\-Iterations used to terminate iterations on cluster center coordinates when the relative tolerance can not be met.  
\item {\ttfamily obj.\-Set\-Tolerance (double )} -\/ Set/get the relative {\itshape Tolerance} used to terminate iterations on cluster center coordinates.  
\item {\ttfamily double = obj.\-Get\-Tolerance ()} -\/ Set/get the relative {\itshape Tolerance} used to terminate iterations on cluster center coordinates.  
\item {\ttfamily obj.\-Aggregate (vtk\-Data\-Object\-Collection , vtk\-Data\-Object )} -\/ Given a collection of models, calculate aggregate model N\-B\-: not implemented  
\end{DoxyItemize}\hypertarget{vtkinfovis_vtkmatricizearray}{}\section{vtk\-Matricize\-Array}\label{vtkinfovis_vtkmatricizearray}
Section\-: \hyperlink{sec_vtkinfovis}{Visualization Toolkit Infovis Classes} \hypertarget{vtkwidgets_vtkxyplotwidget_Usage}{}\subsection{Usage}\label{vtkwidgets_vtkxyplotwidget_Usage}
Given a sparse input array of arbitrary dimension, creates a sparse output matrix (vtk\-Sparse\-Array$<$double$>$) where each column is a slice along an arbitrary dimension from the source.

.S\-E\-C\-T\-I\-O\-N Thanks Developed by Timothy M. Shead (\href{mailto:tshead@sandia.gov}{\tt tshead@sandia.\-gov}) at Sandia National Laboratories.

To create an instance of class vtk\-Matricize\-Array, simply invoke its constructor as follows \begin{DoxyVerb}  obj = vtkMatricizeArray
\end{DoxyVerb}
 \hypertarget{vtkwidgets_vtkxyplotwidget_Methods}{}\subsection{Methods}\label{vtkwidgets_vtkxyplotwidget_Methods}
The class vtk\-Matricize\-Array has several methods that can be used. They are listed below. Note that the documentation is translated automatically from the V\-T\-K sources, and may not be completely intelligible. When in doubt, consult the V\-T\-K website. In the methods listed below, {\ttfamily obj} is an instance of the vtk\-Matricize\-Array class. 
\begin{DoxyItemize}
\item {\ttfamily string = obj.\-Get\-Class\-Name ()}  
\item {\ttfamily int = obj.\-Is\-A (string name)}  
\item {\ttfamily vtk\-Matricize\-Array = obj.\-New\-Instance ()}  
\item {\ttfamily vtk\-Matricize\-Array = obj.\-Safe\-Down\-Cast (vtk\-Object o)}  
\item {\ttfamily vtk\-Id\-Type = obj.\-Get\-Slice\-Dimension ()} -\/ Returns the 0-\/numbered dimension that will be mapped to columns in the output  
\item {\ttfamily obj.\-Set\-Slice\-Dimension (vtk\-Id\-Type )} -\/ Sets the 0-\/numbered dimension that will be mapped to columns in the output  
\end{DoxyItemize}\hypertarget{vtkinfovis_vtkmergecolumns}{}\section{vtk\-Merge\-Columns}\label{vtkinfovis_vtkmergecolumns}
Section\-: \hyperlink{sec_vtkinfovis}{Visualization Toolkit Infovis Classes} \hypertarget{vtkwidgets_vtkxyplotwidget_Usage}{}\subsection{Usage}\label{vtkwidgets_vtkxyplotwidget_Usage}
vtk\-Merge\-Columns replaces two columns in a table with a single column containing data in both columns. The columns are set using

Set\-Input\-Array\-To\-Process(0, 0, 0, vtk\-Data\-Object\-::\-F\-I\-E\-L\-D\-\_\-\-A\-S\-S\-O\-C\-I\-A\-T\-I\-O\-N\-\_\-\-R\-O\-W\-S, \char`\"{}col1\char`\"{})

and

Set\-Input\-Array\-To\-Process(1, 0, 0, vtk\-Data\-Object\-::\-F\-I\-E\-L\-D\-\_\-\-A\-S\-S\-O\-C\-I\-A\-T\-I\-O\-N\-\_\-\-R\-O\-W\-S, \char`\"{}col2\char`\"{})

where \char`\"{}col1\char`\"{} and \char`\"{}col2\char`\"{} are the names of the columns to merge. The user may also specify the name of the merged column. The arrays must be of the same type. If the arrays are numeric, the values are summed in the merged column. If the arrays are strings, the values are concatenated. The strings are separated by a space if they are both nonempty.

To create an instance of class vtk\-Merge\-Columns, simply invoke its constructor as follows \begin{DoxyVerb}  obj = vtkMergeColumns
\end{DoxyVerb}
 \hypertarget{vtkwidgets_vtkxyplotwidget_Methods}{}\subsection{Methods}\label{vtkwidgets_vtkxyplotwidget_Methods}
The class vtk\-Merge\-Columns has several methods that can be used. They are listed below. Note that the documentation is translated automatically from the V\-T\-K sources, and may not be completely intelligible. When in doubt, consult the V\-T\-K website. In the methods listed below, {\ttfamily obj} is an instance of the vtk\-Merge\-Columns class. 
\begin{DoxyItemize}
\item {\ttfamily string = obj.\-Get\-Class\-Name ()}  
\item {\ttfamily int = obj.\-Is\-A (string name)}  
\item {\ttfamily vtk\-Merge\-Columns = obj.\-New\-Instance ()}  
\item {\ttfamily vtk\-Merge\-Columns = obj.\-Safe\-Down\-Cast (vtk\-Object o)}  
\item {\ttfamily obj.\-Set\-Merged\-Column\-Name (string )} -\/ The name to give the merged column created by this filter.  
\item {\ttfamily string = obj.\-Get\-Merged\-Column\-Name ()} -\/ The name to give the merged column created by this filter.  
\end{DoxyItemize}\hypertarget{vtkinfovis_vtkmergegraphs}{}\section{vtk\-Merge\-Graphs}\label{vtkinfovis_vtkmergegraphs}
Section\-: \hyperlink{sec_vtkinfovis}{Visualization Toolkit Infovis Classes} \hypertarget{vtkwidgets_vtkxyplotwidget_Usage}{}\subsection{Usage}\label{vtkwidgets_vtkxyplotwidget_Usage}
vtk\-Merge\-Graphs combines information from two graphs into one. Both graphs must have pedigree ids assigned to the vertices. The output will contain the vertices/edges in the first graph, in addition to\-:


\begin{DoxyItemize}
\item vertices in the second graph whose pedigree id does not match a vertex in the first input
\end{DoxyItemize}


\begin{DoxyItemize}
\item edges in the second graph
\end{DoxyItemize}

The output will contain the same attribute structure as the input; fields associated only with the second input graph will not be passed to the output. When possible, the vertex/edge data for new vertices and edges will be populated with matching attributes on the second graph. To be considered a matching attribute, the array must have the same name, type, and number of components.

To create an instance of class vtk\-Merge\-Graphs, simply invoke its constructor as follows \begin{DoxyVerb}  obj = vtkMergeGraphs
\end{DoxyVerb}
 \hypertarget{vtkwidgets_vtkxyplotwidget_Methods}{}\subsection{Methods}\label{vtkwidgets_vtkxyplotwidget_Methods}
The class vtk\-Merge\-Graphs has several methods that can be used. They are listed below. Note that the documentation is translated automatically from the V\-T\-K sources, and may not be completely intelligible. When in doubt, consult the V\-T\-K website. In the methods listed below, {\ttfamily obj} is an instance of the vtk\-Merge\-Graphs class. 
\begin{DoxyItemize}
\item {\ttfamily string = obj.\-Get\-Class\-Name ()}  
\item {\ttfamily int = obj.\-Is\-A (string name)}  
\item {\ttfamily vtk\-Merge\-Graphs = obj.\-New\-Instance ()}  
\item {\ttfamily vtk\-Merge\-Graphs = obj.\-Safe\-Down\-Cast (vtk\-Object o)}  
\item {\ttfamily int = obj.\-Extend\-Graph (vtk\-Mutable\-Graph\-Helper g1, vtk\-Graph g2)} -\/ This is the core functionality of the algorithm. Adds edges and vertices from g2 into g1.  
\item {\ttfamily obj.\-Set\-Max\-Edges (vtk\-Id\-Type )} -\/ The maximum number of edges in the combined graph. Default is -\/1, which specifies that there should be no limit on the number of edges.  
\item {\ttfamily vtk\-Id\-Type = obj.\-Get\-Max\-Edges ()} -\/ The maximum number of edges in the combined graph. Default is -\/1, which specifies that there should be no limit on the number of edges.  
\end{DoxyItemize}\hypertarget{vtkinfovis_vtkmergetables}{}\section{vtk\-Merge\-Tables}\label{vtkinfovis_vtkmergetables}
Section\-: \hyperlink{sec_vtkinfovis}{Visualization Toolkit Infovis Classes} \hypertarget{vtkwidgets_vtkxyplotwidget_Usage}{}\subsection{Usage}\label{vtkwidgets_vtkxyplotwidget_Usage}
Combines the columns of two tables into one larger table. The number of rows in the resulting table is the sum of the number of rows in each of the input tables. The number of columns in the output is generally the sum of the number of columns in each input table, except in the case where column names are duplicated in both tables. In this case, if Merge\-Columns\-By\-Name is on (the default), the two columns will be merged into a single column of the same name. If Merge\-Columns\-By\-Name is off, both columns will exist in the output. You may set the First\-Table\-Prefix and Second\-Table\-Prefix to define how the columns named are modified. One of these prefixes may be the empty string, but they must be different.

To create an instance of class vtk\-Merge\-Tables, simply invoke its constructor as follows \begin{DoxyVerb}  obj = vtkMergeTables
\end{DoxyVerb}
 \hypertarget{vtkwidgets_vtkxyplotwidget_Methods}{}\subsection{Methods}\label{vtkwidgets_vtkxyplotwidget_Methods}
The class vtk\-Merge\-Tables has several methods that can be used. They are listed below. Note that the documentation is translated automatically from the V\-T\-K sources, and may not be completely intelligible. When in doubt, consult the V\-T\-K website. In the methods listed below, {\ttfamily obj} is an instance of the vtk\-Merge\-Tables class. 
\begin{DoxyItemize}
\item {\ttfamily string = obj.\-Get\-Class\-Name ()}  
\item {\ttfamily int = obj.\-Is\-A (string name)}  
\item {\ttfamily vtk\-Merge\-Tables = obj.\-New\-Instance ()}  
\item {\ttfamily vtk\-Merge\-Tables = obj.\-Safe\-Down\-Cast (vtk\-Object o)}  
\item {\ttfamily obj.\-Set\-First\-Table\-Prefix (string )} -\/ The prefix to give to same-\/named fields from the first table. Default is \char`\"{}\-Table1.\char`\"{}.  
\item {\ttfamily string = obj.\-Get\-First\-Table\-Prefix ()} -\/ The prefix to give to same-\/named fields from the first table. Default is \char`\"{}\-Table1.\char`\"{}.  
\item {\ttfamily obj.\-Set\-Second\-Table\-Prefix (string )} -\/ The prefix to give to same-\/named fields from the second table. Default is \char`\"{}\-Table2.\char`\"{}.  
\item {\ttfamily string = obj.\-Get\-Second\-Table\-Prefix ()} -\/ The prefix to give to same-\/named fields from the second table. Default is \char`\"{}\-Table2.\char`\"{}.  
\item {\ttfamily obj.\-Set\-Merge\-Columns\-By\-Name (bool )} -\/ If on, merges columns with the same name. If off, keeps both columns, but calls one First\-Table\-Prefix + name, and the other Second\-Table\-Prefix + name. Default is on.  
\item {\ttfamily bool = obj.\-Get\-Merge\-Columns\-By\-Name ()} -\/ If on, merges columns with the same name. If off, keeps both columns, but calls one First\-Table\-Prefix + name, and the other Second\-Table\-Prefix + name. Default is on.  
\item {\ttfamily obj.\-Merge\-Columns\-By\-Name\-On ()} -\/ If on, merges columns with the same name. If off, keeps both columns, but calls one First\-Table\-Prefix + name, and the other Second\-Table\-Prefix + name. Default is on.  
\item {\ttfamily obj.\-Merge\-Columns\-By\-Name\-Off ()} -\/ If on, merges columns with the same name. If off, keeps both columns, but calls one First\-Table\-Prefix + name, and the other Second\-Table\-Prefix + name. Default is on.  
\item {\ttfamily obj.\-Set\-Prefix\-All\-But\-Merged (bool )} -\/ If on, all columns will have prefixes except merged columns. If off, only unmerged columns with the same name will have prefixes. Default is off.  
\item {\ttfamily bool = obj.\-Get\-Prefix\-All\-But\-Merged ()} -\/ If on, all columns will have prefixes except merged columns. If off, only unmerged columns with the same name will have prefixes. Default is off.  
\item {\ttfamily obj.\-Prefix\-All\-But\-Merged\-On ()} -\/ If on, all columns will have prefixes except merged columns. If off, only unmerged columns with the same name will have prefixes. Default is off.  
\item {\ttfamily obj.\-Prefix\-All\-But\-Merged\-Off ()} -\/ If on, all columns will have prefixes except merged columns. If off, only unmerged columns with the same name will have prefixes. Default is off.  
\end{DoxyItemize}\hypertarget{vtkinfovis_vtkmulticorrelativestatistics}{}\section{vtk\-Multi\-Correlative\-Statistics}\label{vtkinfovis_vtkmulticorrelativestatistics}
Section\-: \hyperlink{sec_vtkinfovis}{Visualization Toolkit Infovis Classes} \hypertarget{vtkwidgets_vtkxyplotwidget_Usage}{}\subsection{Usage}\label{vtkwidgets_vtkxyplotwidget_Usage}
Given a selection of sets of columns of interest, this class provides the following functionalities, depending on the execution mode it is executed in\-: Learn\-: calculates means, unbiased variance and covariance estimators of column pairs coefficient. More precisely, Learn calculates the averages and centered variance/covariance sums; if {\ttfamily finalize} is set to true (default), the final statistics are calculated. The output metadata on port O\-U\-T\-P\-U\-T\-\_\-\-M\-O\-D\-E\-L is a multiblock dataset containing at a minimum one vtk\-Table holding the raw sums in a sparse matrix style. If {\itshape finalize} is true, then one additional vtk\-Table will be present for each requested set of column correlations. These additional tables contain column averages, the upper triangular portion of the covariance matrix (in the upper right hand portion of the table) and the Cholesky decomposition of the covariance matrix (in the lower portion of the table beneath the covariance triangle). The leftmost column will be a vector of column averages. The last entry in the column averages vector is the number of samples. As an example, consider a request for a 3-\/column correlation with columns named Col\-A, Col\-B, and Col\-C. The resulting table will look like this\-: 
\begin{DoxyPre}
      Column  |Mean     |ColA     |ColB     |ColC
      --------+---------+---------+---------+---------
      ColA    |avg(A)   |cov(A,A) |cov(A,B) |cov(A,C)
      ColB    |avg(B)   |chol(1,1)|cov(B,B) |cov(B,C)
      ColC    |avg(C)   |chol(2,1)|chol(2,2)|cov(C,C)
      Cholesky|length(A)|chol(3,1)|chol(3,2)|chol(3,3)
   \end{DoxyPre}
 Assess\-: given a set of results matrices as specified above in input port I\-N\-P\-U\-T\-\_\-\-M\-O\-D\-E\-L and tabular data on input port I\-N\-P\-U\-T\-\_\-\-D\-A\-T\-A that contains column names matching those of the tables on input port I\-N\-P\-U\-T\-\_\-\-M\-O\-D\-E\-L, the assess mode computes the relative deviation of each observation in port I\-N\-P\-U\-T\-\_\-\-D\-A\-T\-A's table according to the linear correlations implied by each table in port I\-N\-P\-U\-T\-\_\-\-M\-O\-D\-E\-L.

.S\-E\-C\-T\-I\-O\-N Thanks Thanks to Philippe Pebay, Jackson Mayo, and David Thompson of Sandia National Laboratories for implementing this class.

To create an instance of class vtk\-Multi\-Correlative\-Statistics, simply invoke its constructor as follows \begin{DoxyVerb}  obj = vtkMultiCorrelativeStatistics
\end{DoxyVerb}
 \hypertarget{vtkwidgets_vtkxyplotwidget_Methods}{}\subsection{Methods}\label{vtkwidgets_vtkxyplotwidget_Methods}
The class vtk\-Multi\-Correlative\-Statistics has several methods that can be used. They are listed below. Note that the documentation is translated automatically from the V\-T\-K sources, and may not be completely intelligible. When in doubt, consult the V\-T\-K website. In the methods listed below, {\ttfamily obj} is an instance of the vtk\-Multi\-Correlative\-Statistics class. 
\begin{DoxyItemize}
\item {\ttfamily string = obj.\-Get\-Class\-Name ()}  
\item {\ttfamily int = obj.\-Is\-A (string name)}  
\item {\ttfamily vtk\-Multi\-Correlative\-Statistics = obj.\-New\-Instance ()}  
\item {\ttfamily vtk\-Multi\-Correlative\-Statistics = obj.\-Safe\-Down\-Cast (vtk\-Object o)}  
\item {\ttfamily obj.\-Aggregate (vtk\-Data\-Object\-Collection , vtk\-Data\-Object )} -\/ Given a collection of models, calculate aggregate model  
\end{DoxyItemize}\hypertarget{vtkinfovis_vtkmutablegraphhelper}{}\section{vtk\-Mutable\-Graph\-Helper}\label{vtkinfovis_vtkmutablegraphhelper}
Section\-: \hyperlink{sec_vtkinfovis}{Visualization Toolkit Infovis Classes} \hypertarget{vtkwidgets_vtkxyplotwidget_Usage}{}\subsection{Usage}\label{vtkwidgets_vtkxyplotwidget_Usage}
vtk\-Mutable\-Graph\-Helper has helper methods Add\-Vertex and Add\-Edge which add vertices/edges to the underlying mutable graph. This is helpful in filters which need to (re)construct graphs which may be either directed or undirected.

To create an instance of class vtk\-Mutable\-Graph\-Helper, simply invoke its constructor as follows \begin{DoxyVerb}  obj = vtkMutableGraphHelper
\end{DoxyVerb}
 \hypertarget{vtkwidgets_vtkxyplotwidget_Methods}{}\subsection{Methods}\label{vtkwidgets_vtkxyplotwidget_Methods}
The class vtk\-Mutable\-Graph\-Helper has several methods that can be used. They are listed below. Note that the documentation is translated automatically from the V\-T\-K sources, and may not be completely intelligible. When in doubt, consult the V\-T\-K website. In the methods listed below, {\ttfamily obj} is an instance of the vtk\-Mutable\-Graph\-Helper class. 
\begin{DoxyItemize}
\item {\ttfamily string = obj.\-Get\-Class\-Name ()}  
\item {\ttfamily int = obj.\-Is\-A (string name)}  
\item {\ttfamily vtk\-Mutable\-Graph\-Helper = obj.\-New\-Instance ()}  
\item {\ttfamily vtk\-Mutable\-Graph\-Helper = obj.\-Safe\-Down\-Cast (vtk\-Object o)}  
\item {\ttfamily obj.\-Set\-Graph (vtk\-Graph g)} -\/ Set the underlying graph that you want to modify with this helper. The graph must be an instance of vtk\-Mutable\-Directed\-Graph or vtk\-Mutable\-Undirected\-Graph.  
\item {\ttfamily vtk\-Graph = obj.\-Get\-Graph ()} -\/ Set the underlying graph that you want to modify with this helper. The graph must be an instance of vtk\-Mutable\-Directed\-Graph or vtk\-Mutable\-Undirected\-Graph.  
\item {\ttfamily vtk\-Graph\-Edge = obj.\-Add\-Graph\-Edge (vtk\-Id\-Type u, vtk\-Id\-Type v)} -\/ Add an edge to the underlying mutable graph.  
\item {\ttfamily vtk\-Id\-Type = obj.\-Add\-Vertex ()} -\/ Add a vertex to the underlying mutable graph.  
\item {\ttfamily obj.\-Remove\-Vertex (vtk\-Id\-Type v)} -\/ Remove a vertex from the underlying mutable graph.  
\item {\ttfamily obj.\-Remove\-Vertices (vtk\-Id\-Type\-Array verts)} -\/ Remove a collection of vertices from the underlying mutable graph.  
\item {\ttfamily obj.\-Remove\-Edge (vtk\-Id\-Type e)} -\/ Remove an edge from the underlying mutable graph.  
\item {\ttfamily obj.\-Remove\-Edges (vtk\-Id\-Type\-Array edges)} -\/ Remove a collection of edges from the underlying mutable graph.  
\end{DoxyItemize}\hypertarget{vtkinfovis_vtknetworkhierarchy}{}\section{vtk\-Network\-Hierarchy}\label{vtkinfovis_vtknetworkhierarchy}
Section\-: \hyperlink{sec_vtkinfovis}{Visualization Toolkit Infovis Classes} \hypertarget{vtkwidgets_vtkxyplotwidget_Usage}{}\subsection{Usage}\label{vtkwidgets_vtkxyplotwidget_Usage}
Use Set\-Input\-Array\-To\-Process(0, ...) to set the array to that has the network ip addresses. Currently this array must be a vtk\-String\-Array.

To create an instance of class vtk\-Network\-Hierarchy, simply invoke its constructor as follows \begin{DoxyVerb}  obj = vtkNetworkHierarchy
\end{DoxyVerb}
 \hypertarget{vtkwidgets_vtkxyplotwidget_Methods}{}\subsection{Methods}\label{vtkwidgets_vtkxyplotwidget_Methods}
The class vtk\-Network\-Hierarchy has several methods that can be used. They are listed below. Note that the documentation is translated automatically from the V\-T\-K sources, and may not be completely intelligible. When in doubt, consult the V\-T\-K website. In the methods listed below, {\ttfamily obj} is an instance of the vtk\-Network\-Hierarchy class. 
\begin{DoxyItemize}
\item {\ttfamily string = obj.\-Get\-Class\-Name ()}  
\item {\ttfamily int = obj.\-Is\-A (string name)}  
\item {\ttfamily vtk\-Network\-Hierarchy = obj.\-New\-Instance ()}  
\item {\ttfamily vtk\-Network\-Hierarchy = obj.\-Safe\-Down\-Cast (vtk\-Object o)}  
\item {\ttfamily string = obj.\-Get\-I\-P\-Array\-Name ()} -\/ Used to store the ip array name  
\item {\ttfamily obj.\-Set\-I\-P\-Array\-Name (string )} -\/ Used to store the ip array name  
\end{DoxyItemize}\hypertarget{vtkinfovis_vtkorderstatistics}{}\section{vtk\-Order\-Statistics}\label{vtkinfovis_vtkorderstatistics}
Section\-: \hyperlink{sec_vtkinfovis}{Visualization Toolkit Infovis Classes} \hypertarget{vtkwidgets_vtkxyplotwidget_Usage}{}\subsection{Usage}\label{vtkwidgets_vtkxyplotwidget_Usage}
Given a selection of columns of interest in an input data table, this class provides the following functionalities, depending on the execution mode it is executed in\-: Learn\-: calculate 5-\/point statistics (minimum, 1st quartile, median, third quartile, maximum) and all other deciles (1,2,3,4,6,7,8,9). Assess\-: given an input data set in port I\-N\-P\-U\-T\-\_\-\-D\-A\-T\-A, and two percentiles p1 $<$ p2, assess all entries in the data set which are outside of \mbox{[}p1,p2\mbox{]}.

.S\-E\-C\-T\-I\-O\-N Thanks Thanks to Philippe Pebay and David Thompson from Sandia National Laboratories for implementing this class.

To create an instance of class vtk\-Order\-Statistics, simply invoke its constructor as follows \begin{DoxyVerb}  obj = vtkOrderStatistics
\end{DoxyVerb}
 \hypertarget{vtkwidgets_vtkxyplotwidget_Methods}{}\subsection{Methods}\label{vtkwidgets_vtkxyplotwidget_Methods}
The class vtk\-Order\-Statistics has several methods that can be used. They are listed below. Note that the documentation is translated automatically from the V\-T\-K sources, and may not be completely intelligible. When in doubt, consult the V\-T\-K website. In the methods listed below, {\ttfamily obj} is an instance of the vtk\-Order\-Statistics class. 
\begin{DoxyItemize}
\item {\ttfamily string = obj.\-Get\-Class\-Name ()}  
\item {\ttfamily int = obj.\-Is\-A (string name)}  
\item {\ttfamily vtk\-Order\-Statistics = obj.\-New\-Instance ()}  
\item {\ttfamily vtk\-Order\-Statistics = obj.\-Safe\-Down\-Cast (vtk\-Object o)}  
\item {\ttfamily obj.\-Set\-Number\-Of\-Intervals (vtk\-Id\-Type )} -\/ Set the number of quantiles (with uniform spacing).  
\item {\ttfamily vtk\-Id\-Type = obj.\-Get\-Number\-Of\-Intervals ()} -\/ Get the number of quantiles (with uniform spacing).  
\item {\ttfamily obj.\-Set\-Quantile\-Definition (int )} -\/ Set the quantile definition.  
\item {\ttfamily vtk\-Id\-Type = obj.\-Get\-Quantile\-Definition ()} -\/ Given a collection of models, calculate aggregate model N\-B\-: not implemented  
\item {\ttfamily obj.\-Aggregate (vtk\-Data\-Object\-Collection , vtk\-Data\-Object )} -\/ Given a collection of models, calculate aggregate model N\-B\-: not implemented  
\end{DoxyItemize}\hypertarget{vtkinfovis_vtkpairwiseextracthistogram2d}{}\section{vtk\-Pairwise\-Extract\-Histogram2\-D}\label{vtkinfovis_vtkpairwiseextracthistogram2d}
Section\-: \hyperlink{sec_vtkinfovis}{Visualization Toolkit Infovis Classes} \hypertarget{vtkwidgets_vtkxyplotwidget_Usage}{}\subsection{Usage}\label{vtkwidgets_vtkxyplotwidget_Usage}
This class computes a 2\-D histogram between all adjacent pairs of columns of an input vtk\-Table. Internally it creates multiple vtk\-Extract\-Histogram2\-D instances (one for each pair of adjacent table columns). It also manages updating histogram computations intelligently, only recomputing those histograms for whom a relevant property has been altered.

Note that there are two different outputs from this filter. One is a table for which each column contains a flattened 2\-D histogram array. The other is a vtk\-Multi\-Block\-Data\-Set for which each block is a vtk\-Image\-Data representation of the 2\-D histogram.

To create an instance of class vtk\-Pairwise\-Extract\-Histogram2\-D, simply invoke its constructor as follows \begin{DoxyVerb}  obj = vtkPairwiseExtractHistogram2D
\end{DoxyVerb}
 \hypertarget{vtkwidgets_vtkxyplotwidget_Methods}{}\subsection{Methods}\label{vtkwidgets_vtkxyplotwidget_Methods}
The class vtk\-Pairwise\-Extract\-Histogram2\-D has several methods that can be used. They are listed below. Note that the documentation is translated automatically from the V\-T\-K sources, and may not be completely intelligible. When in doubt, consult the V\-T\-K website. In the methods listed below, {\ttfamily obj} is an instance of the vtk\-Pairwise\-Extract\-Histogram2\-D class. 
\begin{DoxyItemize}
\item {\ttfamily string = obj.\-Get\-Class\-Name ()}  
\item {\ttfamily int = obj.\-Is\-A (string name)}  
\item {\ttfamily vtk\-Pairwise\-Extract\-Histogram2\-D = obj.\-New\-Instance ()}  
\item {\ttfamily vtk\-Pairwise\-Extract\-Histogram2\-D = obj.\-Safe\-Down\-Cast (vtk\-Object o)}  
\item {\ttfamily obj.\-Set\-Number\-Of\-Bins (int , int )} -\/ Set/get the bin dimensions of the histograms to compute  
\item {\ttfamily obj.\-Set\-Number\-Of\-Bins (int a\mbox{[}2\mbox{]})} -\/ Set/get the bin dimensions of the histograms to compute  
\item {\ttfamily int = obj. Get\-Number\-Of\-Bins ()} -\/ Set/get the bin dimensions of the histograms to compute  
\item {\ttfamily obj.\-Set\-Custom\-Column\-Range\-Index (int )} -\/ Strange method for setting an index to be used for setting custom column range. This was (probably) necessary to get this class to interact with the Para\-View client/server message passing interface.  
\item {\ttfamily obj.\-Set\-Custom\-Column\-Range\-By\-Index (double , double )} -\/ Strange method for setting an index to be used for setting custom column range. This was (probably) necessary to get this class to interact with the Para\-View client/server message passing interface.  
\item {\ttfamily obj.\-Set\-Custom\-Column\-Range (int col, double range\mbox{[}2\mbox{]})} -\/ More standard way to set the custom range for a particular column. This makes sure that only the affected histograms know that they need to be updated.  
\item {\ttfamily obj.\-Set\-Custom\-Column\-Range (int col, double rmin, double rmax)} -\/ More standard way to set the custom range for a particular column. This makes sure that only the affected histograms know that they need to be updated.  
\item {\ttfamily obj.\-Set\-Scalar\-Type (int )} -\/ Set the scalar type for each of the computed histograms.  
\item {\ttfamily obj.\-Set\-Scalar\-Type\-To\-Unsigned\-Int ()} -\/ Set the scalar type for each of the computed histograms.  
\item {\ttfamily obj.\-Set\-Scalar\-Type\-To\-Unsigned\-Long ()} -\/ Set the scalar type for each of the computed histograms.  
\item {\ttfamily obj.\-Set\-Scalar\-Type\-To\-Unsigned\-Short ()} -\/ Set the scalar type for each of the computed histograms.  
\item {\ttfamily obj.\-Set\-Scalar\-Type\-To\-Unsigned\-Char ()} -\/ Set the scalar type for each of the computed histograms.  
\item {\ttfamily int = obj.\-Get\-Scalar\-Type ()} -\/ Set the scalar type for each of the computed histograms.  
\item {\ttfamily double = obj.\-Get\-Maximum\-Bin\-Count (int idx)} -\/ Get the maximum bin count for a single histogram  
\item {\ttfamily double = obj.\-Get\-Maximum\-Bin\-Count ()} -\/ Get the maximum bin count over all histograms  
\item {\ttfamily int = obj.\-Get\-Bin\-Range (int idx, vtk\-Id\-Type bin\-X, vtk\-Id\-Type bin\-Y, double range\mbox{[}4\mbox{]})} -\/ Compute the range of the bin located at position (bin\-X,bin\-Y) in the 2\-D histogram at idx.  
\item {\ttfamily int = obj.\-Get\-Bin\-Range (int idx, vtk\-Id\-Type bin, double range\mbox{[}4\mbox{]})} -\/ Get the range of the of the bin located at 1\-D position index bin in the 2\-D histogram array at idx.  
\item {\ttfamily obj.\-Get\-Bin\-Width (int idx, double bw\mbox{[}2\mbox{]})} -\/ Get the width of all of the bins. Also stored in the spacing ivar of the histogram image output at idx.  
\item {\ttfamily vtk\-Image\-Data = obj.\-Get\-Output\-Histogram\-Image (int idx)} -\/ Get the vtk\-Image\-Data output of the idx'th histogram filter  
\item {\ttfamily vtk\-Extract\-Histogram2\-D = obj.\-Get\-Histogram\-Filter (int idx)} -\/ Get a pointer to the idx'th histogram filter.  
\item {\ttfamily obj.\-Aggregate (vtk\-Data\-Object\-Collection , vtk\-Data\-Object )} -\/ Given a collection of models, calculate aggregate model. Not used  
\end{DoxyItemize}\hypertarget{vtkinfovis_vtkpassarrays}{}\section{vtk\-Pass\-Arrays}\label{vtkinfovis_vtkpassarrays}
Section\-: \hyperlink{sec_vtkinfovis}{Visualization Toolkit Infovis Classes} \hypertarget{vtkwidgets_vtkxyplotwidget_Usage}{}\subsection{Usage}\label{vtkwidgets_vtkxyplotwidget_Usage}
This filter preserves all the topology of the input, but only a subset of arrays are passed to the output. Add an array to be passed to the output data object with Add\-Array(). If Remove\-Arrays is on, the specified arrays will be the ones that are removed instead of the ones that are kept.

Arrays with special attributes (scalars, pedigree ids, etc.) will retain those attributes in the output.

By default, only those field types with at least one array specified through Add\-Array will be processed. If instead Use\-Field\-Types is turned on, you explicitly set which field types to process with Add\-Field\-Type.

Example 1\-:


\begin{DoxyPre}
 passArray->AddArray(vtkDataObject::POINT, "velocity");
 \end{DoxyPre}


The output will have only that one array \char`\"{}velocity\char`\"{} in the point data, but cell and field data will be untouched.

Example 2\-:


\begin{DoxyPre}
 passArray->AddArray(vtkDataObject::POINT, "velocity");
 passArray->UseFieldTypesOn();
 passArray->AddFieldType(vtkDataObject::POINT);
 passArray->AddFieldType(vtkDataObject::CELL);
 \end{DoxyPre}


The point data would still contain the single array, but the cell data would be cleared since you did not specify any arrays to pass. Field data would still be untouched.

To create an instance of class vtk\-Pass\-Arrays, simply invoke its constructor as follows \begin{DoxyVerb}  obj = vtkPassArrays
\end{DoxyVerb}
 \hypertarget{vtkwidgets_vtkxyplotwidget_Methods}{}\subsection{Methods}\label{vtkwidgets_vtkxyplotwidget_Methods}
The class vtk\-Pass\-Arrays has several methods that can be used. They are listed below. Note that the documentation is translated automatically from the V\-T\-K sources, and may not be completely intelligible. When in doubt, consult the V\-T\-K website. In the methods listed below, {\ttfamily obj} is an instance of the vtk\-Pass\-Arrays class. 
\begin{DoxyItemize}
\item {\ttfamily string = obj.\-Get\-Class\-Name ()}  
\item {\ttfamily int = obj.\-Is\-A (string name)}  
\item {\ttfamily vtk\-Pass\-Arrays = obj.\-New\-Instance ()}  
\item {\ttfamily vtk\-Pass\-Arrays = obj.\-Safe\-Down\-Cast (vtk\-Object o)}  
\item {\ttfamily obj.\-Add\-Array (int field\-Type, string name)} -\/ Adds an array to pass through. field\-Type where the array that should be passed (point data, cell data, etc.). It should be one of the constants defined in the vtk\-Data\-Object\-::\-Attribute\-Types enumeration.  
\item {\ttfamily obj.\-Clear\-Arrays ()} -\/ Clear all arrays to pass through.  
\item {\ttfamily obj.\-Set\-Remove\-Arrays (bool )} -\/ Instead of passing only the specified arrays, remove the specified arrays and keep all other arrays. Default is off.  
\item {\ttfamily bool = obj.\-Get\-Remove\-Arrays ()} -\/ Instead of passing only the specified arrays, remove the specified arrays and keep all other arrays. Default is off.  
\item {\ttfamily obj.\-Remove\-Arrays\-On ()} -\/ Instead of passing only the specified arrays, remove the specified arrays and keep all other arrays. Default is off.  
\item {\ttfamily obj.\-Remove\-Arrays\-Off ()} -\/ Instead of passing only the specified arrays, remove the specified arrays and keep all other arrays. Default is off.  
\item {\ttfamily obj.\-Set\-Use\-Field\-Types (bool )} -\/ Process only those field types explicitly specified with Add\-Field\-Type. Otherwise, processes field types associated with at least one specified array. Default is off.  
\item {\ttfamily bool = obj.\-Get\-Use\-Field\-Types ()} -\/ Process only those field types explicitly specified with Add\-Field\-Type. Otherwise, processes field types associated with at least one specified array. Default is off.  
\item {\ttfamily obj.\-Use\-Field\-Types\-On ()} -\/ Process only those field types explicitly specified with Add\-Field\-Type. Otherwise, processes field types associated with at least one specified array. Default is off.  
\item {\ttfamily obj.\-Use\-Field\-Types\-Off ()} -\/ Process only those field types explicitly specified with Add\-Field\-Type. Otherwise, processes field types associated with at least one specified array. Default is off.  
\item {\ttfamily obj.\-Add\-Field\-Type (int field\-Type)} -\/ Add a field type to process. field\-Type where the array that should be passed (point data, cell data, etc.). It should be one of the constants defined in the vtk\-Data\-Object\-::\-Attribute\-Types enumeration. N\-O\-T\-E\-: These are only used if Use\-Field\-Type is turned on.  
\item {\ttfamily obj.\-Clear\-Field\-Types ()} -\/ Clear all field types to process.  
\end{DoxyItemize}\hypertarget{vtkinfovis_vtkpassthrough}{}\section{vtk\-Pass\-Through}\label{vtkinfovis_vtkpassthrough}
Section\-: \hyperlink{sec_vtkinfovis}{Visualization Toolkit Infovis Classes} \hypertarget{vtkwidgets_vtkxyplotwidget_Usage}{}\subsection{Usage}\label{vtkwidgets_vtkxyplotwidget_Usage}
The output type is always the same as the input object type.

To create an instance of class vtk\-Pass\-Through, simply invoke its constructor as follows \begin{DoxyVerb}  obj = vtkPassThrough
\end{DoxyVerb}
 \hypertarget{vtkwidgets_vtkxyplotwidget_Methods}{}\subsection{Methods}\label{vtkwidgets_vtkxyplotwidget_Methods}
The class vtk\-Pass\-Through has several methods that can be used. They are listed below. Note that the documentation is translated automatically from the V\-T\-K sources, and may not be completely intelligible. When in doubt, consult the V\-T\-K website. In the methods listed below, {\ttfamily obj} is an instance of the vtk\-Pass\-Through class. 
\begin{DoxyItemize}
\item {\ttfamily string = obj.\-Get\-Class\-Name ()}  
\item {\ttfamily int = obj.\-Is\-A (string name)}  
\item {\ttfamily vtk\-Pass\-Through = obj.\-New\-Instance ()}  
\item {\ttfamily vtk\-Pass\-Through = obj.\-Safe\-Down\-Cast (vtk\-Object o)}  
\item {\ttfamily int = obj.\-Fill\-Input\-Port\-Information (int port, vtk\-Information info)} -\/ Specify the first input port as optional  
\item {\ttfamily obj.\-Set\-Deep\-Copy\-Input (int )} -\/ Whether or not to deep copy the input. This can be useful if you want to create a copy of a data object. You can then disconnect this filter's input connections and it will act like a source. Defaults to O\-F\-F.  
\item {\ttfamily int = obj.\-Get\-Deep\-Copy\-Input ()} -\/ Whether or not to deep copy the input. This can be useful if you want to create a copy of a data object. You can then disconnect this filter's input connections and it will act like a source. Defaults to O\-F\-F.  
\item {\ttfamily obj.\-Deep\-Copy\-Input\-On ()} -\/ Whether or not to deep copy the input. This can be useful if you want to create a copy of a data object. You can then disconnect this filter's input connections and it will act like a source. Defaults to O\-F\-F.  
\item {\ttfamily obj.\-Deep\-Copy\-Input\-Off ()} -\/ Whether or not to deep copy the input. This can be useful if you want to create a copy of a data object. You can then disconnect this filter's input connections and it will act like a source. Defaults to O\-F\-F.  
\end{DoxyItemize}\hypertarget{vtkinfovis_vtkpassthroughedgestrategy}{}\section{vtk\-Pass\-Through\-Edge\-Strategy}\label{vtkinfovis_vtkpassthroughedgestrategy}
Section\-: \hyperlink{sec_vtkinfovis}{Visualization Toolkit Infovis Classes} \hypertarget{vtkwidgets_vtkxyplotwidget_Usage}{}\subsection{Usage}\label{vtkwidgets_vtkxyplotwidget_Usage}
Simply passes existing edge layout information from the input to the output without making changes.

To create an instance of class vtk\-Pass\-Through\-Edge\-Strategy, simply invoke its constructor as follows \begin{DoxyVerb}  obj = vtkPassThroughEdgeStrategy
\end{DoxyVerb}
 \hypertarget{vtkwidgets_vtkxyplotwidget_Methods}{}\subsection{Methods}\label{vtkwidgets_vtkxyplotwidget_Methods}
The class vtk\-Pass\-Through\-Edge\-Strategy has several methods that can be used. They are listed below. Note that the documentation is translated automatically from the V\-T\-K sources, and may not be completely intelligible. When in doubt, consult the V\-T\-K website. In the methods listed below, {\ttfamily obj} is an instance of the vtk\-Pass\-Through\-Edge\-Strategy class. 
\begin{DoxyItemize}
\item {\ttfamily string = obj.\-Get\-Class\-Name ()}  
\item {\ttfamily int = obj.\-Is\-A (string name)}  
\item {\ttfamily vtk\-Pass\-Through\-Edge\-Strategy = obj.\-New\-Instance ()}  
\item {\ttfamily vtk\-Pass\-Through\-Edge\-Strategy = obj.\-Safe\-Down\-Cast (vtk\-Object o)}  
\item {\ttfamily obj.\-Layout ()} -\/ This is the layout method where the graph that was set in Set\-Graph() is laid out.  
\end{DoxyItemize}\hypertarget{vtkinfovis_vtkpassthroughlayoutstrategy}{}\section{vtk\-Pass\-Through\-Layout\-Strategy}\label{vtkinfovis_vtkpassthroughlayoutstrategy}
Section\-: \hyperlink{sec_vtkinfovis}{Visualization Toolkit Infovis Classes} \hypertarget{vtkwidgets_vtkxyplotwidget_Usage}{}\subsection{Usage}\label{vtkwidgets_vtkxyplotwidget_Usage}
Yes, this incredible strategy does absoluted nothing to the data so in affect passes through the graph untouched. This strategy is useful in the cases where the graph is already laid out.

To create an instance of class vtk\-Pass\-Through\-Layout\-Strategy, simply invoke its constructor as follows \begin{DoxyVerb}  obj = vtkPassThroughLayoutStrategy
\end{DoxyVerb}
 \hypertarget{vtkwidgets_vtkxyplotwidget_Methods}{}\subsection{Methods}\label{vtkwidgets_vtkxyplotwidget_Methods}
The class vtk\-Pass\-Through\-Layout\-Strategy has several methods that can be used. They are listed below. Note that the documentation is translated automatically from the V\-T\-K sources, and may not be completely intelligible. When in doubt, consult the V\-T\-K website. In the methods listed below, {\ttfamily obj} is an instance of the vtk\-Pass\-Through\-Layout\-Strategy class. 
\begin{DoxyItemize}
\item {\ttfamily string = obj.\-Get\-Class\-Name ()}  
\item {\ttfamily int = obj.\-Is\-A (string name)}  
\item {\ttfamily vtk\-Pass\-Through\-Layout\-Strategy = obj.\-New\-Instance ()}  
\item {\ttfamily vtk\-Pass\-Through\-Layout\-Strategy = obj.\-Safe\-Down\-Cast (vtk\-Object o)}  
\item {\ttfamily obj.\-Initialize ()} -\/ This strategy sets up some data structures for faster processing of each Layout() call  
\item {\ttfamily obj.\-Layout ()} -\/ This is the layout method where the graph that was set in Set\-Graph() is laid out. The method can either entirely layout the graph or iteratively lay out the graph. If you have an iterative layout please implement the Is\-Layout\-Complete() method.  
\item {\ttfamily int = obj.\-Is\-Layout\-Complete ()}  
\end{DoxyItemize}\hypertarget{vtkinfovis_vtkpbivariatelineartablethreshold}{}\section{vtk\-P\-Bivariate\-Linear\-Table\-Threshold}\label{vtkinfovis_vtkpbivariatelineartablethreshold}
Section\-: \hyperlink{sec_vtkinfovis}{Visualization Toolkit Infovis Classes} \hypertarget{vtkwidgets_vtkxyplotwidget_Usage}{}\subsection{Usage}\label{vtkwidgets_vtkxyplotwidget_Usage}
Perform the table filtering operations provided by vtk\-Bivariate\-Linear\-Table\-Threshold in parallel.

To create an instance of class vtk\-P\-Bivariate\-Linear\-Table\-Threshold, simply invoke its constructor as follows \begin{DoxyVerb}  obj = vtkPBivariateLinearTableThreshold
\end{DoxyVerb}
 \hypertarget{vtkwidgets_vtkxyplotwidget_Methods}{}\subsection{Methods}\label{vtkwidgets_vtkxyplotwidget_Methods}
The class vtk\-P\-Bivariate\-Linear\-Table\-Threshold has several methods that can be used. They are listed below. Note that the documentation is translated automatically from the V\-T\-K sources, and may not be completely intelligible. When in doubt, consult the V\-T\-K website. In the methods listed below, {\ttfamily obj} is an instance of the vtk\-P\-Bivariate\-Linear\-Table\-Threshold class. 
\begin{DoxyItemize}
\item {\ttfamily string = obj.\-Get\-Class\-Name ()}  
\item {\ttfamily int = obj.\-Is\-A (string name)}  
\item {\ttfamily vtk\-P\-Bivariate\-Linear\-Table\-Threshold = obj.\-New\-Instance ()}  
\item {\ttfamily vtk\-P\-Bivariate\-Linear\-Table\-Threshold = obj.\-Safe\-Down\-Cast (vtk\-Object o)}  
\item {\ttfamily obj.\-Set\-Controller (vtk\-Multi\-Process\-Controller )} -\/ Set the vtk\-Multi\-Process\-Controller to be used for combining filter results from the individual nodes.  
\item {\ttfamily vtk\-Multi\-Process\-Controller = obj.\-Get\-Controller ()} -\/ Set the vtk\-Multi\-Process\-Controller to be used for combining filter results from the individual nodes.  
\end{DoxyItemize}\hypertarget{vtkinfovis_vtkpcastatistics}{}\section{vtk\-P\-C\-A\-Statistics}\label{vtkinfovis_vtkpcastatistics}
Section\-: \hyperlink{sec_vtkinfovis}{Visualization Toolkit Infovis Classes} \hypertarget{vtkwidgets_vtkxyplotwidget_Usage}{}\subsection{Usage}\label{vtkwidgets_vtkxyplotwidget_Usage}
This class derives from the multi-\/correlative statistics algorithm and uses the covariance matrix and Cholesky decomposition computed by it. However, when it finalizes the statistics in Learn mode, the P\-C\-A class computes the S\-V\-D of the covariance matrix in order to obtain its eigenvectors.

In the assess mode, the input data are
\begin{DoxyItemize}
\item projected into the basis defined by the eigenvectors,
\item the energy associated with each datum is computed,
\item or some combination thereof. Additionally, the user may specify some threshold energy or eigenvector entry below which the basis is truncated. This allows projection into a lower-\/dimensional state while minimizing (in a least squares sense) the projection error.
\end{DoxyItemize}

.S\-E\-C\-T\-I\-O\-N Thanks Thanks to David Thompson, Philippe Pebay and Jackson Mayo from Sandia National Laboratories for implementing this class.

To create an instance of class vtk\-P\-C\-A\-Statistics, simply invoke its constructor as follows \begin{DoxyVerb}  obj = vtkPCAStatistics
\end{DoxyVerb}
 \hypertarget{vtkwidgets_vtkxyplotwidget_Methods}{}\subsection{Methods}\label{vtkwidgets_vtkxyplotwidget_Methods}
The class vtk\-P\-C\-A\-Statistics has several methods that can be used. They are listed below. Note that the documentation is translated automatically from the V\-T\-K sources, and may not be completely intelligible. When in doubt, consult the V\-T\-K website. In the methods listed below, {\ttfamily obj} is an instance of the vtk\-P\-C\-A\-Statistics class. 
\begin{DoxyItemize}
\item {\ttfamily string = obj.\-Get\-Class\-Name ()}  
\item {\ttfamily int = obj.\-Is\-A (string name)}  
\item {\ttfamily vtk\-P\-C\-A\-Statistics = obj.\-New\-Instance ()}  
\item {\ttfamily vtk\-P\-C\-A\-Statistics = obj.\-Safe\-Down\-Cast (vtk\-Object o)}  
\item {\ttfamily obj.\-Set\-Normalization\-Scheme (int )} -\/ This determines how (or if) the covariance matrix {\itshape cov} is normalized before P\-C\-A.

When set to N\-O\-N\-E, no normalization is performed. This is the default.

When set to T\-R\-I\-A\-N\-G\-L\-E\-\_\-\-S\-P\-E\-C\-I\-F\-I\-E\-D, each entry cov(i,j) is divided by V(i,j). The list V of normalization factors must be set using the Set\-Normalization method before the filter is executed.

When set to D\-I\-A\-G\-O\-N\-A\-L\-\_\-\-S\-P\-E\-C\-I\-F\-I\-E\-D, each entry cov(i,j) is divided by sqrt(V(i)$\ast$\-V(j)). The list V of normalization factors must be set using the Set\-Normalization method before the filter is executed.

When set to D\-I\-A\-G\-O\-N\-A\-L\-\_\-\-V\-A\-R\-I\-A\-N\-C\-E, each entry cov(i,j) is divided by sqrt(cov(i,i)$\ast$cov(j,j)). {\bfseries Warning}\-: Although this is accepted practice in some fields, some people think you should not turn this option on unless there is a good physically-\/based reason for doing so. Much better instead to determine how component magnitudes should be compared using physical reasoning and use D\-I\-A\-G\-O\-N\-A\-L\-\_\-\-S\-P\-E\-C\-I\-F\-I\-E\-D, T\-R\-I\-A\-N\-G\-L\-E\-\_\-\-S\-P\-E\-C\-I\-F\-I\-E\-D, or perform some pre-\/processing to shift and scale input data columns appropriately than to expect magical results from a shady normalization hack.  
\item {\ttfamily int = obj.\-Get\-Normalization\-Scheme ()} -\/ This determines how (or if) the covariance matrix {\itshape cov} is normalized before P\-C\-A.

When set to N\-O\-N\-E, no normalization is performed. This is the default.

When set to T\-R\-I\-A\-N\-G\-L\-E\-\_\-\-S\-P\-E\-C\-I\-F\-I\-E\-D, each entry cov(i,j) is divided by V(i,j). The list V of normalization factors must be set using the Set\-Normalization method before the filter is executed.

When set to D\-I\-A\-G\-O\-N\-A\-L\-\_\-\-S\-P\-E\-C\-I\-F\-I\-E\-D, each entry cov(i,j) is divided by sqrt(V(i)$\ast$\-V(j)). The list V of normalization factors must be set using the Set\-Normalization method before the filter is executed.

When set to D\-I\-A\-G\-O\-N\-A\-L\-\_\-\-V\-A\-R\-I\-A\-N\-C\-E, each entry cov(i,j) is divided by sqrt(cov(i,i)$\ast$cov(j,j)). {\bfseries Warning}\-: Although this is accepted practice in some fields, some people think you should not turn this option on unless there is a good physically-\/based reason for doing so. Much better instead to determine how component magnitudes should be compared using physical reasoning and use D\-I\-A\-G\-O\-N\-A\-L\-\_\-\-S\-P\-E\-C\-I\-F\-I\-E\-D, T\-R\-I\-A\-N\-G\-L\-E\-\_\-\-S\-P\-E\-C\-I\-F\-I\-E\-D, or perform some pre-\/processing to shift and scale input data columns appropriately than to expect magical results from a shady normalization hack.  
\item {\ttfamily obj.\-Set\-Normalization\-Scheme\-By\-Name (string sname)} -\/ This determines how (or if) the covariance matrix {\itshape cov} is normalized before P\-C\-A.

When set to N\-O\-N\-E, no normalization is performed. This is the default.

When set to T\-R\-I\-A\-N\-G\-L\-E\-\_\-\-S\-P\-E\-C\-I\-F\-I\-E\-D, each entry cov(i,j) is divided by V(i,j). The list V of normalization factors must be set using the Set\-Normalization method before the filter is executed.

When set to D\-I\-A\-G\-O\-N\-A\-L\-\_\-\-S\-P\-E\-C\-I\-F\-I\-E\-D, each entry cov(i,j) is divided by sqrt(V(i)$\ast$\-V(j)). The list V of normalization factors must be set using the Set\-Normalization method before the filter is executed.

When set to D\-I\-A\-G\-O\-N\-A\-L\-\_\-\-V\-A\-R\-I\-A\-N\-C\-E, each entry cov(i,j) is divided by sqrt(cov(i,i)$\ast$cov(j,j)). {\bfseries Warning}\-: Although this is accepted practice in some fields, some people think you should not turn this option on unless there is a good physically-\/based reason for doing so. Much better instead to determine how component magnitudes should be compared using physical reasoning and use D\-I\-A\-G\-O\-N\-A\-L\-\_\-\-S\-P\-E\-C\-I\-F\-I\-E\-D, T\-R\-I\-A\-N\-G\-L\-E\-\_\-\-S\-P\-E\-C\-I\-F\-I\-E\-D, or perform some pre-\/processing to shift and scale input data columns appropriately than to expect magical results from a shady normalization hack.  
\item {\ttfamily string = obj.\-Get\-Normalization\-Scheme\-Name (int scheme)} -\/ This determines how (or if) the covariance matrix {\itshape cov} is normalized before P\-C\-A.

When set to N\-O\-N\-E, no normalization is performed. This is the default.

When set to T\-R\-I\-A\-N\-G\-L\-E\-\_\-\-S\-P\-E\-C\-I\-F\-I\-E\-D, each entry cov(i,j) is divided by V(i,j). The list V of normalization factors must be set using the Set\-Normalization method before the filter is executed.

When set to D\-I\-A\-G\-O\-N\-A\-L\-\_\-\-S\-P\-E\-C\-I\-F\-I\-E\-D, each entry cov(i,j) is divided by sqrt(V(i)$\ast$\-V(j)). The list V of normalization factors must be set using the Set\-Normalization method before the filter is executed.

When set to D\-I\-A\-G\-O\-N\-A\-L\-\_\-\-V\-A\-R\-I\-A\-N\-C\-E, each entry cov(i,j) is divided by sqrt(cov(i,i)$\ast$cov(j,j)). {\bfseries Warning}\-: Although this is accepted practice in some fields, some people think you should not turn this option on unless there is a good physically-\/based reason for doing so. Much better instead to determine how component magnitudes should be compared using physical reasoning and use D\-I\-A\-G\-O\-N\-A\-L\-\_\-\-S\-P\-E\-C\-I\-F\-I\-E\-D, T\-R\-I\-A\-N\-G\-L\-E\-\_\-\-S\-P\-E\-C\-I\-F\-I\-E\-D, or perform some pre-\/processing to shift and scale input data columns appropriately than to expect magical results from a shady normalization hack.  
\item {\ttfamily vtk\-Table = obj.\-Get\-Specified\-Normalization ()} -\/ These methods allow you to set/get values used to normalize the covariance matrix before P\-C\-A. The normalization values apply to all requests, so you do not specify a single vector but a 3-\/column table.

The first two columns contain the names of columns from input 0 and the third column contains the value to normalize the corresponding entry in the covariance matrix. The table must always have 3 columns even when the Normalization\-Scheme is D\-I\-A\-G\-O\-N\-A\-L\-\_\-\-S\-P\-E\-C\-I\-F\-I\-E\-D. When only diagonal entries are to be used, only table rows where the first two columns are identical to one another will be employed. If there are multiple rows specifying different values for the same pair of columns, the entry nearest the bottom of the table takes precedence.

These functions are actually convenience methods that set/get the third input of the filter. Because the table is the third input, you may use other filters to produce a table of normalizations and have the pipeline take care of updates.

Any missing entries will be set to 1.\-0 and a warning issued. An error will occur if the third input to the filter is not set and the Normalization\-Scheme is D\-I\-A\-G\-O\-N\-A\-L\-\_\-\-S\-P\-E\-C\-I\-F\-I\-E\-D or T\-R\-I\-A\-N\-G\-L\-E\-\_\-\-S\-P\-E\-C\-I\-F\-I\-E\-D.  
\item {\ttfamily obj.\-Set\-Specified\-Normalization (vtk\-Table )} -\/ These methods allow you to set/get values used to normalize the covariance matrix before P\-C\-A. The normalization values apply to all requests, so you do not specify a single vector but a 3-\/column table.

The first two columns contain the names of columns from input 0 and the third column contains the value to normalize the corresponding entry in the covariance matrix. The table must always have 3 columns even when the Normalization\-Scheme is D\-I\-A\-G\-O\-N\-A\-L\-\_\-\-S\-P\-E\-C\-I\-F\-I\-E\-D. When only diagonal entries are to be used, only table rows where the first two columns are identical to one another will be employed. If there are multiple rows specifying different values for the same pair of columns, the entry nearest the bottom of the table takes precedence.

These functions are actually convenience methods that set/get the third input of the filter. Because the table is the third input, you may use other filters to produce a table of normalizations and have the pipeline take care of updates.

Any missing entries will be set to 1.\-0 and a warning issued. An error will occur if the third input to the filter is not set and the Normalization\-Scheme is D\-I\-A\-G\-O\-N\-A\-L\-\_\-\-S\-P\-E\-C\-I\-F\-I\-E\-D or T\-R\-I\-A\-N\-G\-L\-E\-\_\-\-S\-P\-E\-C\-I\-F\-I\-E\-D.  
\item {\ttfamily obj.\-Set\-Basis\-Scheme (int )} -\/ This variable controls the dimensionality of output tuples in Assess mode. Consider the case where you have requested a P\-C\-A on D columns.

When set to vtk\-P\-C\-A\-Statistics\-::\-F\-U\-L\-L\-\_\-\-B\-A\-S\-I\-S, the entire set of basis vectors is used to derive new coordinates for each tuple being assessed. In this mode, you are guaranteed to have output tuples of the same dimension as the input tuples. (That dimension is D, so there will be D additional columns added to the table for the request.)

When set to vtk\-P\-C\-A\-Statistics\-::\-F\-I\-X\-E\-D\-\_\-\-B\-A\-S\-I\-S\-\_\-\-S\-I\-Z\-E, only the first N basis vectors are used to derive new coordinates for each tuple being assessed. In this mode, you are guaranteed to have output tuples of dimension min(\-N,\-D). You must set N prior to assessing data using the Set\-Fixed\-Basis\-Size() method. When N $<$ D, this turns the P\-C\-A into a projection (instead of change of basis).

When set to vtk\-P\-C\-A\-Statistics\-::\-F\-I\-X\-E\-D\-\_\-\-B\-A\-S\-I\-S\-\_\-\-E\-N\-E\-R\-G\-Y, the number of basis vectors used to derive new coordinates for each tuple will be the minimum number of columns N that satisfy \[ \frac{\sum_{i=1}^{N} \lambda_i}{\sum_{i=1}^{D} \lambda_i} < T \] You must set T prior to assessing data using the Set\-Fixed\-Basis\-Energy() method. When T $<$ 1, this turns the P\-C\-A into a projection (instead of change of basis).

By default Basis\-Scheme is set to vtk\-P\-C\-A\-Statistics\-::\-F\-U\-L\-L\-\_\-\-B\-A\-S\-I\-S.  
\item {\ttfamily int = obj.\-Get\-Basis\-Scheme ()} -\/ This variable controls the dimensionality of output tuples in Assess mode. Consider the case where you have requested a P\-C\-A on D columns.

When set to vtk\-P\-C\-A\-Statistics\-::\-F\-U\-L\-L\-\_\-\-B\-A\-S\-I\-S, the entire set of basis vectors is used to derive new coordinates for each tuple being assessed. In this mode, you are guaranteed to have output tuples of the same dimension as the input tuples. (That dimension is D, so there will be D additional columns added to the table for the request.)

When set to vtk\-P\-C\-A\-Statistics\-::\-F\-I\-X\-E\-D\-\_\-\-B\-A\-S\-I\-S\-\_\-\-S\-I\-Z\-E, only the first N basis vectors are used to derive new coordinates for each tuple being assessed. In this mode, you are guaranteed to have output tuples of dimension min(\-N,\-D). You must set N prior to assessing data using the Set\-Fixed\-Basis\-Size() method. When N $<$ D, this turns the P\-C\-A into a projection (instead of change of basis).

When set to vtk\-P\-C\-A\-Statistics\-::\-F\-I\-X\-E\-D\-\_\-\-B\-A\-S\-I\-S\-\_\-\-E\-N\-E\-R\-G\-Y, the number of basis vectors used to derive new coordinates for each tuple will be the minimum number of columns N that satisfy \[ \frac{\sum_{i=1}^{N} \lambda_i}{\sum_{i=1}^{D} \lambda_i} < T \] You must set T prior to assessing data using the Set\-Fixed\-Basis\-Energy() method. When T $<$ 1, this turns the P\-C\-A into a projection (instead of change of basis).

By default Basis\-Scheme is set to vtk\-P\-C\-A\-Statistics\-::\-F\-U\-L\-L\-\_\-\-B\-A\-S\-I\-S.  
\item {\ttfamily string = obj.\-Get\-Basis\-Scheme\-Name (int scheme\-Index)} -\/ This variable controls the dimensionality of output tuples in Assess mode. Consider the case where you have requested a P\-C\-A on D columns.

When set to vtk\-P\-C\-A\-Statistics\-::\-F\-U\-L\-L\-\_\-\-B\-A\-S\-I\-S, the entire set of basis vectors is used to derive new coordinates for each tuple being assessed. In this mode, you are guaranteed to have output tuples of the same dimension as the input tuples. (That dimension is D, so there will be D additional columns added to the table for the request.)

When set to vtk\-P\-C\-A\-Statistics\-::\-F\-I\-X\-E\-D\-\_\-\-B\-A\-S\-I\-S\-\_\-\-S\-I\-Z\-E, only the first N basis vectors are used to derive new coordinates for each tuple being assessed. In this mode, you are guaranteed to have output tuples of dimension min(\-N,\-D). You must set N prior to assessing data using the Set\-Fixed\-Basis\-Size() method. When N $<$ D, this turns the P\-C\-A into a projection (instead of change of basis).

When set to vtk\-P\-C\-A\-Statistics\-::\-F\-I\-X\-E\-D\-\_\-\-B\-A\-S\-I\-S\-\_\-\-E\-N\-E\-R\-G\-Y, the number of basis vectors used to derive new coordinates for each tuple will be the minimum number of columns N that satisfy \[ \frac{\sum_{i=1}^{N} \lambda_i}{\sum_{i=1}^{D} \lambda_i} < T \] You must set T prior to assessing data using the Set\-Fixed\-Basis\-Energy() method. When T $<$ 1, this turns the P\-C\-A into a projection (instead of change of basis).

By default Basis\-Scheme is set to vtk\-P\-C\-A\-Statistics\-::\-F\-U\-L\-L\-\_\-\-B\-A\-S\-I\-S.  
\item {\ttfamily obj.\-Set\-Basis\-Scheme\-By\-Name (string scheme\-Name)} -\/ This variable controls the dimensionality of output tuples in Assess mode. Consider the case where you have requested a P\-C\-A on D columns.

When set to vtk\-P\-C\-A\-Statistics\-::\-F\-U\-L\-L\-\_\-\-B\-A\-S\-I\-S, the entire set of basis vectors is used to derive new coordinates for each tuple being assessed. In this mode, you are guaranteed to have output tuples of the same dimension as the input tuples. (That dimension is D, so there will be D additional columns added to the table for the request.)

When set to vtk\-P\-C\-A\-Statistics\-::\-F\-I\-X\-E\-D\-\_\-\-B\-A\-S\-I\-S\-\_\-\-S\-I\-Z\-E, only the first N basis vectors are used to derive new coordinates for each tuple being assessed. In this mode, you are guaranteed to have output tuples of dimension min(\-N,\-D). You must set N prior to assessing data using the Set\-Fixed\-Basis\-Size() method. When N $<$ D, this turns the P\-C\-A into a projection (instead of change of basis).

When set to vtk\-P\-C\-A\-Statistics\-::\-F\-I\-X\-E\-D\-\_\-\-B\-A\-S\-I\-S\-\_\-\-E\-N\-E\-R\-G\-Y, the number of basis vectors used to derive new coordinates for each tuple will be the minimum number of columns N that satisfy \[ \frac{\sum_{i=1}^{N} \lambda_i}{\sum_{i=1}^{D} \lambda_i} < T \] You must set T prior to assessing data using the Set\-Fixed\-Basis\-Energy() method. When T $<$ 1, this turns the P\-C\-A into a projection (instead of change of basis).

By default Basis\-Scheme is set to vtk\-P\-C\-A\-Statistics\-::\-F\-U\-L\-L\-\_\-\-B\-A\-S\-I\-S.  
\item {\ttfamily obj.\-Set\-Fixed\-Basis\-Size (int )} -\/ The number of basis vectors to use. See Set\-Basis\-Scheme() for more information. When Fixed\-Basis\-Size $<$= 0 (the default), the fixed basis size scheme is equivalent to the full basis scheme.  
\item {\ttfamily int = obj.\-Get\-Fixed\-Basis\-Size ()} -\/ The number of basis vectors to use. See Set\-Basis\-Scheme() for more information. When Fixed\-Basis\-Size $<$= 0 (the default), the fixed basis size scheme is equivalent to the full basis scheme.  
\item {\ttfamily obj.\-Set\-Fixed\-Basis\-Energy (double )} -\/ The minimum energy the new basis should use, as a fraction. See Set\-Basis\-Scheme() for more information. When Fixed\-Basis\-Energy $>$= 1 (the default), the fixed basis energy scheme is equivalent to the full basis scheme.  
\item {\ttfamily double = obj.\-Get\-Fixed\-Basis\-Energy\-Min\-Value ()} -\/ The minimum energy the new basis should use, as a fraction. See Set\-Basis\-Scheme() for more information. When Fixed\-Basis\-Energy $>$= 1 (the default), the fixed basis energy scheme is equivalent to the full basis scheme.  
\item {\ttfamily double = obj.\-Get\-Fixed\-Basis\-Energy\-Max\-Value ()} -\/ The minimum energy the new basis should use, as a fraction. See Set\-Basis\-Scheme() for more information. When Fixed\-Basis\-Energy $>$= 1 (the default), the fixed basis energy scheme is equivalent to the full basis scheme.  
\item {\ttfamily double = obj.\-Get\-Fixed\-Basis\-Energy ()} -\/ The minimum energy the new basis should use, as a fraction. See Set\-Basis\-Scheme() for more information. When Fixed\-Basis\-Energy $>$= 1 (the default), the fixed basis energy scheme is equivalent to the full basis scheme.  
\end{DoxyItemize}\hypertarget{vtkinfovis_vtkpcomputehistogram2doutliers}{}\section{vtk\-P\-Compute\-Histogram2\-D\-Outliers}\label{vtkinfovis_vtkpcomputehistogram2doutliers}
Section\-: \hyperlink{sec_vtkinfovis}{Visualization Toolkit Infovis Classes} \hypertarget{vtkwidgets_vtkxyplotwidget_Usage}{}\subsection{Usage}\label{vtkwidgets_vtkxyplotwidget_Usage}
This class does exactly the same this as vtk\-Compute\-Histogram2\-D\-Outliers, but does it in a multi-\/process environment. After each node computes their own local outliers, class does an All\-Gather that distributes the outliers to every node. This could probably just be a Gather onto the root node instead.

After this operation, the row selection will only contain local row ids, since I'm not sure how to deal with distributed ids.

To create an instance of class vtk\-P\-Compute\-Histogram2\-D\-Outliers, simply invoke its constructor as follows \begin{DoxyVerb}  obj = vtkPComputeHistogram2DOutliers
\end{DoxyVerb}
 \hypertarget{vtkwidgets_vtkxyplotwidget_Methods}{}\subsection{Methods}\label{vtkwidgets_vtkxyplotwidget_Methods}
The class vtk\-P\-Compute\-Histogram2\-D\-Outliers has several methods that can be used. They are listed below. Note that the documentation is translated automatically from the V\-T\-K sources, and may not be completely intelligible. When in doubt, consult the V\-T\-K website. In the methods listed below, {\ttfamily obj} is an instance of the vtk\-P\-Compute\-Histogram2\-D\-Outliers class. 
\begin{DoxyItemize}
\item {\ttfamily string = obj.\-Get\-Class\-Name ()}  
\item {\ttfamily int = obj.\-Is\-A (string name)}  
\item {\ttfamily vtk\-P\-Compute\-Histogram2\-D\-Outliers = obj.\-New\-Instance ()}  
\item {\ttfamily vtk\-P\-Compute\-Histogram2\-D\-Outliers = obj.\-Safe\-Down\-Cast (vtk\-Object o)}  
\item {\ttfamily obj.\-Set\-Controller (vtk\-Multi\-Process\-Controller )}  
\item {\ttfamily vtk\-Multi\-Process\-Controller = obj.\-Get\-Controller ()}  
\end{DoxyItemize}\hypertarget{vtkinfovis_vtkpcontingencystatistics}{}\section{vtk\-P\-Contingency\-Statistics}\label{vtkinfovis_vtkpcontingencystatistics}
Section\-: \hyperlink{sec_vtkinfovis}{Visualization Toolkit Infovis Classes} \hypertarget{vtkwidgets_vtkxyplotwidget_Usage}{}\subsection{Usage}\label{vtkwidgets_vtkxyplotwidget_Usage}
vtk\-P\-Contingency\-Statistics is vtk\-Contingency\-Statistics subclass for parallel datasets. It learns and derives the global statistical model on each node, but assesses each individual data points on the node that owns it.

To create an instance of class vtk\-P\-Contingency\-Statistics, simply invoke its constructor as follows \begin{DoxyVerb}  obj = vtkPContingencyStatistics
\end{DoxyVerb}
 \hypertarget{vtkwidgets_vtkxyplotwidget_Methods}{}\subsection{Methods}\label{vtkwidgets_vtkxyplotwidget_Methods}
The class vtk\-P\-Contingency\-Statistics has several methods that can be used. They are listed below. Note that the documentation is translated automatically from the V\-T\-K sources, and may not be completely intelligible. When in doubt, consult the V\-T\-K website. In the methods listed below, {\ttfamily obj} is an instance of the vtk\-P\-Contingency\-Statistics class. 
\begin{DoxyItemize}
\item {\ttfamily string = obj.\-Get\-Class\-Name ()}  
\item {\ttfamily int = obj.\-Is\-A (string name)}  
\item {\ttfamily vtk\-P\-Contingency\-Statistics = obj.\-New\-Instance ()}  
\item {\ttfamily vtk\-P\-Contingency\-Statistics = obj.\-Safe\-Down\-Cast (vtk\-Object o)}  
\item {\ttfamily obj.\-Set\-Controller (vtk\-Multi\-Process\-Controller )} -\/ Get/\-Set the multiprocess controller. If no controller is set, single process is assumed.  
\item {\ttfamily vtk\-Multi\-Process\-Controller = obj.\-Get\-Controller ()} -\/ Get/\-Set the multiprocess controller. If no controller is set, single process is assumed.  
\item {\ttfamily obj.\-Learn (vtk\-Table in\-Data, vtk\-Table in\-Parameters, vtk\-Data\-Object out\-Meta)} -\/ Execute the parallel calculations required by the Learn option.  
\end{DoxyItemize}\hypertarget{vtkinfovis_vtkpcorrelativestatistics}{}\section{vtk\-P\-Correlative\-Statistics}\label{vtkinfovis_vtkpcorrelativestatistics}
Section\-: \hyperlink{sec_vtkinfovis}{Visualization Toolkit Infovis Classes} \hypertarget{vtkwidgets_vtkxyplotwidget_Usage}{}\subsection{Usage}\label{vtkwidgets_vtkxyplotwidget_Usage}
vtk\-P\-Correlative\-Statistics is vtk\-Correlative\-Statistics subclass for parallel datasets. It learns and derives the global statistical model on each node, but assesses each individual data points on the node that owns it.

To create an instance of class vtk\-P\-Correlative\-Statistics, simply invoke its constructor as follows \begin{DoxyVerb}  obj = vtkPCorrelativeStatistics
\end{DoxyVerb}
 \hypertarget{vtkwidgets_vtkxyplotwidget_Methods}{}\subsection{Methods}\label{vtkwidgets_vtkxyplotwidget_Methods}
The class vtk\-P\-Correlative\-Statistics has several methods that can be used. They are listed below. Note that the documentation is translated automatically from the V\-T\-K sources, and may not be completely intelligible. When in doubt, consult the V\-T\-K website. In the methods listed below, {\ttfamily obj} is an instance of the vtk\-P\-Correlative\-Statistics class. 
\begin{DoxyItemize}
\item {\ttfamily string = obj.\-Get\-Class\-Name ()}  
\item {\ttfamily int = obj.\-Is\-A (string name)}  
\item {\ttfamily vtk\-P\-Correlative\-Statistics = obj.\-New\-Instance ()}  
\item {\ttfamily vtk\-P\-Correlative\-Statistics = obj.\-Safe\-Down\-Cast (vtk\-Object o)}  
\item {\ttfamily obj.\-Set\-Controller (vtk\-Multi\-Process\-Controller )} -\/ Get/\-Set the multiprocess controller. If no controller is set, single process is assumed.  
\item {\ttfamily vtk\-Multi\-Process\-Controller = obj.\-Get\-Controller ()} -\/ Get/\-Set the multiprocess controller. If no controller is set, single process is assumed.  
\item {\ttfamily obj.\-Learn (vtk\-Table in\-Data, vtk\-Table in\-Parameters, vtk\-Data\-Object out\-Meta)} -\/ Execute the parallel calculations required by the Learn option.  
\end{DoxyItemize}\hypertarget{vtkinfovis_vtkpdescriptivestatistics}{}\section{vtk\-P\-Descriptive\-Statistics}\label{vtkinfovis_vtkpdescriptivestatistics}
Section\-: \hyperlink{sec_vtkinfovis}{Visualization Toolkit Infovis Classes} \hypertarget{vtkwidgets_vtkxyplotwidget_Usage}{}\subsection{Usage}\label{vtkwidgets_vtkxyplotwidget_Usage}
vtk\-P\-Descriptive\-Statistics is vtk\-Descriptive\-Statistics subclass for parallel datasets. It learns and derives the global statistical model on each node, but assesses each individual data points on the node that owns it.

To create an instance of class vtk\-P\-Descriptive\-Statistics, simply invoke its constructor as follows \begin{DoxyVerb}  obj = vtkPDescriptiveStatistics
\end{DoxyVerb}
 \hypertarget{vtkwidgets_vtkxyplotwidget_Methods}{}\subsection{Methods}\label{vtkwidgets_vtkxyplotwidget_Methods}
The class vtk\-P\-Descriptive\-Statistics has several methods that can be used. They are listed below. Note that the documentation is translated automatically from the V\-T\-K sources, and may not be completely intelligible. When in doubt, consult the V\-T\-K website. In the methods listed below, {\ttfamily obj} is an instance of the vtk\-P\-Descriptive\-Statistics class. 
\begin{DoxyItemize}
\item {\ttfamily string = obj.\-Get\-Class\-Name ()}  
\item {\ttfamily int = obj.\-Is\-A (string name)}  
\item {\ttfamily vtk\-P\-Descriptive\-Statistics = obj.\-New\-Instance ()}  
\item {\ttfamily vtk\-P\-Descriptive\-Statistics = obj.\-Safe\-Down\-Cast (vtk\-Object o)}  
\item {\ttfamily obj.\-Set\-Controller (vtk\-Multi\-Process\-Controller )} -\/ Get/\-Set the multiprocess controller. If no controller is set, single process is assumed.  
\item {\ttfamily vtk\-Multi\-Process\-Controller = obj.\-Get\-Controller ()} -\/ Get/\-Set the multiprocess controller. If no controller is set, single process is assumed.  
\item {\ttfamily obj.\-Learn (vtk\-Table in\-Data, vtk\-Table in\-Parameters, vtk\-Data\-Object out\-Meta)} -\/ Execute the parallel calculations required by the Learn option.  
\end{DoxyItemize}\hypertarget{vtkinfovis_vtkperturbcoincidentvertices}{}\section{vtk\-Perturb\-Coincident\-Vertices}\label{vtkinfovis_vtkperturbcoincidentvertices}
Section\-: \hyperlink{sec_vtkinfovis}{Visualization Toolkit Infovis Classes} \hypertarget{vtkwidgets_vtkxyplotwidget_Usage}{}\subsection{Usage}\label{vtkwidgets_vtkxyplotwidget_Usage}
This filter perturbs vertices in a graph that have coincident coordinates. In particular this happens all the time with graphs that are georeferenced, so we need a nice scheme to perturb the vertices so that when the user zooms in the vertices can be distiquished.

To create an instance of class vtk\-Perturb\-Coincident\-Vertices, simply invoke its constructor as follows \begin{DoxyVerb}  obj = vtkPerturbCoincidentVertices
\end{DoxyVerb}
 \hypertarget{vtkwidgets_vtkxyplotwidget_Methods}{}\subsection{Methods}\label{vtkwidgets_vtkxyplotwidget_Methods}
The class vtk\-Perturb\-Coincident\-Vertices has several methods that can be used. They are listed below. Note that the documentation is translated automatically from the V\-T\-K sources, and may not be completely intelligible. When in doubt, consult the V\-T\-K website. In the methods listed below, {\ttfamily obj} is an instance of the vtk\-Perturb\-Coincident\-Vertices class. 
\begin{DoxyItemize}
\item {\ttfamily string = obj.\-Get\-Class\-Name ()}  
\item {\ttfamily int = obj.\-Is\-A (string name)}  
\item {\ttfamily vtk\-Perturb\-Coincident\-Vertices = obj.\-New\-Instance ()}  
\item {\ttfamily vtk\-Perturb\-Coincident\-Vertices = obj.\-Safe\-Down\-Cast (vtk\-Object o)}  
\item {\ttfamily obj.\-Set\-Perturb\-Factor (double )} -\/ Specify the perturbation factor (defaults to 1.\-0)  
\item {\ttfamily double = obj.\-Get\-Perturb\-Factor ()} -\/ Specify the perturbation factor (defaults to 1.\-0)  
\end{DoxyItemize}\hypertarget{vtkinfovis_vtkpextracthistogram2d}{}\section{vtk\-P\-Extract\-Histogram2\-D}\label{vtkinfovis_vtkpextracthistogram2d}
Section\-: \hyperlink{sec_vtkinfovis}{Visualization Toolkit Infovis Classes} \hypertarget{vtkwidgets_vtkxyplotwidget_Usage}{}\subsection{Usage}\label{vtkwidgets_vtkxyplotwidget_Usage}
This class does exactly the same this as vtk\-Extract\-Histogram2\-D, but does it in a multi-\/process environment. After each node computes their own local histograms, this class does an All\-Reduce that distributes the sum of all local histograms onto each node.

To create an instance of class vtk\-P\-Extract\-Histogram2\-D, simply invoke its constructor as follows \begin{DoxyVerb}  obj = vtkPExtractHistogram2D
\end{DoxyVerb}
 \hypertarget{vtkwidgets_vtkxyplotwidget_Methods}{}\subsection{Methods}\label{vtkwidgets_vtkxyplotwidget_Methods}
The class vtk\-P\-Extract\-Histogram2\-D has several methods that can be used. They are listed below. Note that the documentation is translated automatically from the V\-T\-K sources, and may not be completely intelligible. When in doubt, consult the V\-T\-K website. In the methods listed below, {\ttfamily obj} is an instance of the vtk\-P\-Extract\-Histogram2\-D class. 
\begin{DoxyItemize}
\item {\ttfamily string = obj.\-Get\-Class\-Name ()}  
\item {\ttfamily int = obj.\-Is\-A (string name)}  
\item {\ttfamily vtk\-P\-Extract\-Histogram2\-D = obj.\-New\-Instance ()}  
\item {\ttfamily vtk\-P\-Extract\-Histogram2\-D = obj.\-Safe\-Down\-Cast (vtk\-Object o)}  
\item {\ttfamily obj.\-Set\-Controller (vtk\-Multi\-Process\-Controller )}  
\item {\ttfamily vtk\-Multi\-Process\-Controller = obj.\-Get\-Controller ()}  
\end{DoxyItemize}\hypertarget{vtkinfovis_vtkpkmeansstatistics}{}\section{vtk\-P\-K\-Means\-Statistics}\label{vtkinfovis_vtkpkmeansstatistics}
Section\-: \hyperlink{sec_vtkinfovis}{Visualization Toolkit Infovis Classes} \hypertarget{vtkwidgets_vtkxyplotwidget_Usage}{}\subsection{Usage}\label{vtkwidgets_vtkxyplotwidget_Usage}
vtk\-P\-K\-Means\-Statistics is vtk\-K\-Means\-Statistics subclass for parallel datasets. It learns and derives the global statistical model on each node, but assesses each individual data points on the node that owns it.

To create an instance of class vtk\-P\-K\-Means\-Statistics, simply invoke its constructor as follows \begin{DoxyVerb}  obj = vtkPKMeansStatistics
\end{DoxyVerb}
 \hypertarget{vtkwidgets_vtkxyplotwidget_Methods}{}\subsection{Methods}\label{vtkwidgets_vtkxyplotwidget_Methods}
The class vtk\-P\-K\-Means\-Statistics has several methods that can be used. They are listed below. Note that the documentation is translated automatically from the V\-T\-K sources, and may not be completely intelligible. When in doubt, consult the V\-T\-K website. In the methods listed below, {\ttfamily obj} is an instance of the vtk\-P\-K\-Means\-Statistics class. 
\begin{DoxyItemize}
\item {\ttfamily string = obj.\-Get\-Class\-Name ()}  
\item {\ttfamily int = obj.\-Is\-A (string name)}  
\item {\ttfamily vtk\-P\-K\-Means\-Statistics = obj.\-New\-Instance ()}  
\item {\ttfamily vtk\-P\-K\-Means\-Statistics = obj.\-Safe\-Down\-Cast (vtk\-Object o)}  
\item {\ttfamily obj.\-Set\-Controller (vtk\-Multi\-Process\-Controller )} -\/ Get/\-Set the multiprocess controller. If no controller is set, single process is assumed.  
\item {\ttfamily vtk\-Multi\-Process\-Controller = obj.\-Get\-Controller ()} -\/ Get/\-Set the multiprocess controller. If no controller is set, single process is assumed.  
\item {\ttfamily obj.\-Update\-Cluster\-Centers (vtk\-Table new\-Cluster\-Elements, vtk\-Table cur\-Cluster\-Elements, vtk\-Id\-Type\-Array num\-Membership\-Changes, vtk\-Id\-Type\-Array num\-Elements\-In\-Cluster, vtk\-Double\-Array error, vtk\-Id\-Type\-Array start\-Run\-I\-D, vtk\-Id\-Type\-Array end\-Run\-I\-D, vtk\-Int\-Array compute\-Run)} -\/ Subroutine to update new cluster centers from the old centers.  
\item {\ttfamily vtk\-Id\-Type = obj.\-Get\-Total\-Number\-Of\-Observations (vtk\-Id\-Type num\-Observations)} -\/ Subroutine to get the total number of data objects.  
\item {\ttfamily obj.\-Create\-Initial\-Cluster\-Centers (vtk\-Id\-Type num\-To\-Allocate, vtk\-Id\-Type\-Array number\-Of\-Clusters, vtk\-Table in\-Data, vtk\-Table cur\-Cluster\-Elements, vtk\-Table new\-Cluster\-Elements)} -\/ Subroutine to initialize cluster centerss if not provided by the user.  
\end{DoxyItemize}\hypertarget{vtkinfovis_vtkpmulticorrelativestatistics}{}\section{vtk\-P\-Multi\-Correlative\-Statistics}\label{vtkinfovis_vtkpmulticorrelativestatistics}
Section\-: \hyperlink{sec_vtkinfovis}{Visualization Toolkit Infovis Classes} \hypertarget{vtkwidgets_vtkxyplotwidget_Usage}{}\subsection{Usage}\label{vtkwidgets_vtkxyplotwidget_Usage}
vtk\-P\-Multi\-Correlative\-Statistics is vtk\-Multi\-Correlative\-Statistics subclass for parallel datasets. It learns and derives the global statistical model on each node, but assesses each individual data points on the node that owns it.

To create an instance of class vtk\-P\-Multi\-Correlative\-Statistics, simply invoke its constructor as follows \begin{DoxyVerb}  obj = vtkPMultiCorrelativeStatistics
\end{DoxyVerb}
 \hypertarget{vtkwidgets_vtkxyplotwidget_Methods}{}\subsection{Methods}\label{vtkwidgets_vtkxyplotwidget_Methods}
The class vtk\-P\-Multi\-Correlative\-Statistics has several methods that can be used. They are listed below. Note that the documentation is translated automatically from the V\-T\-K sources, and may not be completely intelligible. When in doubt, consult the V\-T\-K website. In the methods listed below, {\ttfamily obj} is an instance of the vtk\-P\-Multi\-Correlative\-Statistics class. 
\begin{DoxyItemize}
\item {\ttfamily string = obj.\-Get\-Class\-Name ()}  
\item {\ttfamily int = obj.\-Is\-A (string name)}  
\item {\ttfamily vtk\-P\-Multi\-Correlative\-Statistics = obj.\-New\-Instance ()}  
\item {\ttfamily vtk\-P\-Multi\-Correlative\-Statistics = obj.\-Safe\-Down\-Cast (vtk\-Object o)}  
\item {\ttfamily obj.\-Set\-Controller (vtk\-Multi\-Process\-Controller )} -\/ Get/\-Set the multiprocess controller. If no controller is set, single process is assumed.  
\item {\ttfamily vtk\-Multi\-Process\-Controller = obj.\-Get\-Controller ()} -\/ Get/\-Set the multiprocess controller. If no controller is set, single process is assumed.  
\end{DoxyItemize}\hypertarget{vtkinfovis_vtkppairwiseextracthistogram2d}{}\section{vtk\-P\-Pairwise\-Extract\-Histogram2\-D}\label{vtkinfovis_vtkppairwiseextracthistogram2d}
Section\-: \hyperlink{sec_vtkinfovis}{Visualization Toolkit Infovis Classes} \hypertarget{vtkwidgets_vtkxyplotwidget_Usage}{}\subsection{Usage}\label{vtkwidgets_vtkxyplotwidget_Usage}
This class does exactly the same this as vtk\-Pairwise\-Extract\-Histogram2\-D, but does it in a multi-\/process environment. After each node computes their own local histograms, this class does an All\-Reduce that distributes the sum of all local histograms onto each node.

Because vtk\-Pairwise\-Extract\-Histogram2\-D is a light wrapper around a series of vtk\-Extract\-Histogram2\-D classes, this class just overrides the function that instantiates new histogram filters and returns the parallel version (vtk\-P\-Extract\-Histogram2\-D).

To create an instance of class vtk\-P\-Pairwise\-Extract\-Histogram2\-D, simply invoke its constructor as follows \begin{DoxyVerb}  obj = vtkPPairwiseExtractHistogram2D
\end{DoxyVerb}
 \hypertarget{vtkwidgets_vtkxyplotwidget_Methods}{}\subsection{Methods}\label{vtkwidgets_vtkxyplotwidget_Methods}
The class vtk\-P\-Pairwise\-Extract\-Histogram2\-D has several methods that can be used. They are listed below. Note that the documentation is translated automatically from the V\-T\-K sources, and may not be completely intelligible. When in doubt, consult the V\-T\-K website. In the methods listed below, {\ttfamily obj} is an instance of the vtk\-P\-Pairwise\-Extract\-Histogram2\-D class. 
\begin{DoxyItemize}
\item {\ttfamily string = obj.\-Get\-Class\-Name ()}  
\item {\ttfamily int = obj.\-Is\-A (string name)}  
\item {\ttfamily vtk\-P\-Pairwise\-Extract\-Histogram2\-D = obj.\-New\-Instance ()}  
\item {\ttfamily vtk\-P\-Pairwise\-Extract\-Histogram2\-D = obj.\-Safe\-Down\-Cast (vtk\-Object o)}  
\item {\ttfamily obj.\-Set\-Controller (vtk\-Multi\-Process\-Controller )}  
\item {\ttfamily vtk\-Multi\-Process\-Controller = obj.\-Get\-Controller ()}  
\end{DoxyItemize}\hypertarget{vtkinfovis_vtkppcastatistics}{}\section{vtk\-P\-P\-C\-A\-Statistics}\label{vtkinfovis_vtkppcastatistics}
Section\-: \hyperlink{sec_vtkinfovis}{Visualization Toolkit Infovis Classes} \hypertarget{vtkwidgets_vtkxyplotwidget_Usage}{}\subsection{Usage}\label{vtkwidgets_vtkxyplotwidget_Usage}
vtk\-P\-P\-C\-A\-Statistics is vtk\-P\-C\-A\-Statistics subclass for parallel datasets. It learns and derives the global statistical model on each node, but assesses each individual data points on the node that owns it.

To create an instance of class vtk\-P\-P\-C\-A\-Statistics, simply invoke its constructor as follows \begin{DoxyVerb}  obj = vtkPPCAStatistics
\end{DoxyVerb}
 \hypertarget{vtkwidgets_vtkxyplotwidget_Methods}{}\subsection{Methods}\label{vtkwidgets_vtkxyplotwidget_Methods}
The class vtk\-P\-P\-C\-A\-Statistics has several methods that can be used. They are listed below. Note that the documentation is translated automatically from the V\-T\-K sources, and may not be completely intelligible. When in doubt, consult the V\-T\-K website. In the methods listed below, {\ttfamily obj} is an instance of the vtk\-P\-P\-C\-A\-Statistics class. 
\begin{DoxyItemize}
\item {\ttfamily string = obj.\-Get\-Class\-Name ()}  
\item {\ttfamily int = obj.\-Is\-A (string name)}  
\item {\ttfamily vtk\-P\-P\-C\-A\-Statistics = obj.\-New\-Instance ()}  
\item {\ttfamily vtk\-P\-P\-C\-A\-Statistics = obj.\-Safe\-Down\-Cast (vtk\-Object o)}  
\item {\ttfamily obj.\-Set\-Controller (vtk\-Multi\-Process\-Controller )} -\/ Get/\-Set the multiprocess controller. If no controller is set, single process is assumed.  
\item {\ttfamily vtk\-Multi\-Process\-Controller = obj.\-Get\-Controller ()} -\/ Get/\-Set the multiprocess controller. If no controller is set, single process is assumed.  
\end{DoxyItemize}\hypertarget{vtkinfovis_vtkprunetreefilter}{}\section{vtk\-Prune\-Tree\-Filter}\label{vtkinfovis_vtkprunetreefilter}
Section\-: \hyperlink{sec_vtkinfovis}{Visualization Toolkit Infovis Classes} \hypertarget{vtkwidgets_vtkxyplotwidget_Usage}{}\subsection{Usage}\label{vtkwidgets_vtkxyplotwidget_Usage}
Removes a subtree rooted at a particular vertex in a vtk\-Tree.

To create an instance of class vtk\-Prune\-Tree\-Filter, simply invoke its constructor as follows \begin{DoxyVerb}  obj = vtkPruneTreeFilter
\end{DoxyVerb}
 \hypertarget{vtkwidgets_vtkxyplotwidget_Methods}{}\subsection{Methods}\label{vtkwidgets_vtkxyplotwidget_Methods}
The class vtk\-Prune\-Tree\-Filter has several methods that can be used. They are listed below. Note that the documentation is translated automatically from the V\-T\-K sources, and may not be completely intelligible. When in doubt, consult the V\-T\-K website. In the methods listed below, {\ttfamily obj} is an instance of the vtk\-Prune\-Tree\-Filter class. 
\begin{DoxyItemize}
\item {\ttfamily string = obj.\-Get\-Class\-Name ()}  
\item {\ttfamily int = obj.\-Is\-A (string name)}  
\item {\ttfamily vtk\-Prune\-Tree\-Filter = obj.\-New\-Instance ()}  
\item {\ttfamily vtk\-Prune\-Tree\-Filter = obj.\-Safe\-Down\-Cast (vtk\-Object o)}  
\item {\ttfamily vtk\-Id\-Type = obj.\-Get\-Parent\-Vertex ()} -\/ Set the parent vertex of the subtree to remove.  
\item {\ttfamily obj.\-Set\-Parent\-Vertex (vtk\-Id\-Type )} -\/ Set the parent vertex of the subtree to remove.  
\end{DoxyItemize}\hypertarget{vtkinfovis_vtkrandomgraphsource}{}\section{vtk\-Random\-Graph\-Source}\label{vtkinfovis_vtkrandomgraphsource}
Section\-: \hyperlink{sec_vtkinfovis}{Visualization Toolkit Infovis Classes} \hypertarget{vtkwidgets_vtkxyplotwidget_Usage}{}\subsection{Usage}\label{vtkwidgets_vtkxyplotwidget_Usage}
Generates a graph with a specified number of vertices, with the density of edges specified by either an exact number of edges or the probability of an edge. You may additionally specify whether to begin with a random tree (which enforces graph connectivity).

To create an instance of class vtk\-Random\-Graph\-Source, simply invoke its constructor as follows \begin{DoxyVerb}  obj = vtkRandomGraphSource
\end{DoxyVerb}
 \hypertarget{vtkwidgets_vtkxyplotwidget_Methods}{}\subsection{Methods}\label{vtkwidgets_vtkxyplotwidget_Methods}
The class vtk\-Random\-Graph\-Source has several methods that can be used. They are listed below. Note that the documentation is translated automatically from the V\-T\-K sources, and may not be completely intelligible. When in doubt, consult the V\-T\-K website. In the methods listed below, {\ttfamily obj} is an instance of the vtk\-Random\-Graph\-Source class. 
\begin{DoxyItemize}
\item {\ttfamily string = obj.\-Get\-Class\-Name ()}  
\item {\ttfamily int = obj.\-Is\-A (string name)}  
\item {\ttfamily vtk\-Random\-Graph\-Source = obj.\-New\-Instance ()}  
\item {\ttfamily vtk\-Random\-Graph\-Source = obj.\-Safe\-Down\-Cast (vtk\-Object o)}  
\item {\ttfamily int = obj.\-Get\-Number\-Of\-Vertices ()} -\/ The number of vertices in the graph.  
\item {\ttfamily obj.\-Set\-Number\-Of\-Vertices (int )} -\/ The number of vertices in the graph.  
\item {\ttfamily int = obj.\-Get\-Number\-Of\-Vertices\-Min\-Value ()} -\/ The number of vertices in the graph.  
\item {\ttfamily int = obj.\-Get\-Number\-Of\-Vertices\-Max\-Value ()} -\/ The number of vertices in the graph.  
\item {\ttfamily int = obj.\-Get\-Number\-Of\-Edges ()} -\/ If Use\-Edge\-Probability is off, creates a graph with the specified number of edges. Duplicate (parallel) edges are allowed.  
\item {\ttfamily obj.\-Set\-Number\-Of\-Edges (int )} -\/ If Use\-Edge\-Probability is off, creates a graph with the specified number of edges. Duplicate (parallel) edges are allowed.  
\item {\ttfamily int = obj.\-Get\-Number\-Of\-Edges\-Min\-Value ()} -\/ If Use\-Edge\-Probability is off, creates a graph with the specified number of edges. Duplicate (parallel) edges are allowed.  
\item {\ttfamily int = obj.\-Get\-Number\-Of\-Edges\-Max\-Value ()} -\/ If Use\-Edge\-Probability is off, creates a graph with the specified number of edges. Duplicate (parallel) edges are allowed.  
\item {\ttfamily double = obj.\-Get\-Edge\-Probability ()} -\/ If Use\-Edge\-Probability is on, adds an edge with this probability between 0 and 1 for each pair of vertices in the graph.  
\item {\ttfamily obj.\-Set\-Edge\-Probability (double )} -\/ If Use\-Edge\-Probability is on, adds an edge with this probability between 0 and 1 for each pair of vertices in the graph.  
\item {\ttfamily double = obj.\-Get\-Edge\-Probability\-Min\-Value ()} -\/ If Use\-Edge\-Probability is on, adds an edge with this probability between 0 and 1 for each pair of vertices in the graph.  
\item {\ttfamily double = obj.\-Get\-Edge\-Probability\-Max\-Value ()} -\/ If Use\-Edge\-Probability is on, adds an edge with this probability between 0 and 1 for each pair of vertices in the graph.  
\item {\ttfamily obj.\-Set\-Include\-Edge\-Weights (bool )} -\/ When set, includes edge weights in an array named \char`\"{}edge\-\_\-weights\char`\"{}. Defaults to off. Weights are random between 0 and 1.  
\item {\ttfamily bool = obj.\-Get\-Include\-Edge\-Weights ()} -\/ When set, includes edge weights in an array named \char`\"{}edge\-\_\-weights\char`\"{}. Defaults to off. Weights are random between 0 and 1.  
\item {\ttfamily obj.\-Include\-Edge\-Weights\-On ()} -\/ When set, includes edge weights in an array named \char`\"{}edge\-\_\-weights\char`\"{}. Defaults to off. Weights are random between 0 and 1.  
\item {\ttfamily obj.\-Include\-Edge\-Weights\-Off ()} -\/ When set, includes edge weights in an array named \char`\"{}edge\-\_\-weights\char`\"{}. Defaults to off. Weights are random between 0 and 1.  
\item {\ttfamily obj.\-Set\-Edge\-Weight\-Array\-Name (string )} -\/ The name of the edge weight array. Default \char`\"{}edge weight\char`\"{}.  
\item {\ttfamily string = obj.\-Get\-Edge\-Weight\-Array\-Name ()} -\/ The name of the edge weight array. Default \char`\"{}edge weight\char`\"{}.  
\item {\ttfamily obj.\-Set\-Directed (bool )} -\/ When set, creates a directed graph, as opposed to an undirected graph.  
\item {\ttfamily bool = obj.\-Get\-Directed ()} -\/ When set, creates a directed graph, as opposed to an undirected graph.  
\item {\ttfamily obj.\-Directed\-On ()} -\/ When set, creates a directed graph, as opposed to an undirected graph.  
\item {\ttfamily obj.\-Directed\-Off ()} -\/ When set, creates a directed graph, as opposed to an undirected graph.  
\item {\ttfamily obj.\-Set\-Use\-Edge\-Probability (bool )} -\/ When set, uses the Edge\-Probability parameter to determine the density of edges. Otherwise, Number\-Of\-Edges is used.  
\item {\ttfamily bool = obj.\-Get\-Use\-Edge\-Probability ()} -\/ When set, uses the Edge\-Probability parameter to determine the density of edges. Otherwise, Number\-Of\-Edges is used.  
\item {\ttfamily obj.\-Use\-Edge\-Probability\-On ()} -\/ When set, uses the Edge\-Probability parameter to determine the density of edges. Otherwise, Number\-Of\-Edges is used.  
\item {\ttfamily obj.\-Use\-Edge\-Probability\-Off ()} -\/ When set, uses the Edge\-Probability parameter to determine the density of edges. Otherwise, Number\-Of\-Edges is used.  
\item {\ttfamily obj.\-Set\-Start\-With\-Tree (bool )} -\/ When set, builds a random tree structure first, then adds additional random edges.  
\item {\ttfamily bool = obj.\-Get\-Start\-With\-Tree ()} -\/ When set, builds a random tree structure first, then adds additional random edges.  
\item {\ttfamily obj.\-Start\-With\-Tree\-On ()} -\/ When set, builds a random tree structure first, then adds additional random edges.  
\item {\ttfamily obj.\-Start\-With\-Tree\-Off ()} -\/ When set, builds a random tree structure first, then adds additional random edges.  
\item {\ttfamily obj.\-Set\-Allow\-Self\-Loops (bool )} -\/ If this flag is set to true, edges where the source and target vertex are the same can be generated. The default is to forbid such loops.  
\item {\ttfamily bool = obj.\-Get\-Allow\-Self\-Loops ()} -\/ If this flag is set to true, edges where the source and target vertex are the same can be generated. The default is to forbid such loops.  
\item {\ttfamily obj.\-Allow\-Self\-Loops\-On ()} -\/ If this flag is set to true, edges where the source and target vertex are the same can be generated. The default is to forbid such loops.  
\item {\ttfamily obj.\-Allow\-Self\-Loops\-Off ()} -\/ If this flag is set to true, edges where the source and target vertex are the same can be generated. The default is to forbid such loops.  
\item {\ttfamily obj.\-Set\-Allow\-Parallel\-Edges (bool )} -\/ When set, multiple edges from a source to a target vertex are allowed. The default is to forbid such loops.  
\item {\ttfamily bool = obj.\-Get\-Allow\-Parallel\-Edges ()} -\/ When set, multiple edges from a source to a target vertex are allowed. The default is to forbid such loops.  
\item {\ttfamily obj.\-Allow\-Parallel\-Edges\-On ()} -\/ When set, multiple edges from a source to a target vertex are allowed. The default is to forbid such loops.  
\item {\ttfamily obj.\-Allow\-Parallel\-Edges\-Off ()} -\/ When set, multiple edges from a source to a target vertex are allowed. The default is to forbid such loops.  
\item {\ttfamily obj.\-Set\-Generate\-Pedigree\-Ids (bool )} -\/ Add pedigree ids to vertex and edge data.  
\item {\ttfamily bool = obj.\-Get\-Generate\-Pedigree\-Ids ()} -\/ Add pedigree ids to vertex and edge data.  
\item {\ttfamily obj.\-Generate\-Pedigree\-Ids\-On ()} -\/ Add pedigree ids to vertex and edge data.  
\item {\ttfamily obj.\-Generate\-Pedigree\-Ids\-Off ()} -\/ Add pedigree ids to vertex and edge data.  
\item {\ttfamily obj.\-Set\-Vertex\-Pedigree\-Id\-Array\-Name (string )} -\/ The name of the vertex pedigree id array. Default \char`\"{}vertex id\char`\"{}.  
\item {\ttfamily string = obj.\-Get\-Vertex\-Pedigree\-Id\-Array\-Name ()} -\/ The name of the vertex pedigree id array. Default \char`\"{}vertex id\char`\"{}.  
\item {\ttfamily obj.\-Set\-Edge\-Pedigree\-Id\-Array\-Name (string )} -\/ The name of the edge pedigree id array. Default \char`\"{}edge id\char`\"{}.  
\item {\ttfamily string = obj.\-Get\-Edge\-Pedigree\-Id\-Array\-Name ()} -\/ The name of the edge pedigree id array. Default \char`\"{}edge id\char`\"{}.  
\item {\ttfamily obj.\-Set\-Seed (int )} -\/ Control the seed used for pseudo-\/random-\/number generation. This ensures that vtk\-Random\-Graph\-Source can produce repeatable results.  
\item {\ttfamily int = obj.\-Get\-Seed ()} -\/ Control the seed used for pseudo-\/random-\/number generation. This ensures that vtk\-Random\-Graph\-Source can produce repeatable results.  
\end{DoxyItemize}\hypertarget{vtkinfovis_vtkrandomlayoutstrategy}{}\section{vtk\-Random\-Layout\-Strategy}\label{vtkinfovis_vtkrandomlayoutstrategy}
Section\-: \hyperlink{sec_vtkinfovis}{Visualization Toolkit Infovis Classes} \hypertarget{vtkwidgets_vtkxyplotwidget_Usage}{}\subsection{Usage}\label{vtkwidgets_vtkxyplotwidget_Usage}
Assigns points to the vertices of a graph randomly within a bounded range.

.S\-E\-C\-I\-O\-N Thanks Thanks to Brian Wylie from Sandia National Laboratories for adding incremental layout capabilities.

To create an instance of class vtk\-Random\-Layout\-Strategy, simply invoke its constructor as follows \begin{DoxyVerb}  obj = vtkRandomLayoutStrategy
\end{DoxyVerb}
 \hypertarget{vtkwidgets_vtkxyplotwidget_Methods}{}\subsection{Methods}\label{vtkwidgets_vtkxyplotwidget_Methods}
The class vtk\-Random\-Layout\-Strategy has several methods that can be used. They are listed below. Note that the documentation is translated automatically from the V\-T\-K sources, and may not be completely intelligible. When in doubt, consult the V\-T\-K website. In the methods listed below, {\ttfamily obj} is an instance of the vtk\-Random\-Layout\-Strategy class. 
\begin{DoxyItemize}
\item {\ttfamily string = obj.\-Get\-Class\-Name ()}  
\item {\ttfamily int = obj.\-Is\-A (string name)}  
\item {\ttfamily vtk\-Random\-Layout\-Strategy = obj.\-New\-Instance ()}  
\item {\ttfamily vtk\-Random\-Layout\-Strategy = obj.\-Safe\-Down\-Cast (vtk\-Object o)}  
\item {\ttfamily obj.\-Set\-Random\-Seed (int )} -\/ Seed the random number generator used to compute point positions. This has a significant effect on their final positions when the layout is complete.  
\item {\ttfamily int = obj.\-Get\-Random\-Seed\-Min\-Value ()} -\/ Seed the random number generator used to compute point positions. This has a significant effect on their final positions when the layout is complete.  
\item {\ttfamily int = obj.\-Get\-Random\-Seed\-Max\-Value ()} -\/ Seed the random number generator used to compute point positions. This has a significant effect on their final positions when the layout is complete.  
\item {\ttfamily int = obj.\-Get\-Random\-Seed ()} -\/ Seed the random number generator used to compute point positions. This has a significant effect on their final positions when the layout is complete.  
\item {\ttfamily obj.\-Set\-Graph\-Bounds (double , double , double , double , double , double )} -\/ Set / get the region in space in which to place the final graph. The Graph\-Bounds only affects the results if Automatic\-Bounds\-Computation is off.  
\item {\ttfamily obj.\-Set\-Graph\-Bounds (double a\mbox{[}6\mbox{]})} -\/ Set / get the region in space in which to place the final graph. The Graph\-Bounds only affects the results if Automatic\-Bounds\-Computation is off.  
\item {\ttfamily double = obj. Get\-Graph\-Bounds ()} -\/ Set / get the region in space in which to place the final graph. The Graph\-Bounds only affects the results if Automatic\-Bounds\-Computation is off.  
\item {\ttfamily obj.\-Set\-Automatic\-Bounds\-Computation (int )} -\/ Turn on/off automatic graph bounds calculation. If this boolean is off, then the manually specified Graph\-Bounds is used. If on, then the input's bounds us used as the graph bounds.  
\item {\ttfamily int = obj.\-Get\-Automatic\-Bounds\-Computation ()} -\/ Turn on/off automatic graph bounds calculation. If this boolean is off, then the manually specified Graph\-Bounds is used. If on, then the input's bounds us used as the graph bounds.  
\item {\ttfamily obj.\-Automatic\-Bounds\-Computation\-On ()} -\/ Turn on/off automatic graph bounds calculation. If this boolean is off, then the manually specified Graph\-Bounds is used. If on, then the input's bounds us used as the graph bounds.  
\item {\ttfamily obj.\-Automatic\-Bounds\-Computation\-Off ()} -\/ Turn on/off automatic graph bounds calculation. If this boolean is off, then the manually specified Graph\-Bounds is used. If on, then the input's bounds us used as the graph bounds.  
\item {\ttfamily obj.\-Set\-Three\-Dimensional\-Layout (int )} -\/ Turn on/off layout of graph in three dimensions. If off, graph layout occurs in two dimensions. By default, three dimensional layout is on.  
\item {\ttfamily int = obj.\-Get\-Three\-Dimensional\-Layout ()} -\/ Turn on/off layout of graph in three dimensions. If off, graph layout occurs in two dimensions. By default, three dimensional layout is on.  
\item {\ttfamily obj.\-Three\-Dimensional\-Layout\-On ()} -\/ Turn on/off layout of graph in three dimensions. If off, graph layout occurs in two dimensions. By default, three dimensional layout is on.  
\item {\ttfamily obj.\-Three\-Dimensional\-Layout\-Off ()} -\/ Turn on/off layout of graph in three dimensions. If off, graph layout occurs in two dimensions. By default, three dimensional layout is on.  
\item {\ttfamily obj.\-Set\-Graph (vtk\-Graph graph)} -\/ Set the graph to layout.  
\item {\ttfamily obj.\-Layout ()} -\/ Perform the random layout.  
\end{DoxyItemize}\hypertarget{vtkinfovis_vtkremovehiddendata}{}\section{vtk\-Remove\-Hidden\-Data}\label{vtkinfovis_vtkremovehiddendata}
Section\-: \hyperlink{sec_vtkinfovis}{Visualization Toolkit Infovis Classes} \hypertarget{vtkwidgets_vtkxyplotwidget_Usage}{}\subsection{Usage}\label{vtkwidgets_vtkxyplotwidget_Usage}
Output only those rows/vertices/edges of the input vtk\-Data\-Object that are visible, as defined by the vtk\-Annotation\-::\-H\-I\-D\-E() flag of the input vtk\-Annotation\-Layers. Inputs\-: Port 0 -\/ vtk\-Data\-Object Port 1 -\/ vtk\-Annotation\-Layers (optional)

To create an instance of class vtk\-Remove\-Hidden\-Data, simply invoke its constructor as follows \begin{DoxyVerb}  obj = vtkRemoveHiddenData
\end{DoxyVerb}
 \hypertarget{vtkwidgets_vtkxyplotwidget_Methods}{}\subsection{Methods}\label{vtkwidgets_vtkxyplotwidget_Methods}
The class vtk\-Remove\-Hidden\-Data has several methods that can be used. They are listed below. Note that the documentation is translated automatically from the V\-T\-K sources, and may not be completely intelligible. When in doubt, consult the V\-T\-K website. In the methods listed below, {\ttfamily obj} is an instance of the vtk\-Remove\-Hidden\-Data class. 
\begin{DoxyItemize}
\item {\ttfamily string = obj.\-Get\-Class\-Name ()}  
\item {\ttfamily int = obj.\-Is\-A (string name)}  
\item {\ttfamily vtk\-Remove\-Hidden\-Data = obj.\-New\-Instance ()}  
\item {\ttfamily vtk\-Remove\-Hidden\-Data = obj.\-Safe\-Down\-Cast (vtk\-Object o)}  
\end{DoxyItemize}\hypertarget{vtkinfovis_vtkremoveisolatedvertices}{}\section{vtk\-Remove\-Isolated\-Vertices}\label{vtkinfovis_vtkremoveisolatedvertices}
Section\-: \hyperlink{sec_vtkinfovis}{Visualization Toolkit Infovis Classes} \hypertarget{vtkwidgets_vtkxyplotwidget_Usage}{}\subsection{Usage}\label{vtkwidgets_vtkxyplotwidget_Usage}
To create an instance of class vtk\-Remove\-Isolated\-Vertices, simply invoke its constructor as follows \begin{DoxyVerb}  obj = vtkRemoveIsolatedVertices
\end{DoxyVerb}
 \hypertarget{vtkwidgets_vtkxyplotwidget_Methods}{}\subsection{Methods}\label{vtkwidgets_vtkxyplotwidget_Methods}
The class vtk\-Remove\-Isolated\-Vertices has several methods that can be used. They are listed below. Note that the documentation is translated automatically from the V\-T\-K sources, and may not be completely intelligible. When in doubt, consult the V\-T\-K website. In the methods listed below, {\ttfamily obj} is an instance of the vtk\-Remove\-Isolated\-Vertices class. 
\begin{DoxyItemize}
\item {\ttfamily string = obj.\-Get\-Class\-Name ()}  
\item {\ttfamily int = obj.\-Is\-A (string name)}  
\item {\ttfamily vtk\-Remove\-Isolated\-Vertices = obj.\-New\-Instance ()}  
\item {\ttfamily vtk\-Remove\-Isolated\-Vertices = obj.\-Safe\-Down\-Cast (vtk\-Object o)}  
\end{DoxyItemize}\hypertarget{vtkinfovis_vtkrisreader}{}\section{vtk\-R\-I\-S\-Reader}\label{vtkinfovis_vtkrisreader}
Section\-: \hyperlink{sec_vtkinfovis}{Visualization Toolkit Infovis Classes} \hypertarget{vtkwidgets_vtkxyplotwidget_Usage}{}\subsection{Usage}\label{vtkwidgets_vtkxyplotwidget_Usage}
R\-I\-S is a tagged format for expressing bibliographic citations. Data is structured as a collection of records with each record composed of one-\/to-\/many fields. See

\href{http://en.wikipedia.org/wiki/RIS_(file_format)}{\tt http\-://en.\-wikipedia.\-org/wiki/\-R\-I\-S\-\_\-(file\-\_\-format)} \href{http://www.refman.com/support/risformat_intro.asp}{\tt http\-://www.\-refman.\-com/support/risformat\-\_\-intro.\-asp} \href{http://www.adeptscience.co.uk/kb/article/A626}{\tt http\-://www.\-adeptscience.\-co.\-uk/kb/article/\-A626}

for details. vtk\-R\-I\-S\-Reader will convert an R\-I\-S file into a vtk\-Table, with the set of table columns determined dynamically from the contents of the file.

To create an instance of class vtk\-R\-I\-S\-Reader, simply invoke its constructor as follows \begin{DoxyVerb}  obj = vtkRISReader
\end{DoxyVerb}
 \hypertarget{vtkwidgets_vtkxyplotwidget_Methods}{}\subsection{Methods}\label{vtkwidgets_vtkxyplotwidget_Methods}
The class vtk\-R\-I\-S\-Reader has several methods that can be used. They are listed below. Note that the documentation is translated automatically from the V\-T\-K sources, and may not be completely intelligible. When in doubt, consult the V\-T\-K website. In the methods listed below, {\ttfamily obj} is an instance of the vtk\-R\-I\-S\-Reader class. 
\begin{DoxyItemize}
\item {\ttfamily string = obj.\-Get\-Class\-Name ()}  
\item {\ttfamily int = obj.\-Is\-A (string name)}  
\item {\ttfamily vtk\-R\-I\-S\-Reader = obj.\-New\-Instance ()}  
\item {\ttfamily vtk\-R\-I\-S\-Reader = obj.\-Safe\-Down\-Cast (vtk\-Object o)}  
\item {\ttfamily string = obj.\-Get\-File\-Name ()} -\/ Set/get the file to load  
\item {\ttfamily obj.\-Set\-File\-Name (string )} -\/ Set/get the file to load  
\item {\ttfamily string = obj.\-Get\-Delimiter ()} -\/ Set/get the delimiter to be used for concatenating field data (default\-: \char`\"{};\char`\"{})  
\item {\ttfamily obj.\-Set\-Delimiter (string )} -\/ Set/get the delimiter to be used for concatenating field data (default\-: \char`\"{};\char`\"{})  
\item {\ttfamily int = obj.\-Get\-Max\-Records ()} -\/ Set/get the maximum number of records to read from the file (zero = unlimited)  
\item {\ttfamily obj.\-Set\-Max\-Records (int )} -\/ Set/get the maximum number of records to read from the file (zero = unlimited)  
\end{DoxyItemize}\hypertarget{vtkinfovis_vtkscurvespline}{}\section{vtk\-S\-Curve\-Spline}\label{vtkinfovis_vtkscurvespline}
Section\-: \hyperlink{sec_vtkinfovis}{Visualization Toolkit Infovis Classes} \hypertarget{vtkwidgets_vtkxyplotwidget_Usage}{}\subsection{Usage}\label{vtkwidgets_vtkxyplotwidget_Usage}
vtk\-S\-Curve\-Spline is a concrete implementation of vtk\-Spline using a S\-Curve basis.

To create an instance of class vtk\-S\-Curve\-Spline, simply invoke its constructor as follows \begin{DoxyVerb}  obj = vtkSCurveSpline
\end{DoxyVerb}
 \hypertarget{vtkwidgets_vtkxyplotwidget_Methods}{}\subsection{Methods}\label{vtkwidgets_vtkxyplotwidget_Methods}
The class vtk\-S\-Curve\-Spline has several methods that can be used. They are listed below. Note that the documentation is translated automatically from the V\-T\-K sources, and may not be completely intelligible. When in doubt, consult the V\-T\-K website. In the methods listed below, {\ttfamily obj} is an instance of the vtk\-S\-Curve\-Spline class. 
\begin{DoxyItemize}
\item {\ttfamily string = obj.\-Get\-Class\-Name ()}  
\item {\ttfamily int = obj.\-Is\-A (string name)}  
\item {\ttfamily vtk\-S\-Curve\-Spline = obj.\-New\-Instance ()}  
\item {\ttfamily vtk\-S\-Curve\-Spline = obj.\-Safe\-Down\-Cast (vtk\-Object o)}  
\item {\ttfamily obj.\-Compute ()}  
\item {\ttfamily double = obj.\-Evaluate (double t)} -\/ Evaluate a 1\-D S\-Curve spline.  
\item {\ttfamily obj.\-Deep\-Copy (vtk\-Spline s)} -\/ Deep copy of S\-Curve spline data.  
\item {\ttfamily obj.\-Set\-Node\-Weight (double )}  
\item {\ttfamily double = obj.\-Get\-Node\-Weight ()}  
\end{DoxyItemize}\hypertarget{vtkinfovis_vtksimple2dlayoutstrategy}{}\section{vtk\-Simple2\-D\-Layout\-Strategy}\label{vtkinfovis_vtksimple2dlayoutstrategy}
Section\-: \hyperlink{sec_vtkinfovis}{Visualization Toolkit Infovis Classes} \hypertarget{vtkwidgets_vtkxyplotwidget_Usage}{}\subsection{Usage}\label{vtkwidgets_vtkxyplotwidget_Usage}
This class is an implementation of the work presented in\-: Fruchterman \& Reingold \char`\"{}\-Graph Drawing by Force-\/directed Placement\char`\"{} Software-\/\-Practice and Experience 21(11) 1991). The class includes some optimizations but nothing too fancy.

.S\-E\-C\-T\-I\-O\-N Thanks Thanks to Brian Wylie from Sandia National Laboratories for creating this class.

To create an instance of class vtk\-Simple2\-D\-Layout\-Strategy, simply invoke its constructor as follows \begin{DoxyVerb}  obj = vtkSimple2DLayoutStrategy
\end{DoxyVerb}
 \hypertarget{vtkwidgets_vtkxyplotwidget_Methods}{}\subsection{Methods}\label{vtkwidgets_vtkxyplotwidget_Methods}
The class vtk\-Simple2\-D\-Layout\-Strategy has several methods that can be used. They are listed below. Note that the documentation is translated automatically from the V\-T\-K sources, and may not be completely intelligible. When in doubt, consult the V\-T\-K website. In the methods listed below, {\ttfamily obj} is an instance of the vtk\-Simple2\-D\-Layout\-Strategy class. 
\begin{DoxyItemize}
\item {\ttfamily string = obj.\-Get\-Class\-Name ()}  
\item {\ttfamily int = obj.\-Is\-A (string name)}  
\item {\ttfamily vtk\-Simple2\-D\-Layout\-Strategy = obj.\-New\-Instance ()}  
\item {\ttfamily vtk\-Simple2\-D\-Layout\-Strategy = obj.\-Safe\-Down\-Cast (vtk\-Object o)}  
\item {\ttfamily obj.\-Set\-Random\-Seed (int )} -\/ Seed the random number generator used to jitter point positions. This has a significant effect on their final positions when the layout is complete.  
\item {\ttfamily int = obj.\-Get\-Random\-Seed\-Min\-Value ()} -\/ Seed the random number generator used to jitter point positions. This has a significant effect on their final positions when the layout is complete.  
\item {\ttfamily int = obj.\-Get\-Random\-Seed\-Max\-Value ()} -\/ Seed the random number generator used to jitter point positions. This has a significant effect on their final positions when the layout is complete.  
\item {\ttfamily int = obj.\-Get\-Random\-Seed ()} -\/ Seed the random number generator used to jitter point positions. This has a significant effect on their final positions when the layout is complete.  
\item {\ttfamily obj.\-Set\-Max\-Number\-Of\-Iterations (int )} -\/ Set/\-Get the maximum number of iterations to be used. The higher this number, the more iterations through the algorithm is possible, and thus, the more the graph gets modified. The default is '100' for no particular reason Note\-: The strong recommendation is that you do not change this parameter. \-:)  
\item {\ttfamily int = obj.\-Get\-Max\-Number\-Of\-Iterations\-Min\-Value ()} -\/ Set/\-Get the maximum number of iterations to be used. The higher this number, the more iterations through the algorithm is possible, and thus, the more the graph gets modified. The default is '100' for no particular reason Note\-: The strong recommendation is that you do not change this parameter. \-:)  
\item {\ttfamily int = obj.\-Get\-Max\-Number\-Of\-Iterations\-Max\-Value ()} -\/ Set/\-Get the maximum number of iterations to be used. The higher this number, the more iterations through the algorithm is possible, and thus, the more the graph gets modified. The default is '100' for no particular reason Note\-: The strong recommendation is that you do not change this parameter. \-:)  
\item {\ttfamily int = obj.\-Get\-Max\-Number\-Of\-Iterations ()} -\/ Set/\-Get the maximum number of iterations to be used. The higher this number, the more iterations through the algorithm is possible, and thus, the more the graph gets modified. The default is '100' for no particular reason Note\-: The strong recommendation is that you do not change this parameter. \-:)  
\item {\ttfamily obj.\-Set\-Iterations\-Per\-Layout (int )} -\/ Set/\-Get the number of iterations per layout. The only use for this ivar is for the application to do visualizations of the layout before it's complete. The default is '100' to match the default 'Max\-Number\-Of\-Iterations' Note\-: Changing this parameter is just fine \-:)  
\item {\ttfamily int = obj.\-Get\-Iterations\-Per\-Layout\-Min\-Value ()} -\/ Set/\-Get the number of iterations per layout. The only use for this ivar is for the application to do visualizations of the layout before it's complete. The default is '100' to match the default 'Max\-Number\-Of\-Iterations' Note\-: Changing this parameter is just fine \-:)  
\item {\ttfamily int = obj.\-Get\-Iterations\-Per\-Layout\-Max\-Value ()} -\/ Set/\-Get the number of iterations per layout. The only use for this ivar is for the application to do visualizations of the layout before it's complete. The default is '100' to match the default 'Max\-Number\-Of\-Iterations' Note\-: Changing this parameter is just fine \-:)  
\item {\ttfamily int = obj.\-Get\-Iterations\-Per\-Layout ()} -\/ Set/\-Get the number of iterations per layout. The only use for this ivar is for the application to do visualizations of the layout before it's complete. The default is '100' to match the default 'Max\-Number\-Of\-Iterations' Note\-: Changing this parameter is just fine \-:)  
\item {\ttfamily obj.\-Set\-Initial\-Temperature (float )} -\/ Set the initial temperature. The temperature default is '5' for no particular reason Note\-: The strong recommendation is that you do not change this parameter. \-:)  
\item {\ttfamily float = obj.\-Get\-Initial\-Temperature\-Min\-Value ()} -\/ Set the initial temperature. The temperature default is '5' for no particular reason Note\-: The strong recommendation is that you do not change this parameter. \-:)  
\item {\ttfamily float = obj.\-Get\-Initial\-Temperature\-Max\-Value ()} -\/ Set the initial temperature. The temperature default is '5' for no particular reason Note\-: The strong recommendation is that you do not change this parameter. \-:)  
\item {\ttfamily float = obj.\-Get\-Initial\-Temperature ()} -\/ Set the initial temperature. The temperature default is '5' for no particular reason Note\-: The strong recommendation is that you do not change this parameter. \-:)  
\item {\ttfamily obj.\-Set\-Cool\-Down\-Rate (double )} -\/ Set/\-Get the Cool-\/down rate. The higher this number is, the longer it will take to \char`\"{}cool-\/down\char`\"{}, and thus, the more the graph will be modified. The default is '10' for no particular reason. Note\-: The strong recommendation is that you do not change this parameter. \-:)  
\item {\ttfamily double = obj.\-Get\-Cool\-Down\-Rate\-Min\-Value ()} -\/ Set/\-Get the Cool-\/down rate. The higher this number is, the longer it will take to \char`\"{}cool-\/down\char`\"{}, and thus, the more the graph will be modified. The default is '10' for no particular reason. Note\-: The strong recommendation is that you do not change this parameter. \-:)  
\item {\ttfamily double = obj.\-Get\-Cool\-Down\-Rate\-Max\-Value ()} -\/ Set/\-Get the Cool-\/down rate. The higher this number is, the longer it will take to \char`\"{}cool-\/down\char`\"{}, and thus, the more the graph will be modified. The default is '10' for no particular reason. Note\-: The strong recommendation is that you do not change this parameter. \-:)  
\item {\ttfamily double = obj.\-Get\-Cool\-Down\-Rate ()} -\/ Set/\-Get the Cool-\/down rate. The higher this number is, the longer it will take to \char`\"{}cool-\/down\char`\"{}, and thus, the more the graph will be modified. The default is '10' for no particular reason. Note\-: The strong recommendation is that you do not change this parameter. \-:)  
\item {\ttfamily obj.\-Set\-Jitter (bool )} -\/ Set Random jitter of the nodes at initialization to on or off. Note\-: It's strongly recommendation to have jitter O\-N even if you have initial coordinates in your graph. Default is O\-N  
\item {\ttfamily bool = obj.\-Get\-Jitter ()} -\/ Set Random jitter of the nodes at initialization to on or off. Note\-: It's strongly recommendation to have jitter O\-N even if you have initial coordinates in your graph. Default is O\-N  
\item {\ttfamily obj.\-Set\-Rest\-Distance (float )} -\/ Manually set the resting distance. Otherwise the distance is computed automatically.  
\item {\ttfamily float = obj.\-Get\-Rest\-Distance ()} -\/ Manually set the resting distance. Otherwise the distance is computed automatically.  
\item {\ttfamily obj.\-Initialize ()} -\/ This strategy sets up some data structures for faster processing of each Layout() call  
\item {\ttfamily obj.\-Layout ()} -\/ This is the layout method where the graph that was set in Set\-Graph() is laid out. The method can either entirely layout the graph or iteratively lay out the graph. If you have an iterative layout please implement the Is\-Layout\-Complete() method.  
\item {\ttfamily int = obj.\-Is\-Layout\-Complete ()}  
\end{DoxyItemize}\hypertarget{vtkinfovis_vtksimple3dcirclesstrategy}{}\section{vtk\-Simple3\-D\-Circles\-Strategy}\label{vtkinfovis_vtksimple3dcirclesstrategy}
Section\-: \hyperlink{sec_vtkinfovis}{Visualization Toolkit Infovis Classes} \hypertarget{vtkwidgets_vtkxyplotwidget_Usage}{}\subsection{Usage}\label{vtkwidgets_vtkxyplotwidget_Usage}
Places vertices on circles depending on the graph vertices hierarchy level. The source graph could be vtk\-Directed\-Acyclic\-Graph or vtk\-Directed\-Graph if Marked\-Start\-Points array was added. The algorithm collects the standalone points, too and take them to a separated circle. If method is Fixed\-Radius\-Method, the radius of the circles will be equal. If method is Fixed\-Distance\-Method, the distance beetwen the points on circles will be equal.

In first step initial points are searched. A point is initial, if its in degree equal zero and out degree is greater than zero (or marked by Marked\-Start\-Vertices and out degree is greater than zero). Independent vertices (in and out degree equal zero) are collected separatelly. In second step the hierarchical level is generated for every vertex. In third step the hierarchical order is generated. If a vertex has no hierarchical level and it is not independent, the graph has loop so the algorithm exit with error message. Finally the vertices positions are calculated by the hierarchical order and by the vertices hierarchy levels.

.S\-E\-C\-T\-I\-O\-N Thanks Ferenc Nasztanovics, \href{mailto:naszta@naszta.hu}{\tt naszta@naszta.\-hu}, Budapest University of Technology and Economics, Department of Structural Mechanics

.S\-E\-C\-T\-I\-O\-N References in 3\-D rotation was used\-: \href{http://en.citizendium.org/wiki/Rotation_matrix}{\tt http\-://en.\-citizendium.\-org/wiki/\-Rotation\-\_\-matrix}

To create an instance of class vtk\-Simple3\-D\-Circles\-Strategy, simply invoke its constructor as follows \begin{DoxyVerb}  obj = vtkSimple3DCirclesStrategy
\end{DoxyVerb}
 \hypertarget{vtkwidgets_vtkxyplotwidget_Methods}{}\subsection{Methods}\label{vtkwidgets_vtkxyplotwidget_Methods}
The class vtk\-Simple3\-D\-Circles\-Strategy has several methods that can be used. They are listed below. Note that the documentation is translated automatically from the V\-T\-K sources, and may not be completely intelligible. When in doubt, consult the V\-T\-K website. In the methods listed below, {\ttfamily obj} is an instance of the vtk\-Simple3\-D\-Circles\-Strategy class. 
\begin{DoxyItemize}
\item {\ttfamily string = obj.\-Get\-Class\-Name ()}  
\item {\ttfamily int = obj.\-Is\-A (string name)}  
\item {\ttfamily vtk\-Simple3\-D\-Circles\-Strategy = obj.\-New\-Instance ()}  
\item {\ttfamily vtk\-Simple3\-D\-Circles\-Strategy = obj.\-Safe\-Down\-Cast (vtk\-Object o)}  
\item {\ttfamily obj.\-Set\-Method (int )} -\/ Set or get cicrle generating method (Fixed\-Radius\-Method/\-Fixed\-Distance\-Method). Default is Fixed\-Radius\-Method.  
\item {\ttfamily int = obj.\-Get\-Method ()} -\/ Set or get cicrle generating method (Fixed\-Radius\-Method/\-Fixed\-Distance\-Method). Default is Fixed\-Radius\-Method.  
\item {\ttfamily obj.\-Set\-Radius (double )} -\/ If Method is Fixed\-Radius\-Method\-: Set or get the radius of the circles. If Method is Fixed\-Distance\-Method\-: Set or get the distance of the points in the circle.  
\item {\ttfamily double = obj.\-Get\-Radius ()} -\/ If Method is Fixed\-Radius\-Method\-: Set or get the radius of the circles. If Method is Fixed\-Distance\-Method\-: Set or get the distance of the points in the circle.  
\item {\ttfamily obj.\-Set\-Height (double )} -\/ Set or get the vertical (local z) distance between the circles. If Auto\-Height is on, this is the minimal height between the circle layers  
\item {\ttfamily double = obj.\-Get\-Height ()} -\/ Set or get the vertical (local z) distance between the circles. If Auto\-Height is on, this is the minimal height between the circle layers  
\item {\ttfamily obj.\-Set\-Orign (double , double , double )} -\/ Set or get the orign of the geometry. This is the center of the first circle. Set\-Orign(x,y,z)  
\item {\ttfamily obj.\-Set\-Orign (double a\mbox{[}3\mbox{]})} -\/ Set or get the orign of the geometry. This is the center of the first circle. Set\-Orign(x,y,z)  
\item {\ttfamily double = obj. Get\-Orign ()} -\/ Set or get the orign of the geometry. This is the center of the first circle. Set\-Orign(x,y,z)  
\item {\ttfamily obj.\-Set\-Direction (double dx, double dy, double dz)} -\/ Set or get the normal vector of the circles plain. The height is growing in this direction. The direction must not be zero vector. The default vector is (0.\-0,0.\-0,1.\-0)  
\item {\ttfamily obj.\-Set\-Direction (double d\mbox{[}3\mbox{]})} -\/ Set or get the normal vector of the circles plain. The height is growing in this direction. The direction must not be zero vector. The default vector is (0.\-0,0.\-0,1.\-0)  
\item {\ttfamily double = obj. Get\-Direction ()} -\/ Set or get the normal vector of the circles plain. The height is growing in this direction. The direction must not be zero vector. The default vector is (0.\-0,0.\-0,1.\-0)  
\item {\ttfamily obj.\-Set\-Marked\-Start\-Vertices (vtk\-Int\-Array \-\_\-arg)} -\/ Set or get initial vertices. If Marked\-Start\-Vertices is added, loop is accepted in the graph. (If all of the loop start vertices are marked in Marked\-Start\-Vertices array.) Marked\-Start\-Vertices size must be equal with the number of the vertices in the graph. Start vertices must be marked by Marked\-Value. (E.\-g.\-: if Marked\-Value=3 and Marked\-Start\-Points is \{ 0, 3, 5, 3 \}, the start points ids will be \{1,3\}.) )  
\item {\ttfamily vtk\-Int\-Array = obj.\-Get\-Marked\-Start\-Vertices ()} -\/ Set or get initial vertices. If Marked\-Start\-Vertices is added, loop is accepted in the graph. (If all of the loop start vertices are marked in Marked\-Start\-Vertices array.) Marked\-Start\-Vertices size must be equal with the number of the vertices in the graph. Start vertices must be marked by Marked\-Value. (E.\-g.\-: if Marked\-Value=3 and Marked\-Start\-Points is \{ 0, 3, 5, 3 \}, the start points ids will be \{1,3\}.) )  
\item {\ttfamily obj.\-Set\-Marked\-Value (int )} -\/ Set or get Marked\-Value. See\-: Marked\-Start\-Vertices.  
\item {\ttfamily int = obj.\-Get\-Marked\-Value ()} -\/ Set or get Marked\-Value. See\-: Marked\-Start\-Vertices.  
\item {\ttfamily obj.\-Set\-Force\-To\-Use\-Universal\-Start\-Points\-Finder (int )} -\/ Set or get Force\-To\-Use\-Universal\-Start\-Points\-Finder. If Force\-To\-Use\-Universal\-Start\-Points\-Finder is true, Marked\-Start\-Vertices won't be used. In this case the input graph must be vtk\-Directed\-Acyclic\-Graph (Defualt\-: false).  
\item {\ttfamily int = obj.\-Get\-Force\-To\-Use\-Universal\-Start\-Points\-Finder ()} -\/ Set or get Force\-To\-Use\-Universal\-Start\-Points\-Finder. If Force\-To\-Use\-Universal\-Start\-Points\-Finder is true, Marked\-Start\-Vertices won't be used. In this case the input graph must be vtk\-Directed\-Acyclic\-Graph (Defualt\-: false).  
\item {\ttfamily obj.\-Force\-To\-Use\-Universal\-Start\-Points\-Finder\-On ()} -\/ Set or get Force\-To\-Use\-Universal\-Start\-Points\-Finder. If Force\-To\-Use\-Universal\-Start\-Points\-Finder is true, Marked\-Start\-Vertices won't be used. In this case the input graph must be vtk\-Directed\-Acyclic\-Graph (Defualt\-: false).  
\item {\ttfamily obj.\-Force\-To\-Use\-Universal\-Start\-Points\-Finder\-Off ()} -\/ Set or get Force\-To\-Use\-Universal\-Start\-Points\-Finder. If Force\-To\-Use\-Universal\-Start\-Points\-Finder is true, Marked\-Start\-Vertices won't be used. In this case the input graph must be vtk\-Directed\-Acyclic\-Graph (Defualt\-: false).  
\item {\ttfamily obj.\-Set\-Auto\-Height (int )} -\/ Set or get auto height (Default\-: false). If Auto\-Height is true, (r(i+1) -\/ r(i-\/1))/\-Height will be smaller than tan(\-Minimum\-Radian). If you want equal distances and parallel circles, you should turn off Auto\-Height.  
\item {\ttfamily int = obj.\-Get\-Auto\-Height ()} -\/ Set or get auto height (Default\-: false). If Auto\-Height is true, (r(i+1) -\/ r(i-\/1))/\-Height will be smaller than tan(\-Minimum\-Radian). If you want equal distances and parallel circles, you should turn off Auto\-Height.  
\item {\ttfamily obj.\-Auto\-Height\-On ()} -\/ Set or get auto height (Default\-: false). If Auto\-Height is true, (r(i+1) -\/ r(i-\/1))/\-Height will be smaller than tan(\-Minimum\-Radian). If you want equal distances and parallel circles, you should turn off Auto\-Height.  
\item {\ttfamily obj.\-Auto\-Height\-Off ()} -\/ Set or get auto height (Default\-: false). If Auto\-Height is true, (r(i+1) -\/ r(i-\/1))/\-Height will be smaller than tan(\-Minimum\-Radian). If you want equal distances and parallel circles, you should turn off Auto\-Height.  
\item {\ttfamily obj.\-Set\-Minimum\-Radian (double )} -\/ Set or get minimum radian (used by auto height).  
\item {\ttfamily double = obj.\-Get\-Minimum\-Radian ()} -\/ Set or get minimum radian (used by auto height).  
\item {\ttfamily obj.\-Set\-Minimum\-Degree (double degree)} -\/ Set or get minimum degree (used by auto height). There is no separated minimum degree, so minimum radian will be changed.  
\item {\ttfamily double = obj.\-Get\-Minimum\-Degree (void )} -\/ Set or get minimum degree (used by auto height). There is no separated minimum degree, so minimum radian will be changed.  
\item {\ttfamily obj.\-Set\-Hierarchical\-Layers (vtk\-Int\-Array \-\_\-arg)} -\/ Set or get hierarchical layers id by vertices (An usual vertex's layer id is greater or equal to zero. If a vertex is standalone, its layer id is -\/2.) If no Hierarchical\-Layers array is defined, vtk\-Simple3\-D\-Circles\-Strategy will generate it automatically (default).  
\item {\ttfamily vtk\-Int\-Array = obj.\-Get\-Hierarchical\-Layers ()} -\/ Set or get hierarchical layers id by vertices (An usual vertex's layer id is greater or equal to zero. If a vertex is standalone, its layer id is -\/2.) If no Hierarchical\-Layers array is defined, vtk\-Simple3\-D\-Circles\-Strategy will generate it automatically (default).  
\item {\ttfamily obj.\-Set\-Hierarchical\-Order (vtk\-Id\-Type\-Array \-\_\-arg)} -\/ Set or get hierarchical ordering of vertices (The array starts from the first vertex's id. All id must be greater or equal to zero!) If no Hierarchical\-Order is defined, vtk\-Simple3\-D\-Circles\-Strategy will generate it automatically (default).  
\item {\ttfamily vtk\-Id\-Type\-Array = obj.\-Get\-Hierarchical\-Order ()} -\/ Set or get hierarchical ordering of vertices (The array starts from the first vertex's id. All id must be greater or equal to zero!) If no Hierarchical\-Order is defined, vtk\-Simple3\-D\-Circles\-Strategy will generate it automatically (default).  
\item {\ttfamily obj.\-Layout (void )} -\/ Standard layout method  
\item {\ttfamily obj.\-Set\-Graph (vtk\-Graph graph)} -\/ Set graph (warning\-: Hierarchical\-Order and Hierarchical\-Layers will set to zero. These reference counts will be decreased!)  
\end{DoxyItemize}\hypertarget{vtkinfovis_vtksliceanddicelayoutstrategy}{}\section{vtk\-Slice\-And\-Dice\-Layout\-Strategy}\label{vtkinfovis_vtksliceanddicelayoutstrategy}
Section\-: \hyperlink{sec_vtkinfovis}{Visualization Toolkit Infovis Classes} \hypertarget{vtkwidgets_vtkxyplotwidget_Usage}{}\subsection{Usage}\label{vtkwidgets_vtkxyplotwidget_Usage}
Lays out a tree-\/map alternating between horizontal and vertical slices, taking into account the relative size of each vertex.

.S\-E\-C\-T\-I\-O\-N Thanks Slice and dice algorithm comes from\-: Shneiderman, B. 1992. Tree visualization with tree-\/maps\-: 2-\/d space-\/filling approach. A\-C\-M Trans. Graph. 11, 1 (Jan. 1992), 92-\/99.

To create an instance of class vtk\-Slice\-And\-Dice\-Layout\-Strategy, simply invoke its constructor as follows \begin{DoxyVerb}  obj = vtkSliceAndDiceLayoutStrategy
\end{DoxyVerb}
 \hypertarget{vtkwidgets_vtkxyplotwidget_Methods}{}\subsection{Methods}\label{vtkwidgets_vtkxyplotwidget_Methods}
The class vtk\-Slice\-And\-Dice\-Layout\-Strategy has several methods that can be used. They are listed below. Note that the documentation is translated automatically from the V\-T\-K sources, and may not be completely intelligible. When in doubt, consult the V\-T\-K website. In the methods listed below, {\ttfamily obj} is an instance of the vtk\-Slice\-And\-Dice\-Layout\-Strategy class. 
\begin{DoxyItemize}
\item {\ttfamily string = obj.\-Get\-Class\-Name ()}  
\item {\ttfamily int = obj.\-Is\-A (string name)}  
\item {\ttfamily vtk\-Slice\-And\-Dice\-Layout\-Strategy = obj.\-New\-Instance ()}  
\item {\ttfamily vtk\-Slice\-And\-Dice\-Layout\-Strategy = obj.\-Safe\-Down\-Cast (vtk\-Object o)}  
\item {\ttfamily obj.\-Layout (vtk\-Tree input\-Tree, vtk\-Data\-Array coords\-Array, vtk\-Data\-Array size\-Array)} -\/ Perform the layout of a tree and place the results as 4-\/tuples in coords\-Array (Xmin, Xmax, Ymin, Ymax).  
\end{DoxyItemize}\hypertarget{vtkinfovis_vtkspantreelayoutstrategy}{}\section{vtk\-Span\-Tree\-Layout\-Strategy}\label{vtkinfovis_vtkspantreelayoutstrategy}
Section\-: \hyperlink{sec_vtkinfovis}{Visualization Toolkit Infovis Classes} \hypertarget{vtkwidgets_vtkxyplotwidget_Usage}{}\subsection{Usage}\label{vtkwidgets_vtkxyplotwidget_Usage}
vtk\-Span\-Tree\-Layout is a strategy for drawing directed graphs that works by first extracting a spanning tree (more accurately, a spanning forest), and using this both to position graph vertices and to plan the placement of non-\/tree edges. The latter are drawn with the aid of edge points to produce a tidy drawing.

The approach is best suited to \char`\"{}quasi-\/trees\char`\"{}, graphs where the number of edges is of the same order as the number of nodes; it is less well suited to denser graphs. The boolean flag Depth\-First\-Spanning\-Tree determines whether a depth-\/first or breadth-\/first strategy is used to construct the underlying forest, and the choice of strategy affects the output layout significantly. Informal experiments suggest that the breadth-\/first strategy is better for denser graphs.

Different layouts could also be produced by plugging in alternative tree layout strategies. To work with the method of routing non-\/tree edges, any strategy should draw a tree so that levels are equally spaced along the z-\/axis, precluding for example the use of a radial or balloon layout.

vtk\-Span\-Tree\-Layout is based on an approach to 3\-D graph layout first developed as part of the \char`\"{}tulip\char`\"{} tool by Dr. David Auber at La\-B\-R\-I, U.\-Bordeaux\-: see www.\-tulip-\/software.\-org

This implementation departs from the original version in that\-: (a) it is reconstructed to use Titan/\-V\-T\-K data structures; (b) it uses a faster method for dealing with non-\/tree edges, requiring at most two edge points per edge (c) allows for plugging in different tree layout methods (d) allows selection of two different strategies for building the underlying layout tree, which can yield significantly different results depending on the data.

.S\-E\-C\-T\-I\-O\-N Thanks Thanks to David Duke from the University of Leeds for providing this implementation.

To create an instance of class vtk\-Span\-Tree\-Layout\-Strategy, simply invoke its constructor as follows \begin{DoxyVerb}  obj = vtkSpanTreeLayoutStrategy
\end{DoxyVerb}
 \hypertarget{vtkwidgets_vtkxyplotwidget_Methods}{}\subsection{Methods}\label{vtkwidgets_vtkxyplotwidget_Methods}
The class vtk\-Span\-Tree\-Layout\-Strategy has several methods that can be used. They are listed below. Note that the documentation is translated automatically from the V\-T\-K sources, and may not be completely intelligible. When in doubt, consult the V\-T\-K website. In the methods listed below, {\ttfamily obj} is an instance of the vtk\-Span\-Tree\-Layout\-Strategy class. 
\begin{DoxyItemize}
\item {\ttfamily string = obj.\-Get\-Class\-Name ()}  
\item {\ttfamily int = obj.\-Is\-A (string name)}  
\item {\ttfamily vtk\-Span\-Tree\-Layout\-Strategy = obj.\-New\-Instance ()}  
\item {\ttfamily vtk\-Span\-Tree\-Layout\-Strategy = obj.\-Safe\-Down\-Cast (vtk\-Object o)}  
\item {\ttfamily obj.\-Set\-Depth\-First\-Spanning\-Tree (bool )} -\/ If set, base the layout on a depth-\/first spanning tree, rather than the default breadth-\/first spanning tree. Switching between D\-F\-T and B\-F\-T may significantly change the layout, and choice must be made on a per-\/graph basis. Default value is off.  
\item {\ttfamily bool = obj.\-Get\-Depth\-First\-Spanning\-Tree ()} -\/ If set, base the layout on a depth-\/first spanning tree, rather than the default breadth-\/first spanning tree. Switching between D\-F\-T and B\-F\-T may significantly change the layout, and choice must be made on a per-\/graph basis. Default value is off.  
\item {\ttfamily obj.\-Depth\-First\-Spanning\-Tree\-On ()} -\/ If set, base the layout on a depth-\/first spanning tree, rather than the default breadth-\/first spanning tree. Switching between D\-F\-T and B\-F\-T may significantly change the layout, and choice must be made on a per-\/graph basis. Default value is off.  
\item {\ttfamily obj.\-Depth\-First\-Spanning\-Tree\-Off ()} -\/ If set, base the layout on a depth-\/first spanning tree, rather than the default breadth-\/first spanning tree. Switching between D\-F\-T and B\-F\-T may significantly change the layout, and choice must be made on a per-\/graph basis. Default value is off.  
\item {\ttfamily obj.\-Layout ()} -\/ Perform the layout.  
\end{DoxyItemize}\hypertarget{vtkinfovis_vtksparsearraytotable}{}\section{vtk\-Sparse\-Array\-To\-Table}\label{vtkinfovis_vtksparsearraytotable}
Section\-: \hyperlink{sec_vtkinfovis}{Visualization Toolkit Infovis Classes} \hypertarget{vtkwidgets_vtkxyplotwidget_Usage}{}\subsection{Usage}\label{vtkwidgets_vtkxyplotwidget_Usage}
Converts any sparse array to a vtk\-Table containing one row for each value stored in the array. The table will contain one column of coordinates for each dimension in the source array, plus one column of array values. A common use-\/case for vtk\-Sparse\-Array\-To\-Table would be converting a sparse array into a table suitable for use as an input to vtk\-Table\-To\-Graph.

The coordinate columns in the output table will be named using the dimension labels from the source array, The value column name can be explicitly set using Set\-Value\-Column().

.S\-E\-C\-T\-I\-O\-N Thanks Developed by Timothy M. Shead (\href{mailto:tshead@sandia.gov}{\tt tshead@sandia.\-gov}) at Sandia National Laboratories.

To create an instance of class vtk\-Sparse\-Array\-To\-Table, simply invoke its constructor as follows \begin{DoxyVerb}  obj = vtkSparseArrayToTable
\end{DoxyVerb}
 \hypertarget{vtkwidgets_vtkxyplotwidget_Methods}{}\subsection{Methods}\label{vtkwidgets_vtkxyplotwidget_Methods}
The class vtk\-Sparse\-Array\-To\-Table has several methods that can be used. They are listed below. Note that the documentation is translated automatically from the V\-T\-K sources, and may not be completely intelligible. When in doubt, consult the V\-T\-K website. In the methods listed below, {\ttfamily obj} is an instance of the vtk\-Sparse\-Array\-To\-Table class. 
\begin{DoxyItemize}
\item {\ttfamily string = obj.\-Get\-Class\-Name ()}  
\item {\ttfamily int = obj.\-Is\-A (string name)}  
\item {\ttfamily vtk\-Sparse\-Array\-To\-Table = obj.\-New\-Instance ()}  
\item {\ttfamily vtk\-Sparse\-Array\-To\-Table = obj.\-Safe\-Down\-Cast (vtk\-Object o)}  
\item {\ttfamily string = obj.\-Get\-Value\-Column ()} -\/ Specify the name of the output table column that contains array values. Default\-: \char`\"{}value\char`\"{}  
\item {\ttfamily obj.\-Set\-Value\-Column (string )} -\/ Specify the name of the output table column that contains array values. Default\-: \char`\"{}value\char`\"{}  
\end{DoxyItemize}\hypertarget{vtkinfovis_vtksplinegraphedges}{}\section{vtk\-Spline\-Graph\-Edges}\label{vtkinfovis_vtksplinegraphedges}
Section\-: \hyperlink{sec_vtkinfovis}{Visualization Toolkit Infovis Classes} \hypertarget{vtkwidgets_vtkxyplotwidget_Usage}{}\subsection{Usage}\label{vtkwidgets_vtkxyplotwidget_Usage}
vtk\-Spline\-Graph\-Edges uses a vtk\-Spline to make edges into nicely sampled splines. By default, the filter will use an optimized b-\/spline. Otherwise, it will use a custom vtk\-Spline instance set by the user.

To create an instance of class vtk\-Spline\-Graph\-Edges, simply invoke its constructor as follows \begin{DoxyVerb}  obj = vtkSplineGraphEdges
\end{DoxyVerb}
 \hypertarget{vtkwidgets_vtkxyplotwidget_Methods}{}\subsection{Methods}\label{vtkwidgets_vtkxyplotwidget_Methods}
The class vtk\-Spline\-Graph\-Edges has several methods that can be used. They are listed below. Note that the documentation is translated automatically from the V\-T\-K sources, and may not be completely intelligible. When in doubt, consult the V\-T\-K website. In the methods listed below, {\ttfamily obj} is an instance of the vtk\-Spline\-Graph\-Edges class. 
\begin{DoxyItemize}
\item {\ttfamily string = obj.\-Get\-Class\-Name ()}  
\item {\ttfamily int = obj.\-Is\-A (string name)}  
\item {\ttfamily vtk\-Spline\-Graph\-Edges = obj.\-New\-Instance ()}  
\item {\ttfamily vtk\-Spline\-Graph\-Edges = obj.\-Safe\-Down\-Cast (vtk\-Object o)}  
\item {\ttfamily obj.\-Set\-Spline (vtk\-Spline s)} -\/ If Spline\-Type is C\-U\-S\-T\-O\-M, uses this spline.  
\item {\ttfamily vtk\-Spline = obj.\-Get\-Spline ()} -\/ If Spline\-Type is C\-U\-S\-T\-O\-M, uses this spline.  
\item {\ttfamily obj.\-Set\-Spline\-Type (int )} -\/ Spline type used by the filter. B\-S\-P\-L\-I\-N\-E (0) -\/ Use optimized b-\/spline (default). C\-U\-S\-T\-O\-M (1) -\/ Use spline set with Set\-Spline.  
\item {\ttfamily int = obj.\-Get\-Spline\-Type ()} -\/ Spline type used by the filter. B\-S\-P\-L\-I\-N\-E (0) -\/ Use optimized b-\/spline (default). C\-U\-S\-T\-O\-M (1) -\/ Use spline set with Set\-Spline.  
\item {\ttfamily obj.\-Set\-Number\-Of\-Subdivisions (vtk\-Id\-Type )} -\/ The number of subdivisions in the spline.  
\item {\ttfamily vtk\-Id\-Type = obj.\-Get\-Number\-Of\-Subdivisions ()} -\/ The number of subdivisions in the spline.  
\end{DoxyItemize}\hypertarget{vtkinfovis_vtksplitcolumncomponents}{}\section{vtk\-Split\-Column\-Components}\label{vtkinfovis_vtksplitcolumncomponents}
Section\-: \hyperlink{sec_vtkinfovis}{Visualization Toolkit Infovis Classes} \hypertarget{vtkwidgets_vtkxyplotwidget_Usage}{}\subsection{Usage}\label{vtkwidgets_vtkxyplotwidget_Usage}
Splits any columns in a table that have more than one component into individual columns. Single component columns are passed through without any data duplication. So if column names \char`\"{}\-Points\char`\"{} had three components this column would be split into \char`\"{}\-Points (0)\char`\"{}, \char`\"{}\-Points (1)\char`\"{} and Points (2)".

To create an instance of class vtk\-Split\-Column\-Components, simply invoke its constructor as follows \begin{DoxyVerb}  obj = vtkSplitColumnComponents
\end{DoxyVerb}
 \hypertarget{vtkwidgets_vtkxyplotwidget_Methods}{}\subsection{Methods}\label{vtkwidgets_vtkxyplotwidget_Methods}
The class vtk\-Split\-Column\-Components has several methods that can be used. They are listed below. Note that the documentation is translated automatically from the V\-T\-K sources, and may not be completely intelligible. When in doubt, consult the V\-T\-K website. In the methods listed below, {\ttfamily obj} is an instance of the vtk\-Split\-Column\-Components class. 
\begin{DoxyItemize}
\item {\ttfamily string = obj.\-Get\-Class\-Name ()}  
\item {\ttfamily int = obj.\-Is\-A (string name)}  
\item {\ttfamily vtk\-Split\-Column\-Components = obj.\-New\-Instance ()}  
\item {\ttfamily vtk\-Split\-Column\-Components = obj.\-Safe\-Down\-Cast (vtk\-Object o)}  
\item {\ttfamily obj.\-Set\-Calculate\-Magnitudes (bool )} -\/ If on this filter will calculate an additional magnitude column for all columns it splits with two or more components. Default is on.  
\item {\ttfamily bool = obj.\-Get\-Calculate\-Magnitudes ()} -\/ If on this filter will calculate an additional magnitude column for all columns it splits with two or more components. Default is on.  
\end{DoxyItemize}\hypertarget{vtkinfovis_vtksqldatabasegraphsource}{}\section{vtk\-S\-Q\-L\-Database\-Graph\-Source}\label{vtkinfovis_vtksqldatabasegraphsource}
Section\-: \hyperlink{sec_vtkinfovis}{Visualization Toolkit Infovis Classes} \hypertarget{vtkwidgets_vtkxyplotwidget_Usage}{}\subsection{Usage}\label{vtkwidgets_vtkxyplotwidget_Usage}
This class combines vtk\-S\-Q\-L\-Database, vtk\-S\-Q\-L\-Query, and vtk\-Query\-To\-Graph to provide a convenience class for generating graphs from databases. Also this class can be easily wrapped and used within Para\-View / Over\-View.

To create an instance of class vtk\-S\-Q\-L\-Database\-Graph\-Source, simply invoke its constructor as follows \begin{DoxyVerb}  obj = vtkSQLDatabaseGraphSource
\end{DoxyVerb}
 \hypertarget{vtkwidgets_vtkxyplotwidget_Methods}{}\subsection{Methods}\label{vtkwidgets_vtkxyplotwidget_Methods}
The class vtk\-S\-Q\-L\-Database\-Graph\-Source has several methods that can be used. They are listed below. Note that the documentation is translated automatically from the V\-T\-K sources, and may not be completely intelligible. When in doubt, consult the V\-T\-K website. In the methods listed below, {\ttfamily obj} is an instance of the vtk\-S\-Q\-L\-Database\-Graph\-Source class. 
\begin{DoxyItemize}
\item {\ttfamily string = obj.\-Get\-Class\-Name ()}  
\item {\ttfamily int = obj.\-Is\-A (string name)}  
\item {\ttfamily vtk\-S\-Q\-L\-Database\-Graph\-Source = obj.\-New\-Instance ()}  
\item {\ttfamily vtk\-S\-Q\-L\-Database\-Graph\-Source = obj.\-Safe\-Down\-Cast (vtk\-Object o)}  
\item {\ttfamily vtk\-Std\-String = obj.\-Get\-U\-R\-L ()}  
\item {\ttfamily obj.\-Set\-U\-R\-L (vtk\-Std\-String \&url)}  
\item {\ttfamily obj.\-Set\-Password (vtk\-Std\-String \&password)}  
\item {\ttfamily vtk\-Std\-String = obj.\-Get\-Edge\-Query ()}  
\item {\ttfamily obj.\-Set\-Edge\-Query (vtk\-Std\-String \&query)}  
\item {\ttfamily vtk\-Std\-String = obj.\-Get\-Vertex\-Query ()}  
\item {\ttfamily obj.\-Set\-Vertex\-Query (vtk\-Std\-String \&query)}  
\item {\ttfamily obj.\-Add\-Link\-Vertex (string column, string domain, int hidden)}  
\item {\ttfamily obj.\-Clear\-Link\-Vertices ()}  
\item {\ttfamily obj.\-Add\-Link\-Edge (string column1, string column2)}  
\item {\ttfamily obj.\-Clear\-Link\-Edges ()}  
\item {\ttfamily bool = obj.\-Get\-Generate\-Edge\-Pedigree\-Ids ()} -\/ If on (default), generate edge pedigree ids. If off, assign an array to be edge pedigree ids.  
\item {\ttfamily obj.\-Set\-Generate\-Edge\-Pedigree\-Ids (bool )} -\/ If on (default), generate edge pedigree ids. If off, assign an array to be edge pedigree ids.  
\item {\ttfamily obj.\-Generate\-Edge\-Pedigree\-Ids\-On ()} -\/ If on (default), generate edge pedigree ids. If off, assign an array to be edge pedigree ids.  
\item {\ttfamily obj.\-Generate\-Edge\-Pedigree\-Ids\-Off ()} -\/ If on (default), generate edge pedigree ids. If off, assign an array to be edge pedigree ids.  
\item {\ttfamily obj.\-Set\-Edge\-Pedigree\-Id\-Array\-Name (string )} -\/ Use this array name for setting or generating edge pedigree ids.  
\item {\ttfamily string = obj.\-Get\-Edge\-Pedigree\-Id\-Array\-Name ()} -\/ Use this array name for setting or generating edge pedigree ids.  
\item {\ttfamily obj.\-Set\-Directed (bool )} -\/ If on (default), generate a directed output graph. If off, generate an undirected output graph.  
\item {\ttfamily bool = obj.\-Get\-Directed ()} -\/ If on (default), generate a directed output graph. If off, generate an undirected output graph.  
\item {\ttfamily obj.\-Directed\-On ()} -\/ If on (default), generate a directed output graph. If off, generate an undirected output graph.  
\item {\ttfamily obj.\-Directed\-Off ()} -\/ If on (default), generate a directed output graph. If off, generate an undirected output graph.  
\end{DoxyItemize}\hypertarget{vtkinfovis_vtksqldatabasetablesource}{}\section{vtk\-S\-Q\-L\-Database\-Table\-Source}\label{vtkinfovis_vtksqldatabasetablesource}
Section\-: \hyperlink{sec_vtkinfovis}{Visualization Toolkit Infovis Classes} \hypertarget{vtkwidgets_vtkxyplotwidget_Usage}{}\subsection{Usage}\label{vtkwidgets_vtkxyplotwidget_Usage}
This class combines vtk\-S\-Q\-L\-Database, vtk\-S\-Q\-L\-Query, and vtk\-Query\-To\-Table to provide a convenience class for generating tables from databases. Also this class can be easily wrapped and used within Para\-View / Over\-View.

To create an instance of class vtk\-S\-Q\-L\-Database\-Table\-Source, simply invoke its constructor as follows \begin{DoxyVerb}  obj = vtkSQLDatabaseTableSource
\end{DoxyVerb}
 \hypertarget{vtkwidgets_vtkxyplotwidget_Methods}{}\subsection{Methods}\label{vtkwidgets_vtkxyplotwidget_Methods}
The class vtk\-S\-Q\-L\-Database\-Table\-Source has several methods that can be used. They are listed below. Note that the documentation is translated automatically from the V\-T\-K sources, and may not be completely intelligible. When in doubt, consult the V\-T\-K website. In the methods listed below, {\ttfamily obj} is an instance of the vtk\-S\-Q\-L\-Database\-Table\-Source class. 
\begin{DoxyItemize}
\item {\ttfamily string = obj.\-Get\-Class\-Name ()}  
\item {\ttfamily int = obj.\-Is\-A (string name)}  
\item {\ttfamily vtk\-S\-Q\-L\-Database\-Table\-Source = obj.\-New\-Instance ()}  
\item {\ttfamily vtk\-S\-Q\-L\-Database\-Table\-Source = obj.\-Safe\-Down\-Cast (vtk\-Object o)}  
\item {\ttfamily vtk\-Std\-String = obj.\-Get\-U\-R\-L ()}  
\item {\ttfamily obj.\-Set\-U\-R\-L (vtk\-Std\-String \&url)}  
\item {\ttfamily obj.\-Set\-Password (vtk\-Std\-String \&password)}  
\item {\ttfamily vtk\-Std\-String = obj.\-Get\-Query ()}  
\item {\ttfamily obj.\-Set\-Query (vtk\-Std\-String \&query)}  
\item {\ttfamily obj.\-Set\-Pedigree\-Id\-Array\-Name (string )} -\/ The name of the array for generating or assigning pedigree ids (default \char`\"{}id\char`\"{}).  
\item {\ttfamily string = obj.\-Get\-Pedigree\-Id\-Array\-Name ()} -\/ The name of the array for generating or assigning pedigree ids (default \char`\"{}id\char`\"{}).  
\item {\ttfamily obj.\-Set\-Generate\-Pedigree\-Ids (bool )} -\/ If on (default), generates pedigree ids automatically. If off, assign one of the arrays to be the pedigree id.  
\item {\ttfamily bool = obj.\-Get\-Generate\-Pedigree\-Ids ()} -\/ If on (default), generates pedigree ids automatically. If off, assign one of the arrays to be the pedigree id.  
\item {\ttfamily obj.\-Generate\-Pedigree\-Ids\-On ()} -\/ If on (default), generates pedigree ids automatically. If off, assign one of the arrays to be the pedigree id.  
\item {\ttfamily obj.\-Generate\-Pedigree\-Ids\-Off ()} -\/ If on (default), generates pedigree ids automatically. If off, assign one of the arrays to be the pedigree id.  
\end{DoxyItemize}\hypertarget{vtkinfovis_vtksqlgraphreader}{}\section{vtk\-S\-Q\-L\-Graph\-Reader}\label{vtkinfovis_vtksqlgraphreader}
Section\-: \hyperlink{sec_vtkinfovis}{Visualization Toolkit Infovis Classes} \hypertarget{vtkwidgets_vtkxyplotwidget_Usage}{}\subsection{Usage}\label{vtkwidgets_vtkxyplotwidget_Usage}
Creates a vtk\-Graph using one or two vtk\-S\-Q\-L\-Query's. The first (required) query must have one row for each arc in the graph. The query must have two columns which represent the source and target node ids.

The second (optional) query has one row for each node in the graph. The table must have a field whose values match those in the arc table. If the node table is not given, a node will be created for each unique source or target identifier in the arc table.

The source, target, and node I\-D fields must be of the same type, and must be either vtk\-String\-Array or a subclass of vtk\-Data\-Array.

All columns in the queries, including the source, target, and node index fields, are copied into the arc data and node data of the resulting vtk\-Graph. If the node query is not given, the node data will contain a single \char`\"{}id\char`\"{} column with the same type as the source/target id arrays.

If parallel arcs are collected, not all the arc data is not copied into the output. Only the source and target id arrays will be transferred. An additional vtk\-Id\-Type\-Array column called \char`\"{}weight\char`\"{} is created which contains the number of times each arc appeared in the input.

If the node query contains positional data, the user may specify the names of these fields. These arrays must be data arrays. The z-\/coordinate array is optional, and if not given the z-\/coordinates are set to zero.

To create an instance of class vtk\-S\-Q\-L\-Graph\-Reader, simply invoke its constructor as follows \begin{DoxyVerb}  obj = vtkSQLGraphReader
\end{DoxyVerb}
 \hypertarget{vtkwidgets_vtkxyplotwidget_Methods}{}\subsection{Methods}\label{vtkwidgets_vtkxyplotwidget_Methods}
The class vtk\-S\-Q\-L\-Graph\-Reader has several methods that can be used. They are listed below. Note that the documentation is translated automatically from the V\-T\-K sources, and may not be completely intelligible. When in doubt, consult the V\-T\-K website. In the methods listed below, {\ttfamily obj} is an instance of the vtk\-S\-Q\-L\-Graph\-Reader class. 
\begin{DoxyItemize}
\item {\ttfamily string = obj.\-Get\-Class\-Name ()}  
\item {\ttfamily int = obj.\-Is\-A (string name)}  
\item {\ttfamily vtk\-S\-Q\-L\-Graph\-Reader = obj.\-New\-Instance ()}  
\item {\ttfamily vtk\-S\-Q\-L\-Graph\-Reader = obj.\-Safe\-Down\-Cast (vtk\-Object o)}  
\item {\ttfamily obj.\-Set\-Directed (bool )} -\/ When set, creates a directed graph, as opposed to an undirected graph.  
\item {\ttfamily bool = obj.\-Get\-Directed ()} -\/ When set, creates a directed graph, as opposed to an undirected graph.  
\item {\ttfamily obj.\-Directed\-On ()} -\/ When set, creates a directed graph, as opposed to an undirected graph.  
\item {\ttfamily obj.\-Directed\-Off ()} -\/ When set, creates a directed graph, as opposed to an undirected graph.  
\item {\ttfamily obj.\-Set\-Vertex\-Query (vtk\-S\-Q\-L\-Query q)} -\/ The query that retrieves the node information.  
\item {\ttfamily vtk\-S\-Q\-L\-Query = obj.\-Get\-Vertex\-Query ()} -\/ The query that retrieves the node information.  
\item {\ttfamily obj.\-Set\-Edge\-Query (vtk\-S\-Q\-L\-Query q)} -\/ The query that retrieves the arc information.  
\item {\ttfamily vtk\-S\-Q\-L\-Query = obj.\-Get\-Edge\-Query ()} -\/ The query that retrieves the arc information.  
\item {\ttfamily obj.\-Set\-Source\-Field (string )} -\/ The name of the field in the arc query for the source node of each arc.  
\item {\ttfamily string = obj.\-Get\-Source\-Field ()} -\/ The name of the field in the arc query for the source node of each arc.  
\item {\ttfamily obj.\-Set\-Target\-Field (string )} -\/ The name of the field in the arc query for the target node of each arc.  
\item {\ttfamily string = obj.\-Get\-Target\-Field ()} -\/ The name of the field in the arc query for the target node of each arc.  
\item {\ttfamily obj.\-Set\-Vertex\-Id\-Field (string )} -\/ The name of the field in the node query for the node I\-D.  
\item {\ttfamily string = obj.\-Get\-Vertex\-Id\-Field ()} -\/ The name of the field in the node query for the node I\-D.  
\item {\ttfamily obj.\-Set\-X\-Field (string )} -\/ The name of the field in the node query for the node's x coordinate.  
\item {\ttfamily string = obj.\-Get\-X\-Field ()} -\/ The name of the field in the node query for the node's x coordinate.  
\item {\ttfamily obj.\-Set\-Y\-Field (string )} -\/ The name of the field in the node query for the node's y coordinate.  
\item {\ttfamily string = obj.\-Get\-Y\-Field ()} -\/ The name of the field in the node query for the node's y coordinate.  
\item {\ttfamily obj.\-Set\-Z\-Field (string )} -\/ The name of the field in the node query for the node's z coordinate.  
\item {\ttfamily string = obj.\-Get\-Z\-Field ()} -\/ The name of the field in the node query for the node's z coordinate.  
\item {\ttfamily obj.\-Set\-Collapse\-Edges (bool )} -\/ When set, creates a graph with no parallel arcs. Parallel arcs are combined into one arc. No cell fields are passed to the output, except the vtk\-Ghost\-Levels array if it exists, but a new field \char`\"{}weight\char`\"{} is created that holds the number of duplicates of that arc in the input.  
\item {\ttfamily bool = obj.\-Get\-Collapse\-Edges ()} -\/ When set, creates a graph with no parallel arcs. Parallel arcs are combined into one arc. No cell fields are passed to the output, except the vtk\-Ghost\-Levels array if it exists, but a new field \char`\"{}weight\char`\"{} is created that holds the number of duplicates of that arc in the input.  
\item {\ttfamily obj.\-Collapse\-Edges\-On ()} -\/ When set, creates a graph with no parallel arcs. Parallel arcs are combined into one arc. No cell fields are passed to the output, except the vtk\-Ghost\-Levels array if it exists, but a new field \char`\"{}weight\char`\"{} is created that holds the number of duplicates of that arc in the input.  
\item {\ttfamily obj.\-Collapse\-Edges\-Off ()} -\/ When set, creates a graph with no parallel arcs. Parallel arcs are combined into one arc. No cell fields are passed to the output, except the vtk\-Ghost\-Levels array if it exists, but a new field \char`\"{}weight\char`\"{} is created that holds the number of duplicates of that arc in the input.  
\end{DoxyItemize}\hypertarget{vtkinfovis_vtksquarifylayoutstrategy}{}\section{vtk\-Squarify\-Layout\-Strategy}\label{vtkinfovis_vtksquarifylayoutstrategy}
Section\-: \hyperlink{sec_vtkinfovis}{Visualization Toolkit Infovis Classes} \hypertarget{vtkwidgets_vtkxyplotwidget_Usage}{}\subsection{Usage}\label{vtkwidgets_vtkxyplotwidget_Usage}
vtk\-Squarify\-Layout\-Strategy partitions the space for child vertices into regions that use all avaliable space and are as close to squares as possible. The algorithm also takes into account the relative vertex size.

.S\-E\-C\-T\-I\-O\-N Thanks The squarified tree map algorithm comes from\-: Bruls, D.\-M., C. Huizing, J.\-J. van Wijk. Squarified Treemaps. In\-: W. de Leeuw, R. van Liere (eds.), Data Visualization 2000, Proceedings of the joint Eurographics and I\-E\-E\-E T\-C\-V\-G Symposium on Visualization, 2000, Springer, Vienna, p. 33-\/42.

To create an instance of class vtk\-Squarify\-Layout\-Strategy, simply invoke its constructor as follows \begin{DoxyVerb}  obj = vtkSquarifyLayoutStrategy
\end{DoxyVerb}
 \hypertarget{vtkwidgets_vtkxyplotwidget_Methods}{}\subsection{Methods}\label{vtkwidgets_vtkxyplotwidget_Methods}
The class vtk\-Squarify\-Layout\-Strategy has several methods that can be used. They are listed below. Note that the documentation is translated automatically from the V\-T\-K sources, and may not be completely intelligible. When in doubt, consult the V\-T\-K website. In the methods listed below, {\ttfamily obj} is an instance of the vtk\-Squarify\-Layout\-Strategy class. 
\begin{DoxyItemize}
\item {\ttfamily string = obj.\-Get\-Class\-Name ()}  
\item {\ttfamily int = obj.\-Is\-A (string name)}  
\item {\ttfamily vtk\-Squarify\-Layout\-Strategy = obj.\-New\-Instance ()}  
\item {\ttfamily vtk\-Squarify\-Layout\-Strategy = obj.\-Safe\-Down\-Cast (vtk\-Object o)}  
\item {\ttfamily obj.\-Layout (vtk\-Tree input\-Tree, vtk\-Data\-Array coords\-Array, vtk\-Data\-Array size\-Array)} -\/ Perform the layout of a tree and place the results as 4-\/tuples in coords\-Array (Xmin, Xmax, Ymin, Ymax).  
\end{DoxyItemize}\hypertarget{vtkinfovis_vtkstackedtreelayoutstrategy}{}\section{vtk\-Stacked\-Tree\-Layout\-Strategy}\label{vtkinfovis_vtkstackedtreelayoutstrategy}
Section\-: \hyperlink{sec_vtkinfovis}{Visualization Toolkit Infovis Classes} \hypertarget{vtkwidgets_vtkxyplotwidget_Usage}{}\subsection{Usage}\label{vtkwidgets_vtkxyplotwidget_Usage}
Performs a tree ring layout or \char`\"{}icicle\char`\"{} layout on a tree. This involves assigning a sector region to each vertex in the tree, and placing that information in a data array with four components per tuple representing (inner\-Radius, outer\-Radius, start\-Angle, end\-Angle).

This class may be assigned as the layout strategy to vtk\-Area\-Layout.

.S\-E\-C\-T\-I\-O\-N Thanks Thanks to Jason Shepherd from Sandia National Laboratories for help developing this class.

To create an instance of class vtk\-Stacked\-Tree\-Layout\-Strategy, simply invoke its constructor as follows \begin{DoxyVerb}  obj = vtkStackedTreeLayoutStrategy
\end{DoxyVerb}
 \hypertarget{vtkwidgets_vtkxyplotwidget_Methods}{}\subsection{Methods}\label{vtkwidgets_vtkxyplotwidget_Methods}
The class vtk\-Stacked\-Tree\-Layout\-Strategy has several methods that can be used. They are listed below. Note that the documentation is translated automatically from the V\-T\-K sources, and may not be completely intelligible. When in doubt, consult the V\-T\-K website. In the methods listed below, {\ttfamily obj} is an instance of the vtk\-Stacked\-Tree\-Layout\-Strategy class. 
\begin{DoxyItemize}
\item {\ttfamily string = obj.\-Get\-Class\-Name ()}  
\item {\ttfamily int = obj.\-Is\-A (string name)}  
\item {\ttfamily vtk\-Stacked\-Tree\-Layout\-Strategy = obj.\-New\-Instance ()}  
\item {\ttfamily vtk\-Stacked\-Tree\-Layout\-Strategy = obj.\-Safe\-Down\-Cast (vtk\-Object o)}  
\item {\ttfamily obj.\-Layout (vtk\-Tree input\-Tree, vtk\-Data\-Array sector\-Array, vtk\-Data\-Array size\-Array)} -\/ Perform the layout of the input tree, and store the sector bounds of each vertex as a tuple (inner\-Radius, outer\-Radius, start\-Angle, end\-Angle) in a data array.  
\item {\ttfamily obj.\-Layout\-Edge\-Points (vtk\-Tree input\-Tree, vtk\-Data\-Array sector\-Array, vtk\-Data\-Array size\-Array, vtk\-Tree edge\-Routing\-Tree)} -\/ Fill edge\-Routing\-Tree with points suitable for routing edges of an overlaid graph.  
\item {\ttfamily obj.\-Set\-Interior\-Radius (double )} -\/ Define the tree ring's interior radius.  
\item {\ttfamily double = obj.\-Get\-Interior\-Radius ()} -\/ Define the tree ring's interior radius.  
\item {\ttfamily obj.\-Set\-Ring\-Thickness (double )} -\/ Define the thickness of each of the tree rings.  
\item {\ttfamily double = obj.\-Get\-Ring\-Thickness ()} -\/ Define the thickness of each of the tree rings.  
\item {\ttfamily obj.\-Set\-Root\-Start\-Angle (double )} -\/ Define the start angle for the root node. N\-O\-T\-E\-: It is assumed that the root end angle is greater than the root start angle and subtends no more than 360 degrees.  
\item {\ttfamily double = obj.\-Get\-Root\-Start\-Angle ()} -\/ Define the start angle for the root node. N\-O\-T\-E\-: It is assumed that the root end angle is greater than the root start angle and subtends no more than 360 degrees.  
\item {\ttfamily obj.\-Set\-Root\-End\-Angle (double )} -\/ Define the end angle for the root node. N\-O\-T\-E\-: It is assumed that the root end angle is greater than the root start angle and subtends no more than 360 degrees.  
\item {\ttfamily double = obj.\-Get\-Root\-End\-Angle ()} -\/ Define the end angle for the root node. N\-O\-T\-E\-: It is assumed that the root end angle is greater than the root start angle and subtends no more than 360 degrees.  
\item {\ttfamily obj.\-Set\-Use\-Rectangular\-Coordinates (bool )} -\/ Define whether or not rectangular coordinates are being used (as opposed to polar coordinates).  
\item {\ttfamily bool = obj.\-Get\-Use\-Rectangular\-Coordinates ()} -\/ Define whether or not rectangular coordinates are being used (as opposed to polar coordinates).  
\item {\ttfamily obj.\-Use\-Rectangular\-Coordinates\-On ()} -\/ Define whether or not rectangular coordinates are being used (as opposed to polar coordinates).  
\item {\ttfamily obj.\-Use\-Rectangular\-Coordinates\-Off ()} -\/ Define whether or not rectangular coordinates are being used (as opposed to polar coordinates).  
\item {\ttfamily obj.\-Set\-Reverse (bool )} -\/ Define whether to reverse the order of the tree stacks from low to high.  
\item {\ttfamily bool = obj.\-Get\-Reverse ()} -\/ Define whether to reverse the order of the tree stacks from low to high.  
\item {\ttfamily obj.\-Reverse\-On ()} -\/ Define whether to reverse the order of the tree stacks from low to high.  
\item {\ttfamily obj.\-Reverse\-Off ()} -\/ Define whether to reverse the order of the tree stacks from low to high.  
\item {\ttfamily obj.\-Set\-Interior\-Log\-Spacing\-Value (double )} -\/ The spacing of tree levels in the edge routing tree. Levels near zero give more space to levels near the root, while levels near one (the default) create evenly-\/spaced levels. Levels above one give more space to levels near the leaves.  
\item {\ttfamily double = obj.\-Get\-Interior\-Log\-Spacing\-Value ()} -\/ The spacing of tree levels in the edge routing tree. Levels near zero give more space to levels near the root, while levels near one (the default) create evenly-\/spaced levels. Levels above one give more space to levels near the leaves.  
\item {\ttfamily vtk\-Id\-Type = obj.\-Find\-Vertex (vtk\-Tree tree, vtk\-Data\-Array array, float pnt\mbox{[}2\mbox{]})} -\/ Returns the vertex id that contains pnt (or -\/1 if no one contains it).  
\end{DoxyItemize}\hypertarget{vtkinfovis_vtkstatisticsalgorithm}{}\section{vtk\-Statistics\-Algorithm}\label{vtkinfovis_vtkstatisticsalgorithm}
Section\-: \hyperlink{sec_vtkinfovis}{Visualization Toolkit Infovis Classes} \hypertarget{vtkwidgets_vtkxyplotwidget_Usage}{}\subsection{Usage}\label{vtkwidgets_vtkxyplotwidget_Usage}
All statistics algorithms can conceptually be operated with several options\-: Learn\-: given an input data set, calculate a minimal statistical model (e.\-g., sums, raw moments, joint probabilities). Derive\-: given an input minimal statistical model, derive the full model (e.\-g., descriptive statistics, quantiles, correlations, conditional probabilities). N\-B\-: It may be, or not be, a problem that a full model was not derived. For instance, when doing parallel calculations, one only wants to derive the full model after all partial calculations have completed. On the other hand, one can also directly provide a full model, that was previously calculated or guessed, and not derive a new one. Assess\-: given an input data set, input statistics, and some form of threshold, assess a subset of the data set. Test\-: perform at least one statistical test. Therefore, a vtk\-Statistics\-Algorithm has the following vtk\-Table ports 3 input ports\-: Data (mandatory) Parameters to the learn phase (optional) Input model (optional) 3 output port (called Output)\-: Data (annotated with assessments when the Assess option is O\-N). Output model (identical to the the input model when Learn option is O\-F\-F). Meta information about the model and/or the overall fit of the data to the model; is filled only when the Assess option is O\-N.

.S\-E\-C\-T\-I\-O\-N Thanks Thanks to Philippe Pebay and David Thompson from Sandia National Laboratories for implementing this class.

To create an instance of class vtk\-Statistics\-Algorithm, simply invoke its constructor as follows \begin{DoxyVerb}  obj = vtkStatisticsAlgorithm
\end{DoxyVerb}
 \hypertarget{vtkwidgets_vtkxyplotwidget_Methods}{}\subsection{Methods}\label{vtkwidgets_vtkxyplotwidget_Methods}
The class vtk\-Statistics\-Algorithm has several methods that can be used. They are listed below. Note that the documentation is translated automatically from the V\-T\-K sources, and may not be completely intelligible. When in doubt, consult the V\-T\-K website. In the methods listed below, {\ttfamily obj} is an instance of the vtk\-Statistics\-Algorithm class. 
\begin{DoxyItemize}
\item {\ttfamily string = obj.\-Get\-Class\-Name ()}  
\item {\ttfamily int = obj.\-Is\-A (string name)}  
\item {\ttfamily vtk\-Statistics\-Algorithm = obj.\-New\-Instance ()}  
\item {\ttfamily vtk\-Statistics\-Algorithm = obj.\-Safe\-Down\-Cast (vtk\-Object o)}  
\item {\ttfamily obj.\-Set\-Learn\-Option\-Parameter\-Connection (vtk\-Algorithm\-Output params)} -\/ A convenience method for setting learn input parameters (if one is expected or allowed). It is equivalent to calling Set\-Input( 1, params );  
\item {\ttfamily obj.\-Set\-Learn\-Option\-Parameters (vtk\-Data\-Object params)} -\/ A convenience method for setting the input model (if one is expected or allowed). It is equivalent to calling Set\-Input\-Connection( 2, model );  
\item {\ttfamily obj.\-Set\-Input\-Model\-Connection (vtk\-Algorithm\-Output model)} -\/ //  {\ttfamily obj.\-Set\-Input\-Model (vtk\-Data\-Object model)} -\/ Set/\-Get the Learn option.  
\item {\ttfamily obj.\-Set\-Learn\-Option (bool )} -\/ Set/\-Get the Learn option.  
\item {\ttfamily bool = obj.\-Get\-Learn\-Option ()} -\/ Set/\-Get the Learn option.  
\item {\ttfamily obj.\-Set\-Derive\-Option (bool )} -\/ Set/\-Get the Derive option.  
\item {\ttfamily bool = obj.\-Get\-Derive\-Option ()} -\/ Set/\-Get the Derive option.  
\item {\ttfamily obj.\-Set\-Assess\-Option (bool )} -\/ Set/\-Get the Assess option.  
\item {\ttfamily bool = obj.\-Get\-Assess\-Option ()} -\/ Set/\-Get the Assess option.  
\item {\ttfamily obj.\-Set\-Test\-Option (bool )} -\/ Set/\-Get the Test option.  
\item {\ttfamily bool = obj.\-Get\-Test\-Option ()} -\/ Set/\-Get the Test option.  
\item {\ttfamily obj.\-Set\-Assess\-Parameters (vtk\-String\-Array )} -\/ Set/get assessment parameters.  
\item {\ttfamily vtk\-String\-Array = obj.\-Get\-Assess\-Parameters ()} -\/ Set/get assessment parameters.  
\item {\ttfamily obj.\-Set\-Assess\-Names (vtk\-String\-Array )} -\/ Set/get assessment names.  
\item {\ttfamily vtk\-String\-Array = obj.\-Get\-Assess\-Names ()} -\/ Set/get assessment names.  
\item {\ttfamily obj.\-Set\-Column\-Status (string nam\-Col, int status)} -\/ Add or remove a column from the current analysis request. Once all the column status values are set, call Request\-Selected\-Columns() before selecting another set of columns for a different analysis request. The way that columns selections are used varies from algorithm to algorithm.

Note\-: the set of selected columns is maintained in vtk\-Statistics\-Algorithm\-Private\-::\-Buffer until Request\-Selected\-Columns() is called, at which point the set is appended to vtk\-Statistics\-Algorithm\-Private\-::\-Requests. If there are any columns in vtk\-Statistics\-Algorithm\-Private\-::\-Buffer at the time Request\-Data() is called, Request\-Selected\-Columns() will be called and the selection added to the list of requests.  
\item {\ttfamily obj.\-Reset\-All\-Column\-States ()} -\/ Set the the status of each and every column in the current request to O\-F\-F (0).  
\item {\ttfamily int = obj.\-Request\-Selected\-Columns ()} -\/ Use the current column status values to produce a new request for statistics to be produced when Request\-Data() is called. See Set\-Column\-Status() for more information.  
\item {\ttfamily obj.\-Reset\-Requests ()} -\/ Empty the list of current requests.  
\item {\ttfamily vtk\-Id\-Type = obj.\-Get\-Number\-Of\-Requests ()} -\/ Return the number of requests. This does not include any request that is in the column-\/status buffer but for which Request\-Selected\-Columns() has not yet been called (even though it is possible this request will be honored when the filter is run -- see Set\-Column\-Status() for more information).  
\item {\ttfamily vtk\-Id\-Type = obj.\-Get\-Number\-Of\-Columns\-For\-Request (vtk\-Id\-Type request)} -\/ Return the number of columns for a given request.  
\item {\ttfamily string = obj.\-Get\-Column\-For\-Request (vtk\-Id\-Type r, vtk\-Id\-Type c)} -\/ Provide the name of the {\itshape c-\/th} column for the {\itshape r-\/th} request.

For the version of this routine that returns an integer, if the request or column does not exist because {\itshape r} or {\itshape c} is out of bounds, this routine returns 0 and the value of {\itshape column\-Name} is unspecified. Otherwise, it returns 1 and the value of {\itshape column\-Name} is set.

For the version of this routine that returns const char$\ast$, if the request or column does not exist because {\itshape r} or {\itshape c} is out of bounds, the routine returns N\-U\-L\-L. Otherwise it returns the column name. This version is not thread-\/safe.  
\item {\ttfamily obj.\-Aggregate (vtk\-Data\-Object\-Collection , vtk\-Data\-Object )} -\/ Given a collection of models, calculate aggregate model  
\end{DoxyItemize}\hypertarget{vtkinfovis_vtkstrahlermetric}{}\section{vtk\-Strahler\-Metric}\label{vtkinfovis_vtkstrahlermetric}
Section\-: \hyperlink{sec_vtkinfovis}{Visualization Toolkit Infovis Classes} \hypertarget{vtkwidgets_vtkxyplotwidget_Usage}{}\subsection{Usage}\label{vtkwidgets_vtkxyplotwidget_Usage}
The Strahler metric is a value assigned to each vertex of a tree that characterizes the structural complexity of the sub-\/tree rooted at that node. The metric originated in the study of river systems, but has been applied to other tree-\/ structured systes, Details of the metric and the rationale for using it in infovis can be found in\-:

Tree Visualization and Navigation Clues for Information Visualization, I. Herman, M. Delest, and G. Melancon, Computer Graphics Forum, Vol 17(2), Blackwell, 1998.

The input tree is copied to the output, but with a new array added to the output vertex data.

.S\-E\-C\-T\-I\-O\-N Thanks Thanks to David Duke from the University of Leeds for providing this implementation.

To create an instance of class vtk\-Strahler\-Metric, simply invoke its constructor as follows \begin{DoxyVerb}  obj = vtkStrahlerMetric
\end{DoxyVerb}
 \hypertarget{vtkwidgets_vtkxyplotwidget_Methods}{}\subsection{Methods}\label{vtkwidgets_vtkxyplotwidget_Methods}
The class vtk\-Strahler\-Metric has several methods that can be used. They are listed below. Note that the documentation is translated automatically from the V\-T\-K sources, and may not be completely intelligible. When in doubt, consult the V\-T\-K website. In the methods listed below, {\ttfamily obj} is an instance of the vtk\-Strahler\-Metric class. 
\begin{DoxyItemize}
\item {\ttfamily string = obj.\-Get\-Class\-Name ()}  
\item {\ttfamily int = obj.\-Is\-A (string name)}  
\item {\ttfamily vtk\-Strahler\-Metric = obj.\-New\-Instance ()}  
\item {\ttfamily vtk\-Strahler\-Metric = obj.\-Safe\-Down\-Cast (vtk\-Object o)}  
\item {\ttfamily obj.\-Set\-Metric\-Array\-Name (string )} -\/ Set the name of the array in which the Strahler values will be stored within the output vertex data. Default is \char`\"{}\-Strahler\char`\"{}  
\item {\ttfamily obj.\-Set\-Normalize (int )} -\/ Set/get setting of normalize flag. If this is set, the Strahler values are scaled into the range \mbox{[}0..1\mbox{]}. Default is for normalization to be O\-F\-F.  
\item {\ttfamily int = obj.\-Get\-Normalize ()} -\/ Set/get setting of normalize flag. If this is set, the Strahler values are scaled into the range \mbox{[}0..1\mbox{]}. Default is for normalization to be O\-F\-F.  
\item {\ttfamily obj.\-Normalize\-On ()} -\/ Set/get setting of normalize flag. If this is set, the Strahler values are scaled into the range \mbox{[}0..1\mbox{]}. Default is for normalization to be O\-F\-F.  
\item {\ttfamily obj.\-Normalize\-Off ()} -\/ Set/get setting of normalize flag. If this is set, the Strahler values are scaled into the range \mbox{[}0..1\mbox{]}. Default is for normalization to be O\-F\-F.  
\item {\ttfamily float = obj.\-Get\-Max\-Strahler ()} -\/ Get the maximum strahler value for the tree.  
\end{DoxyItemize}\hypertarget{vtkinfovis_vtkstreamgraph}{}\section{vtk\-Stream\-Graph}\label{vtkinfovis_vtkstreamgraph}
Section\-: \hyperlink{sec_vtkinfovis}{Visualization Toolkit Infovis Classes} \hypertarget{vtkwidgets_vtkxyplotwidget_Usage}{}\subsection{Usage}\label{vtkwidgets_vtkxyplotwidget_Usage}
vtk\-Stream\-Graph iteratively collects information from the input graph and combines it in the output graph. It internally maintains a graph instance that is incrementally updated every time the filter is called.

Each update, vtk\-Merge\-Graphs is used to combine this filter's input with the internal graph.

To create an instance of class vtk\-Stream\-Graph, simply invoke its constructor as follows \begin{DoxyVerb}  obj = vtkStreamGraph
\end{DoxyVerb}
 \hypertarget{vtkwidgets_vtkxyplotwidget_Methods}{}\subsection{Methods}\label{vtkwidgets_vtkxyplotwidget_Methods}
The class vtk\-Stream\-Graph has several methods that can be used. They are listed below. Note that the documentation is translated automatically from the V\-T\-K sources, and may not be completely intelligible. When in doubt, consult the V\-T\-K website. In the methods listed below, {\ttfamily obj} is an instance of the vtk\-Stream\-Graph class. 
\begin{DoxyItemize}
\item {\ttfamily string = obj.\-Get\-Class\-Name ()}  
\item {\ttfamily int = obj.\-Is\-A (string name)}  
\item {\ttfamily vtk\-Stream\-Graph = obj.\-New\-Instance ()}  
\item {\ttfamily vtk\-Stream\-Graph = obj.\-Safe\-Down\-Cast (vtk\-Object o)}  
\item {\ttfamily obj.\-Set\-Max\-Edges (vtk\-Id\-Type )} -\/ The maximum number of edges in the combined graph. Default is -\/1, which specifies that there should be no limit on the number of edges.  
\item {\ttfamily vtk\-Id\-Type = obj.\-Get\-Max\-Edges ()} -\/ The maximum number of edges in the combined graph. Default is -\/1, which specifies that there should be no limit on the number of edges.  
\end{DoxyItemize}\hypertarget{vtkinfovis_vtkstringtocategory}{}\section{vtk\-String\-To\-Category}\label{vtkinfovis_vtkstringtocategory}
Section\-: \hyperlink{sec_vtkinfovis}{Visualization Toolkit Infovis Classes} \hypertarget{vtkwidgets_vtkxyplotwidget_Usage}{}\subsection{Usage}\label{vtkwidgets_vtkxyplotwidget_Usage}
vtk\-String\-To\-Category creates an integer array named \char`\"{}category\char`\"{} based on the values in a string array. You may use this filter to create an array that you may use to color points/cells by the values in a string array. Currently there is not support to color by a string array directly. The category values will range from zero to N-\/1, where N is the number of distinct strings in the string array. Set the string array to process with Set\-Input\-Array\-To\-Process(0,0,0,...). The array may be in the point, cell, or field data of the data object.

To create an instance of class vtk\-String\-To\-Category, simply invoke its constructor as follows \begin{DoxyVerb}  obj = vtkStringToCategory
\end{DoxyVerb}
 \hypertarget{vtkwidgets_vtkxyplotwidget_Methods}{}\subsection{Methods}\label{vtkwidgets_vtkxyplotwidget_Methods}
The class vtk\-String\-To\-Category has several methods that can be used. They are listed below. Note that the documentation is translated automatically from the V\-T\-K sources, and may not be completely intelligible. When in doubt, consult the V\-T\-K website. In the methods listed below, {\ttfamily obj} is an instance of the vtk\-String\-To\-Category class. 
\begin{DoxyItemize}
\item {\ttfamily string = obj.\-Get\-Class\-Name ()}  
\item {\ttfamily int = obj.\-Is\-A (string name)}  
\item {\ttfamily vtk\-String\-To\-Category = obj.\-New\-Instance ()}  
\item {\ttfamily vtk\-String\-To\-Category = obj.\-Safe\-Down\-Cast (vtk\-Object o)}  
\item {\ttfamily obj.\-Set\-Category\-Array\-Name (string )} -\/ The name to give to the output vtk\-Int\-Array of category values.  
\item {\ttfamily string = obj.\-Get\-Category\-Array\-Name ()} -\/ The name to give to the output vtk\-Int\-Array of category values.  
\end{DoxyItemize}\hypertarget{vtkinfovis_vtkstringtonumeric}{}\section{vtk\-String\-To\-Numeric}\label{vtkinfovis_vtkstringtonumeric}
Section\-: \hyperlink{sec_vtkinfovis}{Visualization Toolkit Infovis Classes} \hypertarget{vtkwidgets_vtkxyplotwidget_Usage}{}\subsection{Usage}\label{vtkwidgets_vtkxyplotwidget_Usage}
vtk\-String\-To\-Numeric is a filter for converting a string array into a numeric arrays.

To create an instance of class vtk\-String\-To\-Numeric, simply invoke its constructor as follows \begin{DoxyVerb}  obj = vtkStringToNumeric
\end{DoxyVerb}
 \hypertarget{vtkwidgets_vtkxyplotwidget_Methods}{}\subsection{Methods}\label{vtkwidgets_vtkxyplotwidget_Methods}
The class vtk\-String\-To\-Numeric has several methods that can be used. They are listed below. Note that the documentation is translated automatically from the V\-T\-K sources, and may not be completely intelligible. When in doubt, consult the V\-T\-K website. In the methods listed below, {\ttfamily obj} is an instance of the vtk\-String\-To\-Numeric class. 
\begin{DoxyItemize}
\item {\ttfamily string = obj.\-Get\-Class\-Name ()}  
\item {\ttfamily int = obj.\-Is\-A (string name)}  
\item {\ttfamily vtk\-String\-To\-Numeric = obj.\-New\-Instance ()}  
\item {\ttfamily vtk\-String\-To\-Numeric = obj.\-Safe\-Down\-Cast (vtk\-Object o)}  
\item {\ttfamily obj.\-Set\-Convert\-Field\-Data (bool )} -\/ Whether to detect and convert field data arrays. Default is on.  
\item {\ttfamily bool = obj.\-Get\-Convert\-Field\-Data ()} -\/ Whether to detect and convert field data arrays. Default is on.  
\item {\ttfamily obj.\-Convert\-Field\-Data\-On ()} -\/ Whether to detect and convert field data arrays. Default is on.  
\item {\ttfamily obj.\-Convert\-Field\-Data\-Off ()} -\/ Whether to detect and convert field data arrays. Default is on.  
\item {\ttfamily obj.\-Set\-Convert\-Point\-Data (bool )} -\/ Whether to detect and convert cell data arrays. Default is on.  
\item {\ttfamily bool = obj.\-Get\-Convert\-Point\-Data ()} -\/ Whether to detect and convert cell data arrays. Default is on.  
\item {\ttfamily obj.\-Convert\-Point\-Data\-On ()} -\/ Whether to detect and convert cell data arrays. Default is on.  
\item {\ttfamily obj.\-Convert\-Point\-Data\-Off ()} -\/ Whether to detect and convert cell data arrays. Default is on.  
\item {\ttfamily obj.\-Set\-Convert\-Cell\-Data (bool )} -\/ Whether to detect and convert point data arrays. Default is on.  
\item {\ttfamily bool = obj.\-Get\-Convert\-Cell\-Data ()} -\/ Whether to detect and convert point data arrays. Default is on.  
\item {\ttfamily obj.\-Convert\-Cell\-Data\-On ()} -\/ Whether to detect and convert point data arrays. Default is on.  
\item {\ttfamily obj.\-Convert\-Cell\-Data\-Off ()} -\/ Whether to detect and convert point data arrays. Default is on.  
\item {\ttfamily obj.\-Set\-Convert\-Vertex\-Data (bool b)} -\/ Whether to detect and convert vertex data arrays. Default is on.  
\item {\ttfamily bool = obj.\-Get\-Convert\-Vertex\-Data ()} -\/ Whether to detect and convert vertex data arrays. Default is on.  
\item {\ttfamily obj.\-Convert\-Vertex\-Data\-On ()} -\/ Whether to detect and convert vertex data arrays. Default is on.  
\item {\ttfamily obj.\-Convert\-Vertex\-Data\-Off ()} -\/ Whether to detect and convert vertex data arrays. Default is on.  
\item {\ttfamily obj.\-Set\-Convert\-Edge\-Data (bool b)} -\/ Whether to detect and convert edge data arrays. Default is on.  
\item {\ttfamily bool = obj.\-Get\-Convert\-Edge\-Data ()} -\/ Whether to detect and convert edge data arrays. Default is on.  
\item {\ttfamily obj.\-Convert\-Edge\-Data\-On ()} -\/ Whether to detect and convert edge data arrays. Default is on.  
\item {\ttfamily obj.\-Convert\-Edge\-Data\-Off ()} -\/ Whether to detect and convert edge data arrays. Default is on.  
\item {\ttfamily obj.\-Set\-Convert\-Row\-Data (bool b)} -\/ Whether to detect and convert row data arrays. Default is on.  
\item {\ttfamily bool = obj.\-Get\-Convert\-Row\-Data ()} -\/ Whether to detect and convert row data arrays. Default is on.  
\item {\ttfamily obj.\-Convert\-Row\-Data\-On ()} -\/ Whether to detect and convert row data arrays. Default is on.  
\item {\ttfamily obj.\-Convert\-Row\-Data\-Off ()} -\/ Whether to detect and convert row data arrays. Default is on.  
\end{DoxyItemize}\hypertarget{vtkinfovis_vtkstringtotimepoint}{}\section{vtk\-String\-To\-Time\-Point}\label{vtkinfovis_vtkstringtotimepoint}
Section\-: \hyperlink{sec_vtkinfovis}{Visualization Toolkit Infovis Classes} \hypertarget{vtkwidgets_vtkxyplotwidget_Usage}{}\subsection{Usage}\label{vtkwidgets_vtkxyplotwidget_Usage}
vtk\-String\-To\-Time\-Point is a filter for converting a string array into a datetime, time or date array. The input strings must conform to one of the I\-S\-O8601 formats defined in vtk\-Time\-Point\-Utility.

The input array specified by Set\-Input\-Array\-To\-Process(...) indicates the array to process. This array must be of type vtk\-String\-Array.

The output array will be of type vtk\-Type\-U\-Int64\-Array.

To create an instance of class vtk\-String\-To\-Time\-Point, simply invoke its constructor as follows \begin{DoxyVerb}  obj = vtkStringToTimePoint
\end{DoxyVerb}
 \hypertarget{vtkwidgets_vtkxyplotwidget_Methods}{}\subsection{Methods}\label{vtkwidgets_vtkxyplotwidget_Methods}
The class vtk\-String\-To\-Time\-Point has several methods that can be used. They are listed below. Note that the documentation is translated automatically from the V\-T\-K sources, and may not be completely intelligible. When in doubt, consult the V\-T\-K website. In the methods listed below, {\ttfamily obj} is an instance of the vtk\-String\-To\-Time\-Point class. 
\begin{DoxyItemize}
\item {\ttfamily string = obj.\-Get\-Class\-Name ()}  
\item {\ttfamily int = obj.\-Is\-A (string name)}  
\item {\ttfamily vtk\-String\-To\-Time\-Point = obj.\-New\-Instance ()}  
\item {\ttfamily vtk\-String\-To\-Time\-Point = obj.\-Safe\-Down\-Cast (vtk\-Object o)}  
\item {\ttfamily obj.\-Set\-Output\-Array\-Name (string )} -\/ The name of the output array. If this is not specified, the name will be the same as the input array name with either \char`\"{} \mbox{[}to datetime\mbox{]}\char`\"{}, \char`\"{} \mbox{[}to date\mbox{]}\char`\"{}, or \char`\"{} \mbox{[}to time\mbox{]}\char`\"{} appended.  
\item {\ttfamily string = obj.\-Get\-Output\-Array\-Name ()} -\/ The name of the output array. If this is not specified, the name will be the same as the input array name with either \char`\"{} \mbox{[}to datetime\mbox{]}\char`\"{}, \char`\"{} \mbox{[}to date\mbox{]}\char`\"{}, or \char`\"{} \mbox{[}to time\mbox{]}\char`\"{} appended.  
\end{DoxyItemize}\hypertarget{vtkinfovis_vtktabletoarray}{}\section{vtk\-Table\-To\-Array}\label{vtkinfovis_vtktabletoarray}
Section\-: \hyperlink{sec_vtkinfovis}{Visualization Toolkit Infovis Classes} \hypertarget{vtkwidgets_vtkxyplotwidget_Usage}{}\subsection{Usage}\label{vtkwidgets_vtkxyplotwidget_Usage}
Converts a vtk\-Table into a dense matrix. Use Add\-Column() to designate one-\/to-\/many table columns that will become columns in the output matrix.

To create an instance of class vtk\-Table\-To\-Array, simply invoke its constructor as follows \begin{DoxyVerb}  obj = vtkTableToArray
\end{DoxyVerb}
 \hypertarget{vtkwidgets_vtkxyplotwidget_Methods}{}\subsection{Methods}\label{vtkwidgets_vtkxyplotwidget_Methods}
The class vtk\-Table\-To\-Array has several methods that can be used. They are listed below. Note that the documentation is translated automatically from the V\-T\-K sources, and may not be completely intelligible. When in doubt, consult the V\-T\-K website. In the methods listed below, {\ttfamily obj} is an instance of the vtk\-Table\-To\-Array class. 
\begin{DoxyItemize}
\item {\ttfamily string = obj.\-Get\-Class\-Name ()}  
\item {\ttfamily int = obj.\-Is\-A (string name)}  
\item {\ttfamily vtk\-Table\-To\-Array = obj.\-New\-Instance ()}  
\item {\ttfamily vtk\-Table\-To\-Array = obj.\-Safe\-Down\-Cast (vtk\-Object o)}  
\item {\ttfamily obj.\-Clear\-Columns ()} -\/ Specify the set of input table columns that will be mapped to columns in the output matrix.  
\item {\ttfamily obj.\-Add\-Column (string name)} -\/ Specify the set of input table columns that will be mapped to columns in the output matrix.  
\end{DoxyItemize}\hypertarget{vtkinfovis_vtktabletograph}{}\section{vtk\-Table\-To\-Graph}\label{vtkinfovis_vtktabletograph}
Section\-: \hyperlink{sec_vtkinfovis}{Visualization Toolkit Infovis Classes} \hypertarget{vtkwidgets_vtkxyplotwidget_Usage}{}\subsection{Usage}\label{vtkwidgets_vtkxyplotwidget_Usage}
vtk\-Table\-To\-Graph converts a table to a graph using an auxilliary link graph. The link graph specifies how each row in the table should be converted to an edge, or a collection of edges. It also specifies which columns of the table should be considered part of the same domain, and which columns should be hidden.

A second, optional, table may be provided as the vertex table. This vertex table must have one or more domain columns whose values match values in the edge table. The linked column name is specified in the domain array in the link graph. The output graph will only contain vertices corresponding to a row in the vertex table. For heterogenous graphs, you may want to use vtk\-Merge\-Tables to create a single vertex table.

The link graph contains the following arrays\-:

(1) The \char`\"{}column\char`\"{} array has the names of the columns to connect in each table row. This array is required.

(2) The optional \char`\"{}domain\char`\"{} array provides user-\/defined domain names for each column. Matching domains in multiple columns will merge vertices with the same value from those columns. By default, all columns are in the same domain. If a vertex table is supplied, the domain indicates the column in the vertex table that the edge table column associates with. If the user provides a vertex table but no domain names, the output will be an empty graph. Hidden columns do not need valid domain names.

(3) The optional \char`\"{}hidden\char`\"{} array is a bit array specifying whether the column should be hidden. The resulting graph will contain edges representing connections \char`\"{}through\char`\"{} the hidden column, but the vertices for that column will not be present. By default, no columns are hidden. Hiding a column in a particular domain hides all columns in that domain.

The output graph will contain three additional arrays in the vertex data. The \char`\"{}domain\char`\"{} column is a string array containing the domain of each vertex. The \char`\"{}label\char`\"{} column is a string version of the distinct value that, along with the domain, defines that vertex. The \char`\"{}ids\char`\"{} column also contains the distinguishing value, but as a vtk\-Variant holding the raw value instead of being converted to a string. The \char`\"{}ids\char`\"{} column is set as the vertex pedigree I\-D attribute.

To create an instance of class vtk\-Table\-To\-Graph, simply invoke its constructor as follows \begin{DoxyVerb}  obj = vtkTableToGraph
\end{DoxyVerb}
 \hypertarget{vtkwidgets_vtkxyplotwidget_Methods}{}\subsection{Methods}\label{vtkwidgets_vtkxyplotwidget_Methods}
The class vtk\-Table\-To\-Graph has several methods that can be used. They are listed below. Note that the documentation is translated automatically from the V\-T\-K sources, and may not be completely intelligible. When in doubt, consult the V\-T\-K website. In the methods listed below, {\ttfamily obj} is an instance of the vtk\-Table\-To\-Graph class. 
\begin{DoxyItemize}
\item {\ttfamily string = obj.\-Get\-Class\-Name ()}  
\item {\ttfamily int = obj.\-Is\-A (string name)}  
\item {\ttfamily vtk\-Table\-To\-Graph = obj.\-New\-Instance ()}  
\item {\ttfamily vtk\-Table\-To\-Graph = obj.\-Safe\-Down\-Cast (vtk\-Object o)}  
\item {\ttfamily obj.\-Add\-Link\-Vertex (string column, string domain, int hidden)} -\/ Add a vertex to the link graph. Specify the column name, the domain name for the column, and whether the column is hidden.  
\item {\ttfamily obj.\-Clear\-Link\-Vertices ()} -\/ Clear the link graph vertices. This also clears all edges.  
\item {\ttfamily obj.\-Add\-Link\-Edge (string column1, string column2)} -\/ Add an edge to the link graph. Specify the names of the columns to link.  
\item {\ttfamily obj.\-Clear\-Link\-Edges ()} -\/ Clear the link graph edges. The graph vertices will remain.  
\item {\ttfamily vtk\-Mutable\-Directed\-Graph = obj.\-Get\-Link\-Graph ()} -\/ The graph describing how to link the columns in the table.  
\item {\ttfamily obj.\-Set\-Link\-Graph (vtk\-Mutable\-Directed\-Graph g)} -\/ The graph describing how to link the columns in the table.  
\item {\ttfamily obj.\-Link\-Column\-Path (vtk\-String\-Array column, vtk\-String\-Array domain, vtk\-Bit\-Array hidden)} -\/ Links the columns in a specific order. This creates a simple path as the link graph.  
\item {\ttfamily obj.\-Set\-Directed (bool )} -\/ Specify the directedness of the output graph.  
\item {\ttfamily bool = obj.\-Get\-Directed ()} -\/ Specify the directedness of the output graph.  
\item {\ttfamily obj.\-Directed\-On ()} -\/ Specify the directedness of the output graph.  
\item {\ttfamily obj.\-Directed\-Off ()} -\/ Specify the directedness of the output graph.  
\item {\ttfamily long = obj.\-Get\-M\-Time ()} -\/ Get the current modified time.  
\item {\ttfamily obj.\-Set\-Vertex\-Table\-Connection (vtk\-Algorithm\-Output in)} -\/ A convenience method for setting the vertex table input. This is mainly for the benefit of the V\-T\-K client/server layer, vanilla V\-T\-K code should use e.\-g\-:

table\-\_\-to\-\_\-graph-\/$>$Set\-Input\-Connection(1, vertex\-\_\-table-\/$>$output());


\end{DoxyItemize}\hypertarget{vtkinfovis_vtktabletosparsearray}{}\section{vtk\-Table\-To\-Sparse\-Array}\label{vtkinfovis_vtktabletosparsearray}
Section\-: \hyperlink{sec_vtkinfovis}{Visualization Toolkit Infovis Classes} \hypertarget{vtkwidgets_vtkxyplotwidget_Usage}{}\subsection{Usage}\label{vtkwidgets_vtkxyplotwidget_Usage}
Converts a vtk\-Table into a sparse array. Use Add\-Coordinate\-Column() to designate one-\/to-\/many table columns that contain coordinates for each array value, and Set\-Value\-Column() to designate the table column that contains array values.

Thus, the number of dimensions in the output array will equal the number of calls to Add\-Coordinate\-Column().

The coordinate columns will also be used to populate dimension labels in the output array.

.S\-E\-C\-T\-I\-O\-N Thanks Developed by Timothy M. Shead (\href{mailto:tshead@sandia.gov}{\tt tshead@sandia.\-gov}) at Sandia National Laboratories.

To create an instance of class vtk\-Table\-To\-Sparse\-Array, simply invoke its constructor as follows \begin{DoxyVerb}  obj = vtkTableToSparseArray
\end{DoxyVerb}
 \hypertarget{vtkwidgets_vtkxyplotwidget_Methods}{}\subsection{Methods}\label{vtkwidgets_vtkxyplotwidget_Methods}
The class vtk\-Table\-To\-Sparse\-Array has several methods that can be used. They are listed below. Note that the documentation is translated automatically from the V\-T\-K sources, and may not be completely intelligible. When in doubt, consult the V\-T\-K website. In the methods listed below, {\ttfamily obj} is an instance of the vtk\-Table\-To\-Sparse\-Array class. 
\begin{DoxyItemize}
\item {\ttfamily string = obj.\-Get\-Class\-Name ()}  
\item {\ttfamily int = obj.\-Is\-A (string name)}  
\item {\ttfamily vtk\-Table\-To\-Sparse\-Array = obj.\-New\-Instance ()}  
\item {\ttfamily vtk\-Table\-To\-Sparse\-Array = obj.\-Safe\-Down\-Cast (vtk\-Object o)}  
\item {\ttfamily obj.\-Clear\-Coordinate\-Columns ()} -\/ Specify the set of input table columns that will be mapped to coordinates in the output sparse array.  
\item {\ttfamily obj.\-Add\-Coordinate\-Column (string name)} -\/ Specify the set of input table columns that will be mapped to coordinates in the output sparse array.  
\item {\ttfamily obj.\-Set\-Value\-Column (string name)} -\/ Specify the input table column that will be mapped to values in the output array.  
\item {\ttfamily string = obj.\-Get\-Value\-Column ()} -\/ Specify the input table column that will be mapped to values in the output array.  
\end{DoxyItemize}\hypertarget{vtkinfovis_vtktabletotreefilter}{}\section{vtk\-Table\-To\-Tree\-Filter}\label{vtkinfovis_vtktabletotreefilter}
Section\-: \hyperlink{sec_vtkinfovis}{Visualization Toolkit Infovis Classes} \hypertarget{vtkwidgets_vtkxyplotwidget_Usage}{}\subsection{Usage}\label{vtkwidgets_vtkxyplotwidget_Usage}
vtk\-Table\-To\-Tree\-Filter is a filter for converting a vtk\-Table data structure into a vtk\-Tree datastructure. Currently, this will convert the table into a star, with each row of the table as a child of a new root node. The columns of the table are passed as node fields of the tree.

To create an instance of class vtk\-Table\-To\-Tree\-Filter, simply invoke its constructor as follows \begin{DoxyVerb}  obj = vtkTableToTreeFilter
\end{DoxyVerb}
 \hypertarget{vtkwidgets_vtkxyplotwidget_Methods}{}\subsection{Methods}\label{vtkwidgets_vtkxyplotwidget_Methods}
The class vtk\-Table\-To\-Tree\-Filter has several methods that can be used. They are listed below. Note that the documentation is translated automatically from the V\-T\-K sources, and may not be completely intelligible. When in doubt, consult the V\-T\-K website. In the methods listed below, {\ttfamily obj} is an instance of the vtk\-Table\-To\-Tree\-Filter class. 
\begin{DoxyItemize}
\item {\ttfamily string = obj.\-Get\-Class\-Name ()}  
\item {\ttfamily int = obj.\-Is\-A (string name)}  
\item {\ttfamily vtk\-Table\-To\-Tree\-Filter = obj.\-New\-Instance ()}  
\item {\ttfamily vtk\-Table\-To\-Tree\-Filter = obj.\-Safe\-Down\-Cast (vtk\-Object o)}  
\end{DoxyItemize}\hypertarget{vtkinfovis_vtkthresholdtable}{}\section{vtk\-Threshold\-Table}\label{vtkinfovis_vtkthresholdtable}
Section\-: \hyperlink{sec_vtkinfovis}{Visualization Toolkit Infovis Classes} \hypertarget{vtkwidgets_vtkxyplotwidget_Usage}{}\subsection{Usage}\label{vtkwidgets_vtkxyplotwidget_Usage}
vtk\-Threshold\-Table uses minimum and/or maximum values to threshold table rows based on the values in a particular column. The column to threshold is specified using Set\-Input\-Array\-To\-Process(0, ...).

To create an instance of class vtk\-Threshold\-Table, simply invoke its constructor as follows \begin{DoxyVerb}  obj = vtkThresholdTable
\end{DoxyVerb}
 \hypertarget{vtkwidgets_vtkxyplotwidget_Methods}{}\subsection{Methods}\label{vtkwidgets_vtkxyplotwidget_Methods}
The class vtk\-Threshold\-Table has several methods that can be used. They are listed below. Note that the documentation is translated automatically from the V\-T\-K sources, and may not be completely intelligible. When in doubt, consult the V\-T\-K website. In the methods listed below, {\ttfamily obj} is an instance of the vtk\-Threshold\-Table class. 
\begin{DoxyItemize}
\item {\ttfamily string = obj.\-Get\-Class\-Name ()}  
\item {\ttfamily int = obj.\-Is\-A (string name)}  
\item {\ttfamily vtk\-Threshold\-Table = obj.\-New\-Instance ()}  
\item {\ttfamily vtk\-Threshold\-Table = obj.\-Safe\-Down\-Cast (vtk\-Object o)}  
\item {\ttfamily obj.\-Set\-Mode (int )} -\/ The mode of the threshold filter. Options are\-: A\-C\-C\-E\-P\-T\-\_\-\-L\-E\-S\-S\-\_\-\-T\-H\-A\-N (0) accepts rows with values $<$ Max\-Value; A\-C\-C\-E\-P\-T\-\_\-\-G\-R\-E\-A\-T\-E\-R\-\_\-\-T\-H\-A\-N (1) accepts rows with values $>$ Min\-Value; A\-C\-C\-E\-P\-T\-\_\-\-B\-E\-T\-W\-E\-E\-N (2) accepts rows with values $>$ Min\-Value and $<$ Max\-Value; A\-C\-C\-E\-P\-T\-\_\-\-O\-U\-T\-S\-I\-D\-E (3) accepts rows with values $<$ Min\-Value or $>$ Max\-Value.  
\item {\ttfamily int = obj.\-Get\-Mode\-Min\-Value ()} -\/ The mode of the threshold filter. Options are\-: A\-C\-C\-E\-P\-T\-\_\-\-L\-E\-S\-S\-\_\-\-T\-H\-A\-N (0) accepts rows with values $<$ Max\-Value; A\-C\-C\-E\-P\-T\-\_\-\-G\-R\-E\-A\-T\-E\-R\-\_\-\-T\-H\-A\-N (1) accepts rows with values $>$ Min\-Value; A\-C\-C\-E\-P\-T\-\_\-\-B\-E\-T\-W\-E\-E\-N (2) accepts rows with values $>$ Min\-Value and $<$ Max\-Value; A\-C\-C\-E\-P\-T\-\_\-\-O\-U\-T\-S\-I\-D\-E (3) accepts rows with values $<$ Min\-Value or $>$ Max\-Value.  
\item {\ttfamily int = obj.\-Get\-Mode\-Max\-Value ()} -\/ The mode of the threshold filter. Options are\-: A\-C\-C\-E\-P\-T\-\_\-\-L\-E\-S\-S\-\_\-\-T\-H\-A\-N (0) accepts rows with values $<$ Max\-Value; A\-C\-C\-E\-P\-T\-\_\-\-G\-R\-E\-A\-T\-E\-R\-\_\-\-T\-H\-A\-N (1) accepts rows with values $>$ Min\-Value; A\-C\-C\-E\-P\-T\-\_\-\-B\-E\-T\-W\-E\-E\-N (2) accepts rows with values $>$ Min\-Value and $<$ Max\-Value; A\-C\-C\-E\-P\-T\-\_\-\-O\-U\-T\-S\-I\-D\-E (3) accepts rows with values $<$ Min\-Value or $>$ Max\-Value.  
\item {\ttfamily int = obj.\-Get\-Mode ()} -\/ The mode of the threshold filter. Options are\-: A\-C\-C\-E\-P\-T\-\_\-\-L\-E\-S\-S\-\_\-\-T\-H\-A\-N (0) accepts rows with values $<$ Max\-Value; A\-C\-C\-E\-P\-T\-\_\-\-G\-R\-E\-A\-T\-E\-R\-\_\-\-T\-H\-A\-N (1) accepts rows with values $>$ Min\-Value; A\-C\-C\-E\-P\-T\-\_\-\-B\-E\-T\-W\-E\-E\-N (2) accepts rows with values $>$ Min\-Value and $<$ Max\-Value; A\-C\-C\-E\-P\-T\-\_\-\-O\-U\-T\-S\-I\-D\-E (3) accepts rows with values $<$ Min\-Value or $>$ Max\-Value.  
\item {\ttfamily obj.\-Set\-Min\-Value (double v)} -\/ The maximum value for the threshold as a double.  
\item {\ttfamily obj.\-Set\-Max\-Value (double v)} -\/ Criterion is rows whose scalars are between lower and upper thresholds (inclusive of the end values).  
\item {\ttfamily obj.\-Threshold\-Between (double lower, double upper)}  
\end{DoxyItemize}\hypertarget{vtkinfovis_vtktimepointtostring}{}\section{vtk\-Time\-Point\-To\-String}\label{vtkinfovis_vtktimepointtostring}
Section\-: \hyperlink{sec_vtkinfovis}{Visualization Toolkit Infovis Classes} \hypertarget{vtkwidgets_vtkxyplotwidget_Usage}{}\subsection{Usage}\label{vtkwidgets_vtkxyplotwidget_Usage}
vtk\-Time\-Point\-To\-String is a filter for converting a timestamp array into string array using one of the formats defined in vtk\-Time\-Point\-Utility.\-h.

Use Set\-Input\-Array\-To\-Process to indicate the array to process. This array must be an unsigned 64-\/bit integer array for D\-A\-T\-E\-T\-I\-M\-E formats, and may be either an unsigned 32-\/bit or unsigned 64-\/bit array for D\-A\-T\-E and T\-I\-M\-E formats.

If the new array name is not specified, the array name will be the old name appended by \char`\"{} \mbox{[}to string\mbox{]}\char`\"{}.

To create an instance of class vtk\-Time\-Point\-To\-String, simply invoke its constructor as follows \begin{DoxyVerb}  obj = vtkTimePointToString
\end{DoxyVerb}
 \hypertarget{vtkwidgets_vtkxyplotwidget_Methods}{}\subsection{Methods}\label{vtkwidgets_vtkxyplotwidget_Methods}
The class vtk\-Time\-Point\-To\-String has several methods that can be used. They are listed below. Note that the documentation is translated automatically from the V\-T\-K sources, and may not be completely intelligible. When in doubt, consult the V\-T\-K website. In the methods listed below, {\ttfamily obj} is an instance of the vtk\-Time\-Point\-To\-String class. 
\begin{DoxyItemize}
\item {\ttfamily string = obj.\-Get\-Class\-Name ()}  
\item {\ttfamily int = obj.\-Is\-A (string name)}  
\item {\ttfamily vtk\-Time\-Point\-To\-String = obj.\-New\-Instance ()}  
\item {\ttfamily vtk\-Time\-Point\-To\-String = obj.\-Safe\-Down\-Cast (vtk\-Object o)}  
\item {\ttfamily obj.\-Set\-I\-S\-O8601\-Format (int )} -\/ The format to use when converting the timestamp to a string.  
\item {\ttfamily int = obj.\-Get\-I\-S\-O8601\-Format ()} -\/ The format to use when converting the timestamp to a string.  
\item {\ttfamily obj.\-Set\-Output\-Array\-Name (string )} -\/ The name of the output array. If this is not specified, the name will be the input array name with \char`\"{} \mbox{[}to string\mbox{]}\char`\"{} appended to it.  
\item {\ttfamily string = obj.\-Get\-Output\-Array\-Name ()} -\/ The name of the output array. If this is not specified, the name will be the input array name with \char`\"{} \mbox{[}to string\mbox{]}\char`\"{} appended to it.  
\end{DoxyItemize}\hypertarget{vtkinfovis_vtktransferattributes}{}\section{vtk\-Transfer\-Attributes}\label{vtkinfovis_vtktransferattributes}
Section\-: \hyperlink{sec_vtkinfovis}{Visualization Toolkit Infovis Classes} \hypertarget{vtkwidgets_vtkxyplotwidget_Usage}{}\subsection{Usage}\label{vtkwidgets_vtkxyplotwidget_Usage}
The filter requires both a vtk\-Graph and vtk\-Tree as input. The tree vertices must be a superset of the graph vertices. A common example is when the graph vertices correspond to the leaves of the tree, but the internal vertices of the tree represent groupings of graph vertices. The algorithm matches the vertices using the array \char`\"{}\-Pedigree\-Id\char`\"{}. The user may alternately set the Direct\-Mapping flag to indicate that the two structures must have directly corresponding offsets (i.\-e. node i in the graph must correspond to node i in the tree).

.S\-E\-C\-T\-I\-O\-N Thanks

To create an instance of class vtk\-Transfer\-Attributes, simply invoke its constructor as follows \begin{DoxyVerb}  obj = vtkTransferAttributes
\end{DoxyVerb}
 \hypertarget{vtkwidgets_vtkxyplotwidget_Methods}{}\subsection{Methods}\label{vtkwidgets_vtkxyplotwidget_Methods}
The class vtk\-Transfer\-Attributes has several methods that can be used. They are listed below. Note that the documentation is translated automatically from the V\-T\-K sources, and may not be completely intelligible. When in doubt, consult the V\-T\-K website. In the methods listed below, {\ttfamily obj} is an instance of the vtk\-Transfer\-Attributes class. 
\begin{DoxyItemize}
\item {\ttfamily string = obj.\-Get\-Class\-Name ()}  
\item {\ttfamily int = obj.\-Is\-A (string name)}  
\item {\ttfamily vtk\-Transfer\-Attributes = obj.\-New\-Instance ()}  
\item {\ttfamily vtk\-Transfer\-Attributes = obj.\-Safe\-Down\-Cast (vtk\-Object o)}  
\item {\ttfamily obj.\-Set\-Direct\-Mapping (bool )} -\/ If on, uses direct mapping from tree to graph vertices. If off, both the graph and tree must contain Pedigree\-Id arrays which are used to match graph and tree vertices. Default is off.  
\item {\ttfamily bool = obj.\-Get\-Direct\-Mapping ()} -\/ If on, uses direct mapping from tree to graph vertices. If off, both the graph and tree must contain Pedigree\-Id arrays which are used to match graph and tree vertices. Default is off.  
\item {\ttfamily obj.\-Direct\-Mapping\-On ()} -\/ If on, uses direct mapping from tree to graph vertices. If off, both the graph and tree must contain Pedigree\-Id arrays which are used to match graph and tree vertices. Default is off.  
\item {\ttfamily obj.\-Direct\-Mapping\-Off ()} -\/ If on, uses direct mapping from tree to graph vertices. If off, both the graph and tree must contain Pedigree\-Id arrays which are used to match graph and tree vertices. Default is off.  
\item {\ttfamily string = obj.\-Get\-Source\-Array\-Name ()} -\/ The field name to use for storing the source array.  
\item {\ttfamily obj.\-Set\-Source\-Array\-Name (string )} -\/ The field name to use for storing the source array.  
\item {\ttfamily string = obj.\-Get\-Target\-Array\-Name ()} -\/ The field name to use for storing the source array.  
\item {\ttfamily obj.\-Set\-Target\-Array\-Name (string )} -\/ The field name to use for storing the source array.  
\item {\ttfamily int = obj.\-Get\-Source\-Field\-Type ()} -\/ The source field type for accessing the source array. Valid values are those from enum vtk\-Data\-Object\-::\-Field\-Associations.  
\item {\ttfamily obj.\-Set\-Source\-Field\-Type (int )} -\/ The source field type for accessing the source array. Valid values are those from enum vtk\-Data\-Object\-::\-Field\-Associations.  
\item {\ttfamily int = obj.\-Get\-Target\-Field\-Type ()} -\/ The target field type for accessing the target array. Valid values are those from enum vtk\-Data\-Object\-::\-Field\-Associations.  
\item {\ttfamily obj.\-Set\-Target\-Field\-Type (int )} -\/ The target field type for accessing the target array. Valid values are those from enum vtk\-Data\-Object\-::\-Field\-Associations.  
\item {\ttfamily int = obj.\-Fill\-Input\-Port\-Information (int port, vtk\-Information info)} -\/ Set the input type of the algorithm to vtk\-Graph.  
\end{DoxyItemize}\hypertarget{vtkinfovis_vtktreefieldaggregator}{}\section{vtk\-Tree\-Field\-Aggregator}\label{vtkinfovis_vtktreefieldaggregator}
Section\-: \hyperlink{sec_vtkinfovis}{Visualization Toolkit Infovis Classes} \hypertarget{vtkwidgets_vtkxyplotwidget_Usage}{}\subsection{Usage}\label{vtkwidgets_vtkxyplotwidget_Usage}
vtk\-Tree\-Field\-Aggregator may be used to assign sizes to all the vertices in the tree, based on the sizes of the leaves. The size of a vertex will equal the sum of the sizes of the child vertices. If you have a data array with values for all leaves, you may specify that array, and the values will be filled in for interior tree vertices. If you do not yet have an array, you may tell the filter to create a new array, assuming that the size of each leaf vertex is 1. You may optionally set a flag to first take the log of all leaf values before aggregating.

To create an instance of class vtk\-Tree\-Field\-Aggregator, simply invoke its constructor as follows \begin{DoxyVerb}  obj = vtkTreeFieldAggregator
\end{DoxyVerb}
 \hypertarget{vtkwidgets_vtkxyplotwidget_Methods}{}\subsection{Methods}\label{vtkwidgets_vtkxyplotwidget_Methods}
The class vtk\-Tree\-Field\-Aggregator has several methods that can be used. They are listed below. Note that the documentation is translated automatically from the V\-T\-K sources, and may not be completely intelligible. When in doubt, consult the V\-T\-K website. In the methods listed below, {\ttfamily obj} is an instance of the vtk\-Tree\-Field\-Aggregator class. 
\begin{DoxyItemize}
\item {\ttfamily string = obj.\-Get\-Class\-Name ()}  
\item {\ttfamily int = obj.\-Is\-A (string name)}  
\item {\ttfamily vtk\-Tree\-Field\-Aggregator = obj.\-New\-Instance ()}  
\item {\ttfamily vtk\-Tree\-Field\-Aggregator = obj.\-Safe\-Down\-Cast (vtk\-Object o)}  
\item {\ttfamily string = obj.\-Get\-Field ()} -\/ The field to aggregate. If this is a string array, the entries are converted to double. T\-O\-D\-O\-: Remove this field and use the Array\-To\-Process in vtk\-Algorithm.  
\item {\ttfamily obj.\-Set\-Field (string )} -\/ The field to aggregate. If this is a string array, the entries are converted to double. T\-O\-D\-O\-: Remove this field and use the Array\-To\-Process in vtk\-Algorithm.  
\item {\ttfamily double = obj.\-Get\-Min\-Value ()} -\/ If the value of the vertex is less than Min\-Value then consider it's value to be min\-Val.  
\item {\ttfamily obj.\-Set\-Min\-Value (double )} -\/ If the value of the vertex is less than Min\-Value then consider it's value to be min\-Val.  
\item {\ttfamily obj.\-Set\-Leaf\-Vertex\-Unit\-Size (bool )} -\/ If set, the algorithm will assume a size of 1 for each leaf vertex.  
\item {\ttfamily bool = obj.\-Get\-Leaf\-Vertex\-Unit\-Size ()} -\/ If set, the algorithm will assume a size of 1 for each leaf vertex.  
\item {\ttfamily obj.\-Leaf\-Vertex\-Unit\-Size\-On ()} -\/ If set, the algorithm will assume a size of 1 for each leaf vertex.  
\item {\ttfamily obj.\-Leaf\-Vertex\-Unit\-Size\-Off ()} -\/ If set, the algorithm will assume a size of 1 for each leaf vertex.  
\item {\ttfamily obj.\-Set\-Log\-Scale (bool )} -\/ If set, the leaf values in the tree will be logarithmically scaled (base 10).  
\item {\ttfamily bool = obj.\-Get\-Log\-Scale ()} -\/ If set, the leaf values in the tree will be logarithmically scaled (base 10).  
\item {\ttfamily obj.\-Log\-Scale\-On ()} -\/ If set, the leaf values in the tree will be logarithmically scaled (base 10).  
\item {\ttfamily obj.\-Log\-Scale\-Off ()} -\/ If set, the leaf values in the tree will be logarithmically scaled (base 10).  
\end{DoxyItemize}\hypertarget{vtkinfovis_vtktreelayoutstrategy}{}\section{vtk\-Tree\-Layout\-Strategy}\label{vtkinfovis_vtktreelayoutstrategy}
Section\-: \hyperlink{sec_vtkinfovis}{Visualization Toolkit Infovis Classes} \hypertarget{vtkwidgets_vtkxyplotwidget_Usage}{}\subsection{Usage}\label{vtkwidgets_vtkxyplotwidget_Usage}
Assigns points to the nodes of a tree in either a standard or radial layout. The standard layout places each level on a horizontal line, while the radial layout places each level on a concentric circle. You may specify the sweep angle of the tree which constrains the tree to be contained within a wedge. Also, you may indicate the log scale of the tree, which diminishes the length of arcs at lower levels of the tree. Values near zero give a large proportion of the space to the tree levels near the root, while values near one give nearly equal proportions of space to all tree levels.

The user may also specify an array to use to indicate the distance from the root, either vertically (for standard layout) or radially (for radial layout). You specify this with Set\-Distance\-Array\-Name().

To create an instance of class vtk\-Tree\-Layout\-Strategy, simply invoke its constructor as follows \begin{DoxyVerb}  obj = vtkTreeLayoutStrategy
\end{DoxyVerb}
 \hypertarget{vtkwidgets_vtkxyplotwidget_Methods}{}\subsection{Methods}\label{vtkwidgets_vtkxyplotwidget_Methods}
The class vtk\-Tree\-Layout\-Strategy has several methods that can be used. They are listed below. Note that the documentation is translated automatically from the V\-T\-K sources, and may not be completely intelligible. When in doubt, consult the V\-T\-K website. In the methods listed below, {\ttfamily obj} is an instance of the vtk\-Tree\-Layout\-Strategy class. 
\begin{DoxyItemize}
\item {\ttfamily string = obj.\-Get\-Class\-Name ()}  
\item {\ttfamily int = obj.\-Is\-A (string name)}  
\item {\ttfamily vtk\-Tree\-Layout\-Strategy = obj.\-New\-Instance ()}  
\item {\ttfamily vtk\-Tree\-Layout\-Strategy = obj.\-Safe\-Down\-Cast (vtk\-Object o)}  
\item {\ttfamily obj.\-Layout ()} -\/ Perform the tree layout.  
\item {\ttfamily obj.\-Set\-Angle (double )} -\/ The sweep angle of the tree. For a standard tree layout, this should be between 0 and 180. For a radial tree layout, this can be between 0 and 360.  
\item {\ttfamily double = obj.\-Get\-Angle\-Min\-Value ()} -\/ The sweep angle of the tree. For a standard tree layout, this should be between 0 and 180. For a radial tree layout, this can be between 0 and 360.  
\item {\ttfamily double = obj.\-Get\-Angle\-Max\-Value ()} -\/ The sweep angle of the tree. For a standard tree layout, this should be between 0 and 180. For a radial tree layout, this can be between 0 and 360.  
\item {\ttfamily double = obj.\-Get\-Angle ()} -\/ The sweep angle of the tree. For a standard tree layout, this should be between 0 and 180. For a radial tree layout, this can be between 0 and 360.  
\item {\ttfamily obj.\-Set\-Radial (bool )} -\/ If set, the tree is laid out with levels on concentric circles around the root. If unset (default), the tree is laid out with levels on horizontal lines.  
\item {\ttfamily bool = obj.\-Get\-Radial ()} -\/ If set, the tree is laid out with levels on concentric circles around the root. If unset (default), the tree is laid out with levels on horizontal lines.  
\item {\ttfamily obj.\-Radial\-On ()} -\/ If set, the tree is laid out with levels on concentric circles around the root. If unset (default), the tree is laid out with levels on horizontal lines.  
\item {\ttfamily obj.\-Radial\-Off ()} -\/ If set, the tree is laid out with levels on concentric circles around the root. If unset (default), the tree is laid out with levels on horizontal lines.  
\item {\ttfamily obj.\-Set\-Log\-Spacing\-Value (double )} -\/ The spacing of tree levels. Levels near zero give more space to levels near the root, while levels near one (the default) create evenly-\/spaced levels. Levels above one give more space to levels near the leaves.  
\item {\ttfamily double = obj.\-Get\-Log\-Spacing\-Value ()} -\/ The spacing of tree levels. Levels near zero give more space to levels near the root, while levels near one (the default) create evenly-\/spaced levels. Levels above one give more space to levels near the leaves.  
\item {\ttfamily obj.\-Set\-Leaf\-Spacing (double )} -\/ The spacing of leaves. Levels near one evenly space leaves with no gaps between subtrees. Levels near zero creates large gaps between subtrees.  
\item {\ttfamily double = obj.\-Get\-Leaf\-Spacing\-Min\-Value ()} -\/ The spacing of leaves. Levels near one evenly space leaves with no gaps between subtrees. Levels near zero creates large gaps between subtrees.  
\item {\ttfamily double = obj.\-Get\-Leaf\-Spacing\-Max\-Value ()} -\/ The spacing of leaves. Levels near one evenly space leaves with no gaps between subtrees. Levels near zero creates large gaps between subtrees.  
\item {\ttfamily double = obj.\-Get\-Leaf\-Spacing ()} -\/ The spacing of leaves. Levels near one evenly space leaves with no gaps between subtrees. Levels near zero creates large gaps between subtrees.  
\item {\ttfamily obj.\-Set\-Distance\-Array\-Name (string )} -\/ Get/\-Set the array to use to determine the distance from the root.  
\item {\ttfamily string = obj.\-Get\-Distance\-Array\-Name ()} -\/ Get/\-Set the array to use to determine the distance from the root.  
\end{DoxyItemize}\hypertarget{vtkinfovis_vtktreelevelsfilter}{}\section{vtk\-Tree\-Levels\-Filter}\label{vtkinfovis_vtktreelevelsfilter}
Section\-: \hyperlink{sec_vtkinfovis}{Visualization Toolkit Infovis Classes} \hypertarget{vtkwidgets_vtkxyplotwidget_Usage}{}\subsection{Usage}\label{vtkwidgets_vtkxyplotwidget_Usage}
The filter currently add two arrays to the incoming vtk\-Tree datastructure. 1) \char`\"{}levels\char`\"{} this is the distance from the root of the vertex. Root = 0 and you add 1 for each level down from the root 2) \char`\"{}leaf\char`\"{} this array simply indicates whether the vertex is a leaf or not

.S\-E\-C\-T\-I\-O\-N Thanks Thanks to Brian Wylie from Sandia National Laboratories for creating this class.

To create an instance of class vtk\-Tree\-Levels\-Filter, simply invoke its constructor as follows \begin{DoxyVerb}  obj = vtkTreeLevelsFilter
\end{DoxyVerb}
 \hypertarget{vtkwidgets_vtkxyplotwidget_Methods}{}\subsection{Methods}\label{vtkwidgets_vtkxyplotwidget_Methods}
The class vtk\-Tree\-Levels\-Filter has several methods that can be used. They are listed below. Note that the documentation is translated automatically from the V\-T\-K sources, and may not be completely intelligible. When in doubt, consult the V\-T\-K website. In the methods listed below, {\ttfamily obj} is an instance of the vtk\-Tree\-Levels\-Filter class. 
\begin{DoxyItemize}
\item {\ttfamily string = obj.\-Get\-Class\-Name ()}  
\item {\ttfamily int = obj.\-Is\-A (string name)}  
\item {\ttfamily vtk\-Tree\-Levels\-Filter = obj.\-New\-Instance ()}  
\item {\ttfamily vtk\-Tree\-Levels\-Filter = obj.\-Safe\-Down\-Cast (vtk\-Object o)}  
\end{DoxyItemize}\hypertarget{vtkinfovis_vtktreemaplayout}{}\section{vtk\-Tree\-Map\-Layout}\label{vtkinfovis_vtktreemaplayout}
Section\-: \hyperlink{sec_vtkinfovis}{Visualization Toolkit Infovis Classes} \hypertarget{vtkwidgets_vtkxyplotwidget_Usage}{}\subsection{Usage}\label{vtkwidgets_vtkxyplotwidget_Usage}
vtk\-Tree\-Map\-Layout assigns rectangular regions to each vertex in the tree, creating a tree map. The data is added as a data array with four components per tuple representing the location and size of the rectangle using the format (Xmin, Xmax, Ymin, Ymax).

This algorithm relies on a helper class to perform the actual layout. This helper class is a subclass of vtk\-Tree\-Map\-Layout\-Strategy.

.S\-E\-C\-T\-I\-O\-N Thanks Thanks to Brian Wylie and Ken Moreland from Sandia National Laboratories for help developing this class.

Tree map concept comes from\-: Shneiderman, B. 1992. Tree visualization with tree-\/maps\-: 2-\/d space-\/filling approach. A\-C\-M Trans. Graph. 11, 1 (Jan. 1992), 92-\/99.

To create an instance of class vtk\-Tree\-Map\-Layout, simply invoke its constructor as follows \begin{DoxyVerb}  obj = vtkTreeMapLayout
\end{DoxyVerb}
 \hypertarget{vtkwidgets_vtkxyplotwidget_Methods}{}\subsection{Methods}\label{vtkwidgets_vtkxyplotwidget_Methods}
The class vtk\-Tree\-Map\-Layout has several methods that can be used. They are listed below. Note that the documentation is translated automatically from the V\-T\-K sources, and may not be completely intelligible. When in doubt, consult the V\-T\-K website. In the methods listed below, {\ttfamily obj} is an instance of the vtk\-Tree\-Map\-Layout class. 
\begin{DoxyItemize}
\item {\ttfamily string = obj.\-Get\-Class\-Name ()}  
\item {\ttfamily int = obj.\-Is\-A (string name)}  
\item {\ttfamily vtk\-Tree\-Map\-Layout = obj.\-New\-Instance ()}  
\item {\ttfamily vtk\-Tree\-Map\-Layout = obj.\-Safe\-Down\-Cast (vtk\-Object o)}  
\item {\ttfamily string = obj.\-Get\-Rectangles\-Field\-Name ()} -\/ The field name to use for storing the rectangles for each vertex. The rectangles are stored in a quadruple float array (min\-X, max\-X, min\-Y, max\-Y).  
\item {\ttfamily obj.\-Set\-Rectangles\-Field\-Name (string )} -\/ The field name to use for storing the rectangles for each vertex. The rectangles are stored in a quadruple float array (min\-X, max\-X, min\-Y, max\-Y).  
\item {\ttfamily obj.\-Set\-Size\-Array\-Name (string name)} -\/ The strategy to use when laying out the tree map.  
\item {\ttfamily vtk\-Tree\-Map\-Layout\-Strategy = obj.\-Get\-Layout\-Strategy ()} -\/ The strategy to use when laying out the tree map.  
\item {\ttfamily obj.\-Set\-Layout\-Strategy (vtk\-Tree\-Map\-Layout\-Strategy strategy)} -\/ The strategy to use when laying out the tree map.  
\item {\ttfamily vtk\-Id\-Type = obj.\-Find\-Vertex (float pnt\mbox{[}2\mbox{]}, float binfo)} -\/ Returns the vertex id that contains pnt (or -\/1 if no one contains it)  
\item {\ttfamily obj.\-Get\-Bounding\-Box (vtk\-Id\-Type id, float binfo)} -\/ Return the min and max 2\-D points of the vertex's bounding box  
\item {\ttfamily long = obj.\-Get\-M\-Time ()} -\/ Get the modification time of the layout algorithm.  
\end{DoxyItemize}\hypertarget{vtkinfovis_vtktreemaplayoutstrategy}{}\section{vtk\-Tree\-Map\-Layout\-Strategy}\label{vtkinfovis_vtktreemaplayoutstrategy}
Section\-: \hyperlink{sec_vtkinfovis}{Visualization Toolkit Infovis Classes} \hypertarget{vtkwidgets_vtkxyplotwidget_Usage}{}\subsection{Usage}\label{vtkwidgets_vtkxyplotwidget_Usage}
All subclasses of this class perform a tree map layout on a tree. This involves assigning a rectangular region to each vertex in the tree, and placing that information in a data array with four components per tuple representing (Xmin, Xmax, Ymin, Ymax).

Instances of subclasses of this class may be assigned as the layout strategy to vtk\-Tree\-Map\-Layout

.S\-E\-C\-T\-I\-O\-N Thanks Thanks to Brian Wylie and Ken Moreland from Sandia National Laboratories for help developing this class.

To create an instance of class vtk\-Tree\-Map\-Layout\-Strategy, simply invoke its constructor as follows \begin{DoxyVerb}  obj = vtkTreeMapLayoutStrategy
\end{DoxyVerb}
 \hypertarget{vtkwidgets_vtkxyplotwidget_Methods}{}\subsection{Methods}\label{vtkwidgets_vtkxyplotwidget_Methods}
The class vtk\-Tree\-Map\-Layout\-Strategy has several methods that can be used. They are listed below. Note that the documentation is translated automatically from the V\-T\-K sources, and may not be completely intelligible. When in doubt, consult the V\-T\-K website. In the methods listed below, {\ttfamily obj} is an instance of the vtk\-Tree\-Map\-Layout\-Strategy class. 
\begin{DoxyItemize}
\item {\ttfamily string = obj.\-Get\-Class\-Name ()}  
\item {\ttfamily int = obj.\-Is\-A (string name)}  
\item {\ttfamily vtk\-Tree\-Map\-Layout\-Strategy = obj.\-New\-Instance ()}  
\item {\ttfamily vtk\-Tree\-Map\-Layout\-Strategy = obj.\-Safe\-Down\-Cast (vtk\-Object o)}  
\item {\ttfamily vtk\-Id\-Type = obj.\-Find\-Vertex (vtk\-Tree tree, vtk\-Data\-Array area\-Array, float pnt\mbox{[}2\mbox{]})} -\/ Find the vertex at a certain location, or -\/1 if none found.  
\end{DoxyItemize}\hypertarget{vtkinfovis_vtktreemaptopolydata}{}\section{vtk\-Tree\-Map\-To\-Poly\-Data}\label{vtkinfovis_vtktreemaptopolydata}
Section\-: \hyperlink{sec_vtkinfovis}{Visualization Toolkit Infovis Classes} \hypertarget{vtkwidgets_vtkxyplotwidget_Usage}{}\subsection{Usage}\label{vtkwidgets_vtkxyplotwidget_Usage}
This algorithm requires that the vtk\-Tree\-Map\-Layout filter has already applied to the data in order to create the quadruple array (min x, max x, min y, max y) of bounds for each vertex of the tree.

To create an instance of class vtk\-Tree\-Map\-To\-Poly\-Data, simply invoke its constructor as follows \begin{DoxyVerb}  obj = vtkTreeMapToPolyData
\end{DoxyVerb}
 \hypertarget{vtkwidgets_vtkxyplotwidget_Methods}{}\subsection{Methods}\label{vtkwidgets_vtkxyplotwidget_Methods}
The class vtk\-Tree\-Map\-To\-Poly\-Data has several methods that can be used. They are listed below. Note that the documentation is translated automatically from the V\-T\-K sources, and may not be completely intelligible. When in doubt, consult the V\-T\-K website. In the methods listed below, {\ttfamily obj} is an instance of the vtk\-Tree\-Map\-To\-Poly\-Data class. 
\begin{DoxyItemize}
\item {\ttfamily string = obj.\-Get\-Class\-Name ()}  
\item {\ttfamily int = obj.\-Is\-A (string name)}  
\item {\ttfamily vtk\-Tree\-Map\-To\-Poly\-Data = obj.\-New\-Instance ()}  
\item {\ttfamily vtk\-Tree\-Map\-To\-Poly\-Data = obj.\-Safe\-Down\-Cast (vtk\-Object o)}  
\item {\ttfamily obj.\-Set\-Rectangles\-Array\-Name (string name)} -\/ The field containing the level of each tree node. This can be added using vtk\-Tree\-Levels\-Filter before this filter. If this is not present, the filter simply calls tree-\/$>$Get\-Level(v) for each vertex, which will produce the same result, but may not be as efficient.  
\item {\ttfamily obj.\-Set\-Level\-Array\-Name (string name)} -\/ The spacing along the z-\/axis between tree map levels.  
\item {\ttfamily double = obj.\-Get\-Level\-Delta\-Z ()} -\/ The spacing along the z-\/axis between tree map levels.  
\item {\ttfamily obj.\-Set\-Level\-Delta\-Z (double )} -\/ The spacing along the z-\/axis between tree map levels.  
\item {\ttfamily bool = obj.\-Get\-Add\-Normals ()} -\/ The spacing along the z-\/axis between tree map levels.  
\item {\ttfamily obj.\-Set\-Add\-Normals (bool )} -\/ The spacing along the z-\/axis between tree map levels.  
\item {\ttfamily int = obj.\-Fill\-Input\-Port\-Information (int port, vtk\-Information info)}  
\end{DoxyItemize}\hypertarget{vtkinfovis_vtktreeorbitlayoutstrategy}{}\section{vtk\-Tree\-Orbit\-Layout\-Strategy}\label{vtkinfovis_vtktreeorbitlayoutstrategy}
Section\-: \hyperlink{sec_vtkinfovis}{Visualization Toolkit Infovis Classes} \hypertarget{vtkwidgets_vtkxyplotwidget_Usage}{}\subsection{Usage}\label{vtkwidgets_vtkxyplotwidget_Usage}
Assigns points to the nodes of a tree to an orbital layout. Each parent is orbited by its children, recursively.

To create an instance of class vtk\-Tree\-Orbit\-Layout\-Strategy, simply invoke its constructor as follows \begin{DoxyVerb}  obj = vtkTreeOrbitLayoutStrategy
\end{DoxyVerb}
 \hypertarget{vtkwidgets_vtkxyplotwidget_Methods}{}\subsection{Methods}\label{vtkwidgets_vtkxyplotwidget_Methods}
The class vtk\-Tree\-Orbit\-Layout\-Strategy has several methods that can be used. They are listed below. Note that the documentation is translated automatically from the V\-T\-K sources, and may not be completely intelligible. When in doubt, consult the V\-T\-K website. In the methods listed below, {\ttfamily obj} is an instance of the vtk\-Tree\-Orbit\-Layout\-Strategy class. 
\begin{DoxyItemize}
\item {\ttfamily string = obj.\-Get\-Class\-Name ()}  
\item {\ttfamily int = obj.\-Is\-A (string name)}  
\item {\ttfamily vtk\-Tree\-Orbit\-Layout\-Strategy = obj.\-New\-Instance ()}  
\item {\ttfamily vtk\-Tree\-Orbit\-Layout\-Strategy = obj.\-Safe\-Down\-Cast (vtk\-Object o)}  
\item {\ttfamily obj.\-Layout ()} -\/ Perform the orbital layout.  
\item {\ttfamily obj.\-Set\-Log\-Spacing\-Value (double )} -\/ The spacing of orbital levels. Levels near zero give more space to levels near the root, while levels near one (the default) create evenly-\/spaced levels. Levels above one give more space to levels near the leaves.  
\item {\ttfamily double = obj.\-Get\-Log\-Spacing\-Value ()} -\/ The spacing of orbital levels. Levels near zero give more space to levels near the root, while levels near one (the default) create evenly-\/spaced levels. Levels above one give more space to levels near the leaves.  
\item {\ttfamily obj.\-Set\-Leaf\-Spacing (double )} -\/ The spacing of leaves. Levels near one evenly space leaves with no gaps between subtrees. Levels near zero creates large gaps between subtrees.  
\item {\ttfamily double = obj.\-Get\-Leaf\-Spacing\-Min\-Value ()} -\/ The spacing of leaves. Levels near one evenly space leaves with no gaps between subtrees. Levels near zero creates large gaps between subtrees.  
\item {\ttfamily double = obj.\-Get\-Leaf\-Spacing\-Max\-Value ()} -\/ The spacing of leaves. Levels near one evenly space leaves with no gaps between subtrees. Levels near zero creates large gaps between subtrees.  
\item {\ttfamily double = obj.\-Get\-Leaf\-Spacing ()} -\/ The spacing of leaves. Levels near one evenly space leaves with no gaps between subtrees. Levels near zero creates large gaps between subtrees.  
\item {\ttfamily obj.\-Set\-Child\-Radius\-Factor (double )} -\/ This is a magic number right now. Controls the radius of the child layout, all of this should be fixed at some point with a more logical layout. Defaults to .5 \-:)  
\item {\ttfamily double = obj.\-Get\-Child\-Radius\-Factor ()} -\/ This is a magic number right now. Controls the radius of the child layout, all of this should be fixed at some point with a more logical layout. Defaults to .5 \-:)  
\end{DoxyItemize}\hypertarget{vtkinfovis_vtktreeringtopolydata}{}\section{vtk\-Tree\-Ring\-To\-Poly\-Data}\label{vtkinfovis_vtktreeringtopolydata}
Section\-: \hyperlink{sec_vtkinfovis}{Visualization Toolkit Infovis Classes} \hypertarget{vtkwidgets_vtkxyplotwidget_Usage}{}\subsection{Usage}\label{vtkwidgets_vtkxyplotwidget_Usage}
This algorithm requires that the vtk\-Tree\-Ring\-Layout filter has already been applied to the data in order to create the quadruple array (start angle, end angle, inner radius, outer radius) of bounds for each vertex of the tree.

To create an instance of class vtk\-Tree\-Ring\-To\-Poly\-Data, simply invoke its constructor as follows \begin{DoxyVerb}  obj = vtkTreeRingToPolyData
\end{DoxyVerb}
 \hypertarget{vtkwidgets_vtkxyplotwidget_Methods}{}\subsection{Methods}\label{vtkwidgets_vtkxyplotwidget_Methods}
The class vtk\-Tree\-Ring\-To\-Poly\-Data has several methods that can be used. They are listed below. Note that the documentation is translated automatically from the V\-T\-K sources, and may not be completely intelligible. When in doubt, consult the V\-T\-K website. In the methods listed below, {\ttfamily obj} is an instance of the vtk\-Tree\-Ring\-To\-Poly\-Data class. 
\begin{DoxyItemize}
\item {\ttfamily string = obj.\-Get\-Class\-Name ()}  
\item {\ttfamily int = obj.\-Is\-A (string name)}  
\item {\ttfamily vtk\-Tree\-Ring\-To\-Poly\-Data = obj.\-New\-Instance ()}  
\item {\ttfamily vtk\-Tree\-Ring\-To\-Poly\-Data = obj.\-Safe\-Down\-Cast (vtk\-Object o)}  
\item {\ttfamily obj.\-Set\-Sectors\-Array\-Name (string name)} -\/ Define a shrink percentage for each of the sectors.  
\item {\ttfamily obj.\-Set\-Shrink\-Percentage (double )} -\/ Define a shrink percentage for each of the sectors.  
\item {\ttfamily double = obj.\-Get\-Shrink\-Percentage ()} -\/ Define a shrink percentage for each of the sectors.  
\item {\ttfamily int = obj.\-Fill\-Input\-Port\-Information (int port, vtk\-Information info)}  
\end{DoxyItemize}\hypertarget{vtkinfovis_vtktulipreader}{}\section{vtk\-Tulip\-Reader}\label{vtkinfovis_vtktulipreader}
Section\-: \hyperlink{sec_vtkinfovis}{Visualization Toolkit Infovis Classes} \hypertarget{vtkwidgets_vtkxyplotwidget_Usage}{}\subsection{Usage}\label{vtkwidgets_vtkxyplotwidget_Usage}
vtk\-Tulip\-Reader reads in files in the Tulip format. Definition of the Tulip file format can be found online at\-: \href{http://tulip.labri.fr/tlpformat.php}{\tt http\-://tulip.\-labri.\-fr/tlpformat.\-php} An example is the following {\ttfamily  (nodes 0 1 2 3 4 5 6 7 8 9) (edge 0 0 1) (edge 1 1 2) (edge 2 2 3) (edge 3 3 4) (edge 4 4 5) (edge 5 5 6) (edge 6 6 7) (edge 7 7 8) (edge 8 8 9) (edge 9 9 0) (edge 10 0 5) (edge 11 2 7) (edge 12 4 9) } where \char`\"{}nodes\char`\"{} defines all the nodes ids in the graph, and \char`\"{}edge\char`\"{} is a triple of edge id, source vertex id, and target vertex id. The graph is read in as undirected graph. N\-O\-T\-E\-: This currently only supports reading connectivity information. Display information is discarded.

To create an instance of class vtk\-Tulip\-Reader, simply invoke its constructor as follows \begin{DoxyVerb}  obj = vtkTulipReader
\end{DoxyVerb}
 \hypertarget{vtkwidgets_vtkxyplotwidget_Methods}{}\subsection{Methods}\label{vtkwidgets_vtkxyplotwidget_Methods}
The class vtk\-Tulip\-Reader has several methods that can be used. They are listed below. Note that the documentation is translated automatically from the V\-T\-K sources, and may not be completely intelligible. When in doubt, consult the V\-T\-K website. In the methods listed below, {\ttfamily obj} is an instance of the vtk\-Tulip\-Reader class. 
\begin{DoxyItemize}
\item {\ttfamily string = obj.\-Get\-Class\-Name ()}  
\item {\ttfamily int = obj.\-Is\-A (string name)}  
\item {\ttfamily vtk\-Tulip\-Reader = obj.\-New\-Instance ()}  
\item {\ttfamily vtk\-Tulip\-Reader = obj.\-Safe\-Down\-Cast (vtk\-Object o)}  
\item {\ttfamily string = obj.\-Get\-File\-Name ()} -\/ The Tulip file name.  
\item {\ttfamily obj.\-Set\-File\-Name (string )} -\/ The Tulip file name.  
\end{DoxyItemize}\hypertarget{vtkinfovis_vtkunivariatestatisticsalgorithm}{}\section{vtk\-Univariate\-Statistics\-Algorithm}\label{vtkinfovis_vtkunivariatestatisticsalgorithm}
Section\-: \hyperlink{sec_vtkinfovis}{Visualization Toolkit Infovis Classes} \hypertarget{vtkwidgets_vtkxyplotwidget_Usage}{}\subsection{Usage}\label{vtkwidgets_vtkxyplotwidget_Usage}
This class specializes statistics algorithms to the univariate case, where a number of columns of interest can be selected in the input data set. This is done by the means of the following functions\-:

Reset\-Columns() -\/ reset the list of columns of interest. Add/\-Remove\-Colum( nam\-Col ) -\/ try to add/remove column with name nam\-Col to/from the list. Set\-Column\-Status ( nam\-Col, status ) -\/ mostly for U\-I wrapping purposes, try to add/remove (depending on status) nam\-Col from the list of columns of interest. The verb \char`\"{}try\char`\"{} is used in the sense that neither attempting to repeat an existing entry nor to remove a non-\/existent entry will work.

.S\-E\-C\-T\-I\-O\-N Thanks Thanks to Philippe Pebay and David Thompson from Sandia National Laboratories for implementing this class.

To create an instance of class vtk\-Univariate\-Statistics\-Algorithm, simply invoke its constructor as follows \begin{DoxyVerb}  obj = vtkUnivariateStatisticsAlgorithm
\end{DoxyVerb}
 \hypertarget{vtkwidgets_vtkxyplotwidget_Methods}{}\subsection{Methods}\label{vtkwidgets_vtkxyplotwidget_Methods}
The class vtk\-Univariate\-Statistics\-Algorithm has several methods that can be used. They are listed below. Note that the documentation is translated automatically from the V\-T\-K sources, and may not be completely intelligible. When in doubt, consult the V\-T\-K website. In the methods listed below, {\ttfamily obj} is an instance of the vtk\-Univariate\-Statistics\-Algorithm class. 
\begin{DoxyItemize}
\item {\ttfamily string = obj.\-Get\-Class\-Name ()}  
\item {\ttfamily int = obj.\-Is\-A (string name)}  
\item {\ttfamily vtk\-Univariate\-Statistics\-Algorithm = obj.\-New\-Instance ()}  
\item {\ttfamily vtk\-Univariate\-Statistics\-Algorithm = obj.\-Safe\-Down\-Cast (vtk\-Object o)}  
\item {\ttfamily obj.\-Add\-Column (string nam\-Col)} -\/ Convenience method to create a request with a single column name {\ttfamily nam\-Col} in a single call; this is the preferred method to select columns, ensuring selection consistency (a single column per request). Warning\-: no name checking is performed on {\ttfamily nam\-Col}; it is the user's responsibility to use valid column names.  
\item {\ttfamily int = obj.\-Request\-Selected\-Columns ()} -\/ Use the current column status values to produce a new request for statistics to be produced when Request\-Data() is called. Unlike the superclass implementation, this version adds a new request for each selected column instead of a single request containing all the columns.  
\end{DoxyItemize}\hypertarget{vtkinfovis_vtkvertexdegree}{}\section{vtk\-Vertex\-Degree}\label{vtkinfovis_vtkvertexdegree}
Section\-: \hyperlink{sec_vtkinfovis}{Visualization Toolkit Infovis Classes} \hypertarget{vtkwidgets_vtkxyplotwidget_Usage}{}\subsection{Usage}\label{vtkwidgets_vtkxyplotwidget_Usage}
Adds an attribute array with the degree of each vertex. By default the name of the array will be \char`\"{}\-Vertex\-Degree\char`\"{}, but that can be changed by calling Set\-Output\-Array\-Name(\char`\"{}foo\char`\"{});

To create an instance of class vtk\-Vertex\-Degree, simply invoke its constructor as follows \begin{DoxyVerb}  obj = vtkVertexDegree
\end{DoxyVerb}
 \hypertarget{vtkwidgets_vtkxyplotwidget_Methods}{}\subsection{Methods}\label{vtkwidgets_vtkxyplotwidget_Methods}
The class vtk\-Vertex\-Degree has several methods that can be used. They are listed below. Note that the documentation is translated automatically from the V\-T\-K sources, and may not be completely intelligible. When in doubt, consult the V\-T\-K website. In the methods listed below, {\ttfamily obj} is an instance of the vtk\-Vertex\-Degree class. 
\begin{DoxyItemize}
\item {\ttfamily string = obj.\-Get\-Class\-Name ()}  
\item {\ttfamily int = obj.\-Is\-A (string name)}  
\item {\ttfamily vtk\-Vertex\-Degree = obj.\-New\-Instance ()}  
\item {\ttfamily vtk\-Vertex\-Degree = obj.\-Safe\-Down\-Cast (vtk\-Object o)}  
\item {\ttfamily obj.\-Set\-Output\-Array\-Name (string )} -\/ Set the output array name. If no output array name is set then the name 'Vertex\-Degree' is used.  
\end{DoxyItemize}\hypertarget{vtkinfovis_vtkxmltreereader}{}\section{vtk\-X\-M\-L\-Tree\-Reader}\label{vtkinfovis_vtkxmltreereader}
Section\-: \hyperlink{sec_vtkinfovis}{Visualization Toolkit Infovis Classes} \hypertarget{vtkwidgets_vtkxyplotwidget_Usage}{}\subsection{Usage}\label{vtkwidgets_vtkxyplotwidget_Usage}
vtk\-X\-M\-L\-Tree\-Reader parses an X\-M\-L file and uses the nesting structure of the X\-M\-L tags to generate a tree. Node attributes are assigned to node arrays, and the special arrays .tagname and .chardata contain the tag type and the text internal to the tag, respectively. The arrays are of type vtk\-String\-Array. There is an array for each attribute type in the X\-M\-L file, even if it appears in only one tag. If an attribute is missing from a tag, its value is the empty string.

If Mask\-Arrays is on (the default is off), the filter will additionally make bit arrays whose names are prepended with \char`\"{}.\-valid.\char`\"{} which are 1 if the element contains that attribute, and 0 otherwise.

For example, the X\-M\-L file containing the text\-: 
\begin{DoxyPre}
 <node name="jeff" age="26">
   this is text in jeff's node
   <node name="joe">
     <node name="al" initials="amb" other="something"/>
     <node name="dave" age="30"/>
   </node>
   <node name="lisa">this is text in lisa's node</node>
   <node name="darlene" age="29"/>
 </node>
 \end{DoxyPre}


would be parsed into a tree with the following node I\-Ds and structure\-:


\begin{DoxyPre}
 0 (jeff) - children: 1 (joe), 4 (lisa), 5 (darlene)
 1 (joe)  - children: 2 (al), 3 (dave)
 2 (al)
 3 (dave)
 4 (lisa)
 5 (darlene)
 \end{DoxyPre}


and the node data arrays would be as follows\-:


\begin{DoxyPre}
 name      initials  other     age       .tagname  .chardata
 ------------------------------------------------------------------------------------------------
 jeff      (empty)   (empty)   26         node     "  this is text in jeff's node\(\backslash\)n  \(\backslash\)n  \(\backslash\)n  \(\backslash\)n"
 joe       (empty)   (empty)   (empty)    node     "\(\backslash\)n    \(\backslash\)n    \(\backslash\)n  "
 al        amb       something (empty)    node     (empty)
 dave      (empty)   (empty)   30         node     (empty)
 lisa      (empty)   (empty)   (empty)    node     "this is text in lisa's node"
 darlene   (empty)   (empty)   29         node     (empty)
 \end{DoxyPre}


There would also be the following bit arrays if Mask\-Arrays is on\-:


\begin{DoxyPre}
 .valid.name   .valid.initials   .valid.other   .valid.age
 ---------------------------------------------------------
 1             0                 0              1
 1             0                 0              0
 1             1                 1              0
 1             0                 0              1
 1             0                 0              0
 1             0                 0              1
 \end{DoxyPre}


To create an instance of class vtk\-X\-M\-L\-Tree\-Reader, simply invoke its constructor as follows \begin{DoxyVerb}  obj = vtkXMLTreeReader
\end{DoxyVerb}
 \hypertarget{vtkwidgets_vtkxyplotwidget_Methods}{}\subsection{Methods}\label{vtkwidgets_vtkxyplotwidget_Methods}
The class vtk\-X\-M\-L\-Tree\-Reader has several methods that can be used. They are listed below. Note that the documentation is translated automatically from the V\-T\-K sources, and may not be completely intelligible. When in doubt, consult the V\-T\-K website. In the methods listed below, {\ttfamily obj} is an instance of the vtk\-X\-M\-L\-Tree\-Reader class. 
\begin{DoxyItemize}
\item {\ttfamily string = obj.\-Get\-Class\-Name ()}  
\item {\ttfamily int = obj.\-Is\-A (string name)}  
\item {\ttfamily vtk\-X\-M\-L\-Tree\-Reader = obj.\-New\-Instance ()}  
\item {\ttfamily vtk\-X\-M\-L\-Tree\-Reader = obj.\-Safe\-Down\-Cast (vtk\-Object o)}  
\item {\ttfamily string = obj.\-Get\-File\-Name ()} -\/ If set, reads in the X\-M\-L file specified.  
\item {\ttfamily obj.\-Set\-File\-Name (string )} -\/ If set, reads in the X\-M\-L file specified.  
\item {\ttfamily string = obj.\-Get\-X\-M\-L\-String ()} -\/ If set, and File\-Name is not set, reads in the X\-M\-L string.  
\item {\ttfamily obj.\-Set\-X\-M\-L\-String (string )} -\/ If set, and File\-Name is not set, reads in the X\-M\-L string.  
\item {\ttfamily string = obj.\-Get\-Edge\-Pedigree\-Id\-Array\-Name ()} -\/ The name of the edge pedigree ids. Default is \char`\"{}edge id\char`\"{}.  
\item {\ttfamily obj.\-Set\-Edge\-Pedigree\-Id\-Array\-Name (string )} -\/ The name of the edge pedigree ids. Default is \char`\"{}edge id\char`\"{}.  
\item {\ttfamily string = obj.\-Get\-Vertex\-Pedigree\-Id\-Array\-Name ()} -\/ The name of the vertex pedigree ids. Default is \char`\"{}vertex id\char`\"{}.  
\item {\ttfamily obj.\-Set\-Vertex\-Pedigree\-Id\-Array\-Name (string )} -\/ The name of the vertex pedigree ids. Default is \char`\"{}vertex id\char`\"{}.  
\item {\ttfamily obj.\-Set\-Generate\-Edge\-Pedigree\-Ids (bool )} -\/ Set whether to use an property from the X\-M\-L file as pedigree ids (off), or generate a new array with integer values starting at zero (on). Default is on.  
\item {\ttfamily bool = obj.\-Get\-Generate\-Edge\-Pedigree\-Ids ()} -\/ Set whether to use an property from the X\-M\-L file as pedigree ids (off), or generate a new array with integer values starting at zero (on). Default is on.  
\item {\ttfamily obj.\-Generate\-Edge\-Pedigree\-Ids\-On ()} -\/ Set whether to use an property from the X\-M\-L file as pedigree ids (off), or generate a new array with integer values starting at zero (on). Default is on.  
\item {\ttfamily obj.\-Generate\-Edge\-Pedigree\-Ids\-Off ()} -\/ Set whether to use an property from the X\-M\-L file as pedigree ids (off), or generate a new array with integer values starting at zero (on). Default is on.  
\item {\ttfamily obj.\-Set\-Generate\-Vertex\-Pedigree\-Ids (bool )} -\/ Set whether to use an property from the X\-M\-L file as pedigree ids (off), or generate a new array with integer values starting at zero (on). Default is on.  
\item {\ttfamily bool = obj.\-Get\-Generate\-Vertex\-Pedigree\-Ids ()} -\/ Set whether to use an property from the X\-M\-L file as pedigree ids (off), or generate a new array with integer values starting at zero (on). Default is on.  
\item {\ttfamily obj.\-Generate\-Vertex\-Pedigree\-Ids\-On ()} -\/ Set whether to use an property from the X\-M\-L file as pedigree ids (off), or generate a new array with integer values starting at zero (on). Default is on.  
\item {\ttfamily obj.\-Generate\-Vertex\-Pedigree\-Ids\-Off ()} -\/ Set whether to use an property from the X\-M\-L file as pedigree ids (off), or generate a new array with integer values starting at zero (on). Default is on.  
\item {\ttfamily bool = obj.\-Get\-Mask\-Arrays ()} -\/ If on, makes bit arrays for each attribute with name .valid.\-attribute\-\_\-name for each attribute. Default is off.  
\item {\ttfamily obj.\-Set\-Mask\-Arrays (bool )} -\/ If on, makes bit arrays for each attribute with name .valid.\-attribute\-\_\-name for each attribute. Default is off.  
\item {\ttfamily obj.\-Mask\-Arrays\-On ()} -\/ If on, makes bit arrays for each attribute with name .valid.\-attribute\-\_\-name for each attribute. Default is off.  
\item {\ttfamily obj.\-Mask\-Arrays\-Off ()} -\/ If on, makes bit arrays for each attribute with name .valid.\-attribute\-\_\-name for each attribute. Default is off.  
\item {\ttfamily bool = obj.\-Get\-Read\-Char\-Data ()} -\/ If on, stores the X\-M\-L character data (i.\-e. textual data between tags) into an array named Char\-Data\-Field, otherwise this field is skipped. Default is off.  
\item {\ttfamily obj.\-Set\-Read\-Char\-Data (bool )} -\/ If on, stores the X\-M\-L character data (i.\-e. textual data between tags) into an array named Char\-Data\-Field, otherwise this field is skipped. Default is off.  
\item {\ttfamily obj.\-Read\-Char\-Data\-On ()} -\/ If on, stores the X\-M\-L character data (i.\-e. textual data between tags) into an array named Char\-Data\-Field, otherwise this field is skipped. Default is off.  
\item {\ttfamily obj.\-Read\-Char\-Data\-Off ()} -\/ If on, stores the X\-M\-L character data (i.\-e. textual data between tags) into an array named Char\-Data\-Field, otherwise this field is skipped. Default is off.  
\item {\ttfamily bool = obj.\-Get\-Read\-Tag\-Name ()} -\/ If on, stores the X\-M\-L tag name data in a field called .tagname otherwise this field is skipped. Default is on.  
\item {\ttfamily obj.\-Set\-Read\-Tag\-Name (bool )} -\/ If on, stores the X\-M\-L tag name data in a field called .tagname otherwise this field is skipped. Default is on.  
\item {\ttfamily obj.\-Read\-Tag\-Name\-On ()} -\/ If on, stores the X\-M\-L tag name data in a field called .tagname otherwise this field is skipped. Default is on.  
\item {\ttfamily obj.\-Read\-Tag\-Name\-Off ()} -\/ If on, stores the X\-M\-L tag name data in a field called .tagname otherwise this field is skipped. Default is on.  
\end{DoxyItemize}